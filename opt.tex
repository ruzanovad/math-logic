\documentclass[a4paper]{article}
\usepackage{cmap}
\usepackage[utf8]{inputenc}
\usepackage[T2A]{fontenc}
\usepackage{amsfonts}
\usepackage{amsmath, amsthm}
\usepackage{amssymb}
\usepackage{hyperref}
\usepackage{multicol}
\usepackage{xcolor}

\newcommand\letsymbol{\mathord{\sqsupset}}
\usepackage[russian]{babel}
\renewcommand\qedsymbol{$\blacktriangleright$}
\newtheorem{theorem}{Теорема}[section]
\newtheorem{lemma}{Лемма}[section]
\theoremstyle{definition}
\newtheorem*{example}{Пример}
\newtheorem*{definition}{Определение}
\newtheorem*{statement}{Утверждение}
\theoremstyle{remark}
\newtheorem*{remark}{Замечание}

\setlength{\topmargin}{-0.5in}
\setlength{\oddsidemargin}{-0.5in}
\textwidth 185mm
\textheight 250mm

\begin{document}
\tableofcontents
\section{Введение}
\begin{definition}[Методы оптимизации]
    раздел прикладной математики, содержание которого составляет теория и методы решения оптимизационных задач
\end{definition}
\begin{definition}[Оптимизационная задача]
    задача выбора наилучшего варианта (в некотором смысле) из имеющихся
\end{definition}
\begin{definition}[Задача оптимизации]
        $\begin{cases}
        f(x)\to \min(\max) \\
        x\in D
    \end{cases}$
\end{definition}
D - множество допустимых решений, $f:D\to \mathbb{R}$
\begin{definition}[Задача МП]
    $\begin{cases}
        (1) f(x)\to \min (\max) [extr] (opt)\\
        (2) g_i(x) \# 0, i=1,\dots, m - \text{ограничения} \\
        (3)x\in \mathbb{R}^n 
    \end{cases}$
    \(x = (x_1, ..., x_n) \, f(x): \mathbb{R}^n \to \mathbb{R}, \, g_i(x) : \mathbb{R}^n \to \mathbb{R}\)
\end{definition}
\begin{definition}[Допустимое решение]
    $x\in \mathbb{R}^n$, удовл (2), называется допустимым решением задачи.
\end{definition}
\begin{definition}[Оптимальное решение]
    Допустимое решение $x^*\in D$ задачи 1 - 3 называется оптимальным решением, если $f(x) \leq f(x^*) \, \forall x\in D$ в случае задачи максимизации и $f(x) \geq f(x^*) \, \forall x\in D$ в случае задачи минимизации
\end{definition}
Глобальный оптимум -  $x^*$
\begin{definition}[Локальный оптимум]
    Допустимое решение $\widetilde{x}\in D$ задачи 1 - 3 называется локальным оптимумом, если $f(x) \leq f(\widetilde{x})$ для всех х из некоторой окрестности $\widetilde{x}$ в случае задачи максимизации и $f(x) \geq f(\widetilde{x})$ для всех х из некоторой окрестности $\widetilde{x}$ в случае задачи минимизации
\end{definition}
\begin{definition}[Разрешимая/неразрешимая]
    Задача 1 - 3, которая обладает оптимальным решением, называется разрешимой, иначе неразрешимой
\end{definition}
\section{Линейное программирование}
\subsection{Постановка задачи (ЛП), теоремы эквивалентности}
\begin{definition}[Общая задача ЛП]
$    \begin{cases}
        f(x) = c_0 + \sum_{j = 1}^n c_j x_j \to \max (\min) \\ 
        \sum_{j = 1}^{n} a_{ij}x_j \# b_i, \, i = 1, \dots, m \\
        x_j \geq 0, j \in J\subseteq \{1, \dots, n\}
    \end{cases}$, где \(x = (x_1, ..., x_n)\in \mathbb{R}^n\) -  вектор переменных

    Матричная запись:

    $\begin{cases}
        f(x) = (c, x) \to \max(\min)\\
        Ax \# b \\
        x_j \geq 0, j \in J\subseteq \{1, \dots, n\}
    \end{cases}$, $x = \begin{pmatrix}
        x_1 \\ \vdots \\ x_n
        \end{pmatrix}, b = \begin{pmatrix}
            b_1 \\ \vdots \\ b_m
            \end{pmatrix}, 
            A = \begin{pmatrix} 
                a_{11} & \dots  & a_{1n}\\
    \vdots & \ddots & \vdots\\
    a_{m1} & \dots  & a_{mn} 
    \end{pmatrix}$
\end{definition}
\begin{definition}[Стандартная (симметрическая) форма]
    $\begin{cases}
        f(x) = c_0 + \sum_{j = 1}^n c_j x_j \to \max (\min) \\
        \sum_{j = 1}^{n} a_{ij}x_j \leq (\geq) b_i, \, i = 1, \dots, m \\ 
        x_j \geq 0, j = 1, \dots, n
    \end{cases}$
\end{definition}
\begin{definition}[КЗЛП]
    $\begin{cases}
        f(x) = c_0 + \sum_{j = 1}^n c_j x_j \to \max\\
        \sum_{j = 1}^{n} a_{ij}x_j =b_i, \, i = 1, \dots, m \\ 
        x_j \geq 0, j = 1, \dots, n
    \end{cases}$
\end{definition}
\begin{definition}[Основная задача ЛП]
    $\begin{cases}
        f(x) = c_0 + \sum_{j = 1}^n c_j x_j \to \max\\
        \sum_{j = 1}^{n} a_{ij}x_j \leq b_i, \, i = 1, \dots, m
    \end{cases}$
\end{definition}
\begin{definition}[Эквивалентные ЗЛП (ЗМП)]
    Две задачи ЛП $P_1, P_2$ называются \textit{эквивалентными}, если любому допустимому решению задачи $P_1$ соответствует некоторое допустимое решение задачи $P_2$ и наоборот, причем оптимальному решению одной задачи соответствует оптимальное решение другой задачи.
\end{definition}
\begin{theorem}[Первая теорема эквивалентности]
    Для любой ЗЛП существует эквивалентная ей каноническая ЗЛП.
\end{theorem}
\begin{theorem}[Вторая теорема эквивалентности]
    Для любой ЗЛП существует эквивалентная ей симметрическая ЗЛП.
\end{theorem}
\begin{theorem}[Критерий разрешимости]
    Если целевая функция задачи ЛП ограничена сверху (снизу), на непустом множестве допустимых решений, то задача максимизации (минимизации) имеет оптимальное решение
\end{theorem}
\subsection{Каноническая задача ЗЛП. Базисные решения}
\begin{definition}[Базисное решение]
    Пусть $\overline{x}$ - решение $Ax = B$. Тогда вектор $\overline{x}$ называется базисным решением СЛАУ, если система вектор-столбцов матрицы А, соответствующая компонентам вектора $\overline{x}$, ЛНЗ
\end{definition}
\begin{remark}
    Если система однородная, то x = $\overline{0}$ - базисное решение
\end{remark}
\begin{definition}
    Неотрицательное базисное решение СЛУ называется базисным решением канонической задачи ЛП
\end{definition}
\begin{definition}[Вырожденное БР]
    $\overline{x}$ - БР КЗЛП называется вырожденным, если число ненулевых компонент меньше ранга матрицы А, иначе невырожденное
\end{definition}
\begin{lemma}
    Если x и x' - Б.Р. КЗЛП, $x\neq x'$, то \[J(x) \neq J(x'), J(x)\subset J(x'), J(x) \supset J(x'),\] где $J(x) = \{j | x_j \neq 0, j = 1\dots n\}$
\end{lemma}
\begin{theorem}[О конечности множества базисных решений]
    Число базисных решений КЗЛП конечно
\end{theorem}
\begin{theorem}[О существовании оптимальных БР]
    Если КЗЛП разрешима, то существует ее оптимальное БР
\end{theorem}
\subsection{Симплекс-метод}
Рассмотрим КЗЛП.
\subsubsection{Симплекс-метод для приведенной ЗЛП}
\begin{definition}[Система с базисом]
    СЛАУ - СЛАУ с базисом, если в каждом уравнении имеется переменная с коэффициентом +1, отсутствующим в других уравнениях. Такие переменные будем называть базисными, остальные не базисными
\end{definition}
\begin{definition}[ПЗЛП]
    КЗЛП называется приведенной, если 
    \begin{enumerate}
        \item СЛАУ $Ax = B$ является системой с базисом
        \item Целевая функция выражена через небазисные переменные
    \end{enumerate}
\end{definition}
\begin{definition}[Прямо допустимая симплексная таблица]
    СТ называется прямо допустимой, если $a_{i0}\geq 0, i = 1, \dots, m$ (bшки)
\end{definition}
\begin{definition}[Двойственно допустимая симплексная таблица]
    СТ называется двойственно допустимой, если $a_{0j}\geq 0, i = 1, \dots, n+m$ (cшки)
\end{definition}
\begin{theorem}
    Если симплекс-таблица является прямо допустимой и $a_{0j}\geq 0, j = 1\dots, n+m$, то соответствующее базисное решение является оптимальным 
\end{theorem}
\begin{theorem}
    Если в симплекс-таблице существует $a_{0q}< 0, a_{iq}\leq 0, \, \forall i = 1\dots, m$, то задача неразрешима, потому что f неограничена на множестве допустимых решений
\end{theorem}
\begin{theorem}
    Если ведущая строка выбирается из условия минимума ключевого отношения, то следующаяя симплексная таблица будет прямо допустимой
\end{theorem}
\begin{theorem}[Об улучшении базисного решения]
    Если $\exists a_{0j}< 0, j = 1\dots n+m$, то возможен переход к новой прямо допустимой симплекс таблице, причем $f(x)\leq f(x')$, где x - БР старой таблицы, x'- БР новой таблицы,
    f(x) = $a_{00}$ старой таблицы, f(x') = $a_{00}-\frac{a_{p0}a_{0q}}{a{pq}}, a_{p0} = 0$ - вырожденное решение
\end{theorem}
\subsection{Каноническая ЗЛП}
\paragraph*{Метод искусственного базиса}
\begin{definition}[искусственные]
    $t_i\geq 0$ - искусственные переменные
\end{definition}
\begin{remark}[Свойства ВЗЛП]
    \begin{enumerate}
        \item ВЗЛП почти приведенная (нужно выразить $t_i$)
        \item $h(x, t) \leq 0 \quad \forall (x, t)\in \widetilde{D}$
        \item $\widetilde{D}\neq 0$ (например, есть $(0, \dots, n, b_1, \dots, b_m)$, n нулей)
        \item ВЗЛП всегда разрешима 
    \end{enumerate}
\end{remark}
\begin{theorem}[О существовании допустимого решения исходной КЗЛП]
    $$D\neq 0 \Leftrightarrow h^*(x, t)=0$$
\end{theorem}
\begin{theorem}[О преобразовании КЗЛП в эквивалентную ей приведенную]
    Если множество допустимых решений исходной КЗЛП непусто, то ПЗЛП, эквивалентная исходной КЗЛП, может быть получена из последней симплекс таблицы - таблицы ВЗЛП
\end{theorem}
\subsection{Двойственность в ЛП}
\begin{definition}
    Будем говорить, что знаки линейных ограничений ЗЛП согласованы с целевой функцией, если в задаче на max ограничения неравенства имеют вид "$\leq$", а в задаче на min ограничения на неравенство имеют вид "$\geq$"
\end{definition}
\begin{definition}[Двойственная задача]
    Для ЗЛП I двойственной задачей II является ЗЛП вида:
        $$f(x) = \sum_{j = 1}^n c_j x_j \to \max \leftrightarrow g(y) = \sum_{i = 1}^m b_i y_i\to \min,$$
        $$\sum_{j = 1}^n a_{ij} x_j \leq b_i, i = 1, \dots, l \leftrightarrow y_i\geq 0, i = 1...l,$$
        $$\sum_{j = 1}^n a_{ij} x_j = b_i, i = l+1, \dots m \leftrightarrow y_i\in \mathbb{R}, i = l+1, \dots, m,$$
        $$x_j\geq 0, i = 1, \dots p\leftrightarrow \sum_{i = 1}^m a_{ij} y_i \leq c_j, j = 1, \dots, p$$
        $$x_j \in \mathbb{R}, j = p+1, \dots n \leftrightarrow \sum_{i = 1}^m a_{ij} y_i \leq c_j, j = p+1, \dots, n$$
        Задачу I называют прямой, а II - двойственной. Стрелки соответствуют сопряженным ограничениям
\end{definition}
\begin{theorem}[Основное неравенство двойственности]
    $$\forall x\in D_I, \forall y\in D_{II}, f(x)\leq g(y)$$
\end{theorem}
\paragraph*{Теоремы двойственности}
\begin{lemma}[основная лемма]
    Пусть $\forall x\in D_I \neq \varnothing, f(x)\leq M < +\infty\implies \exists y \in D_{II}\, g(y)\leq M$
\end{lemma}
\begin{theorem}[Первая теорема двойственности]
    Если одна из пары двойственных задач разрешима, то разрешима и другая, причем оптимальное значение целевых функций совпадают, т.е $f(x^*) = g(y^*)$, где $x^*, y^*$ - оптимальные решения задач I, II соответственно
\end{theorem}
\begin{definition}[Условия дополняющей нежесткости]
    Будем говорить, что $x\in D_I, y \in D_{II}$ удовлетворяют УДН, если при подстановке в любую пару сопряженных неравенств хотя бы одно из них обращается в равенство. Это означает, что следующие характеристические произведения обращаются в 0:
    \[(\sum_{j = 1}^n a_{ij}x_j - b_i)y_i = 0, i  = 1, \dots m\]
    \[x_i (\sum_{i = 1}^m a_{ij}y_i - c_j) = 0, j = 1, \dots n\]
\end{definition}
\begin{theorem}[Вторая теорема двойственности]
    $x^* \in D_I, y^*\in D_{II}.$ оптимальны в задачах I, II тогда и только тогда, когда они удовлетворяют УДН.
\end{theorem}
\end{document}