\documentclass[a4paper, 12pt]{article}
\usepackage[utf8]{inputenc}
\usepackage[T2A]{fontenc}
\usepackage{amsfonts}
\usepackage{amsmath, amsthm}
\usepackage{amssymb}
\usepackage{hyperref}

\newcommand\letsymbol{\mathord{\sqsupset}}
\usepackage[russian]{babel}
\renewcommand\qedsymbol{$\blacktriangleright$}
\newtheorem{theorem}{Теорема}[section]
\newtheorem{lemma}{Лемма}[section]
\theoremstyle{definition}
\newtheorem*{example}{Пример}
\newtheorem*{definition}{Определение}
\theoremstyle{remark}
\newtheorem*{remark}{Замечание}


\setlength{\topmargin}{-0.5in}
\setlength{\oddsidemargin}{-0.5in}
\textwidth 185mm
\textheight 250mm

\begin{document}
\section*{Венгерский метод}
    \begin{enumerate}
        \item Найдем минимальный элемент каждой из строк и вычтем его из элементов этой строки
        Матрица С станет неотрицательной
        \item Найдем мнимальный элемент каждого столбца и вычтем его из элементов этого столбца
        \item Найдем наименьшее возможное число "линий" (строк и столбцов), вычеркнув которые из матрицы
        С, мы вычеркнем все нули из этой матрицы
        \begin{enumerate}
            \item Находим строку с минимальным количеством неотмеченных нулей. 
            Закрашиваем один из нулей

            В строке и столбце, соответствующим этому нулю, вычеркнем все остальные нули.

            Если в таблице еще есть незакрашенные/неотмеченные нули, повторяем процесс

            Если строк с невычеркнутыми нулями не осталось, выделяем все строки, не содержащие
            закрашенных нулей (СТРЕЛОЧКА)
            \item Вычеркнем те столбцы (ежик), которые содержат вычеркнутые нули в выделенной строке (стрелочкой)
            \item Выделим (стрелочкой) те строки, которые содержат закрашенный 0 в (ежике)

            Повторяем пункты 2-3 до тех пор, пока появляются новые выделенные строки и закрашенные столбцы
            \item Вычеркнем (ежиком) все невыделенные знаком (стрелочка) строки

            k = сумма ежиков
        \end{enumerate}
        
        Если k = n, то существует нулевое решение и оно оптимально для исходной матрицы, stop.

        Если k < n, то нулевое решение не существует
        \item (k < n)
        Есть элементы следующие:
        \begin{enumerate}
            \item невычеркнутые
            \item вычеркнутые один раз
            \item вычеркнутые 2 раза
        \end{enumerate}
        $\delta = $минимальный элемент из невычеркнутых.

        Из 1) отнимаем дельта

        К 3) прибавляем дельта

        2) не меняем
        
        переходим на шаг 3
    \end{enumerate}
\end{document}