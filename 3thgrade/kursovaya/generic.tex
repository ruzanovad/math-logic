\documentclass[a4paper, 12pt]{article}
\usepackage[utf8]{inputenc}
\usepackage[T2A]{fontenc}
\usepackage{amsfonts}
\usepackage{amsmath, amsthm}
\usepackage{amssymb}
\usepackage{hyperref}
% \usepackage{mathrm}
% \usepackage{bbold}
% \usepackage{mathrsfs}

\usepackage[russian]{babel}
\renewcommand\qedsymbol{$\blacktriangleright$}
\newtheorem{theorem}{Теорема}
\newtheorem{lemma}{Лемма}
\theoremstyle{definition}
\newtheorem{example}{Пример}
\newtheorem{problem}{Упражнение}[section]
\newtheorem*{definition}{Определение}
\theoremstyle{remark}
\newtheorem*{remark}{Замечание}
\newcommand\letsymbol{\mathord{\sqsupset}}

\DeclareMathOperator{\degg}{deg}

\setlength{\topmargin}{-0.5in}
\setlength{\oddsidemargin}{-0.5in}
\textwidth 185mm
\textheight 250mm
\def\lc{\left\lfloor}
\def\rc{\right\rfloor}
\begin{document}
\section*{Проблема поиска 3-раскраски $\mathcal
G_3$}
Задан неориентированный граф без кратных ребер и петель $G = (V, E)$ c n вершинами. Гарантировано,
что в нем существует 3-раскраска.
\\
Необходимо найти
множества $V_1, V_2, V_3$ такие, что $V_1\sqcup V_2 \sqcup V_3 = V$ и
$\forall x, y\in V_i, i\in \{1, 2, 3\} \implies (x, y)\notin E$

\section*{Подпроблема поиска 3-раскраски $\mathcal
G_3(\gamma)$}
Рассмотрим бесконечную последовательность графов $\gamma = \{G_1, G_2, ..., G_n, ...\}$
такую, что $G_n$ имеет n вершин $\forall n\in \mathbb N$.

Для каждой последовательности графов $\gamma$ определим подпроблему поиска 3-раскраски
как ограничение исходной проблемы на множество входов $\{G: G\cong G_n, G_n \in \gamma\}$.

\begin{lemma}
	Если не существует полиномиального вероятностного алгоритма для
	решения проблемы $\mathcal
G_3$, то найдется последовательность графов $\gamma$ такая,
	что не существует полиномиального вероятностного алгоритма для решения
	проблемы $\mathcal
G_3(\gamma)$.
\end{lemma}

\begin{proof}
	Пусть $P_1, P_2, \dots$ — все полиномиальные вероятностные алгоритмы. Если не существует полиномиального вероятностного алгоритма для проблемы
	$\mathcal
 G_3$, то для любого вероятностного полиномиального алгоритма $P_n$ найдётся бесконечно много графов, для которых $P_n$ не может решить $\mathcal
 G_3$. Из этого следует, что
	можно выбрать такую последовательность $\gamma'
		= \{G_1, G_2, \dots , G_n, \dots\}$, что алгоритм $P_n$
	не может решить $\mathcal
 G_3$ для $G_n$ для всех n. Более того, $\gamma'$
	упорядочена по возрастанию
	числа вершин в графах. Теперь можно расширить последовательность $\gamma'$ до последовательности графов $\gamma$ с графами $G_n$ для всех размеров n. Из построения $\gamma$ следует,
	что не существует полиномиального вероятностного алгоритма для решения проблемы $\mathcal
 G_3(\gamma$).
\end{proof}

Для изучения генерической сложности проблемы 3-раскраски графов будем использовать представление графов с помощью матриц смежности. Под размером графа будем понимать число вершин.

\begin{theorem}
	Пусть $\gamma$ - произвольная последовательность графов. Если существует генерический полиномиальный алгоритм, решающий проблему $\mathcal
 G_3(\gamma)$, то существует вероятностный полиномиальный алгоритм, решающий эту проблему на всём множестве входов.
\end{theorem}
\begin{proof}
	Допустим, что существует генерический полиномиальный алгоритм $A$, решающий проблему 3-раскраски графов $\mathcal G_3(\gamma)$.
	Построим вероятностный полиномиальный алгоритм $B$, решающий эту проблему на всем множестве входов. На графе $G$ с $n$ вершинами алгоритм $B$ работает следующим образом:

	\begin{enumerate}
		\item Запускает алгоритм $A$ на G.
		\item Если $A(G)\neq\  ?
$, то $B$ выдает ответ $A(G)$ и останавливается, иначе идёт на шаг 3.
		\item Генерирует случайно и равномерно перестановку $\pi$ на вершинах $\{1,\dots,n\}$ и вычисляет граф $G' = \pi(G)$.
		\item Запускает алгоритм $A$ на $(G')$.
		\item Если $A(G') =\  ?
$, выдает $
			      V_1 = \{1, 2, ..., \lc n/3 \rc\},
			      V_2 = \{\lc n/3\rc +1, ... \lc n\cdot 2/3\rc \},
			      V_3 = \{\lc n\cdot 2/3 \rc +1 , \dots, n\}$ - возможно, неправильный.
		\item Пусть $A(G') = \{V_1, V_2, V_3\}$ --- решение задачи 3-раскраски для графа $G'$. Тогда
		      \[\pi(V) = \pi^{-1}(V_1) \sqcup \pi^{-1} (V_2) \sqcup \pi^{-1} (V_3)\]
		      является решением задачи 3-раскраски для исходного графа $G = \pi^{-1} (G')$.
	\end{enumerate}
	Для доказательства корректности работы вероятностного алгоритма надо показать,
	что вероятность того, что $A(G') =\  ?
$, меньше 1/3. Заметим, что $\pi (G)$ при варьировании
	перестановки $\pi$ пробегает всё множество входов размера n. Множество $\{G : A(G) =\  ?
\}$
	пренебрежимо, поэтому вероятность того, что $A(G'
		) =\  ?
$, стремится к 0 при увеличении n.
\end{proof}
\begin{theorem}
	Если P $\neq$ NP и P = BPP, то существует последовательность графов $\gamma$ такая, что для решения проблемы 3-раскраски $\mathcal
 G_3(\gamma)$ не существует генерического полиномиального алгоритма.
\end{theorem}
\begin{proof}
	Покажем сначала, что при условиях P$\neq$ NP и P = BPP не существует полиномиального вероятностного алгоритма для решения проблемы $\mathcal
 G_3$. Действительно, пусть такой алгоритм существует. Так как проблема $\mathcal
 G_3$ является NP-трудной, то существует полиномиально эквивалентная ей NP-проблема распознавания A. Из полиномиального вероятностного алгоритма для $G_3$ легко получается полиномиальный вероятностный алгоритм для решения проблемы $A$. А так как P = BPP, то существует и детерминированный полиномиальный алгоритм для $A$, откуда P = NP. Противоречие.

	Теперь нужное утверждение следует из леммы 1 и теоремы 1.
\end{proof}
\end{document}