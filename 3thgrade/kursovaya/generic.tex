\documentclass[a4paper, 12pt]{article}
\usepackage[utf8]{inputenc}
\usepackage[T2A]{fontenc}
\usepackage{amsfonts}
\usepackage{amsmath, amsthm}
\usepackage{amssymb}
\usepackage{hyperref}
% \usepackage{mathrm}
\usepackage{bbold}
\usepackage{mathrsfs}


\usepackage[russian]{babel}
\renewcommand\qedsymbol{$\blacktriangleright$}
\newtheorem*{theorem}{Утверждение}
\newtheorem{lemma}{Лемма}
\theoremstyle{definition}
\newtheorem{example}{Пример}
\newtheorem{problem}{Упражнение}[section]
\newtheorem*{definition}{Определение}
\theoremstyle{remark}
\newtheorem*{remark}{Замечание}
\newcommand\letsymbol{\mathord{\sqsupset}}


\DeclareMathOperator{\degg}{deg}

\setlength{\topmargin}{-0.5in}
\setlength{\oddsidemargin}{-0.5in}
\textwidth 185mm
\textheight 250mm

\begin{document}
\section*{Проблема поиска 3-раскраски $\mathsf{G_3}$}
Задан неориентированный граф без кратных ребер и петель $G = (V, E)$ c n вершинами. Гарантировано,
что в нем существует 3-раскраска.
\\
Надо найти
    множества $V_1, V_2, V_3$ такие, что $V_1\sqcup V_2 \sqcup V_3 = V$ и
    $\forall x, y\in V_i, i\in \{1, 2, 3\} \implies (x, y)\notin E$

\section*{Подпроблема поиска 3-раскраски $\mathsf{G_3}(\gamma)$}
Рассмотрим бесконечную последовательность графов $\gamma = \{G_1, G_2, ..., G_n, ...\}$
такую, что $G_n$ имеет n вершин $\forall n\in \mathbb N$

Для каждой последовательности графов $\gamma$ определим подпроблему поиска 3-раскраски
как ограничение исходной проблемы на множество входов $\{G: G\cong G_n, G_n \in \gamma\}$.

\begin{lemma}
    Если не существует полиномиального вероятностного алгоритма для
решения проблемы $\mathsf{G_3}$, то найдется последовательность графов $\gamma$, такая,
 что не существует полиномиального вероятностного алгоритма для решения
  проблемы $\mathsf{G_3}(\gamma)$.
\end{lemma}

\begin{proof}
    Пусть $P_1, P_2, \dots$ — все полиномиальные вероятностные алгоритмы. Если не существует полиномиального вероятностного алгоритма для проблемы
GC6k, то для любого вероятностного полиномиального алгоритма Pn найдётся бесконечно много графов, для которых Pn не может решить GC6k. Из этого следует, что
можно выбрать такую последовательность $\gamma'
 = \{G1, G2, \dots , Gn, \dots\}$, что алгоритм $P_n$
не может решить GC6k для Gn для всех n. Более того, $\gamma$
0 упорядочена по возрастанию
числа вершин в графах. Теперь можно расширить последовательность $\gamma$
0 до последовательности графов $\gamma$ с графами Gn для всех размеров n. Из построения $\gamma$ следует,
что не существует полиномиального вероятностного алгоритма для решения проблемы GC6k($\gamma$).
\end{proof}
\end{document}