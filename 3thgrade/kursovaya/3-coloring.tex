\documentclass[a4paper, 12pt]{article}
\usepackage[utf8]{inputenc}
\usepackage[T2A]{fontenc}
\usepackage{amsfonts}
\usepackage{amsmath, amsthm}
\usepackage{amssymb}
\usepackage{hyperref}
% \usepackage{mathrm}
\usepackage{bbold}
\usepackage{mathrsfs}


\usepackage[russian]{babel}
\renewcommand\qedsymbol{$\blacktriangleright$}
\newtheorem*{theorem}{Теорема}
\newtheorem{lemma}{Лемма}
\theoremstyle{definition}
\newtheorem{example}{Пример}
\newtheorem{problem}{Упражнение}[section]
\newtheorem*{definition}{Определение}
\theoremstyle{remark}
\newtheorem*{remark}{Замечание}
\newcommand\letsymbol{\mathord{\sqsupset}}


\DeclareMathOperator{\degg}{deg}

\setlength{\topmargin}{-0.5in}
\setlength{\oddsidemargin}{-0.5in}
\textwidth 185mm
\textheight 250mm

\begin{document}
\section*{3-раскраска графа}
\textbf{Вход:} неориентированный граф без кратных ребер и петель $G = (V, E)$ c n вершинами.
\\
\textbf{Выход:} $\begin{cases}
    1,\text{если существует правильная 3-раскраска графа,}\\
    0,\text{ иначе.}
\end{cases}$
\subsection*{Предлагаемый алгоритм} \label{algo}
Если граф является надграфом $K_4$, то возвращаем 0. \\
Иначе "?".
\begin{theorem}
    Алгоритм \ref{algo} является генерическим.
\end{theorem}
\begin{proof}
    Граф $K_4$ нельзя раскрасить в 3 цвета, значит, если он является подграфом некоторого графа, то сам
    граф также нельзя раскрасить тремя цветами.
    
    Пусть $D_i$ - это множество графов размера n,
    которые являются надграфами $K_4$, образованного сочетанием вершин,
    $i = \overline{1, C_n^4}$.
    Тогда $I_n = \bigcup_{i = 1}^{C_n^4} D_i$.

    Воспользуемся формулой включений-исключений:

    $$|I_n| = \sum |D_i| - \sum_{i< j} |D_i\cap D_j| + \dots + (-1)^{C_n^4-1} |D_1 \cap \dots \cap D_{C_n^4}|$$
    
    $|D_i| = 2^{\frac{n(n-1)}{2}-6}$ (6 ребер уже определены выбором 4 вершин для $K_4$)

    % Количество таких пар - $C_{C_n^4}^2$ 
    Для вычисления $|D_i\cap D_j|$ рассмотрим следующие случаи:
    
    \begin{enumerate}
        \item Множества вершин, соответствующих $D_i$ и $D_j$ (обозначим их $V_i, V_j$) не пересекаются.
        Тогда будут однозначно определены $2\cdot 6 $ ребер. Количество пар множеств - $\frac{C_n^4C_{n-4}^4}{2}$
        \item $|V_i\cap V_j|$ = 1. Аналогично предыдущему пункту. Количество пар множеств - $\frac{C_n^1C_{n-1}^3 C_{n-4}^3}{2}$
        \item $|V_i\cap V_j|$ = 2. Однозначно определены $6+5$ ребер. Количество пар множеств - $\frac{C_n^2C_{n-2}^2 C_{n-4}^2}{2}$
        \item $|V_i\cap V_j|$ = 3. Остается соединить только одну вершину с другими тремя. Тем самым, получаем $6+3$ ребра, которые определены
        однозначно. Количество пар множеств - $\frac{C_n^3C_{n-3}^1 C_{n-4}^1}{2}$
    \end{enumerate}
    Итого получаем: $$\sum_{i< j} |D_i\cap D_j| = \frac{C_n^4C_{n-4}^4}{2} \cdot 2^{(n-1)n/2 - 6 - 6} + \frac{C_n^1C_{n-1}^3 C_{n-4}^3}{2}  \cdot 2^{(n-1)n/2 - 6 - 6} + $$
    $$  + \frac{C_n^2C_{n-2}^2 C_{n-4}^2}{2} \cdot 2^{(n-1)n/2 - 6 - 5}  + \frac{C_n^3C_{n-3}^1 C_{n-4}^1}{2} \cdot 2^{(n-1)n/2 - 6 - 3} = $$
    $$ = |D_i| (\frac{C_n^4C_{n-4}^4}{2} \cdot 2^{-6} + \frac{C_n^1C_{n-1}^3 C_{n-4}^3}{2}  \cdot 2^{-6} + $$
    $$+ \frac{C_n^2C_{n-2}^2 C_{n-4}^2}{2} \cdot 2^{- 5}  + \frac{C_n^3C_{n-3}^1 C_{n-4}^1}{2} \cdot 2^{- 3})$$

\end{proof}
\end{document}