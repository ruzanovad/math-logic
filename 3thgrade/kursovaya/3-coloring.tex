\documentclass[a4paper, 12pt]{article}
\usepackage[utf8]{inputenc}
\usepackage[T2A]{fontenc}
\usepackage{amsfonts}
\usepackage{amsmath, amsthm}
\usepackage{amssymb}
\usepackage{hyperref}
% \usepackage{mathrm}
\usepackage{bbold}
\usepackage{mathrsfs}


\usepackage[russian]{babel}
\renewcommand\qedsymbol{$\blacktriangleright$}
\newtheorem{theorem}{Теорема}[section]
\newtheorem{lemma}{Лемма}[section]
\theoremstyle{definition}
\newtheorem{example}{Пример}
\newtheorem{problem}{Упражнение}[section]
\newtheorem*{definition}{Определение}
\theoremstyle{remark}
\newtheorem*{remark}{Замечание}
\newcommand\letsymbol{\mathord{\sqsupset}}


\DeclareMathOperator{\degg}{deg}

\setlength{\topmargin}{-0.5in}
\setlength{\oddsidemargin}{-0.5in}
\textwidth 185mm
\textheight 250mm

\begin{document}
\section*{3-раскраска графа}
\textbf{Вход:} неориентированный граф без кратных ребер и петель $G = (V, E)$ c n вершинами.
\\
\textbf{Выход:} $\begin{cases}
    1,\text{если существует правильная 3-раскраска графа,}\\
    0,\text{ иначе.}
\end{cases}$
\subsection*{Генерический алгоритм}
\begin{enumerate}
    \item Сравнить $|E|$ и $\frac32|V|=\frac32 n$.

    Если $|E| \ge \frac32 n$, то правильной 3-раскраски не существует.
    
    Иначе ответ - "не знаю"
\end{enumerate}
\begin{proof}
    $|I_n \setminus (I_n\cap S)|$ - множество графов такое, что $\forall v\in V \degg(v) \ge 3$, т.е для них не существует правильной 3-раскраски.

    Для множества S: $2|E| = \sum_{v\in V} \degg (v) < 3n$.
    \[S = \{G = (V, E) \;|\; |E| <\frac32 |V|\} = \{M \;|\; \text{число единиц в матрице} < 3|V|\}\]

    \[S_1 = \{G = (V, E) \;|\; \text{множество матриц (смежности) такое, что в каждой строчке}\]
    \[\text{единиц правее главной диагонали меньше или равно 2}\}\]
    Заметим, что $S_1 \supseteq S$.

    Мы рассматриваем симметричные матрицы смежности (т.к неориентированный граф). Тем самым:
    
    \[|I_n| = 2^{\frac{(n-1)n}{2}}\]
    \[|I_n\cap S_1| = |2\prod_{i=1}^{n-2} ((n-i)(n-i-1) + n - i +1)|\]

    \[ \rho (S_1) = \lim_{n\to \infty} \frac{2\prod_{i=1}^{n-2}((n-i)(n-i-1) + n - i +1)}{2^{\frac{(n-1)n}{2}}}\]

    Заметим, что сверху полином $n^{2(n-2)}$ степени, поэтому предел равен 0.

    (т.к $\lim_{n\to \infty}\frac{n}{2^n} = 0$)

    Подмножество пренебрежимого множества пренебрежимо, тем самым $S$ также пренебрежимо. Алгоритм является генерическим.
\end{proof}
\end{document}