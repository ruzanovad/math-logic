\documentclass[a4paper, 12pt]{article}
\usepackage[utf8]{inputenc}
\usepackage[T2A]{fontenc}
\usepackage{amsfonts}
\usepackage{amsmath, amsthm}
\usepackage{amssymb}
\usepackage{hyperref}
% \usepackage{mathrm}
\usepackage{bbold}
\usepackage{mathrsfs}


\usepackage[russian]{babel}
\renewcommand\qedsymbol{$\blacktriangleright$}
\newtheorem{theorem}{Теорема}[section]
\newtheorem{lemma}{Лемма}[section]
\theoremstyle{definition}
\newtheorem{example}{Пример}
\newtheorem{problem}{Упражнение}[section]
\newtheorem*{definition}{Определение}
\theoremstyle{remark}
\newtheorem*{remark}{Замечание}
\newcommand\letsymbol{\mathord{\sqsupset}}

\setlength{\topmargin}{-0.5in}
\setlength{\oddsidemargin}{-0.5in}
\textwidth 185mm
\textheight 250mm

\begin{document}
\section*{3-раскраска графа}
\textbf{Вход:} неориентированный граф без кратных ребер и петель $G = (V, E)$ c n вершинами.
\\
\textbf{Выход:} $\begin{cases}
    1,\text{если существует правильная 3-раскраска графа,}\\
    0,\text{ иначе.}
\end{cases}$
\subsection*{Генерический алгоритм}
\begin{enumerate}
    \item Сравнить $|E|$ и $\frac32|V|=\frac32 n$.

    Если $|E| \ge \frac32 n$, то правильной 3-раскраски не существует.
    
    Иначе ответ - "не знаю".
\end{enumerate}
\begin{proof}
    \[S = \{G = (V, E) \;|\; |E| <\frac32 n\} = \{M \;|\; \text{число единиц <}\frac{3n}{2}\}\]

    Мы рассматриваем верхнетреугольные матрицы (т.е неориентированный граф). Тем самым,
    
    \[|I_n| = 2^{\frac{(n-1)n}{2}}\]
    \[|S \cap I_n| = \sum_{i = 0}^{\lceil \frac{3n}{2} - 1 \rceil} C_{\frac{(n-1)n}{2}}^{i}\]
    \[\rho (S) = \lim_{n\to\infty} \frac{|S \cap I_n|}{|I_n|} = \lim_{n\to \infty} \frac{\sum_{i = 0}^{\lceil \frac{3n}{2} - 1 \rceil} C_{\frac{(n-1)n}{2}}^{i}}{2^{\frac{(n-1)n}{2}}} = \lim_{n\to\infty} \sum_{i=0}^{\lceil \frac{3n}{2} - 1 \rceil}\frac{(\frac{(n-1)n}{2})!}{2^{\frac{(n-1)n}{2}}(\frac{(n-1)n}{2} - i)!(i)! }\]
    Воспользуемся формулой Стирлинга: $n! \sim \sqrt{2 \pi n} \left(\frac{n}{e}\right)^n$
    \[\rho (S) = \lim_{n\to\infty} \sum_{i=0}^{\lceil \frac{3n}{2} - 1 \rceil} \frac{\sqrt{2\pi \frac{(n-1)n}{2}} {(\frac{\frac{(n-1)n}{2}}{e})}^{\frac{(n-1)n}{2}}}{2^{\frac{(n-1)n}{2}} \sqrt{2\pi (\frac{(n-1)n}{2} - i)} {(\frac{\frac{(n-1)n}{2} - i}{e})}^{\frac{(n-1)n}{2} - i}      \sqrt{2\pi i} {(\frac{i}{e})}^{i}} =\]

    \[= \lim_{n\to\infty} \sum_{i=0}^{\lceil \frac{3n}{2} - 1 \rceil} \frac{\sqrt{\frac{(n-1)n}{2}} {({\frac{(n-1)n}{2}})}^{\frac{(n-1)n}{2}} }{2^{\frac{(n-1)n}{2}} \sqrt{2\pi (\frac{(n-1)n}{2} - i)} {({\frac{(n-1)n}{2} - i})}^{\frac{(n-1)n}{2} - i}      \sqrt{i} {({i})}^{i}} = \]

    \[=\lim_{n\to\infty} \sum_{i=0}^{\lceil \frac{3n}{2} - 1 \rceil} \frac{\sqrt{\frac{(n-1)n}{2}} {({\frac{(n-1)n}{2}})}^{\frac{(n-1)n}{2}} {({\frac{(n-1)n}{2} - i})}^{{i}}}{2^{\frac{(n-1)n}{2}} \sqrt{2\pi (\frac{(n-1)n}{2} - i)} {({\frac{(n-1)n}{2} - i})}^{\frac{(n-1)n}{2} }      \sqrt{i} {({i})}^{i}} =\]

    \[=\lim_{n\to\infty} \sum_{i=0}^{\lceil \frac{3n}{2} - 1 \rceil}  \frac{  {({(\frac{\frac{(n-1)n}{2}}{i}) - 1})}^{{i}}}{2^{\frac{(n-1)n}{2}} \sqrt{2\pi (1 - (\frac{i}{\frac{(n-1)n}{2}}))} {({1 - (\frac{i}{\frac{(n-1)n}{2}})})}^{\frac{(n-1)n}{2} }      \sqrt{i} } =\]
    
    \[=\lim_{n\to\infty} \sum_{i=0}^{\lceil \frac{3n}{2} - 1 \rceil} \frac{  e^{i}{({(\frac{\frac{(n-1)n}{2}}{i}) - 1})}^{{i}}}{2^{\frac{(n-1)n}{2}} \sqrt{2\pi}       \sqrt{i} } = 0\]
    (т.к $\lim_{n\to \infty}\frac{n}{2^n} = 0$,$\lim_{n\to \infty} \frac{e^n n^n}{2^{n^2}} = 0$)
\end{proof}
\end{document}