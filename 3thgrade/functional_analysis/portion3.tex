\documentclass[a4paper, 12pt]{report}
\usepackage[utf8]{inputenc}
\usepackage[T2A]{fontenc}
\usepackage{amsfonts}
\usepackage{amsmath, amsthm}
\usepackage{amssymb}
\usepackage{hyperref}

\newcommand\letsymbol{\mathord{\sqsupset}}
\usepackage[russian]{babel}
\renewcommand\qedsymbol{$\blacktriangleright$}
\newtheorem{theorem}{Теорема}[section]
\newtheorem{lemma}{Лемма}[section]
\theoremstyle{definition}
\newtheorem*{example}{Пример}
\newtheorem*{definition}{Определение}
\theoremstyle{remark}
\newtheorem*{remark}{Замечание}


\DeclareMathOperator{\Int}{int}
\DeclareMathOperator{\clo}{cl}


\setlength{\topmargin}{-0.5in}
\setlength{\oddsidemargin}{-0.5in}
\textwidth 185mm
\textheight 250mm

\begin{document}
\begin{large}
    Рассмотрим $f: X \to Y$, где $X = \mathbb{R}$ с дискретной топологией и $Y = \mathbb R$ с естественной на числовой прямой топологией.
    $f = id$. $f$ -- биекция.
\begin{enumerate}


    % \item[21] Привести пример непрерывной биекции, не являющейся гомеоморфизмом.


    
    \item[61] Решить интегральные уравнения:
    \[x(t) = \mu \int_0^1 e^{t-s} x(s) ds + t^2, \quad x(t) = \mu \int_0^1 t^m s^n x(s) ds + t^k\]
    Ядра обоих интегральных уравнений вырождены.
    \begin{enumerate}
        \item \[e^{t-s}x(s) = (e^t)(e^{-s}x(s))\]
        \[x(t) = \mu e^t\int_0^1 e^{-s}x(s) ds + t^2\]
        Пусть $C = \int_0^1 e^{-s}x(s) ds$. Тогда:
        \[x(t) = C\mu e^t + t^2\]
        \[C = \int_0^1 e^{-s} (\mu e^s C+s^2)ds = \mu C + \int_0^1 s^2e^{-s}ds =  \mu C + e - 2\]\
        \[C = \frac{2-5e^{-1}}{1-\mu}\]
        \[x(t) = \mu \frac{2-5e^{-1}}{1-\mu} e^t + t^2, \text{при } \mu = 1\text{уравнение неразрешимо.}\]
    
        \item \[x(t) = \mu \int_0^1 t^m s^n x(s) ds + t^k\]
        \[C = \int_0^1 s^n x(s) ds\]
        \[x(t) = \mu C t^m + t^k\]
        \[C = \int_0^1 s^n (\mu C s^m + s^k)ds = \mu C \int_0^1 s^{n+m} ds + \int_0^1 s^{k+n} ds  = \frac{\mu C}{n+m+1} + \frac{1}{k+n+1}\]
        \[C = \frac{n+m+1}{(k+n+1)(n+m+1-\mu)}\]
        \[x(t) = \frac{\mu(n+m+1)t^m}{(k+n+1)(n+m+1-\mu)} + t^k\]
        при $\mu = m+n+1$ уравнение неразрешимо.
    \end{enumerate}
    \item[62] Решить интегральные уравнения:
    \[x(t) = \mu \int_0^\pi \cos(t+s) x(s) ds + 2\sin t, \quad x(t) = \int_0^t e^{t-s} x(s) ds + t +1\]
    \begin{enumerate}
        \item \[x(t) = \mu \int_0^\pi \cos(t+s) x(s) ds + 2\sin t\]
        \[\cos(t+s) = \frac{e^{-is-it} + e^{is+it}}{2} = \frac{e^{-is}e^{-it} + e^{is}e^{it}}{2}\]
        \[x(t) = \frac{\mu}{2} \int_0^\pi e^{-is}e^{-it}x(s)ds + \frac{\mu}{2} \int_0^\pi e^{is}e^{it} x(s) ds + 2\sin t\]
        \[C_1 = \int_0^\pi e^{-is}x(s)ds, C_2 = \int_0^\pi e^{is} x(s) ds\]
        \[x(t) = \frac{\mu}{2}C_1 e^{-it} + \frac{\mu}{2} C_2 e^{it} + 2\sin t\]
        \[\begin{cases}
            C_1 = \int_0^\pi e^{-is} (\frac{\mu}{2}C_1 e^{-is} + \frac{\mu}{2} C_2 e^{is} + 2\sin s)ds \\
            C_2 = \int_0^\pi e^{is} (\frac{\mu}{2}C_1 e^{-is} + \frac{\mu}{2} C_2 e^{is} + 2\sin s) ds
        \end{cases}\]
        \[\begin{cases}
            C_1 = \int_0^\pi \frac{\mu}{2}C_1 e^{-2is} + \frac{\mu}{2} C_2 + 2e^{-is} \sin s ds \\
            C_2 = \int_0^\pi \frac{\mu}{2}C_1 + \frac{\mu}{2} C_2 e^{2is} + 2e^{is} \sin s ds
        \end{cases}\]
        \[\begin{cases}
            C_1 = \frac{\mu}{2} C_2 \pi + \int_0^\pi \frac{\mu}{2}C_1 e^{-2is} + 2e^{-is} \sin s ds \\
            C_2 = \frac{\mu}{2}C_1 \pi + \int_0^\pi \frac{\mu}{2} C_2 e^{2is} + 2e^{is} \sin s ds
        \end{cases}\]
        \[\sin s = i \frac{e^{-i s}- e^{is}}{2}, e^{-is}\sin s = i\frac{e^{-2is}- 1}{2}, e^{is} \sin s = i\frac{1- e^{-2is}}{2}\]
        \[2\int_0^\pi e^{-is}\sin s ds = -i\pi, 2\int_0^\pi e^{is}\sin s ds = i\pi\]
        \[\begin{cases}
            C_1 = \frac{\mu}{2} C_2 \pi -i\pi \\
            C_2 = \frac{\mu}{2} C_1 \pi +i\pi,
        \end{cases} C_1 = -\frac{2i\pi}{2 + \mu \pi}, C_2 = \frac{2i\pi}{2 + \mu \pi}\]
        \[x(t) = -\frac{\mu i\pi}{2 + \mu \pi} e^{-it} + \frac{\mu i\pi}{2 + \mu \pi} e^{it} + 2\sin t = \frac{\mu i\pi}{2 + \mu \pi}2i \sin t + 2\sin t = \frac{4\sin t}{2+\mu \pi}\]
        \item \[x(t) = \int_0^t e^{t-s} x(s) ds + t +1\]
            \[x'(t) = \int_0^t e^{t-s} x(s) ds +1 + x(t)\]
            \[x'(t)- x(t) =x(t) - t\]
            \[x'(t) - 2x(t) = -t\]
            \[x(t) = C e^{2t} +\frac{t}{2} + \frac{1}{4}\]
            Значение константы найдем из начального условия $x(0) = 1$:
            \[1 = C + \frac{1}{4}\]
            \[C = \frac{3}{4}\]
            \[x(t) = \frac{3}{4}e^{2t} +\frac{t}{2} + \frac{1}{4}\]
    \end{enumerate}
    \item[63] Сведением к дифференциальному уравнению решить уравнение Вольтерра 1-го рода
    \[\int_0^t  \cos(t - s) x(s) ds = \sinh t\]
    \[\cosh t = \int_0^t \sin(t-s)x(s) ds + \cos(0)x(t)\]
    \[\sinh t = -\int_0^t \cos(t-s)x(s) dx + x'(t)\]
    \[x'(t) = 2\sinh t\]
    \[x(t) = 2\cosh t + c\]
    
    Начальное условие определяем из $\int_0^0 \sin(0-s) x(s) ds + x(0)= \cosh(0) = 1$ 
    \[x(0) = 2\cosh(0) + c = 1\]
    \[c=-1\]
    \[x(t) = 2\cosh(t) - 1\]
    
\end{enumerate}
        
\end{large}
\end{document}