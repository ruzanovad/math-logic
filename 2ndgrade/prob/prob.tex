\documentclass[a4paper]{article}
\usepackage{cmap}
\usepackage[utf8]{inputenc}
\usepackage[T2A]{fontenc}
\usepackage{amsfonts}

\usepackage{amsmath, amsthm}
\usepackage{amssymb}
\usepackage{mathrsfs}
\usepackage{hyperref}
\usepackage{multicol}
\usepackage{xcolor}

\newcommand\letsymbol{\mathord{\sqsupset}}
\usepackage[russian]{babel}
\renewcommand\qedsymbol{$\blacktriangleright$}
\newtheorem{theorem}{Теорема}[section]
\newtheorem{lemma}{Лемма}[section]
\newtheorem*{axiom}{Аксиома}
\theoremstyle{definition}
\newtheorem*{definition}{Определение}
\newtheorem*{statement}{Утверждение}
\theoremstyle{remark}
\newtheorem*{remark}{Замечание}
\newtheorem*{example}{Пример}
\newtheorem*{corollary}{Следствие}

\setlength{\topmargin}{-0.5in}
\setlength{\oddsidemargin}{-0.5in}
\textwidth 185mm
\textheight 250mm

\begin{document}
\tableofcontents
\section{Вероятностное пространство}

\begin{definition}[Алгебра]
    Семейство $\mathcal{A}$ подмножеств множества $\Omega$ называется
    алгеброй, если выполнены след. аксиомы:
    \begin{enumerate}
        \item $\varnothing \in \mathcal{A}$
        \item $A\in\mathcal{A} \implies \overline{A}\in \mathbb{A}$
        \item (аддитивность) $A_1, \dots, A_n \in \mathbb{A} \implies A_1 \cup \dots \cup A_n\in \mathbb{A}$ 
    \end{enumerate}
\end{definition}
Т.е алгебра является замкнутой относительно замыкания и объединения
\begin{definition}[$\sigma$-алгебра]
    Алгебра называется $\sigma$-алгеброй, если 
    \[A_1, \dots, A_n \in \mathcal{A} \implies \bigcup\limits_{k = 1}^\infty  A_k \in \mathcal{A}\]
\end{definition}
\begin{definition}[мера]
    $\mu: \mathcal {A} \to [0; \infty)$ - мера, если 
    \[A_1, \dots , A_n \in \mathcal{A}, A_i\cap A_j = \varnothing, i\neq j: \quad \mu(\bigcup\limits_{n = 1}^\infty  A_n) = \sum\limits_{n = 1}^\infty \mu (A_n)\quad \text{счетная аддитивность}\]

    Мера конечная, если $\mu(\Omega) < \infty$

    Мера вероятностная, если $\mu (\Omega) = 1$
\end{definition}
Короче говоря, вероятностная мера:
\begin{enumerate}
    \item неотрицательность 
    \item нормированность 
    \item счетная аддитивность (иногда счетная аддитивность заменяется на аддитивность и непрерывность)
\end{enumerate}
\subsection{Свойства P из учебника}
\begin{enumerate}
    \item $A\subseteq B \implies P(B\setminus A) = P(B) - P(A) \ge 0$
    \[\text{представить B как } B = A + (B\setminus A), а потом аддитивность\]
    \item $A\subseteq B \implies P(A)\le P(B)$
    \item $\forall A \in \mathscr{A}\quad 0 \le P(A) \le 1$
    \item $P(\overline{A}) = 1 - P(A)$
    \item $P(\varnothing) = 0$
    \item Конечная аддитивность (следует из счетной)
    \item \[P(\bigcup\limits_{k = 1}^n A_k)  \le \sum_{k = 1}^n P(A_k) \]
    \item \[B_k = A_k \setminus (A_1 \cup A_2 \dots  \cup A_{k-1}) \implies \bigcup\limits_{k = 1}^n A_k = \sum_{k=1}^n  B_k \implies P (\bigcup\limits_{k = 1}^n A_k) = P(\sum_{k=1}^n  B_k) | P(B_k) \le P(A_k)|\]
    \item \[\forall A, B \quad P(A \cup B) = P(A) + P(B) - P(AB)\]
    раздробить объединение через минус + аддитивность + 1 свойство
\end{enumerate}
\begin{definition}[Вероятностное пространство]
    Тройка $(\Omega, \mathcal{A}, P)$, где 
    \begin{enumerate}
        \item $\Omega$ - пространство элементарных событий;
        \item $\mathcal{A}$ - $\sigma-$алгебра подмножеств $\Omega$ (события);
        \item P - вероятностная счетно-аддитивная мера на $\mathcal{A}$ (вероятность);
        называется вероятностным пространством.
    \end{enumerate}
\end{definition}
Все элементарные исходы равновозможны
\begin{enumerate}
    \item Размещение (упорядоченный набор) $ A^n_N = \frac{N!}{(N-n)!}$
    \item Перестановка (частный случай размещения при N = n)
    \item Сочетание (подмножество) $C^n_N = \frac{A^n_N}{n!} = \frac{N!}{n! (N-n)!}$
\end{enumerate}
\begin{multicols*}{2}
    \begin{definition}[Классическая вероятность]
        Модель вероятностного пространства (A - событие)
        \begin{enumerate}
            \item $\Omega = \{\omega_1, \dots, \omega_n\}$ - конечное пространство
            \item $\mathcal{A}$ все подмножества $\Omega$
            \item $P(A) = \sum\limits_{\omega\in A}p_\omega = \frac{|A|}{|\Omega|}$
        \end{enumerate}
    \end{definition}
    \vfill\null\columnbreak
    \begin{definition}[Геометрическая вероятность]  
        $V\in \mathbb{R}^n$
        \begin{enumerate}
            \item $\Omega = V$
            \item $\mathcal{A}$ - борелевская $\sigma-$алгебра 
            (минимальная $\sigma-$алгебра, содержащая все компакты) подмножеств $V$
            \item $P(A) = \frac{\mu(A)}{\mu(V)}$
        \end{enumerate}
    \end{definition}
\end{multicols*}
\subsection{Некоторые следствия аксиоматики}
\begin{enumerate}
    \item \begin{axiom}[Аксиома непрерывности]
        Если $A_1\supset A_2, \dots, \supset A_n \supset \mathcal{A}, \bigcap\limits_{i = 1}^\infty A_i = \varnothing$, то
        \[\lim\limits_{n\to \infty} P(A_n) = 0\]
    \end{axiom}
    \begin{proof}
        Пусть $B_n\downarrow \varnothing$. Тогда обозначим $A_n = B_n \setminus B_{n+1}, n = 1, \dots, .$. $A_n$ попарно несовместны и 
        \[B_1  =    \sum\limits_{n = 1}^\infty A_n \quad B_n  =  \sum\limits_{k = n}^\infty A_k, \]
        поэтому из счетной аддитивности меры следует сходимость ряда \[P(B_1)=\sum\limits_{n = 1}^\infty P(A_n),\] и сумма остатка ряда \[P(B_n) =  \sum\limits_{k = n}^\infty P(A_k) = 0.\] 
    \end{proof}
    \item (Формула включений и исключений)
    \[P(\bigcup\limits_{k = 1}^n A_k)  = \sum_{k = 1}^n P(A_k) - \sum_{i < j}^n P(A_i \cap A_j) + \dots + {(-1)}^{n-1} P(A_1 \cap \dots  \cap A_n)\]
    \begin{proof}
        Выводится через обычную формулу включений и исключений для множеств по индукции \[P(A \cup B) = P(A) + P(B) - P(AB)\]+ 
        \[\begin{cases}
            A \cup B = A + (B\setminus AB) \\ 
            \text{Счетная аддитивность} \\ 
            P(B \setminus AB) = P(B) - P(AB) (\text{также по счетной аддитивности})
        \end{cases}
        \]
    \end{proof}
\end{enumerate}
\subsubsection{Индикатор}
\begin{definition}
    Индикатор события А - это функция $I_A(\omega) = \begin{cases}
        1,  & \omega \in A \\
        0,  & \omega \notin A
        \end{cases}$
\end{definition}
\paragraph{Свойства индикатора}
\begin{enumerate}
    \item $I_{\bar{A}} = 1 - I_A$
    \item $I_{A_1 \cap A_2} = I_{A_1}I_{A_2}$
    \item $I_{A_1\cup\dots\cup A_n} = 1 - I_{\bar{A_1} \cap \dots\cap \bar{A_n}} = 1  -  I_{\bar{A_1}}\dots I_{\bar{A_n}} = 1 - (1 - I_{A_1})\dots(1 - I_{A_n})$
\end{enumerate}
\section{Условные вероятности и независимость}
\begin{definition}[Условная вероятность]
    Пусть $P(B)>0$. Условной вероятностью $P(A|B)$ события А при условии, что произошло событие B (или просто: при условии B), назовем отношение
    $$P(A|B) = \frac{P(AB)}{P(B)}$$
    Применяется также обозначение $P_B(A)$
\end{definition}
\begin{theorem}[Теорема умножения]
    Пусть события $A_1, \dots, A_n$ таковы, что $P(A_1, \dots, A_{n-1)}>0$. Тогда $$P(A_1, \dots, A_n) = P(A_1)P_{A_1}(A_2)\dots P_{A_1, \dots, A_{n-1}}(A_n)$$
\end{theorem}
\begin{proof}
    Из условия теоремы вытекает, что существуют все условные вероятности из формулы.
    
    База индукции $P(AB) = P(B)P_B(A)$.

    Переход: $B=A_1, \dots, A_{n-1}, A = A_n$, применим формулу выше
\end{proof}
\begin{definition}[Разбиение]
    Систему событий $A_1, \dots, A_n$ будем называть конечным разбиением (в дальнейшем - просто разбиением),
    если они попарно несовместны ($A_i A_j = \varnothing, i\neq j$) и 
    \[A_1+\dots A_n = \Omega\]
\end{definition}
\begin{theorem}[Формула полной вероятности]
    Если $A_1, \dots, A_n$ - разбиение и все $P(A_k)>0$, то для любого события B имеет
    место формула
    \[P(B) = \sum_{k = 1}^n P(A_k)P(B|A_k)\]
\end{theorem}
\begin{proof}
    \[B = B\Omega = B A_1 + B A_2 + \dots + B A_n\]
    сумма попарно несовместных событий. Тогда
    \[P(B) = P(B\Omega = B A_1 + B A_2 + \dots + B A_n) = \sum_{k = 1}^{n} P (B A_k)\]
    \[P(B A_k) = P(A_k) P_{A_k}(B) = P(A_k) P(B|A_k)\]
\end{proof}
\subsection{Формулы Байеса}
\begin{theorem}[Формулы Байеса]
    Если $A_1, \dots, A_n$ - разбиение и все $P(A_k)>0$, то для любого события B ($P(B)> 0$) имеют место формулы:
    \[P(A_k | B) = \frac{P(A_k)P(B | A_k)}{\sum_{i=1}^{n}P(A_i)P(B|A_i)}\]
\end{theorem}
\begin{proof}
    \[P(A_k B) = P(A_k) P (B | A_k) = P(B) P(A_k | B) \implies P (A_k | B) = \frac{P(A_k)P(B|A_k)}{P(B)}\]
    Применяем к $P(B)$ формулу полной вероятности.
\end{proof}
\begin{enumerate}
    \item $P(A_k)$ - априорные вероятности (до опыта)
    \item $P(A_k | B)$ - апостериорные вероятности (после опыта)
\end{enumerate}
\subsection{Независимость событий}
\begin{definition}[независимость 2 событий]
    A и B - независимы, если \[P(AB) = P(A)P(B),\]иначе зависимы
\end{definition}
\begin{definition}[независимость n событий (в совокупности)]
    $A_1, \dots, A_n$ называются независимыми, если для любых $1 \le i_1 < i_2 < \dots < i_m \le n, 2 \le m \le n$
    \[P(A_{i_1}A_{i_2}\dots A_{i_m}) = P(A_{i_1})P(A_{i_2})\dots  P(A_{i_m}),\] иначе зависимы.
\end{definition}
\begin{theorem}
    Если $A_1, \dots, A_n$ независимы, $i_1, \dots, i_r, j_1, \dots, j_s$ - индексы все различны, вероятность $P(A_{i_1}\dots A_{i_r}) > 0$, тогда
    \[P(A_{j_1}\dots A_{j_s}| A_{i_1}\dots A_{i_r}) = P(A_{j_1}\dots A_{j_s})\]
\end{theorem}
\begin{proof}
    $A_1, \dots, A_n$ независимы, поэтому
    \[P (A_{i_1}\dots A_{i_r}) = P(A_{i_1})\dots P(A_{i_r})\]
    \[P (A_{j_1}\dots A_{j_s}) = P(A_{j_1})\dots P(A_{j_s})\]
    \[P(A_{j_1}\dots A_{j_s} A_{i_1}\dots A_{i_r}) = P(A_{i_1})\dots P(A_{i_r}) P(A_{j_1})\dots P(A_{j_s})\]
    поэтому \[P(A_{j_1}\dots A_{j_s} \cap A_{i_1}\dots A_{i_r}) = P(A_{i_1}\dots A_{i_r})\times P(A_{j_1}\dots A_{j_s}) \]
    + формула условной вероятности
\end{proof}
\subsection{Независимость разбиений, алгебр/$\sigma$-алгебр}
\begin{definition}[Порожденная алгебра]
    $\gamma$ - система множеств. Наименьшая алгебра множеств $\mathscr{A}(\gamma)$, содержащая $\gamma$, называется \textbf{алгеброй, порожденной системой $\gamma$}.
\end{definition}
\begin{definition}[Порожденная $\sigma$-алгебра]
    Аналогично.
\end{definition}
\begin{remark}
    Алгебра, порожденная разбиением, является конечной, состоит из пустого множества и множеств вида
    \[A_{i_1} + A_{i_2}+ \dots A_{i_m}\]
\end{remark}
\begin{theorem}
    Каждая конечная алгебра множеств порождается некоторым разбиением
\end{theorem}
\begin{proof}
    $\mathscr{B}$ - конечная алгебра событий. Обозначим $\mathscr{B}_w$ - совокупность событий B из $\mathscr{B}$, для которых w из B.

    Для каждого $w \in \Omega$ введем $B_w = \bigcap_{B \in \mathscr{B}_w} B$

    Покажем, что для двух $\omega \neq \omega '$ 
        \[\left[
            \begin{gathered}
            B_{\omega} =  B_{\omega'}\\
            B_{\omega} \cap B_{\omega'} = \varnothing \\
        \end{gathered}
        \right.\]
    
    Для любых $\omega \in \Omega$ и $B \in \mathscr{B}$ истинно свойство: если $\omega \in B, $ то $B_\omega \subseteq B$ (т.к $B_\omega$ - пересечение всех таких B из алгебры, в которых лежит $\omega$)

    Пусть теперь $\omega \in B_{\omega'}$, тогда $B_\omega \subseteq B_{\omega '}$ (транзитивность $B_\omega \subseteq B \subseteq B_{\omega'}$) 
    
    Далее если $\omega' \in B_\omega, $ то $B_{\omega'} \subseteq B_{\omega}$ и, следовательно, $B_{\omega'} = B_{\omega}$

    Случай $\omega'\in \overline{B_{\omega}}$ невозможен, так как противоречие $B_{\omega'} \subseteq \overline{B_\omega}$ (а уже было доказано, что $B_\omega \subseteq B_{\omega '}$)

    Выберем среди $B_\omega$ разные множества $B_1, \dots, B_r$. Это разбиение, т.к $B_1 + \dots + B_r = \Omega$ и $B_i B_j = \varnothing$ при $i\neq j$.

    Так как $\forall B \in \mathscr{B}$ представимо в виде $B = \bigcup_{\omega \in B} B_\omega$, то это разбиение порождает алгебру $\mathscr{B}$.

\end{proof}
\begin{definition}[Независимые разбиения]
    $\alpha_k : A_{k1}+\dots+A_{kr_k} = \Omega, k = 1, \dots, n$
    независимые, если для любых $i_k, 1\le i_k \le r_k, k = 1, \dots, n$
    \[P(A_{1i_1}A_{2i_2}\dots A_{n i_n}) = P(A_{1i_1})P(A_{2i_2})\dots P(A_{n i_n})\]
\end{definition}
По-русски: есть n разбиений, они могут быть разной мощности.
Берем по любому событию из каждого разбиения. (всего получается n событий)  (то есть вариантов формулы всего $|\alpha_1| \times\dots\times |\alpha_n|$)
\begin{definition}[Независимые алгебры ($\sigma$-алгебры)]
    $\mathscr{A_1}, \dots, \mathscr{A_n}$ - независимы, если $\forall A_i\in \mathscr{A_i}$
    \[P(A_1 A_2 \dots A_n) = P(A_1) \dots P(A_n)\]
\end{definition}
\begin{theorem}
    Конечные алгебры $\mathscr{A_1}, \dots, \mathscr{A_n}$ независимы тогда и только тогда, когда независимы порождающие их разбиения $\alpha_1, \dots, \alpha_n$
\end{theorem}
\begin{proof}
    $\implies$ Разбиение есть подсистема порожденной алгебры. Из независимости алгебр следует независимость разбиений.
    \begin{lemma}
        \begin{enumerate}
            \item Если события A и B независимы, то $\bar{A}$ и B также независимы
            \item Если $A_1$ и B независимы и $A_2$ и B независимы, а $A_1 A_2 = \varnothing$, то $A_1 + A_2$ и B независимы.
        \end{enumerate}
        \begin{proof}
            \begin{enumerate}
                \item A и B независимы, тогда 
                \[P(B \bar{A}) = P(B \setminus AB) = P(B) - P(AB) = P(B) - P(A)P(B) = P(B)(1- P(A)) = P(B)P(\bar{A})\]
                \item $A_i$ и B независимы: $P(A_i B) = P(A_i)P(B)$
                \[P((A_1+A_2)B) = P(A_1)P(B) + P(A_2)P(B) = P(B)(P(A_1)+ P(A_2)) = P(B)P(A_1+A_2)\]
            \end{enumerate}
            
        \end{proof}
    \end{lemma}
    
    $\impliedby$ Каждое событие из алгебры - сумма попарно несовместных событий из соответствующего разбиения.
    
    Обратный вывод по лемме.
\end{proof}
\begin{remark}
    Каждое событие A порождает разбиение $A + \bar{A} = \Omega$, порождающее алгебру $\mathscr{A}(A)$. Из леммы вытекает, что независимость событий $A_1, \dots, A_n$ и независимость порожденных ими алгебр $\mathscr{A} (A_1), \dots,\mathscr{A} (A_n)$ эквивалентны.
\end{remark}
\subsection{Независимые испытания}
Если имеем n независимых испытаний, то можно построить одно большое вероятностное пространство, элементы которого являются прямыми произведениями соответствующих $\Omega_i$ и тд.

Подалгебры должны быть независимы, тогда такое пространство всегда можно построить 

Прямое произведение вероятностей:
\[\omega = (\omega_1, \dots, \omega_n), p(\omega) = p_1(\omega_1)\dots p_n(\omega_n), \quad P(A) = \sum_{\omega\in A} p(\omega)\]
\[P = P_1 \times \dots \times P_n\]

События являются "прямоугольниками":
\[A = A_1 \times \dots \times A_n\]
состоит из векторов $\omega = (\omega_1, \dots, \omega_n), \omega_i\in A_i\in \mathscr{A_i}$ 

Вероятность A:
\[P(A) = \sum_{\omega\in A} p(\omega) = \sum_{\omega\in A_1} p_1(\omega_1)\dots \sum_{\omega\in A_n} p_n(\omega_n) = \prod_{k = 1}^{n}P_k(A_k)\]
Пусть $\mathscr{A_k'}$ - подалгебра $\mathscr{A}$, где все $A_i =\Omega_i$ для всех компонент прямоугольника ($i\neq k$)

События из этой алгебры ($A_i'$) изоморфны событиям из $A_i$.

\[P(A_i') = P_i(A_i)\]

Событие A является пересечением событий $A_k', k = 1, \dots, n$
\[P(\bigcap_{k = 1}^n A_k') = \prod_{k = 1}^{n}P(A_k')\]
Поэтому алгебры $A_j'$ независимы.
\subsubsection{Схема Бернулли}
n испытаний, в котором либо успех, либо неудача (неуспех) (в каждом испытании вероятность успеха и неудачи равны), тогда вероятность элементарного события (вектора из событий каждого испытания, он булев, так как каждое $\omega_i$ либо 0, либо 1) 
\[p(\omega) = \prod_{i = 1}^{n}p^{\omega_i}q^{1 - \omega_i}\]
Обозначим $B_k = \{\omega: \omega_1 + \dots + \omega_n = k\}$

Для $\omega \in B_k$ $p(\omega) = p^k q^{n-k}$
\[P(B_k) = C_n^k p^k q^{n-k}, k = 1, \dots n - \textbf{Биномиальное распределение}\]
Еще есть полиномиальноая схема. там не по 2 исхода, а по r.
\section{Случайные величины}
\begin{definition}[Случайная величина]
    Случайной величиной (СВ) $X(\omega)$ называется функция элементарного события $\omega$ с областью определения $\Omega$
и областью значений $\mathbb{R}$ такая, что событие $\{\omega : X(\omega) \leq x\}$ принадлежит $\sigma$ -алгебре $\mathcal{F}$ при любом действительном $x \in \mathbb{R}$ . Значения x функции $X(\omega)$ называются реализациями СВ $X(\omega)$.    
\end{definition}
\begin{definition}[Алгебра, порожденная случайной величиной]
    Пусть $x_1< \dots< x_k$ - значения, принимаемые случайной величиной $\xi$. Каждая такая
    величина определяет разбиение из событий $A_i = \{\omega : \xi(\omega) = x_i\}$.
    Т.к $x_i\neq x_j$, то $A_i A_j = \varnothing$. Сумма - достоверное событие $\Omega$.

    Разбиение порождает алгебру событий \[\{\xi \in B \} = \{\omega: \xi (\omega) \in B\} \],
    B - числовое множество. 
\end{definition}
\begin{definition}[Закон распределения]
    Любое правило (таблица, функция), позволяющее находить вероятности всех возможных событий, связанных со случайной величиной.
\end{definition}

\subsection{Примеры законов распределения}
\begin{enumerate}
    \item Биномиальный закон
    \item Гипергеометрическое распределение:
    распределение числа белых шаров $\xi$ в выборке без возвращения объема n из урны, содержащей M белых и N-M черных шаров 
    \[P \{\xi = m\} = \frac{C^m_M C^{n-m}_{N-M}}{C_M^n}, m = 0, 1, \dots \min(n, M)\]
    \item Равномерное распределение
\end{enumerate}

\begin{definition}[Математическое ожидание]
    Математическое ожидание случайной величины $\xi = xi(\omega)$ обозначается $M\xi$ и определяется как сумма
    \[M\xi = \sum_{\omega \in \Omega} \xi(\omega) p(\omega)\]
\end{definition}
\textsubscript{среднее значение $\xi$}
\subsection{Свойства мат. ожидания}
\begin{enumerate}
    \item $M I_A = P(A)$
    \begin{proof}
    \[M I_A = \sum_{\omega \in \Omega} I_A(\omega) p (\omega) = \sum_{\omega \in A} p(\omega)  = P(A)\]
    \end{proof}
    \item Аддитивность: $M(\xi + \eta) = M\xi + M\eta$
    \begin{proof}
        
    \end{proof}
    Из этого также следует конечная аддитивность.
    \item Для любой константы C \[M(C\xi) = cM\xi,\quad MC = C\]
    \item Если $\xi \ge \eta$, то $M \xi \ge M \eta$.
    $\xi \ge 0 \& M\xi = 0\implies P\{\xi = 0\} = 1$
    \item Математическое ожидание $\xi$ выражается через закон распределения случайной величины $\xi$ формулой
    \[M\xi = \sum_{i = 1}^k x_k P \{\xi = x_i\}\]

\end{enumerate}
Подставляя в числовую функцию случайную величину, мы также получаем случайную величину. Например, если $\eta = g(\xi)$, то \[M\eta  = M g(\xi)  = \sum_{i = 1}^k g(x_i) P\{\xi = x_i\}\]
При этом \[g(x_i) = \sum_{i = 1}^k g(x_i) I_{\xi = x_i}\]
\begin{definition}[n-ый момент случайной величины]
    Математическое ожидание $M\xi^n$ называется n-ым моментом (или моментом n-ого порядка) случайной величины $\xi$ (или ее закона распределения).
\end{definition}
\begin{definition}[Абсолютный n-ый момент]
    Математическое ожидание $M {|\xi|}^n$.
\end{definition}
\begin{definition}[Центральный момент n-ого порядка]
    $M(\xi-M\xi)^n$
\end{definition}
\begin{definition}[Абсолютный центральный момент n-ого порядка]
    $M|\xi-M\xi|^n$
\end{definition}
\begin{definition}[Дисперсия]
    $D\xi = M(\xi - M\xi)^2$
\end{definition}
\begin{definition}[Среднее квадратическое отклонение (стандартное отклонение)]
    $\sqrt(D\xi)$
\end{definition}
\subsection{Свойства дисперсии}
\begin{enumerate}
    \item $D\xi = M\xi^2 - (M\xi)^2$
    \item $D\xi\leq 0$ и $D\xi = 0$ тогда и только тогда, когда существует такая константа c, что $P\{\xi = c\} = 1$
    \item Для любой константы c $D(c\xi) = c^2 D\xi, \quad D(\xi +c) = D\xi$
\end{enumerate}
\begin{theorem}[Неравенство Иенсена]
    Если числовая функция $g(x)$, то для любой случайной величины $\xi$
    \[Mg(\xi)\leq g(M\xi)\]
\end{theorem}
\begin{theorem}[Неравенство Ляпунова]
    Для любых положительных $\alpha \leq \beta$
    \[(M|\xi|^\alpha)^{1/ \alpha}\leq (M|\xi|^\beta)^{1/ \beta}\]
\end{theorem}
\begin{theorem}[Неравенство Коши-Буняковского]
    
\end{theorem}
\subsection{Джентльменский набор}
\begin{enumerate}
    \item Равномерное дискретное распределение
    \[P\{\xi=k\} = \frac{1}{N}, \quad M\xi = \frac{1+N}{2}, \quad D\xi = \frac{N^2-1}{12}\]
    \item Биномиальное (распределение Бернулли)
    \[P\{n=k\}=C_n^k p^k (1-p)^{n-k}, \quad M\xi = np, \quad D\xi = np(1-p)\]
    \item Геометрическое распределение
    \[P\{n=k\}=(1-p)p^k, \quad M\xi = \frac{p}{1-p}, \quad D\xi = \frac{p}{(1-p)^2}\]
    \item Распределение Пуассона
    \[P\{n=k\}=\frac{\lambda^k}{k!}e^{-\lambda}, \quad M\xi = \lambda, \quad D\xi = \lambda\]
    
\end{enumerate} 
\subsection{Многомерные законы распределения}
\subsection{Независимость случайных величин}
\begin{definition}[Независимость случайных величин]
    $\xi_1, \dots, \xi_n$ называются независимыми, если порожденные ими алгебры \[\mathcal{A}_{\xi_1}, \dots, \mathcal{A}_{\xi_n}\] независимы.
\end{definition}
\begin{definition}[Независимость случайных величин]
    $\xi_1, \dots, \xi_n$ называются независимыми, если для любых $x_{1_{j_1}}\dots, x_{x_{j_n}}$
    \[P\{\xi_1 = x_{1_{j_1}}, \xi_n = x_{1_{j_n}}\} = \prod_{i = 1}^n P\{\xi_i = x_{1_{j_i}}\}\]
\end{definition}
\begin{theorem}
    Если случайные величины \(\xi_1, \dots \xi_n\) независимы, а \(g_i(x)\) - числовые функции, то случайные величины \(\eta_1 = g_1(\xi_1), \dots \eta_n = g_n(\xi_n) \) также независимы.
\end{theorem}
\begin{theorem}[Мультипликативное свойство математических ожиданий]
    Если случайные величины \(\xi_1, \dots \xi_n\) независимы, то 
    \[M \xi_1, \dots, \xi_n = \prod_{i = 1}^n M \xi_i\]
\end{theorem}
\begin{theorem}[Аддитивное свойство дисперсии]
    Если случайные величины \(\xi_1, \dots \xi_n\) независимы, то 
    \[D(\xi_1 + \cdots + \xi_n) = D\xi_1 + \cdots + D\xi_n\]
\end{theorem}

\subsection{Евклидово пространство случайных величин}
\begin{enumerate}
    \item зададим евклидово пространство случайных величин - векторов $(\xi(\omega_1), \dots, \xi(\omega_n))$ с 
    \begin{enumerate}
        \item скалярное произведение \[(\xi, \eta) = \sum_{\omega}\xi(\omega)\eta(\omega)p(\omega)=  M\xi\eta\]
        \item норма \[\left\lVert \xi\right\rVert =\sqrt{(\xi, \xi)}\]
        \item расстояние \[d(\xi, \eta) = \sqrt{M(\xi-\eta)^2} = \left\lVert \xi-\eta\right\rVert\]
    \end{enumerate}
    Рассмотрим прямую констант $l_0 = \{\xi | \xi(\omega_1) = \dots = \xi(\omega_n)\}$ и найдем проекцию $m_\xi$ случайной величины на прямую 
    \[d(\xi, m_\xi) = \min_{c\in l}d(\xi, c)\]
    При любой константе 
    \[M(\xi-c)^2 = M(\xi - M\xi)^2 + (M\xi -c)^2 \ge D\xi\]
    Значит $\sqrt{D\xi} = \min_{c\in l}d(\xi, c) = d(\xi, m_\xi) $ и $m_\xi = M\xi$.
    
    Проекция случайной величины - ее матожидание, $\xi - M\xi$ ортогональнально прямой констант. 
    \[(1, \xi - M\xi) = 0\]
    \item Рассмотрим две случайных величины $\xi, \eta$.
    \[\begin{cases}
        \xi = M\xi +\xi_1 \\
        \eta = M\eta +\eta_1 \\
    \end{cases}\]

    \begin{definition}[Коэффициент корелляции]
        \[\rho(\xi, \eta) = \cos \phi_{\xi_1, \eta_1} = \frac{(\xi_1, \eta_1)}{\left\lVert  \xi_1 \right\rVert \left\lVert \eta_1 \right\rVert } = \frac{(\xi - M\xi, \eta - M\eta)}{\left\lVert \xi - M\xi \right\rVert \left\lVert \eta - M\eta \right\rVert } = \frac{M(\xi - M\xi)(\eta - M\eta)}{\sqrt{D\xi D\eta}}\] - коэффициент корелляции между $\xi$ и $\eta$
    \end{definition}
    \begin{definition}[Ковариация]
        \[\mathop{Cov} (\xi, \eta) = M(\xi - M\xi)(\eta - M\eta)\]
    \end{definition}
    \[\rho(\xi, \eta) = \frac{\mathop{Cov} (\xi, \eta) }{\sqrt{D\xi D\eta}}\]
    \begin{itemize}
        \item По неравенству Коши-Буняковского $(M \xi_1 \eta_1)^2 \le M \xi_1^2 M \eta_1^2\implies |\rho(\xi, \eta)|\le 1$
        \item Если $\xi, \eta$ независимы, то Cov$(\xi, \eta) = 0, \rho(\xi, \eta) = 0$
        \begin{proof}
            $M(\xi - M\xi)(\eta - M\eta) = M(\xi - M\xi)M(\eta - M\eta) = 0$
        \end{proof}
    \end{itemize}
    \begin{definition}[некореллированные случайные величины]
        Если $\rho(\xi, \eta) = 0, $ то $\xi_1 \perp \eta_1$ и случайные величины $eta, \xi$- некореллированные
    \end{definition}
    При $\alpha_1 \alpha_2\neq 0$
    \[\rho(\alpha_1 \xi +\beta_1, \alpha_2 \eta + \beta_2) = \frac{\alpha_1}{|\alpha_1|} \frac{\alpha_2}{|\alpha_2|}\rho(\xi, \eta)\]
    Спроектируем вектор $\eta$ на плоскость, в которой лежат прямая констант $l_0$ и $\xi$. Проекция $\eta = \alpha \xi + \beta$ определяется константами $\alpha, \beta$, при которых 
    \[\begin{cases}
        \eta - \alpha \xi - \beta \perp 1 \\
        \eta - \alpha \xi - \beta \perp \xi \\
    \end{cases}\]
    (вектор-высота)
    \[\begin{cases}
        M(\eta - \alpha \xi - \beta ) \cdot 1  = 0\\
        M(\eta - \alpha \xi - \beta ) \cdot \xi  = 0\\
    \end{cases}\]
    \[\begin{cases}
        \alpha M \xi + \beta = M \eta \\
        \alpha M \xi ^2 + \beta M \xi = M \xi \eta
    \end{cases}\]
    Получаем 
    \[\begin{cases}
        \alpha = \rho \frac{\sigma_\eta}{\sigma_\xi} \\
        \beta = M\eta - \rho \frac{M\xi}{\sigma_\xi}\sigma_\eta \\
    \end{cases}\]
    $\sigma_\xi^2 = D\xi, \sigma_\eta^2 = D\eta, \rho = \rho(\xi, \eta)$
\end{enumerate}
Если случайные величины зависимы, то
\begin{theorem}
    \[D(\xi_1 + \cdots + \xi_n) = \sum_{k = 1}^n  D \xi_k + 2\sum_{1 \le k < l \le n} Cov (\xi_k, \xi_l)\]
\end{theorem}
\begin{proof}
    \[D(\xi+\eta) = M ((\xi - M\xi) + (\eta - M\eta))^2 = M (\xi - M \xi)^2 + M (\eta - M\eta)^2 + 2M(\xi - M\xi)(\eta - M\eta) =  D\xi + D\eta + 2 Cov(\xi, \eta)\]
\end{proof}

Условное мат. ожидание $M (\xi | \eta)$- ортогональная проекция $\xi$ на линейное подпространство $\eta$

\subsection{Условные математические ожидания}
\begin{definition}[Условная вероятность]
    Условная вероятность \(P(B|\mathscr{A}(\alpha))\) относительно \(\mathscr{A}(\alpha)\) как случайную величину, которая принимает значение \(P(B|A_k)\) при \(\omega \in A_k\).
\end{definition}
\begin{definition}[Условный закон распределения]
    Условный закон распределения $\eta $ при заданном значении $\xi = x_k$ назовем набор условных вероятностей
    \[P\{\eta = y_t | \xi = x_k\} = \frac{P(\eta = y_t, \xi = x_k)}{P(\xi = x_k)}, \quad t = 1, \dots, m\]
\end{definition}
\begin{definition}[Условное мат.ожидание]
    Условное мат.ожидание $\eta$ при заданном значении $\xi = x_k$
    \[M\{\eta | \xi = x_k\} = \sum_{t = 1}^{m} P\{\eta = y_t| \xi = x_k\} = \frac{\sum_{t = 1}^{m}y_t P(\eta = y_t, \xi = x_k)}{P(\xi = x_k)}\]
\end{definition}
Условное мат.ожидание является функцией от $\eta$. Случайная величина $M(\eta | \xi)$ - условное мат.ожидание при заданном $\xi$
\begin{definition}
    \[M[M(\eta | \xi)] = \sum_{k = 1}^{n} P\{\xi  = x_k\}M\{\eta | \xi = x_k\}\]
\end{definition}
\begin{theorem}
    \[M[M(\eta | \xi)] = M\eta\]
\end{theorem}
\begin{proof}
    \[M[M(\eta | \xi)] = \sum_{k = 1}^{n} P\{\xi  = x_k\}M\{\eta | \xi = x_k\} = \sum_{k = 1}^{n} P\{\xi  = x_k\}\sum_{t = 1}^{m} P\{\eta = y_t| \xi = x_k\} = \sum_{k = 1}^{n} P\{\xi  = x_k\}\frac{\sum_{t = 1}^{m}y_t P(\eta = y_t, \xi = x_k)}{P(\xi = x_k)}=\]
    \[=\sum_{k = 1}^{n}\sum_{t = 1}^{m}y_t P(\eta = y_t, \xi = x_k)=\sum_{k = 1}^{n}\sum_{t = 1}^{m}y_t P(\eta = y_t, \xi = x_k) = \sum_{l = 1}^m y_l P\{\eta = y_l\} = M \eta\]
\end{proof}
\subsection{Неравенство Чебышева. Закон больших чисел}
\begin{theorem}[Неравенство Чебышева]
    Для любого $x>0$ имеют место неравенства:
    \begin{equation}
        P\{|\xi|\geq x\} \leq \frac{M|\xi|}{x}
    \end{equation}
    \begin{equation}
        P\{|\xi - M\xi| \geq x\} \leq \frac{D \xi}{x^2}
    \end{equation}
\end{theorem}
\begin{proof}
    (1)\[|\xi| = |\xi|I_{|\xi|\ge x}+|\xi|I_{|\xi|< x}\ge |\xi| I_{|\xi| \ge x} \ge x I_{|\xi|\ge x}\]
    \[M|\xi| \ge xM I_{|\xi|\ge x} = x P \{|\xi|\ge x\}\]
    (2)\[\eta = (\xi - M\xi)^2\]
    \[M\eta = D\xi\]
\end{proof}
\paragraph*{Закон больших чисел}
\begin{theorem}[Теорема Чебышева]
    Если $\xi_1, \dots, \xi_n$ независимы и существует такая константа $c > 0$, что $D\xi_n\le c, n = 1, \dots, $ то при любом $\varepsilon > 0$
    \[\lim_{n\to \infty} P\{\left\lvert \frac{\xi_1+\cdots + \xi_n}{n} - \frac{M \xi_1 + \cdots + M \xi_n}{n}\right\rvert  > \varepsilon\} = 1\]
\end{theorem}
\begin{corollary}
    Если $\xi_1, \dots$ независимы и одинаково распределены,

    $M\xi_n = a, D\xi_n = \sigma^2< \infty$

    то при любом $x>0$
    \[\lim_{n\to \infty} P\{\left\lvert \frac{\xi_1+\cdots + \xi_n}{n} - a \right\rvert  < x\} = 1\]
\end{corollary}
Закон больших чисел утверждает, что с вероятностью, приближающейся при $n\to \infty$ к 1, среднее арифметическое сумм независимых слагаемых при определенных условиях становится близким к константе.
\paragraph*{Закон больших чисел в схеме Бернулли}
\begin{theorem}[Теорема Бернулли]
    
\end{theorem}
\section{Случайные величины (общий случай)}
\begin{definition}
    Числовая функция $\xi = \xi (\omega)$ от элементарного события $\omega \in \Omega$ называется случайной величиной, если для любого числа x
    \[\{\xi \leq x\}=\{\omega : \xi(\omega)\leq x\} \in \mathscr{A}\]
\end{definition}
\begin{definition}[Функция распределения случайной величины $\xi$]
    \[F(x) = F_\xi (x) = P \{\xi \le x\}\], определенная при всех $x\in R$
\end{definition}
При помощи этой функции можно выразить вероятность попадания $\xi$ в интервалы.
\[P(x_1 < \xi \le x_2) = F(x_2) - F(x_1)\]
\[\{\xi < x\}: \quad \sum_{n=1}^{\infty} \{x - \frac{1}{n-1}<\xi \le x - \frac{1}{n}\} \]
\[P({\xi = x}) = F(x) - F(x - 0)\]
\[P({x_1 \le \xi \le x_2}) = F(x_2) - F(x_1 - 0)\]
\[P({x_1 < \xi < x_2}) = F(x_2 - 0 )  - F(x_1)\]
\[P (x_1 \le \xi < x_2) = F(x_2 - 0) - F(x_1 - 0)\]
\begin{theorem}[Свойства функции распределения]
    Функция распределения $F(x)$ обладает следующими свойствами:
    \begin{enumerate}
        \item F(x) не убывает
        \item F(x) непрерывна справа
        \item $F(+\infty) = 1$
        \item $F(-\infty) = 0$
    \end{enumerate}
\end{theorem}
\begin{definition}[Борелевская $\sigma$-алгебра]
    $\sigma$-алгебра $\mathcal{A}$ числовых множеств, порожденная всевозможными интервалами вида $x_1<x\leq x_2$, называется борелевской; множества A, входящие в $\mathcal{A}$, называются борелевскими.
\end{definition}
\begin{definition}[$\sigma-$алгебра, порожденная случайной величиной $\xi$]
    Совокупность $\xi^{-1}(B)$ для всех борелевских множеств борелевской алгебры.
\end{definition}
\subsection{Примеры дискретных распределений}
\begin{enumerate}
    \item Биномиальное
    \item Пуассоновское
    \item Геометрическое
\end{enumerate}
\begin{theorem}
    Если $\xi$ - случайная величина, а $g(x)$ - борелевская функция, то $\eta = g(\xi)$ есть случайная величина
\end{theorem}
\begin{definition}[Распределение вероятностей]
    $P_\xi(B)$, определенная для всех $B\in \mathscr{B}$, называется распределением вероятностей случайной величины $\xi$
\end{definition}
\begin{definition}[величина с дискретным распределением]
    величина имеет дискретное распределение, если в точках разрыва
    функции распределения вероятности таковы, что их сумма $\sum_{k = 1}^{\infty} p_k  = 1$

\end{definition}
% \begin{theorem}[Каратеодори]
    
% \end{theorem}
\begin{definition}[Плотность распределения]
    $p(x) = p_\xi(x)$ - плотность распределения случайной величины $\xi$, если для любых $x_1 < x_2$
    \[P\{x_1 < \xi < x_2\} = \int_{x_1}^{x_2}p_\xi (x)dx\]
\end{definition}
\subsection{Свойства}
\[p(x)\le 0, \int_{-\infty}^{\infty} p(x)dx = 1\]
\section{Математическое ожидание}
\begin{definition}[Простая случайная величина]
    Случайная величина простая, если она представима в виде
    \[\xi = \xi (\omega) = \sum_{j = 1}^m x_j I_{A_j}(\omega)\]
    где события $A_1, \dots, A_m$ составляют разбиение, т.е $A_i A_j = \varnothing$ при $i \neq j$ и $\sum_{j = 1}^m A_i = \Omega$
\end{definition}
\begin{definition}[Мат. ожидание простой случайной величины]
    \[M\xi = \sum_{j = 1}^m x_j P(A_j)\]
\end{definition}
\begin{definition}[Мат. ожидание неотрицательной случайной величины]
    \[M\xi = \lim_{n\to \infty} M\xi^n\]
\end{definition}
\begin{definition}[Мат. ожидание в общем случае]
    \[\xi = \xi^+ - \xi^-,\] где \(\xi^+ = \xi I_{\{\xi\geq 0\}}\), \(\xi^+ = |\xi| I_{\{\xi< 0\}}\)
\end{definition}
\subsection{Свойства мат. ожидания}
\begin{enumerate}
    \item Свойство линейности
    \item Свойство положительности
    \item Свойство конечности
\end{enumerate}
\subsection{Джентльменский набор абсолютно непрерывных распределений}
\begin{enumerate}
    \item Нормальное (гауссово распределение)
    \begin{definition}[гауссово распределение]
        Случайная величина $\xi$ имеет нормальное распределение с параметрами $(a, \sigma)$, $-\infty < a < \infty, \sigma > 0$, если она имеет плотность
        \[p_\xi(x) = \frac1{\sqrt{2\pi}\sigma} e^{-\frac{(x-a)^2}{2\sigma^2}}\]
        Нормальное распределение с параметрами (0, 1) называется стандартным.
        \[p(x) = \frac1{\sqrt{2\pi}}e^{-\frac{x^2}2}\]
        Для плотности истинно условие
        \[\int_{-\infty}^{\infty}p(x) dx = \frac1{\sqrt{2\pi}} \int_{-\infty}^{\infty} e^{-\frac{x^2}2}dx = |t = \frac{x}{\sqrt{2}}|= \frac1{\sqrt{\pi}} \int_{-\infty}^{\infty}  e^{-t^2}dt = |\text{гауссов интеграл}| = 1\]
    \end{definition}
    \item Равномерное распределение
    \begin{definition}[равномерное распределение]
        Случайная величина $\xi$ имеет равномерное распределение на отрезке [a, b] если ее плотность имеет вид:
        \[p_\xi (x) = \begin{cases}
            C & \text{при } a \le x \le b \\
            0 &\text{при } x < a \text{ или } x > b
        \end{cases}\]
        Так как 
        \[\int_{-\infty}^{\infty}p(x) dx = C \int_{a}^{b}dx = C(b-a) = 1,\]
        то $C = b - a.$
    \end{definition}
    \item Гамма-распределение 
    \begin{definition}[гамма распределение]
        \[p_\xi (x) = \begin{cases}
            0 & x< 0 \\
            \frac{\lambda^\alpha x^{\alpha-1}}{\Gamma(\alpha)}e^{-\lambda x} & x\ge 0,
        \end{cases}\]
        где $\alpha > 0, \lambda > 0$ - параметры
    \end{definition}
    При $\alpha = 1$ имеем показательное распределение
    \[p_\xi(x) = \begin{cases}
        0 & x< 0 \\
            \lambda e^{-\lambda x} & x\ge 0
    \end{cases}\]
    \[\int_{-\infty}^{\infty}p(x) dx  = \int_{0}^{\infty} \lambda e^{-\lambda x} dx = |-\lambda x = t, dt =  - \lambda dx| = -\int_{-\infty}^{0} \frac{e^t }{-\lambda} dt = e^t {|}_{-\infty}^0 = 1 - 0 = 1\]
\end{enumerate}
\begin{center}
    \begin{tabular}{ |c c c c| }
        \hline
        $p_\xi(x)$ & $F_\xi(x)$ & $M(\xi)$ & $D(\xi)$ \\ 
        \hline
        $\frac1{\sqrt{2\pi}\sigma} e^{-\frac{(x-a)^2}{2\sigma^2}}$ & $\frac12 [1 + erf(\frac{x-a}{\sqrt{2\sigma^2}})]$ & a  & $\sigma^2$\\ 
        
        ${\displaystyle\left\{{\begin{matrix}{\dfrac {1}{b-a}},&x\in [a,b]\\0,&x\not \in [a,b]\end{matrix}}\right..}$ &
        ${\displaystyle \left\{{\begin{matrix}0,&x<a\\{\dfrac {x-a}{b-a}},&a\leqslant x<b\\1,&x\geqslant b\end{matrix}}\right..}$ &
        $\frac{a+b}2$ & $\frac{{(b-a})^2}{12}$ \\


        $\displaystyle\left\{{\begin{matrix}x^{{\alpha-1}}{\frac  {e^{{-x\lambda }}}{\lambda ^{-\alpha}\,\Gamma (\alpha)}},&x\geq 0\\0,&x<0\end{matrix}}\right.$ & $\dots$ & $\alpha \lambda^{-1}$ & $\alpha \lambda^{-2}$ \\
        
     \hline
    \end{tabular}
    \end{center}
    \[\operatorname {erf}\,x={\frac  {2}{{\sqrt  {\pi }}}}\int \limits _{0}^{x}e^{{-t^{2}}}\,{\mathrm  d}t.\]
\subsection{Правила для вычисления}
\[M\xi = \int_{-\infty}^{\infty} x dF_\xi(x)\]
Для непрерывных случайных величин:
\[M\xi = \int_{-\infty}^{\infty} x p_\xi (x)dx\]
\[M g(\xi) = \int_{-\infty}^{\infty} g(x) p_\xi (x) dx\]
\section{Производящие функции}
\begin{definition}[Целочисленная случайная величина]
    Дискретная случайная величина $\xi$, принимающая только целые неотрицательные значения.

    Закон распределения: \[p_n = P\{\xi = n\}, n = 0, 1\dots, \quad \sum_{n = 0}^{\infty} p_n = 1\]
\end{definition}
\begin{definition}[Производящая функция]
    \[\phi_\xi(s) = M s^\xi = \sum_{n = 0}^{\infty} p_n s^n\]
    Ряд абсолютно сходится при $|s|\le 1$
\end{definition}
\subsection{Джентльменский набор}
\begin{enumerate}
    \item Равномерное дискретное распределение
    \[P\{\xi=k\} = \frac{1}{N}, \quad M\xi = \frac{1+N}{2}, \quad D\xi = \frac{N^2-1}{12}, \quad \phi(s) = \sum_{n=1}^{\infty} \frac{s^n}{n}=-\ln(1-s), \quad f_\xi (t) = -\ln (1-e^{it})\]
    \item Биномиальное (распределение Бернулли)
    \[P\{n=k\}=C_n^k p^k {(1-p)}^{n-k}, \quad M\xi = np, \quad D\xi = np(1-p), \quad \phi(s) = \sum_{m = 0}^{\infty} C_n^m p^m {(1-p)}^{n-m} = {(ps +1-p)}^n,\]
    \[f_\xi (t) = {(pe^{it} +1-p)}^n\]
    \item Геометрическое распределение
    \[P\{n=k\}=(1-p)p^k, \quad M\xi = \frac{p}{1-p}, \quad D\xi = \frac{p}{{(1-p)}^2}, \quad \phi(s) = \sum_{n=1}^{\infty}p^k (1-p) s^n =\frac{p}{1-(1-p)s}, \quad f_\xi (t) = \frac{p}{1-(1-p)e^{it}}\]
    \item Распределение Пуассона
    \[P\{n=k\}=\frac{\lambda^k}{k!}e^{-\lambda}, \quad M\xi = \lambda, \quad D\xi = \lambda, \quad \phi(s) = \sum_{n = 0}^{\infty} \frac{\lambda^n s^n}{n!}e^{-\lambda}=e^{\lambda (s-1)}, \quad f_\xi (t) = e^{\lambda (e^{it}-1)}\]
\end{enumerate} 
\section{Характеристические функции}
129-137 
\[\xi(t) = \xi_1(t)+ i\xi_2(t)\]
\[|M\xi| \le M |\xi|\]
\begin{definition}[характеристическая функция]
    Функция $f_\xi(t)$ называется характеристической функцией случайной величины $\xi$, если она имеет вид
    \[f_\xi(t)  = Me^{it\xi}\]
\end{definition}
Если $\xi$ - целочисленная случайная величина, то $\phi_\xi (z) = M z^\xi$
\[f_\xi (t) = M e^{it\xi} = M (e^{it})^\xi = \phi_\xi (e^{it})\]
(Свойства х.ф.)
\begin{enumerate}
    \item $|f_\xi(t)|\le 1, f_\xi(0) = 1$
    \item $f_\xi$ - равномерно непрерывна по t
    \item $f_{a\xi+ b}(t) = M e^{it(a\xi)}e^{itb} = e^{itb} M e^{i\xi(at)} = e^{itb} f_\xi(at)$
    \item $\xi_1, \dots, \xi_n$ - независимы, тогда
    \[f_{\xi_1+ \dots+ \xi_n}(t) = \prod_{i = 1}^n f_{\xi_i}(t)\]
    \[M e^{it (\xi_1+ \dots+ \xi_n)} = M \prod_{i = 1}^{n} e^{it \xi_i} = \prod_{j = 1}^{n} M e^{it \xi_j} = \prod_{j = 1}^n f_{\xi_j} (t)\]
    \item $f_\xi(-t) = \overline{f_\xi(t) }$
    \[M e^{-it\xi} = M \overline{ e^{it\xi}} = \overline{M e^{it\xi}}\]
    \item $\exists m_1, \dots, m_n = M \xi^n$ (существуют первые n моментов), тогда 
    \[f_\xi(t) = \sum_{k = 0}^{n}\frac{(it)^k}{k!}m_k + R_n(t),\]
    где $R_n(t) = o(t^n)$ при $t \to 0$
    \item \[\zeta  \begin{cases}
        \xi, с вероятностью p \\
        \eta, с вероятностью 1 - p,
        \end{cases}
        p \in (0, 1)\]
        
        \[f_\zeta (t)  = p f_\xi (t) + (1 - p) f_\eta(t)\]
\end{enumerate}
\begin{example}
    \begin{enumerate}
        \item $\cos(t)$

        \textsubscript{мы не знаем косинус\dots}
\[\xi = \begin{cases}
    -1 & p = 1/2 \\
    1 & p = 1/2
\end{cases}\quad \textsubscript{Бернуллиевская случайная величина}\]
$M e^{it\xi} = \frac12 e^{-it} = \frac12 e^{-it} = \cos t$
        \item $\cos^3(t)$

        свойство про независимость
        \item $\frac{\cos (t) + \cos(2t)}2$
        
        свойство про независимость, свойство про выпуклую комбинацию (3)
        \item $e^{-t^4}$
        
        шестое свойство, по формуле Тейлора
        \[e^{-t^4} = 1 - t^4 + o(t^4)\]

    \end{enumerate}
\end{example}
функция обращения - слишком тяжко, проверяем по свойствам а потом мучаемся (Фурье, Лаплас?)

$\xi = C $ с вероятностью 1
\[Me^{i\xi t} = e^{iCt}\]
\begin{remark}
    В силу 6 свойства, можно обобщить - если моменты до второго равны 0, то уже не характеристическая функция. (сравниваем с х.ф тождественного нуля, а $t^4$ высоко)
\end{remark}
\begin{enumerate}
    \item Стандартное распределение
    \[f_\xi(t) = e^{-t^2/2}\]
    \paragraph*{Мучаемся}
        \[f(t) = \frac1{\sqrt{2 \pi}}\int_{-\infty}^{\infty} e^{itx - x^2/2}dx\]
        дифференцируем.
        \[f'(t) = \frac{i}{\sqrt{2 \pi}}\int_{-\infty}^{\infty} xe^{itx - x^2/2}dx = \]
        \[u = e^{itx}, du = it e^{itx}dx, dv = \frac{xdx}{e^{x^2/2}} = \frac{d(x^2/2)}{e^{x^2/2}}= - d(-x^2/2)e^{-x^2/2} = -d(e^{-x^2/2}), v = -e^{-x^2/2}\implies\]
        \[ =\frac{i}{\sqrt{2 \pi}} (uv - \int_{-\infty}^{\infty} vdu)=\frac{i}{\sqrt{2 \pi}} ({-e^{itx}e^{-x^2/2}|_{-\infty}^{\infty} } + it\int_{-\infty}^{\infty} e^{itx - x^2/2}dx) = \frac{i^2 t}{\sqrt{2\pi}}\int_{-\infty}^{\infty} e^{itx - x^2/2}dx = -tf(t) \]
        \[f'(t) + tf(t) = 0,\] \[\text{уравнение с разделяющимися перем. с начальным условием} f(0) = 1 \text{(по свойству хар.функции)}\]
        \[f(t) = e^{-t^2/2}\]
    можно получить нормальное при помощи 3 свойства.
    \[f(t) = e^{ita} f_\xi(\sigma t) = e^{ita - (\sigma t)^2 /2} \]
    \item равномерное на [a, b]:
    \[f(t) = \frac1{b - a}\int_{a}^{b}e^{itx}dx\]
    \[f_\xi = \frac{e^{itb} - e^{ita}}{it(b-a)}\]
    \item Гамма распределение с параметром $\alpha$
    \[p(x) = \frac{x^{\alpha-1}}{\Gamma(\alpha)e^{-x}}\]
    \[f_\xi (t) = (1-it)^{-\alpha}\]
    Рассмотрим плотности гамма распределений с параметрами альфа и бета, плотность гамма распределения с параметром (альфа + бета) вычисляется через свертку:
    \[p_{\alpha + \beta}(x) = \int_{0}^{x}p_\beta(x-y)p_\alpha(y)dy = \frac{e^{-x}}{\Gamma(\alpha)\Gamma(\beta)}\int_{0}^{x}y^{\alpha - 1}(x-y)^{\beta - 1}dy = \] 
    \[ = \frac{e^{-x}}{\Gamma(\alpha)\Gamma(\beta)}\int_{0}^{x}\frac{y^{\alpha - 1}(x-y)^{\beta - 1}}{x^{\alpha-1}x^{\beta-1}}x^{\alpha-1}x^{\beta-1}dy = \left\lvert z = y/x, dz = dy/x, dy = xdz\right\rvert\] 
    \[ = \frac{e^{-x}x^{\alpha+\beta - 1}}{\Gamma(\alpha)\Gamma(\beta)}\int_{0}^{1} z^{\alpha-1}(1-z)^{\beta-1}dz = \frac{x^{\alpha+\beta-1}}{\Gamma(\alpha+\beta)}e^{-x}, x\ge 0\]
    определили независимость

    По 4 свойству \[f_{\alpha+\beta}(t) = f_{\alpha}(t) f_{\beta}(t)\]
    \[f_1(x) = \int_{0}^{\infty}e^{itx}p_1(x)dx = \int_{0}^{\infty}e^{itx-x}dx = \left\lvert du = e^{-x}dx, u = -e^{-x}, v = e^{itx}, dv = it e^{itx}\right\rvert \]
    \[= - e^{itx-x}\lvert_{0}^{\infty} + it \int_{0}^{\infty}e^{itx-x}dx = 1 + it f_1(t)\]
    \[f_1(t) = \frac1{1-it}\]
    \[f_n(t) = \frac1{(1-it)^n}\]
    \[f_{1/n} = (1-it)^{-1/n}\]
    \[f_{m/n} = (1-it)^{-m/n}\]
    Формула работает для рациональных чисел.
    Но можно сделать предельный переход и формула будет работать для всех положительных альфа. 
    \textsubscript{многозначная функция - нужно выделять ветвь $f_\alpha(0) = 1$}
\end{enumerate}
\begin{remark}[Вырожденное распределение]
    \[P\{\xi = C\} = 1, \quad f_\xi (t) = e^{itC}\]
\end{remark}
\begin{definition}[свертка]
    Свертка двух функций на прямой (обозначается $f \ast  g$) - это функция

$$f \ast  g : y \mapsto 
\int f(x)g(y - x)dx.$$
\end{definition}


\subsection{Абсолютно непрерывный случай}
\begin{definition}[L1-пространство]
    Пространством $L_1$ называется нормированное пространство, элементами которого служат классы эквивалентных между собой суммируемых функций; сложение элементов в $L_1$ и умножение их на числа определяются как обычное сложение и умножение функций, а норма задается формулой 
    \[\left\lVert f\right\rVert = \int |f(x)|d\mu \]
\end{definition}
\[f_\xi(t) = \int_{-\infty}^{\infty} e^{itx}p_\xi(x)dx \quad f_\xi (t) \textsubscript{ преобразование Фурье функции }p_\xi(x)\]
\[p_\xi(x) = \frac1{2\pi} \int_{-\infty}^{\infty} e^{-itx}f_{\xi}(t) dt\quad \textsubscript{Обратное преобразование Фурье}\]
имеют смысл для функций из $L_1(-\infty, -\infty)$, т.е. с конечным интегралом $\int_{-\infty}^{\infty} |f(t)|dt$

\begin{theorem}
    Пусть $f(t)$ - характеристическая функция и $F(x)$ - соответствующая функция распределения. Тогда, если x-l и x+l являются точками непрерывности функции $F(x)$, то 
    \[F(x+l) - F(x-l) = \lim_{\sigma\to 0}\frac1{\pi} \int_{-\infty}^\infty e^{-ixt}f(t)\frac{\sin{tl}}{t}e^{-\sigma^2t^2/2}dt\]
\end{theorem}
\begin{theorem}
    Каждой хар.функции соответствует только одна функция распределения.
\end{theorem}
с 146-153
\section*{Предельные теоремы}
\begin{theorem}[Центральная предельная теорема для одинаково распределенных с.ч.]
    Пусть $\xi_1, \dots, \xi_n$ - НОР (независимые одинаково распределенные величины), $M\xi_i = a, D \xi_i = \sigma^2$
    Тогда 
    \[P(\frac{\xi_1 + \dots + \xi_n - na}{\sigma \sqrt{n}}\le x)\to_{n\to \infty} \Phi(x) = \frac{1}{\sqrt{2\pi}}\int_{-\infty}^x e^{-t^2/2}dt\]
\end{theorem}
\begin{proof}
    докажем поточечную сходимость хар. функций - из нее будет выходить сходимость по распределению (или слабо)

    \(\widetilde{\xi_i} = \xi_i - a, f_{\widetilde{\xi_i}}(t) = f(t)\)

    Рассмотрим величину дзета
    \[\zeta_n = \frac{\xi_1 + \dots + \xi_n - na}{\sigma \sqrt{n}} = \frac1{\sigma \sqrt{n}}\sum_{i = 1}^n \widetilde{\xi_i}\]

    Воспользуемся свойствами хар. функций.

    \[f_{\zeta_n}(t) = f^n(\frac{t}{\sigma \sqrt{n}})\]
    Есть 2 момента - можно расписать как ряд Тейлора

    \[f(t) = 1 [\text{матожидание}] -\frac{\sigma^2 t^2}{2}+o(t^2)[\text{дисперсия}]\]

    \[f_{\zeta_n}(t) = f^n(\frac{t}{\sigma \sqrt{n}}) = (1 - \frac{t^2}{2n} +o(\frac1{n}))^n\to_{n\to\infty} e^{-t^2/2}\text{[второй замечательный предел]}\]
    
\end{proof}
Частный случай.
\begin{theorem}[Интегральная теорема Муавра-Лапласа]
    \[\xi_i = \begin{cases}
        1, prob = p \\ 
        0, prob = 1- p\\
    \end{cases}, \mu_n = \sum_{i = 1}^n \xi_i\]
    (число успехов)
    $P(\frac{\mu_n - np}{\sqrt{np(1-p)}}\le x) \to_{n \to\infty}\Phi(x)$
\end{theorem}
\begin{theorem}[Пуассона]
    \[C_n^k p^k (1-p)^{n-k} \to_{n\to\infty, np\to a} e^{-a}\frac{a^k}{k!}\]
    Много схем Бернулли, при этом средняя величина успехов с ростом n сходится к константе a
\end{theorem}
теорема о взаимнооднозначном непрерывном соответствии
\begin{proof}
    $(1-p+pz)^n = \phi_{\mu_n}(z)=  (1-\frac{a}{n}(1-z))^n \to_{n\to \infty} e^{-a(1-z)}$ - производящая функция распределения Пуассона.
\end{proof}
Подбор приближения - если np (матожидание) меньше 9 - приближение Пуассона, иначе - центральной предельной теоремой или теоремой Муавра-Лапласа.
\begin{example}
    Известно, что левши составляют 1 процент от популяции. Какова вероятность того, что по меньшей мере 4 левши будет среди
    \begin{enumerate}
        \item 200 людей

        p = 0.01. Нужно оценить вероятность
    \[P(S_n > 4) = 1 - P(S_n\le 3)=\]
    \[np = 2 < 9\]
    По формуле Пуассона 
    \[= 1 - P(S_n = 0) - P(S_n = 1) - P(S_n = 2) - P(S_n = 3) ~~ 1 - e^{-2} - 2e^{-2} - 2e^{-2} - \frac43 e^{-2} = 1 - \frac{19}3e^{-2}\]
    
        \item 10000 людей

        np = 100 > 9. Центральная предельная теорема
        \[P(S_n > 4) = 1 - P(S_n\le 3)\]
        \[P(\frac{S_n - 100}{\sqrt{99}} \le \frac{3 - 100}{\sqrt{99}}) ~~ \Phi(-\frac{97}{\sqrt{99}})\]
        \[P(S_n > 4) = 1 - P(S_n\le 3) = 1 - \Phi(-\frac{97}{\sqrt{99}})\]
    \end{enumerate}
\end{example}
\begin{example}
    sueta
    Бизнес
    Булочки с изюмом. Сколько изюма в расчете на тесто нужно, чтобы вероятность того, что булочка без изюма, была равна 
    0.01?

    Будем опрашивать изюминки, пойдут ли они в булку. Вероятность успеха (изюминка пойдет в булочку)

    N - число булочек. $p = \frac{1}{N}$

    Число изюминок на одну булочку k, число экспериментов в схеме Бернулли :n = $Nk$
    
    Все изюминки оказались идти в булочку.
    \[P(S_n = 0)\le 0.01\]
    Неравенство заменяем на равенство, а $S_n$ на приближение Пуассона(попробуем, так как не знаем np = k)
    \[e^{-k} = 0.01\]
    \[k = -\ln{0.01}\]
    Берем ceil, как раз реализуем неравенство.

    k = 5. Взяли правильное приближение в итоге.
\end{example}
Ляпунова не будет в задачах.
\section{Лемма Бореля-Кантелли}
(174-177)
$A_1, \dots, A_n \in \mathcal{F}$

Определим множество 

\[A^* \text{множество элементарных событий из} \Omega, \text{принадлежащим бесконечному множеству событий}A_n\]

\[A_*\text{множество элементарных событий из} \Omega | \exists N_0 \in \mathbb{N}\forall n> N_0 \]
\[A^* = \limsup A_n = \bigcap^\infty_{n = 1} \bigcup_{k\ge n} A_k\]
\[A_* = \liminf A_n = \bigcup^\infty_{n = 1} \bigcap_{k\ge n} A_k\]

\begin{theorem}[Лемма Бореля-Кантелли]
    Если $\sum_{n\ge 0}P(A_n) < \infty$, то $P(A^*) = 0$

    Если $\sum_{n\ge 0}P(A_n) = \infty$ и $A_1, \dots, A_n$ - независимы, то $P(A^*) = 1$
\end{theorem}
\begin{proof}
    \[\xi = \sum_{i = 1}^{\infty}I_{A_n}\]
    \[M\xi = \sum_{i = 1}^\infty P(A_n) <\infty\]
    С одной стороны
    \[P(\xi = \infty) = 0\]
    С другой стороны, событие $\xi = \infty$ - $A^*$
    
    Вложенная цепочка убывающих по включению
    \[P(A^*)= \lim_{n\to \infty}P( \bigcup_{k\ge n }A_k)=\lim_{n\to\infty}(1 - P(\bigcap_{k\ge n}\overline{A_k})) \]
    независимость
    \[\lim_{n\to\infty}(1 - \prod_{k\ge n}(1 - P(A_k)))\]
    \[\lim_{m\to\infty} \prod_{k = n}^{m+n}(1 - P(A_k)) = e^{-\infty} = 0\]
\[\lim_{n\to\infty}(\ln(1 - P(A_n)) + \dots + \ln(1 - P(A_{n+m})))\]
\[\sum \ln(1 - \alpha_k) = -\infty\]
\end{proof}
Пусть имеется некоторая последовательность независимых случайных величин $\xi_1, \dots, \xi_n ...$

минимальная сигма-алгебра, относительно которой все случайные величины из списка являются измеримыми

$\widetilde{f_{\le n}} = \sigma(\xi_1, \dots, \xi_n)$

$\widetilde{f_{> n}} = \sigma(\xi_n+1, \dots)$

$\widetilde{f_\infty} = \bigcap_{n\ge 1} \widetilde{f_{> n}}$ - сигма алгебра остаточных событий.
\begin{lemma}[закон 0 и 1 Колмогорова]
    вероятности остаточных событий (элементы сигма алгебры остаточных событий) могут принимать либо значение 0, либо 1
\end{lemma}
\begin{proof}
    Пусть $A\in \widetilde{f_\infty}, B \in \widetilde{f_{\le n}}$. Они независимы.

    Это справедливо для любого n. $\forall B\in \sigma(\xi_1, \dots, \xi_n \dots): $ A и B независимы

    $\implies $ A не зависит само от себя $\implies A$ либо 0, либо 1


\end{proof}
\end{document}