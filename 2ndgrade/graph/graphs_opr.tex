\documentclass[a4paper]{article}
\usepackage{cmap}
\usepackage[utf8]{inputenc}
\usepackage[T2A]{fontenc}
\usepackage{amsfonts}
\usepackage{amsmath, amsthm}
\usepackage{amssymb}
\usepackage{hyperref}
\usepackage{multicol}
\usepackage{tikz} 

\newcommand\letsymbol{\mathord{\sqsupset}}
\usepackage[russian]{babel}
\renewcommand\qedsymbol{$\blacktriangleright$}
\newtheorem{theorem}{Теорема}[section]
\newtheorem{lemma}{Лемма}[section]
\theoremstyle{definition}
\newtheorem*{example}{Пример}
\newtheorem*{definition}{Определение}
\newtheorem*{statement}{Утверждение}
\theoremstyle{remark}
\newtheorem*{remark}{Замечание}
\newtheorem*{corollary}{Следствие}

\setlength{\topmargin}{-0.5in}
\setlength{\oddsidemargin}{-0.5in}
\textwidth 185mm
\textheight 250mm

\begin{document}
\begin{multicols*}{2}
    \tableofcontents
    \pagenumbering{arabic}
    \setcounter{page}{1}
    \section{Определение графа. Примеры графов. Степени вершин графа. Лемма о рукопожатиях}
    \begin{definition}
        Граф (неориентированный) состоит из непустого конечно множества V и конечного множества
        E неупорядоченных пар элементов из V (записывается G =  (V, E)).
    \end{definition}
    Элементы множества $V = V_G$ называются \textbf{вершинами}, а элементы множества $E = E_G$ - \textbf{ребрами} графа G.
    Те и другие называются \textbf{элементами} графа.
    \begin{definition}
        Если $\{u, v\}\in E$, то будем записывать $e = uv$ и говорить, что 
        вершины u и v смежны, а вершина u и ребро e инцидентны (так же, как вершина
        v и ребро e). Два ребра называются смежными, если они имеют общую вершину.
    \end{definition}
    \begin{definition}
        Степенью вершины v в графе G называется число ребер, инцидентных вершине v (обозначается $d_G(v) = d(v)$).
    \end{definition}
    Вершина степени 0 - изолированная, вершина степени 1 - висячая.
    Минимальная и максимальная степени вершин графа G обозначаются $\delta(G), \Delta(G)$.

    Последовательность степеней вершин графа G, выписанных в порядке неубывания называется степенной
    последовательностью или вектором степеней графа G.
    \begin{definition}
        Кратные ребра - два и более ребра, соединяющие одну и ту же пару вершин.
    \end{definition}
    \begin{definition}
        Петли - ребра, соединяющие вершины сами с собой.
    \end{definition}
    \begin{definition}
        Мультиграф - граф с кратными ребрами
    \end{definition}
    \begin{definition}
        Обыкновенный граф - граф без петель и кратных ребер.
    \end{definition}
    Примеры графов:
    \begin{enumerate}
        \item Граф G = (V, E) с n вершинами и m ребрами называется (n, m)-графом,
        (1, 0)-граф называется тривиальным.
        \item Пустой граф - $O_n$
        \item Полный граф $K_n, C_n^2 = \frac{n(n-1)}{2}$
        \item Двудольный граф $G = (V_1, V_2; E)$
        \item Полный двудольный граф - $K_{p,q}$
        \item Звезда - полный двудольный граф $K_{1,q}$
        \item Простой цикл $C_n$
        \item Регулярный (однородный) граф - граф, все вершины которого имеют
        одну и ту же степень. 
        Кубические графы - 3-регулярные
        \item Графы многогранников 
    \end{enumerate}
    \begin{lemma}[О рукопожатиях]
        Сумма степеней всех вершин произвольного графа $G = (V, E)$ - четное число, равное удвоенному
        числу его ребер:
        $\sum_{v\in V} d_G(v) = 2|E|$
    \end{lemma}
    \begin{proof}
        Индукция по числу ребер.\\
        База: если в графе G нет ребер, то $\sum_{v\in V} d_G(v) = 0$.
        Предположим, что формула верна для любого графа, число ребер в котором не превосходит $m\leq 0$.\\
        Пусть $|E| = m+1$. Рассмотри произвольное ребро $e = uv\in E$ и удалим его из графа G. Получим граф
        $G' = (V, E'), |E'| = m$. По предположению индукции $\sum_{v\in V} d_{G'}(v) = 2 |E'| = 2m$\\
        Тогда $\sum_{v\in V} d_{G}(v) = \sum_{v\in V} d_G(v) + 2 = 2m + 2 = 2|E|$.
    \end{proof}
    Теорема имеет место быть и для мультиграфов.
    \begin{corollary}
        В любом графе число вершин нечетной степени четно.
    \end{corollary}
    \section{Маршруты, цепи, циклы. Лемма о выделении простой цепи. Лемма об объединении 
    простых цепей}
    \begin{definition}
        Маршрутом, соединяющим вершины u, v (или (u,v)-маршрутом) называется чередующаяся последовательность вершин и ребер графа $$(u = v_1, e_1, \dots, v_k, e_k, v_{k+1} = v)$$ такая, что $e_i = v_i v_{i+1}, i = 1, \dots, k$. Маршрут замкнут, если $v_1 = v_{k+1}$.
    \end{definition}
    \begin{definition}
        Цепь - маршрут, в котором все ребра различны.
    \end{definition}
    \begin{definition}
        Простая цепь - маршрут, в котором все вершины различны(м. б. кроме 1 и последней).  $(P_k)$
    \end{definition}
    \begin{definition}
        Цикл - замкнутая цепь.
    \end{definition}
        Простой цикл - замкнутая простая цепь.  $(C_k)$
    \begin{lemma}[О выделении простой цепи]
        Всякий незамкнутый (u,v)-маршрут содержит простую (u,v)-цепь.
    \end{lemma}
    \begin{corollary}
        Всякий замкнутый маршрут содержит простой цикл.
    \end{corollary}
    \begin{lemma}[Об объединении простых цепей]
        Объединение двух несовпадающих простых (u, v)-цепей содержит простой цикл.
    \end{lemma}
    \begin{corollary}[Свойство циклов]
        Если С и D - два несовпадающих простых цикла, имеющих общее ребро е, то граф $(C\cup D) - e$ содержит простой цикл. 
    \end{corollary}
    Расстояние - это метрика.
    \section{Эйлеровы графы. Критерий существования эйлерова цикла (теорема Эйлера)}
    Пусть $G = (V, E)$ - произвольный, возможно мультиграф.
    \begin{definition}
        Цикл в графе называется эйлеровым, если он содержит все ребра этого графа. Граф эйлеров, если содержит эйлеров цикл.
    \end{definition}
    \begin{theorem}[Эйлер]
        В связном $G = (V, E)$ существует эйлеров цикл, если и только если все его вершины имеют четную степень.
    \end{theorem}
    \begin{theorem}
        Цепь в графе G - эйлерова, если содердит все ребра графа. Граф, содержащий эту цепь, называется полуэйлеровым.
    \end{theorem}
    \begin{corollary}[Эйлер]
        В связном графе существует (незамкнутая) эйлерова цепь тогда и только тогда, когда все его вершины за исключением, быть может, двух, имеют четную степень.
    \end{corollary}
    \section{Гамильтоновы графы. Достаточные условия существования гамильтонова цикла (теоремы 
    Оре и Дирака)}
    \begin{definition}
        Простой цикл называется гамильтоновым, если содержит все вершины графа. Граф называется гамильтоновым, если он содержит гамильтонов цикл.
    \end{definition}
    \begin{theorem}[Оре]
        Пусть $n\geq 3$. Если в n-вершинном графе G для любой пары u, v несмежных вершин выполнено неравенство $d(u)+ d(v)\geq n$, то G - гамильтонов граф.
    \end{theorem}
    \begin{theorem}[Дирак]
        Пусть $n\geq 3$. Если в n-вершинном графе G для любой вершины v выполнено неравенство $d(v)\geq n/2$, то G - гамильтонов граф.
    \end{theorem}
    \section{Изоморфизм графов. Помеченные и непомеченные графы. Теорема о числе помеченных 
    n-вершинных графов}
    \begin{definition}
        Графы $G = (V_G, E_G), H = (V_H, E_H)$ называются изоморфными ($G\cong H$), если между множествами их вершин существует взаимно однозначное соответствие $\phi: V_G \to V_H,$ сохраняющее смежность:
        $$\forall u, v\in V_G\quad u, v\in E_G \Leftrightarrow \phi(u)\phi(v)\in E_H$$
    \end{definition}
    Отношение изоморфизма - отношение эквивалентности 
    \begin{definition}
        Граф \textbf{помеченный}, если его вершины отличаются одна от другой какими-либо метками. Если же графы различаются лишь с точностью до изоморфизма, говорят о \textbf{непомеченных} графах.
    \end{definition}
    \begin{theorem}
        Число $p_n$ различных помеченных n-вершинных графов с фиксированным множеством вершин V равно $2^{\frac{n(n-1)}{2}}$
    \end{theorem}
    \begin{theorem}[Формула Пойа]
        $q_n$(количество непомеченных n-вершинных)$ \sim \frac{2^{\frac{n(n-1)}{2}}}{n!}$
    \end{theorem}
    \section{Проблема изоморфизма. Инварианты графа. Примеры инвариантов. Пример полного 
    инварианта}
    \begin{definition}
        Подграфом графа $G = (V, E)$ называется произвольный граф $H = (U, D)$ такой, что $U\subseteq V, D \subseteq E$ (обозначается $H\subseteq G$)
    \end{definition}
    \begin{definition}
        Подграф $H = (U, D)$ называется порожденным, если 
        $$\forall u, v\in U \quad uv\in D \Leftrightarrow uv \in E$$
    \end{definition}
    \begin{definition}
        Остовный подграф - это подграф графа G, содержащий все его вершины
    \end{definition}
    G - U, G - D тоже подграфы.
    \begin{definition}
        Инвариант графа G - это величина $i(G)$ (число, набор чисел, или функция), связанная с графом и принимающая одно и то же значение на любом графе, изоморфном G, т.е. $G\cong H \Rightarrow i(G)=i(H).$ Инвариант называется полным, если для любых графов G и H $i(G) = i(H) \Rightarrow G\cong H$
    \end{definition}
    \textbf{[Примеры инвариантов]}
    \begin{enumerate}
        \item n(G) вершины
        \item m(G) ребра
        \item $\delta(G)$ минимальная степень
        \item Степенная последовательность (в порядке неубывания)
        \item Определитель матрицы смежности, характеристический многочлен
    \end{enumerate}
    \begin{definition}[Полный инвариант - миникод]
        Пусть $G = (V, E), V = \{1, \dots, n\}, A(G)$ - матрица смежности.
        Выпишем в определенном порядке лишь элементы, расположенные выше главной диагонали, например по столбцам:
        $$a_{12}, a_{13}, a_{23}\dots, a_{1n}, a_{2n}\dots, a_{(n-1)n}$$
        Полученное двоичное слово длины $\frac{(n-1)n}{2}$ называется двоичным кодом матрицы $A(G)$. Число
        $$\mu(G) = a_{12} 2^0 + a_{13} 2^1\dots+ a_{(n-1)n}2^{k-1}$$
        называется каноническим кодом матрицы $A(G)$.
        Канонические коды одного и того же графа зависят от нумерации его вершин. Наименьший канонический код будем называть миникодом $\mu(G)$ графа G.
    \end{definition}
    Миникод - число от 0 до $2^{\frac{(n-1)n}{2}} - 1$
    \begin{theorem}
        $(\mu(G), n(G))$ - полная система инвариантов (миникод и число вершин графа).
    \end{theorem}
    \section{Связные и несвязные графы. Лемма об удалении ребра. Оценки числа ребер связного 
    графа}
    \begin{definition}
        Вершины u, v в графе G называются соединимыми, если в G существует $(u,v)$-маршрут.
    \end{definition}
    \begin{definition}
        Граф G называется связным, если любые его две вершины соединимы. 
    \end{definition}
    Тривиальный (1, 0)-граф по определению считается связным.
    \begin{definition}
        Компонента связности графа G - максимальный(по включению) связный подграф.
    \end{definition}
    \begin{definition}
        Ребро называется циклическим, если оно принадлежит некоторому циклу, и ациклическим в противном случае.
    \end{definition}
    \begin{lemma}[Об удалении ребра]
        Пусть $G = (V, E)$ - связный граф, $e\in E$.
        \begin{enumerate}
            \item Если e - циклическое ребро, то граф G - e связен;
            \item Если e - ациклическое, то граф G - e имеет ровно две компоненты связности
        \end{enumerate}
    \end{lemma}
    \begin{corollary}
        Если граф $G = (V, E)$ имеет k компонент связности и $e\in E,$ то граф G-e имеет k или k+1 компонент.
    \end{corollary}
    \begin{lemma}[Об удалении вершины]
        В любом нетривиальном связном графе G существует вершина, после удаления которой граф остается связным.
    \end{lemma}
    \begin{corollary}
        В любом нетривиальном связном графе существует не менее двух вершин, после удаления каждой из которых граф остается связным.
    \end{corollary}
    \begin{theorem}[Оценки числа ребер связного графа]
        Если G - связный (n, m)-граф, то 
        $$n-1 \leq m \leq \frac{n(n-1)}{2}$$
    \end{theorem}
    \section{Плоские и планарные графы. Графы Куратовского. Формула Эйлера для плоских графов}
    \begin{definition}
        Плоский граф - граф, вершины которого являются точками плоскости, а ребра - непрерывными плоскими линиями без самопересечений, соединяющими вершины так, что никакие два ребра не имеют общих точек вне вершин.
    \end{definition}
    \begin{definition}
        Планарный граф - граф, изоморфный некоторому плоскому.
    \end{definition}
    \begin{definition}
        Два графа называются гомеоморфными, если их можно получить из одного и того же графа с помощью разбиений ребер, т.е. замены некоторых ребер простыми цепями.
    \end{definition}
    \subsection*{Графы Куратовского}
    \begin{remark}
        Графы $K_{3,3}$ и $K_5$ непланарны
    \end{remark}
    \begin{proof}
        $K_{3, 2}$ - плоский, в нем по формуле Эйлера 3 грани независимо от способа изображения.
        Пытаемся добавить 6 вершину, подставляя ее в каждую грань, получаем каждый раз противоречие - 
        невозможность соединить вершину с необходимыми.
        Аналогично для $K_5$.
    \end{proof}
    \begin{theorem}[Понтрягин-Куратовский]
        Граф планарен тогда и только тогда, когда он не содержит подграфов, гомеоморфных $K_{3,3}$ или $K_5$
    \end{theorem}
    \begin{definition}
        Гранью плоского графа называется максимальное множество точек плоскости такое, что каждая пара из которого может быть соединена непрерывной линией, не пересекающей ребер графа.
    \end{definition}
    \begin{theorem}[Формула Эйлера для плоских графов]
        Для любого связного плоского графа G = (V, E) верно $n - m + l = 2$, где n = |V|, m = |E|,
        l - число граней 
    \end{theorem}
    \begin{proof}
        Рассмотрим две операции перехода от связного плоского графа G к его связному 
        плоскому подграфу, не изменяющие величины $n - m + l$
        \begin{enumerate}
            \item удаление ребра, принадлежащего сразу 2 граням (одна из которых может быть внешней) \textbf{уменьшает m и l на 1}
            \item удаление висячей вершины (вместе с инцидентным ребром) \textbf{уменьшает m и n на 1}
        \end{enumerate}
        Очевидно, что любой связный граф после этих операций может быть приведен к тривиальному, а для него формула верна $\implies$
        верна и для данного 
    \end{proof}
    \section{Деревья. Теорема о деревьях (критерии)}
    \begin{definition}
        Граф называется ациклическим, если в нем нет циклов.
    \end{definition}
    \begin{definition}
        Дерево - связный ациклический граф.
    \end{definition}
    \begin{definition}
        Лес - несвязный ациклический граф.
    \end{definition}
    \begin{theorem}[о деревьях №1]
        Для (n, m)-графа G следующие определения эквивалентны:
        \begin{enumerate}
            \item G - дерево
            \item G - связный граф и $m = n - 1$
            \item G - ациклический граф и $m = n - 1$
        \end{enumerate}
    \end{theorem}
    \begin{proof}
        \begin{itemize}
            \item[$1 \to 2$] Дерево - связный, планарный граф (имеет 1 грань) $\implies$ $n - m + 1 = 2\implies m = n - 1$
            \item[$2 \to 3$] Пусть граф не ациклический $\implies$ есть цикл и e - циклическое ребро.
            Тогда по лемме об удалении ребра граф $G - e$ также связен и имеет m - 1 = n - 2 ребер
            $\implies$ противоречие оценке числа ребер связного графа $\implies$ граф ациклический
            \item[$3 \to 1$] Обозначим число компонент связности - k. Пусть $T_i$ - iтая компонента,
            является $(n_i, m_i)$-графом. Т.к $T_i$ - дерево, то по ранее доказанному ($1 \to 2$) $m_i = n_i - 1, i = \overline{1, k}$
            $\implies n - 1 = m = \sum_{i = 1}^k m_i = \sum_{i = 1}^k n_i - k = n - k \implies k = 1\implies$ граф связный 
        \end{itemize}
    \end{proof}
    \begin{theorem}[о деревьях №2]
        Для (n, m)-графа G следующие определения эквивалентны:
        \begin{enumerate}
            \item G - дерево
            \item G - ациклический граф и если $\forall$ пару несмежных вершин соединить ребром, то
            полученный граф будет содержать ровно 1 цикл
            \item $\forall$ 2 вершины графа G соединены единственной простой цепью
        \end{enumerate}
    \end{theorem}
    \begin{proof}
        \begin{itemize}
            \item[$1\to 2$] В связном графе все несмежные вершины соединены простой цепью \textbf{(по лемме о выделении простой цепи)}
            $\implies$ добавление ребра e = uv приведет к образованию цикла, а два цикла образоваться не может в силу свойства циклов
            \item[$2 \to 3$] любые две несмежные вершины u,v графа G соединимы, иначе при добавлении ребра uv не появится цикл$\implies$
            в силу леммы о выделении простой цепи любые две вершины соединены простой цепью. А она единственная, иначе по лемме об 
            объединении простых цепей в графе G был бы цикл.
            \item[$3\to1$]из условия следует, что граф связен, а существование цикла противоречит условию единственности цепи$\implies$
            граф ациклический.
        \end{itemize}
    \end{proof}
    \begin{lemma}[О листьях дерева]
        В любом нетривиальном дереве имеется не менее двух листьев
    \end{lemma}
    \begin{proof}
        $\forall v\in V$ $d(v)\geq 1$

        $\sum_{v\in V} = 2|E|=2m=2(n-1)=2n-2$

        Если 2 листа - то у 2 вершин степень 1 и у остальных n-2 как минимум 2,
        а для меньшего количества листьев оценка суммы неверна

        $\sum_{v\in V} \leq 2 + (n-2)2 = 2n - 2$
    \end{proof}
    \section{Перечисление деревьев. Теорема Кэли о числе помеченных n-вершинных деревьев}
    \begin{theorem}[Кэли]
        Число помеченных деревьев с n вершинами равно $t(n) = n^{n-2}$
    \end{theorem}
    \section{Центр дерева. Центральные и бицентральные деревья. Теорема Жордана}
    \begin{definition}
        Эксцентриситет вершины v в связном графе $G = (V, E)$ - $\epsilon(v)=\max_{u\in V} d(v, u)$ - расстояние от v до самой удаленной вершины.
    \end{definition}
    \begin{definition}
        Радиус связного графа - наименьший из эксцентриситетов $r(G) = \min_{v\in V}\epsilon(v)$
    \end{definition}
    \begin{definition}
        Вершина $v\in V$ называется центральной, если $\epsilon(v) = r(G)$
    \end{definition}
    \begin{definition}
        Множество всех центральных вершин графа называется его центром.
    \end{definition}
    \begin{theorem}[Жордан]
        Центр любого дерева состоит из одной или двух смежных вершин.
    \end{theorem}
    \begin{definition}
        Дерево, центр которого состоит из 1 вершины, называется центральным (если из 2, то бицентральным)
    \end{definition}
    \section{Изоморфизм деревьев. Процедура кортежирования. Теорема Эдмондса}
    \begin{theorem}[Теорема Эдмондса]
        Два дерева изоморфны $\Leftrightarrow$ совпадают их центральные кортежи
    \end{theorem}
    \begin{proof}
        \begin{itemize}
            \item[$\Rightarrow$] Если деревья изоморфны, то любой изоморфизм биективно отображает $V_1$ на $V_1'$ (листья каждого уровня,
            $V_2 = T - V_1 $etc). Поэтому для соответствующих вершин совпадают уровень и их кортежи, в т.ч и центральные.
            \item[$\Leftarrow$] Пусть центральные вершины v и v' произвольных деревьев T и T'
            имеют одинаковые кортежи. По кортежу однозначно восстанавливаются кортежи смежных с v вершин
            низших уровней $(1, \dots s)$, их количество и порядок следования, в поддереве $T_v$ это
            все вершины, смежные с v, то же самое справедливо для вершин из T'. Таким образом,
            из совпадения центральных кортежей следует совпадение кортежей для вершин $v_j, v_j', j = 1, \dots s$.
            После применения к каждой паре невисячих вершин $v_j, v_j'$получаем изоморфизм $T_v, T_v'$

            Для центрального дерева доказательство закончено, а для бицентрального повторяем доказательство для второй пары центральных
            вершин. 
        \end{itemize}
    \end{proof}
    \section{Вершинная и реберная связность графа. Основное неравенство связности}
    \begin{definition}
        Вершинной связностью нетривиального графа G называется наименьшее число $\varkappa(G)$ вершин графа G, в результате удаления которых получается несвязный или тривиальный граф. Для тривиального графа полагаем $\varkappa(O_1) = 0$.
    \end{definition}
    \begin{definition}
        Реберной связностью нетривиального графа G называется наименьшее число $\lambda(G)$ ребер графа G, в результате удаления которых получается несвязный граф. Для тривиального графа полагаем $\lambda(O_1) = 0$.
    \end{definition}
    \begin{theorem}[Основное неравенство связности]
        Для любого графа G = (V, E) $$\varkappa(G) \leq \lambda(G)$$
    \end{theorem}
    \section{Отделимость и соединимость. Теорема Менгера}
    \begin{definition}
        Пусть G = (V, E) - связный граф, s, t - две его несмежные вершины. Говорят, что множество вершин $W \subset V$ \textbf{разделяет} вершины s, t, если они принадлежат разным компонентам связности графа G - W.
    \end{definition}
    \begin{definition}
        Несмежные вершины s, t называются \textbf{k-отделимыми}, если k равно наименьшему числу вершин, разделяющих s и t. Говорят также, что отделимость вершин s, t равна k.
    \end{definition}
    \begin{definition}
        Две простые цепи, соединяющие вершины s, t называются вершинно-независимыми, если они не имеют общих вершин, отличных от s и t.
    \end{definition}
    \begin{definition}
        Вершины s, t называются l-соединимыми, если наибольшее число вершинно-независимых (s,t)-цепей равно l. Говорят также, что в этом случае соединимость вершин s, t равна l.
    \end{definition}
    \begin{theorem}[Менгер]
        В связном графе несмежные вершины k-отделимы тогда и только тогда, когда они k-соединимы.
    \end{theorem}
    \section{Реберный вариант теоремы Менгера}
    \begin{definition}
        G = (V, E) - связный граф, s, t - две его произвольные вершины. Две простые цепи, соединяющие s, t называются реберно независимыми, если они не имеют общих ребер.
    \end{definition}
    \begin{definition}
        Вершины s,t называются l-реберно-соединимыми, если наибольшее число реберно-независимых (s,t)-цепей равно l.
    \end{definition}
    \begin{definition}
        Множество ребер $R\subseteq E$ разделяет s, t, если s и t принадлежат разным компонентам связности графа G-R
    \end{definition}
    \begin{definition}
        Вершины s, t называются k-реберно-отделимыми, если k равно наименьшему числу ребер, разделяющему s и t.
    \end{definition}
    \begin{theorem}
        В связном графе G две вершины k-реберно-отделимы тогда и только тогда, когда они k-реберно соединимы.
    \end{theorem}
    \section{Критерии вершинной и реберной k-связности графа (без доказательства)}
    \begin{theorem}
        Граф G k-связен тогда и только тогда, когда любая пара его вершин соединена не менее, чем k вершинно-независимыми цепями.
    \end{theorem}
    \begin{theorem}
        Граф k-реберно-связен тогда и только тогда, когда любая пара его вершин соединена не менее, чем k-реберно-независимыми цепями.
    \end{theorem}
    \section{Ориентированные графы. Основные понятия. Ормаршруты и полумаршруты. 
    Ориентированые аналоги теоремы Менгера}
    \begin{definition}
        Ориентированный граф(орграф) G состоит из непустого конечного множества V и конечного множества $E\subseteq V \times V$ упорядоченных пар элементов множества V. Элементы множества E - дуги.
    \end{definition}
    Если e = uv - дуга, то u - ее начало, а v - конец.
    \begin{definition}
        Полустепень исхода $d^+(v)$ вершины v - число дуг, выходящих из v, а полустепень захода $d^-(v)$ вершины v - число дуг, заходящих в v.
        Степень вершины - $d(v) = d^+(v) + d^-(v)$
    \end{definition}
    \begin{lemma}[Орлемма о рукопожатиях]
        Сумма полустепеней исхода всех вершин орграфа $G = (V, E)$ равна сумме полустепеней захода и равно числу дуг орграфа:
        $$\sum_{v\in V}d^+(v) = \sum_{v\in V}d^-(v) = |E|$$
    \end{lemma}
    \begin{definition}
        Подграфом орграфа $G = (V, E)$ называется произвольный орграф $H = (U, D)$ такой, что $U\subseteq V, D \subseteq E$ (обозначается $H\subseteq G$).
    \end{definition}
    \begin{definition}
        Подграф $H = (U, D)$ называется порожденным, если 
        $$\forall u, v\in U \quad uv\in D \Leftrightarrow uv \in E$$
    \end{definition}
    \begin{definition}
        Орграфы $G = (V_G, E_G), H = (V_H, E_H)$ называются изоморфными ($G\cong H$), если между множествами их вершин существует взаимно однозначное соответствие $\phi: V_G \to V_H,$ сохраняющее смежность:
        $$\forall u, v\in V_G\quad u, v\in E_G \Leftrightarrow \phi(u)\phi(v)\in E_H$$
    \end{definition}
    \begin{definition}
        Ориентированным $(v_1, v_{k+1})$-маршрутом (ормаршрутом) в орграфе G называется
        чередующаяся последовательность вершин и дуг
        \begin{align*}
            P = (v_1, e_1, \dots, e_k, v_{k+1})
        \end{align*}
        в которой $e_i = v_iv_{i+1}$
    \end{definition}
    Если в орграфе нет кратных дуг, то ормаршрут может быть задан последовательностью
    входящих в него вершин. В любом случае ормаршрут задается дугами.
    \begin{definition}
        Орцепь - ормаршрут, в котором все дуги различны.
    \end{definition}
    \begin{definition}
        Простая орцепь (путь) - ормаршрут, в котором все вершины различны. (м.б кроме 1 и последней)
    \end{definition}
    \begin{definition}
        Орцикл - замкнутая орцепь.
    \end{definition}
    \begin{definition}
        Контур - замкнутый путь.
    \end{definition}
    \begin{definition}
        Полумаршрут - последовательность вершин и дуг орграфа, если для $\forall i = \overline{1, k}$ $e_i = v_iv_{i+1}\in E$
        $\vee e_i = v_{i+1}v_i\in E$
    \end{definition}
    Аналогично определяются \textbf{полупуть и полуконтур}.
    \begin{definition}
        Орграф называется обыкновенным, если он не содержит петель и кратных дуг.
    \end{definition}
    \begin{definition}
        Орграф называется направленным, не имеющий симметричных пар дуг.
    \end{definition}
    \begin{definition}
        Основание орграфа G - неориентированный орграф, полученный снятием ориентации всех дуг.
    \end{definition}
    \begin{definition}
        Два (s, t)-пути называются вершинно-независимыми, если у них нет общих вершин, отличных от s и t, и независимыми по дугам, если они не имеют общих дуг.
    \end{definition}
    \begin{definition}
        Множество W вершин орграфа G называется (s, t)-разделяющим, если в орграфе G - W вершина t не достижима из s.
    \end{definition}
    \begin{definition}
        Множество R дуг орграфа G называется (s, t)-разделяющим, если в орграфе G - R вершина t не достижима из s.
    \end{definition}
    \begin{theorem}
        G = (V, E) - слабо связный орграф. Для любой пары вершин $s, t \in V$ таких, что $st\notin E$, наименьшее число вершин в (s, t)-разделяющем множестве равно наибольшему числу вершинно-независимых (s, t)-цепей.
    \end{theorem}
    \begin{theorem}
        G = (V, E) - слабо связный орграф. Для любой пары вершин $s, t \in V$ наименьшее число дуг в (s, t)-разделяющем множестве равно наибольшему числу независимых по дугам  (s, t)-цепей.
    \end{theorem}
    \section{Ориентированные графы. Достижимость и связность. Три типа связности. Критерии 
    сильной, односторонней и слабой связности орграфа}
    \begin{definition}
        Если в орграфе G = (V, E) существует ориентированный (u, v)-маршрут, то говорят, что вершина v достижима из вершины u. Любая вершина считается достижимой из самой себя.
    \end{definition}
    \begin{definition}
        Орграф называется сильно связным(или сильным), если любые его две вершины взаимно достижимы.
    \end{definition}
    \begin{definition}
        Орграф называется односторонне связным, если для любой пары его вершин хотя бы одна достижима из другой.
    \end{definition}
    \begin{definition}
        Орграф называется слабо связным, если любые две его вершины соединены полумаршрутом.
    \end{definition}
    \begin{definition}
        Орграф называется несвязным, если он даже не является слабым.
    \end{definition}
    \begin{remark}
        Орграф несвязен тогда и только тогда, когда его основание - несвязный граф.
    \end{remark}
    Тривиальный орграф является сильным, односторонним и слабым.
    \begin{theorem}[Критерий сильной связности]
        Орграф является сильно связным тогда и только тогда, когда в нем есть остовный замкнутый ормаршрут.
    \end{theorem}
    \begin{theorem}[Критерий односторонней связности]
        Орграф является односторонне связным тогда и только тогда, когда в нем есть остовный ормаршрут.
    \end{theorem}
    \begin{theorem}[Критерий слабой связности]
        Орграф является слабо связным тогда и только тогда, когда в нем есть остовный полумаршрут.
    \end{theorem}
    \section{Основные структуры данных для представления графов в памяти компьютера. Их 
    достоинства и недостатки}
    \begin{enumerate}
        \item Матрица инцидентности
        
        Классический способ представления графа в теории.

        $\Theta(nm)$ ячеек памяти, большинство из которых занято нулями.

        Ответ на вопрос "смежны ли некоторые вершины u, v?" или "существует ли вершина, смежная с данной вершиной v?" - $O(m)$.
        \item Матрица смежности

        $\Theta(n^2)$ ячеек памяти

        Начальное заполнение матрицы - $\Theta(n^2)$

        "смежны ли некоторые вершины u, v?" - O(1).

        "существует ли вершина, смежная с данной вершиной v?" - $O(n)$.
        \item Массив ребер или дуг
        \item Списки соседних вершин (списки инцидентности)
        \item Списки соседних вершин с перекрестными ссылками
        \item Списки соседних вершин для орграфов
    \end{enumerate}
    \section{Влияние структур данных на трудоемкость алгоритмов (на примере алгоритма 
    отыскания эйлерова цикла)}
    \section{Задача о минимальном остовном дереве. Алгоритм Прима}
    \section{Задача о кратчайших путях. Случай неотрицательных весов дуг. Алгоритм Дейкстры}
    \section{Потоки в сетях. Увеличивающие пути. Лемма об увеличении потока}
    \begin{definition}
        Пусть задана двухполюсная сеть $G = (V, E)$ с источником s, стоком t. Для произвольной вершины $u \in V$ обозначим 
        $$A(u) = \{v\in V : uv\in E\}\quad B(u) = \{v\in V : vu\in E\}$$
        Потоком из s в t в сети G называется функция $f:E\to R$, удовлетворяющая условиям:$$0\leq f(e) \leq c(e) \forall e\in E$$
        (условие допустимости)
        \begin{equation}
            \sum\limits_{v\in A(u)}f(uv) - \sum\limits_{v\in B(u)}f(vu)=
            \begin{cases}
                b & u = s\\
                0 & u\notin\{s, t\}\\
                -b & u = t\\
            \end{cases}
        \end{equation}
        (условие баланса)
    \end{definition}
    \begin{definition}
        $b = b(f)\geq 0$ - величина потока. Поток называется максимальным, если $b(f^*) = \max\limits_{f-\text{поток}}b(f)$
    \end{definition}
    \begin{definition}
        f(e) - дуговые потоки. Дуга $e\in E$ называется насыщенной потоком f, если $f(e) = c(e),$ и пустой, если $f(e) = 0$
    \end{definition}
    \begin{lemma}[об увеличении потока]
        Если для потока f в сети G существует увеличивающий путь P, то поток может быть увеличен.
    \end{lemma}
    \section{Алгоритм Эдмондса-Карпа построения максимального потока}
    \section{Разрезы. Лемма о потоках и разрезах. Следствие}
    \begin{definition}[Разрез]
        Пусть $W\subseteq V, \overline{W} = V \backslash  W$. Разрезом $(W, \overline{W})$ в сети G называется множество всех дуг вида $e = uv, u\in W, v\in \overline{W}$.
    \end{definition}
    \begin{definition}[Разрез разделяет вершины]
        Говорят, что разрез $(W, \overline{W})$ разделяет вершины s и t, если $s\in W, t\in \overline{W}$.
    \end{definition}
    \begin{definition}
        Пропускной способностью разреза $(W, \overline{W})$ называется число $c(W, \overline{W}) = \sum\limits_{e\in (W, \overline{W})} c(e)$
    \end{definition}
    \begin{definition}
        Минимальным разрезом, разделяющим s и t, называется разрез с минимальной пропускной способностью среди всех таких разрезов.
    \end{definition}
    \begin{definition}
        Если $f: E\to R_+$ - поток из s в t, то потоком через разрез $(W, \overline{W})$ называется число $f(W, \overline{W}) = \sum\limits_{e\in (W, \overline{W})} f(e)$
    \end{definition}
    \begin{lemma}[о потоках и разрезах]
        Для любого потока f из s в t и произвольного разреза $(W, \overline{W})$, разделяющего s и t, имеет место равенство
        $$b(f) = f(W, \overline{W}) - f(\overline{W}, W)$$
    \end{lemma}
    \begin{corollary}
        В любой сети величина любого потока из s в t не превосходит пропускной способности любого разреза, разделяющего s и t.
    \end{corollary}
    \section{Теорема Форда-Фалкерсона}
    \begin{theorem}
        В любой конечной сети $G = (V,E)$ величина максимального потока из s в t равна пропускной способности минимального разреза, разделяющего s и t.
    \end{theorem}
    \section{Два критерия максимальности потока.}
    \begin{theorem}[I]
        Поток $f^*$ максимален, если и только если не существует пути, увеличивающего $f^*$.
    \end{theorem}
    \begin{theorem}[II]
        Поток $f^*$ максимален тогда и только тогда, когда он насыщает все дуги некоторого разреза $(W, \overline{W})$, оставляя пустыми все дуги обратного разреза $(\overline{W}, W)$.
    \end{theorem}
    \section{Приложения теории потоков в сетях. Задачи анализа структурно-надежных 
    коммуникационных сетей}
    При проектировании сложных систем сетевой структуры необходимо учитывать требования, предъявленные к надежности этих систем. Важно отметить, что надежность сети не сводится только к функциональной надежности, т.е надежности выполнения отдельных функций, возлагаемых на систему, но и к структурной надежности, т.е свойству непрерывно сохранять работоспособность в некоторых условиях.
    \begin{definition}
        Коммуникационная сеть называется структурно надежной, если она сохраняет работоспособность при выходе из строя определенного количества узлов связи и/или линий связи.
    \end{definition}
    При моделировании КС графом: узлы соответствуют вершинам, линии - ребрам. Выход из строя узла равнозначен удалению вершины и аналогично с удалением ребра. При этом исправной сети соответствует связный граф, а понятие "структурная надежность сети" переносится в понятие вершинной и/или реберной связности графа.
    \begin{definition}
        Локальная вершинная связность пары несмежных вершин s,t связного графа - наименьшее число $\varkappa(s, t)$ вершин, в результате удаления которых получается несвязный граф, причем вершины s, t лежат в разных компонентах связности. Множество удаленных вершин называется \textbf{(s,t)-разделяющим множеством вершин}.
    \end{definition} 
    \begin{definition}
        Локальная реберная связность произвольных вершин s, t связного графа называется минимальное число $\lambda(s, t)$ ребер, в результате удаления которого получается несвязный граф, причем s, t лежат в разных компонентах связности. Множество удаленных ребер называют \textbf{(s,t)-разделяющим множеством ребер}.
    \end{definition}
    \begin{statement}
        Для любого связного графа, не являющегося полным
        $$\varkappa(G)= \min\limits_{s, t\text{ несмежные}} \varkappa(s, t); \quad \lambda(G) = \min\limits_{s, t\in V}\lambda(s, t)$$
        (Для $K_n$ $\varkappa(G) = \lambda(G) = n-1$)
    \end{statement}
    \begin{proof}
    \end{proof}
    Пусть G нетривиальный связный граф, не полный. $\varkappa(s, t)$ для всех несмежных s, t = наибольшее число вершинно независимых простых (s, t)-цепей.
    \begin{theorem}[Реберный аналог теоремы Менгера в новых определениях]
        $\lambda(s, t) \forall s, t\in V$(нетривиальный, связный) = наибольшее число реберно независимых простых (s, t)-цепей
    \end{theorem}
    Локальная связность пары вершин вычисляется с помощью потоковых алгоритмов
    Пусть G = (V, E), != $K_n$, неориентированный связный граф, $n\geq 3, (s, t)\in V$
    \begin{theorem}[Алгоритм 1, $\lambda(s, t)$]
        \begin{enumerate}
            \item Заменим ребра на пару симметричных дуг, получим орграф G' 
            \item Положим пропуск. способность каждой дуги равна 1 
            \item Вычисляем максимальный поток s,t
        \end{enumerate}
    \end{theorem}
    \begin{theorem}
        Величина $max(s,t)$ потока в сети G' = $\lambda(s,t)$ в G
    \end{theorem}
    \begin{proof}
        По теореме Форда-Фалкерсона максимальный поток G' = пропускная способность min разреза. Так как пропускная способность всех дуг = 1, то п.с минимального разреза = числу дуг в разрезе = минимальное число ребер в (s,t)-разделяющем множестве в графе G = $\lambda (s, t)$
    \end{proof}
    G = (V, E) - n-вершинный связный, неориентированный граф, $n\geq 3$, $(s, t)$ несмежные
    \begin{theorem}[Алгоритм 2, $\varkappa(s, t)$]
        \begin{enumerate}
            \item Заменим ребра на пару симметричных дуг, получим орграф G'
            \item Все вершины (не s, t) заменить на дугу v'v''. Все старые дуги входят в v', выходят из v''. Получим орграф G''.
            \item Положим пропуск. способность каждой дуги равна 1 
            \item Вычисляем максимальный поток s,t в G''
        \end{enumerate}
    \end{theorem}
    \begin{theorem}
        Величина $max(s,t)$ потока в сети G'' = $\varkappa(s,t)$ в G
    \end{theorem}
    \begin{proof}
        По теореме Форда-Фалкерсона максимальный поток в G'' = пропускная способность min разреза. Так как пропускная способность всех дуг = 1, то п.с минимального разреза = числу дуг в разрезе.

        Кроме того, заметим что в G'' существует минимальный разрез, состоящий из дуг v' v'', а число дуг в таком разрезе равно минимальному числу вершин в (s, t)-разделяющем множестве в графе G =$\varkappa(s, t)$
    \end{proof}
    \begin{enumerate}
        \item Трудоемкость вычисления $\lambda(s, t)$ определяется трудоемкостью потокового алгоритма в п.3 алгоритма 1.

        $b(f^*) = \lambda(s, t) \leq n - 1\implies$ применяем Эдмондса-Карпа$\implies$
        $T_{A1}(n) = O(n^2b^*) = O(n^3)$
        \item 
        Трудоемкость вычисления $\lambda(G) = \min \lambda(s, t)$, всего $C^2_n$ пар $\implies$ трудоемкость $O(n^5)$
        \item     
        $T_{A2} = O(n^3)$
        Вычислительная сложность $\varkappa(G) = O(n^5)$
    \end{enumerate}
    
    \section{Задачи комбинаторной оптимизации. Массовая и индивидуальная задачи. 
    Трудоемкость алгоритма. Полиномиальные и экспоненциальные алгоритмы}
    Теория сложности вычислений строится для задач распознавания свойств. Такие задачи 
    имеют только два возможных решения ~ «да» и «нет». 
    \begin{definition}
        Массовая задача распознавания P состоит из двух множеств: множества $P_I$
        всевозможных индивидуальных задач и множества $P_Y$ всех индивидуальных задач с 
        ответом «да», $P_Y\subseteq P_I$. 
    \end{definition}    
    Типичным примером массовой задачи распознавания служит известная задача о 
    гамильтоновом цикле, в которой требуется определить, содержит ли заданный 
    обыкновенный неориентированный граф гамильтонов цикл. 
    \begin{definition}
        Будем говорить, что алгоритм решает массовую задачу распознавания P, если он останавливается (т. е. заканчивает работу) на всех индивидуальных задачах I из P (т. е. для всех $ I \in P_I$) и даёт ответ «да» для всех $I \in P_Y$ и только для них. 
    \end{definition}
    \section{Задачи распознавания свойств. Детерминированные и недетерминированные 
    алгоритмы. Классы P и NP. Проблема “P vs NP”}
    \begin{definition}[Класс P]
        Класс P определяется как класс массовых задач распознавания, разрешимых 
    полиномиальными алгоритмами. Другими словами, задача распознавания P принадлежит классу P, если и только если существует полиномиальный алгоритм который решает задачу P.
    \end{definition}
    \begin{definition}[Недетерминированный алгоритм]
        Недетерминированный алгоритм состоит из двух стадий — стадии угадывания и 
        стадии проверки. На стадии угадывания по заданной индивидуальной задаче I
        происходит просто «угадывание» некоторой структуры S — подсказки. 

        Затем I и S вместе подаются на вход стадии проверки, которая представляет собой обычный детерминированный алгоритм и либо заканчивается ответом «да», либо ответом «нет», либо продолжается бесконечно (последние две возможности обычно не различают).
    \end{definition}
    \begin{definition}
        Говорят, что недетерминированный алгоритм решает задачу распознавания Р, если для любой индивидуальной задачи $I \in P_I$ выполнено условие:

        $I\in P_Y$ тогда и только тогда, когда существует такая подсказка S, угадывание которой для входа I приводит к тому, что стадия проверки, начиная работу на входе (I, S), заканчивается ответом «да».
    \end{definition}
    \begin{definition}
        Класс $\mathcal{NP}$ - это класс всех массовых задач распознавания, разрешимых недетерминированными алгоритмами за полиномиальное время.
    \end{definition}
    \begin{theorem}
        $P \subseteq \mathcal{NP}$
    \end{theorem}
    \section{Полиномиальная сводимость задач распознавания. Свойства полиномиальной 
    сводимости}
    \begin{definition}[Полиномиальная сводимость]
        Пусть P, Q - две задачи распознавания и А - такой полиномиальный алгоритм, который для любой индивидуальной задачи $I \in P_I$ строит некоторую задачу $A(I) \in Q_I$. Если при этом $$I\in P_Y \Leftrightarrow A(I) \in Q_Y,$$ то говорят, что задача P полиномиально сводится к задаче Q ($P\propto Q$)
    \end{definition}
    \section{NP-полные задачи распознавания. Теорема о сложности NP-полных задач. Примеры 
    NP-полных задач}
    \begin{definition}[$\mathcal{NP}$-полная задача]
        Задача распознавания Q называется $\mathcal{NP}$-полной, если $Q\in\mathcal{NP} $и для любой задачи $P\in\mathcal{NP}\quad P\propto Q$.
    \end{definition}
    \begin{definition}
        Класс всех $\mathcal{NP}$-полных задач обозначается $\mathcal{NPC}$(NP complete)
    \end{definition}
    \begin{theorem}[О сложности $\mathcal{NP}$-полных задач]
        \begin{enumerate}
            \item Если хотя бы одна $\mathcal{NP}$-полная задача полиномиально разрешима, то $\mathcal{P} = \mathcal{NP}$.
            \item Если хотя бы одна задача класса $\mathcal{NP}$ трудно решаема (т. е $\mathcal{P}\neq\mathcal{NP}$) то все $\mathcal{NP}$-полные задачи труднорешаемы.
        \end{enumerate}
    \end{theorem}
\end{multicols*}
\end{document}