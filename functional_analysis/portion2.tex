\documentclass[a4paper, 12pt]{article}
\usepackage[utf8]{inputenc}
\usepackage[T2A]{fontenc}
\usepackage{amsfonts}
\usepackage{amsmath, amsthm}
\usepackage{amssymb}
\usepackage{hyperref}

\newcommand\letsymbol{\mathord{\sqsupset}}
\usepackage[russian]{babel}
\renewcommand\qedsymbol{$\blacktriangleright$}
\newtheorem{theorem}{Теорема}[section]
\newtheorem{lemma}{Лемма}[section]
\theoremstyle{definition}
\newtheorem*{example}{Пример}
\newtheorem*{definition}{Определение}
\theoremstyle{remark}
\newtheorem*{remark}{Замечание}


\DeclareMathOperator{\Int}{int}
\DeclareMathOperator{\clo}{cl}


\setlength{\topmargin}{-0.5in}
\setlength{\oddsidemargin}{-0.5in}
\textwidth 185mm
\textheight 250mm

\begin{document}
\begin{enumerate}
    \item[43] Доказать, что если шар радиуса 7 содержится в шаре радиуса 3, то они совпадают.
    \begin{proof}
        $B_7(x_0) \subset B_3(y_0) \implies \forall x: d(x, x_0) < 7 \quad d(x, y_0) < 3$

        Сразу получаем, что $d(x_0, y_0) < 3$.

        Пусть $B_3(y_0) \not\subset B_7(x_0)$, тогда существует такое $y\in B_3(y_0)$,
        что $d(y, x_0) \geq 7$

        Применим свойство полуметрики, учитывая, что $d(y_0, y) < 3$:

        \[d(x_0, y_0) \geq |d(x_0, y) - d(y_0, y)| > 4\]
        
        Получили противоречие, значит, имеет место включение и шары совпадают.
    \end{proof}

    \item[..] Задача с лекции 03.10.23.
    Найти $\mu$, при которых уравнение разрешимо и найти решение
    \[x(t) - \mu \int_{0}^{1} t s x(s) ds = t\]
    
    Пусть C = $\int_{0}^{1}s x(s) ds$

    \[x(t) = (\mu C + 1)t\]
    \[t x(t) = (\mu C +1)t^2\]
    \[C = \int_0^1 (\mu C +1)y^2dy\]
    \[C = \frac{(\mu C +1) }{3}\]
    % \[C(1-\frac{\mu}{3}) = \frac{1}{3}\]
    \[C = \frac{1}{3-\mu}\]
    \[x(t) = \frac{3t}{3-\mu}, \quad \mu\neq 3\]

    \item[37] Доказать, что если $(X, d)$ - полуметрическое пространство,
    то
    \begin{enumerate}
        \item $\forall x\in X, R > r > 0\quad B_r[x] \subset B_R(x)$;
        \item $\forall x\in X, r > 0\quad \clo B_r(x) \subset B_r[x]$, а вот равенства может не быть;
        \item $\forall x\in X, r > 0\quad B_r(x)\in Op(X, d), B_r[x]\in Cl(X, d)$;
        \item шар большего радиуса может быть собственным подмножеством шара меньшего радиуса
    \end{enumerate}
    \begin{proof}
        \begin{enumerate}
            \item $B_r[x] = \{y\in X\;|\;d(x, y) \leq r\} \quad$
            $B_R(x) = \{y\in X\;|\;d(x, y)< R\}$

            Для всех у из шара $B_r[x]$, т.е $y\in X$, таких что $d(x, y)\le r < R \implies y\in B_R(x)$ 
            \item 
        \end{enumerate}
    \end{proof}
    % \item[47] 
    % Пусть $d$ - полуметрика на множестве X.
    % Доказать, что функции
    % \[d_1(x, y) = \frac{d(x,y)}{1+d(x, y)}, \quad d_2(x, y) = \min\{1, d(x, y)\}, \quad d_3(x, y) = \ln (1 + d(x, y))\]
    % являются полуметриками на X, причем они все эквивалентны и являются 
    % метриками или нет, одновременно с исходной.
    % \begin{enumerate}
    %     \item {
    %         \begin{enumerate}
    %             \item \[d_1(x, x) = \frac{d(x, x)}{1+ d(x, x)} = 0\]
    %             \item \[d_1(x, y) = \frac{d(x, y)}{1+ d(x, y)} = \frac{d(y, x)}{1+ d(y, x)} = d_1(y, x)\]
    %             \item \[d_1(x, y) + d_1(y, z) = \frac{d(x, y)}{1+ d(x, y)} + \frac{d(y, z)}{1+ d(y, z)}\]
    %         \end{enumerate}
    %     }
    % \end{enumerate}
\end{enumerate}

\end{document}