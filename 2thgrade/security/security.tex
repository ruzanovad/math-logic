\documentclass[a4paper]{article}
\usepackage{cmap}
\usepackage[utf8]{inputenc}
\usepackage[T2A]{fontenc}
\usepackage{amsfonts}
\usepackage{amsmath, amsthm}
\usepackage{amssymb}
\usepackage{hyperref}
\usepackage{multicol}
\usepackage{xcolor}

\newcommand\letsymbol{\mathord{\sqsupset}}
\usepackage[russian]{babel}
\renewcommand\qedsymbol{$\blacktriangleright$}
% \newtheorem{theorem}{Теорема}[section]
% \newtheorem{lemma}{Лемма}[section]
\theoremstyle{definition}
\newtheorem*{example}{Пример}
\newtheorem*{definition}{Определение}
\newtheorem*{statement}{Утверждение}
\theoremstyle{remark}
\newtheorem*{remark}{Замечание}

\setlength{\topmargin}{-0.5in}
\setlength{\oddsidemargin}{-0.5in}
\textwidth 185mm
\textheight 250mm

\begin{document}
\begin{definition}
    Block cipher processes the
    input one block of elements at a time, producing an output block for each
    input block
\end{definition}
\begin{definition}
    Plaintext - an original message.
\end{definition}
\begin{definition}
    Ciphertext - the coded message.
\end{definition}
\begin{definition}
    Attack - an assault on system security that derives from an intelligent threat; that is, an intelligent act that is a
deliberate attempt (especially in the sense of a method or technique) to evade security services and
violate the security policy of a system.
\end{definition}
\begin{definition}
    Brute-force attack - The attacker tries every possible key
     on a piece of ciphertext until an intelligible translation into plaintext is obtained. On average, half
    of all possible keys must be tried to achieve success
\end{definition}
\begin{definition}
    Caesar cipher - substitution cipher, that involves
     replacing each letter of the alphabet with the letter standing three places further down the alphabet.
\end{definition}

\begin{definition}
    Computationally secure - cipher, if these two criteria are met:
    \begin{itemize}
        \item The cost of breaking the cipher exceeds the value of the encrypted information.
        \item The time required to break the cipher exceeds the useful lifetime of the
        information.
    \end{itemize}
\end{definition}

\begin{definition}
    Conventional encryption -  single-key
    encryption, symmetric encryption
\end{definition}

\begin{definition}
    cryptographic system (cipher) - scheme of encryption
    \begin{itemize}
        \item Plaintext: This is the original intelligible message or data that is fed into the
        algorithm as input.
        \item Encryption algorithm: The encryption algorithm performs various substitutions and transformations on the plaintext.
        \item Secret key: The secret key is also input to the encryption algorithm. The key is
        a value independent of the plaintext and of the algorithm. The algorithm will
        produce a different output depending on the specific key being used at the
        time. The exact substitutions and transformations performed by the algorithm
        depend on the key. (In symmetric cipher there is only one key that is used for both encryption and decryption, but in different
        systems like asymmetric ciphers there are two different keys for encryption and decryption)
        \item Ciphertext: This is the scrambled message produced as output. It depends on
        the plaintext and the secret key. For a given message, two different keys will
        produce two different ciphertexts. The ciphertext is an apparently random
        stream of data and, as it stands, is unintelligible.
        \item Decryption algorithm: This is essentially the encryption algorithm run in
        reverse. It takes the ciphertext and the secret key and produces the original
        plaintext
    \end{itemize}
\end{definition}
\begin{definition}
    Cryptography - study, which area are constituted by the many schemes used for
encryption
\end{definition}
\begin{definition}
    Cryptanalysis - study of techniques used for deciphering a
    message without any knowledge of the enciphering details\\
    (AS AN ATTACK):\\
    Cryptanalytic attacks rely on the nature of the algorithm plus
perhaps some knowledge of the general characteristics of the plaintext or
even some sample plaintext-ciphertext pairs. This type of attack exploits the
characteristics of the algorithm to attempt to deduce a specific plaintext or to
deduce the key being used.
\end{definition}
\begin{definition}
    Cryptology - study, which area is the areas
    of cryptography and cryptanalysis
\end{definition}

\begin{definition}
    Deciphering (decryption) - restoring the plaintext from the ciphertext
\end{definition}

\begin{definition}
    Digram - two-letter combinations
\end{definition}

\begin{definition}
    Enciphering (encryption) - the process of converting
from plaintext to ciphertext
\end{definition}

\begin{definition}[Hill cipher]

    This encryption algorithm takes successive plaintext letters
and substitutes for them ciphertext letters. The substitution is determined by 
linear equations in which each character is assigned a numerical value $(a = 0, b = 1, \dots, z = 25)$

$$C = E(K, P) = PK \mod 26$$
$$P = D(K, C) = CK^{-1} \mod 26 = P K K^{-1} = P$$

C and P are row vectors of length 3 representing the plaintext and ciphertext,
and K is a matrix representing the encryption key.
\end{definition}

\begin{definition}
    monoalphabetic (substitution) cipher - type of cipher, where ciphertext and plaintext have
    the same alphabet.
\end{definition}
\begin{definition}[one-time pad]
    An Army Signal Corp officer, Joseph Mauborgne, proposed an improvement to the
Vernam cipher that yields the ultimate in security. Mauborgne suggested using a
random key that is as long as the message, so that the key need not be repeated. In
addition, the key is to be used to encrypt and decrypt a single message, and then is
discarded. Each new message requires a new key of the same length as the new message. 
Such a scheme, known as a one-time pad, is unbreakable. It produces random
output that bears no statistical relationship to the plaintext. Because the ciphertext
contains no information whatsoever about the plaintext, there is simply no way to
break the code.

Vernam cipher:
$$c_i = p_i \oplus k_i$$
$$p_i = c_i \oplus k_i$$
\end{definition}

\begin{definition}
    Playfair cipher.

    The Playfair algorithm is based on the use of a 5 × 5 matrix of letters constructed using a keyword.
    The matrix is constructed by filling in
the letters of the keyword (minus duplicates) from left to right and from top to bottom, and then filling in the remainder of the matrix with the remaining letters in
alphabetic order. The letters I and J count as one letter. Plaintext is encrypted two
letters at a time, according to the following rules:
\begin{enumerate}
    \item Repeating plaintext letters that are in the same pair are separated with a filler
    letter, such as x, so that balloon would be treated as ba lx lo on.
    \item Two plaintext letters that fall in the same row of the matrix are each replaced by
    the letter to the right, with the first element of the row circularly following the
    last. For example, ar is encrypted as RM.
    \item Two plaintext letters that fall in the same column are each replaced by the letter
    beneath, with the top element of the column circularly following the last. For
    example, mu is encrypted as CM.
    \item Otherwise, each plaintext letter in a pair is replaced by the letter that lies in its
    own row and the column occupied by the other plaintext letter. Thus, hs
    becomes BP and ea becomes IM (or JM, as the encipherer wishes).
\end{enumerate}
\end{definition}

\begin{definition}[polyalphabetic cipher]
    Another way to improve on the simple monoalphabetic technique is to use different
monoalphabetic substitutions as one proceeds through the plaintext message. The
general name for this approach is \textbf{polyalphabetic substitution cipher}. All these techniques have the following features in common:
\begin{enumerate}
    \item A set of related monoalphabetic substitution rules is used.
    \item A key determines which particular rule is chosen for a given transformation
\end{enumerate} 
\end{definition}
\begin{definition}[rail fence cipher]
    The simplest such cipher is the rail fence technique, in which the plaintext is
written down as a sequence of diagonals and then read off as a sequence of rows. For
example, to encipher the message “meet me after the toga party” with a rail fence of
depth 2, we write the following:
m e m a t r h t g p r y\\
 e t e f e t e o a a t
The encrypted message is \\ 
MEMATRHTGPRYETEFETEOAAT
\end{definition}

\begin{definition}[steganography]
    is the practice of representing information within another message or physical object,
     in such a manner that the presence of the information is not evident to human inspection.

     A plaintext message may be hidden in one of two ways. The methods of
steganography conceal the existence of the message, whereas the methods of cryptography render the message unintelligible to outsiders by various transformations
of the text
\begin{enumerate}
    \item Character marking: Selected letters of printed or typewritten text are overwritten in pencil. The marks are ordinarily not visible unless the paper is held
    at an angle to bright light.
    \item Invisible ink: A number of substances can be used for writing but leave no
    visible trace until heat or some chemical is applied to the paper.
\end{enumerate}
\end{definition}

\begin{definition}
    Stream cipher processes the input elements continuously,
producing output one element at a time, as it goes along
\end{definition}
\begin{definition}
    Symmetric encryption - is a form of cryptosystem in which encryption and
    decryption are performed using the same key.
\end{definition}

\begin{definition}[transposition cipher]
    All the techniques examined so far involve the substitution of a ciphertext symbol
for a plaintext symbol. A very different kind of mapping is achieved by performing
some sort of permutation on the plaintext letters. This technique is referred to as a
transposition cipher.
\end{definition}

\begin{definition}
    Unconditionally secure - an encryption scheme is
unconditionally secure if the ciphertext generated by the scheme does not contain enough information
 to determine uniquely the corresponding plaintext, no
matter how much ciphertext is available.
\end{definition}

\begin{definition}[Vigenère cipher]
    We can express the Vigenère cipher in the following manner. Assume a
sequence of plaintext letters and a key consisting of the
sequence of letters , where typically < .The sequence of
ciphertext letters is calculated as follows:

$$C_i = (p_i + k_{i \mod m}) \mod 26$$

$$p_i = (C_i - k_{i\mod m}) \mod 26$$
\end{definition}
\end{document}