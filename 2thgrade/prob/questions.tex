\documentclass[a4paper]{article}
\usepackage{cmap}
\usepackage[utf8]{inputenc}
\usepackage[T2A]{fontenc}
\usepackage{amsfonts}
\usepackage{amsmath, amsthm}
\usepackage{amssymb}
\usepackage{hyperref}
\usepackage{multicol}
\usepackage{xcolor}

\newcommand\letsymbol{\mathord{\sqsupset}}
\usepackage[russian]{babel}
\renewcommand\qedsymbol{$\blacktriangleright$}
\newtheorem{theorem}{Теорема}[section]
\newtheorem{lemma}{Лемма}[section]
\theoremstyle{definition}
\newtheorem*{example}{Пример}
\newtheorem*{definition}{Определение}
\newtheorem*{statement}{Утверждение}
\theoremstyle{remark}
\newtheorem*{remark}{Замечание}

\setlength{\topmargin}{-0.5in}
\setlength{\oddsidemargin}{-0.5in}
\textwidth 185mm
\textheight 250mm

\begin{document}
\begin{enumerate}
    \item Аксиомы теории вероятностей и простейшие следствия из них.
    \item Теорема о непрерывности вероятностной меры.
    \item Условные вероятности. Формула полной вероятности.
    \item Условные вероятности. Формулы Байеса.
    \item Независимость случайных событий. 
    \item Классическая схема. Примеры. 
    \item Геометрическая вероятность. Примеры.
    \item Схема Бернулли. 
    \item Теоремы Муавра-Лапласа и Пуассона.
    \item Случайные величины, распределение случайной величины.
    \item Плотность распределения и её свойства. Формула свёртки.
    \item Функция распределения случайной величины и её свойства. Типы распределений.
    \item Взаимно-однозначное соответствие между вероятностными мерами и функциями распределения. 
    \item Случайные векторы и их распределения. Определение и основные свойства.
    \item Функции распределения, плотности распределения случайных векторов.
    \item Независимость классов событий и случайных величин. 
    \item Функции от случайных величин.
    \item Измеримость поточечного предела измеримых функций.
    \item Сходимость по вероятности. Измеримость предела.
    \item Сходимость почти наверное. Измеримость предела.
    \item Соотношение сходимостей по вероятности и почти наверное.
    \item Слабая сходимость и сходимость по распределению.
    \item Интеграл Лебега. Определение и примеры.
    \item Теорема Лебега.
    \item Теорема Фату.
    \item Теорема Леви.
    \item Математическое ожидание и дисперсия. Основные свойства и примеры вычисления.
    \item Ковариация, коэффициент корреляции и их геометрическая интерпретация
    \item Неравенства Чебышёва.
    \item Условное математические ожидание относительно разбиения, геометрическая интерпретация.
    \item Условные математические ожидания: общее определение, теорема Радона-Никодима.
    \item Производящие функции: определение, свойства и примеры вычисления.
    \item Характеристические функции: определение, свойства и примеры вычисления.
\end{enumerate}
\end{document}
