\documentclass[a4paper]{article}
\usepackage{cmap}
\usepackage[utf8]{inputenc}
\usepackage[T2A]{fontenc}
\usepackage{amsfonts}
\usepackage{amsmath, amsthm}
\usepackage{amssymb}
\usepackage{hyperref}
\usepackage{multicol}
\usepackage{xcolor}

\newcommand\letsymbol{\mathord{\sqsupset}}
\usepackage[russian]{babel}
\renewcommand\qedsymbol{$\blacktriangleright$}
\newtheorem{theorem}{Теорема}[section]
\newtheorem{lemma}{Лемма}[section]
\theoremstyle{definition}
\newtheorem*{example}{Пример}
\newtheorem*{definition}{Определение}
\newtheorem*{statement}{Утверждение}
\theoremstyle{remark}
\newtheorem*{remark}{Замечание}

\setlength{\topmargin}{-0.5in}
\setlength{\oddsidemargin}{-0.5in}
\textwidth 185mm
\textheight 250mm

\begin{document}
% \begin{multicols*}{2}
    \tableofcontents
    \pagenumbering{arabic}
    \setcounter{page}{1}
    \section{Теория булевых функций}
    \subsection{Определение булевой функции (БФ). Количество БФ от n переменных. Таблица истинности БФ}
    \begin{definition}
        Булева функция от n переменных - это отображение $\{0,1\}^n \rightarrow \{0, 1\}$
    \end{definition}

    \begin{remark}
        Количество БФ от n переменных - $2^{2^n}$
    \end{remark}
    \begin{proof}
        Каждая булева функция определяется своим столбцом значений.
         Столбец является булевым вектором длины $m=2n$, где n - число аргументов функции.
          Число различных векторов длины m (а значит и число булевых функций, зависящих от n переменных) равно $2^m=2^{2^n}$
    \end{proof}
    \subsection{Булевы функции одной и двух переменных (их таблицы, названия)}
        Булевы функции одной переменной:
         \begin{tabular}{c|cccc}
        x & $f_1$ & $f_2$ & $f_3$ & $f_4$ \\
        \hline
        0 & 0 & 0 & 1 & 1 \\
        1 & 0 & 1 & 0 & 1 \\ 
        \end{tabular}
    $f_1$ - тождественный 0, $f_2$ - тождественная функция, $f_3$ - отрицание ($\neg$), $f_4$ - тождественная 1
\\ \\
    Булевы функции двух переменных
    \begin{tabular}{cc|cccccccccccccccc}
        x & y & 0 & $\wedge$ & $\rightarrow '$ & $x$ & $\leftarrow '$ & $y$ & $+$ & $\vee$ & $\downarrow$ & $\leftrightarrow$ & $y'$ & $\leftarrow$ & $x'$ & $\rightarrow$ & | & 1\\
        \hline
        0 & 0 & 0 & 0 & 0 & 0 & 0 & 0 & 0 & 0 & 1 & 1 & 1 & 1 & 1 & 1 & 1 & 1 \\
        0 & 1 & 0 & 0 & 0 & 0 & 1 & 1 & 1 & 1 & 0 & 0 & 0 & 0 & 1 & 1 & 1 & 1 \\ 
        1 & 0 & 0 & 0 & 1 & 1 & 0 & 0 & 1 & 1 & 0 & 0 & 1 & 1 & 0 & 0 & 1 & 1 \\
        1 & 1 & 0 & 1 & 0 & 1 & 0 & 1 & 0 & 1 & 0 & 1 & 0 & 1 & 0 & 1 & 0 & 1 \\
        \end{tabular}
    \begin{enumerate}
        \item $\wedge$ - конъюнкция
        \item $\leftarrow$ -  антиимпликация
        \item $\rightarrow$ - импликация
        \item $\vee$ - дизъюнкция
        \item | - штрих Шеффера
        \item $\downarrow$ - стрелка Пирса
        \item + -  взаимоисключающее или, сложение по модулю 2 (XOR)
    \end{enumerate}
    \subsection{Формулы логики высказываний. Представление БФ формулами}
    \begin{definition}
        Формула логики высказываний - слово алфавита логики высказываний,
        построенное по следующим правилам:
        \begin{enumerate}
            \item символ переменной - формула
            \item символы 0 и 1 - формулы
            \item если $\Phi_1$ и $\Phi_2$ - формулы, то слова ($\Phi_1 \& \Phi_2$),
            ($\Phi_1 \leftrightarrow \Phi_2$), ($\Phi_1 \rightarrow \Phi_2$), 
            ($\Phi_1 | \Phi_2$), $\dots$ , $\Phi_1 '$ тоже формулы
        \end{enumerate}
    \end{definition}
    Очевидно, что каждой формуле логики высказываний можно поставить в соответствие
    булеву функцию, причем если формуле $F_1$ соответствует функция $f_2$, а 
    формуле $F_2$ функция $f_2$ и $F_1 \equiv F_2$, то $f_1 \equiv f_2$.
    
    Каждая формула $\Phi(x_1, \dots, x_n)$ логики высказываний однозначно
    определяет некоторую булеву функцию $f(x_1, \dots, x_n)$

    Это булева функция, определенная таблицей истинности
    формулы $\Phi$.
    \subsection{Эквивалентные формулы. Основные эквивалентности теории булевых функций}
	\begin{definition}
		Формулы логики высказываний 	$\Phi$($x_1$, $x_2$, \dots , $x_n$) и $\Psi$($x_1$, $x_2$, \dots , $x_n$) эквивалетные, если для всех наборов значений $a_1$, \dots , $a_n\in\{$0, 1$\}$
		$\Phi$($a_1$, \dots , $a_n$) = 1 $\Leftrightarrow \Psi$($a_1$, \dots , $a_n$) = 1
	\end{definition}
    \begin{theorem}[Об эквивалентных формулах]


        \begin{enumerate}
           	\item Если $\Phi(x_1, \dots , x_n) \equiv \Psi(x_1, \dots , x_n)$ и $\theta_i(x_1, \dots , x_k)$, i = 1, \dots , n, - формулы логики высказываний, то $\Phi(\theta_1, \dots , \theta_n) \equiv \Psi(\theta_1, \dots , \theta_n)$
		\item Если в формуле $\Phi$ заменить подформулу $\Psi$ на эквивалетную формулу $\Theta$, то результат замены эквивалентен $\Phi.$
        \end{enumerate}
	\begin{proof}
		\begin{enumerate}
			\item После подстановки в $\Phi(x_1, \dots , x_n)$ формул $\theta_i(x_1, \dots , x_k)$ получим формулу от k переменных: $$\Phi(\theta_1, \dots , \theta_n)(x_1, \dots , x_k) = \Phi(\theta_i(x_1, \dots , x_k), \dots , \theta_n(x_1, \dots , x_k))$$ и аналогично для $\Psi.$ Выберем произвольный набор элементов $a_1, \dots , a_k \in \{0, 1\}$ и подставим: $$\Phi(\theta_1(a_1, \dots , a_k), \dots , \theta_n(a_1, \dots , a_k)) = \Phi(b_1, \dots , b_n), где b_i = \theta_i(a_1, \dots , a_k),$$ $$\Psi(\theta_1(a_1, \dots , a_k), \dots , \theta_n(a_1, \dots , a_k), \dots , \theta_n(a_1, \dots , a_k)) = \Psi(b_1, \dots , b_n).$$ Т.к. $\Phi \equiv \Psi, то \Phi(b_1, \dots , b_n) = 1 \leftrightarrow \Psi(b_1, \dots , b_n) = 1$, значит и $\Phi(\theta_1, \dots , \theta_n)(a_1, \dots , a_k) = 1 \leftrightarrow \Psi(\theta_1, \dots , \theta_n)(a_1, \dots , a_k)$, т.е. $\Phi(\theta_1, \dots , \theta_n) \equiv \Psi(\theta_1, \dots , \theta_n).$
			\item По условию $\Psi \equiv \Theta$. Обозначим результат замены в формуле $\Phi$ подформулы $\Psi$ на $\Theta$ через $\Phi[\Psi/\Theta].$

				Индукцию по числу логических связанок в формуле $\Phi$. Пусть k - число связок в подфомруле $\Psi$.

				Заметим, что, если формула $\Phi$ содержит менее k связок, то в ней нет подформулы $\Psi$. А если формула $\Phi$ имеет ровно k связок, то единственный случай, когда она содержит подформулу $\Psi$ - это $\Phi = \Psi$

				База индукции. 
				
				\begin{enumerate}
					\item Формула $\Phi$ содержит не более k связок и при этом $\Phi \neq \Psi$. Тогда $\Phi$ не содержит подформулы $\Psi$, поэтому при данной операции не меняется: $\Phi[\Psi/\Theta] = \Phi$, отсюда $\Phi[\Psi/\Theta] \equiv \Phi$
					\item Формула $\Phi$ содержит k связок и $\Phi = \Psi$. Тогда $\Phi[\Psi/\Theta] = \Theta$ результат замены эквивалентен исходной формуле $\Phi = \Psi$
				\end{enumerate}

				Шаг индукции.

				Рассмотрим формулу $\Phi(x_1, \dots , x_n)$ содержающую m + 1 связки, считая, что для формул из не более, чем m связок, утверждение доказано. Тогда $\Phi$  имеет вид $\Phi_1 \wedge \Phi_2, \Phi_1 \vee \Phi_2$ и т.д.

				Рассмотрим случай конъюнкции(остальные аналогично). Выберем набор элементов $a_1, \dots , a_n \in \{0, 1\}$ и подставим в формулы: $$\Phi(a_1, \dots , a_n) = \Phi_1(a_1, \dots , a_n) \wedge \Phi_2(a_1, \dots , a_n),$$
				$$\Phi[\Psi/\Theta](a_1, \dots , a_n) = \Phi_1[\Psi/\Theta](a_1, \dots , a_n) \wedge \Phi_2[\Psi/\Theta](a_1, \dots , a_n).$$
				По индукционному допущению формулы $\Phi_1 \equiv \Phi_1[\Psi/\Theta]$ аналогично для $\Phi_2$ Поэтому $$\Phi(a_1, \dots , a_n) = \Phi_1(a_1, \dots , a_n) \wedge \Phi_2(a_1, \dots  , a_n),$$
				$$\Phi[\Psi/\Theta](a_1, \dots , a_n) = \Phi_1[\Psi/\Theta](a_1, \dots , a_n) \wedge \Phi_2[\Psi/\Theta](a_1, \dots , a_n),$$ т.е. $\Phi \equiv \Phi[\Psi/\Theta]$
		\end{enumerate}
	\end{proof}
    \end{theorem}
	\begin{theorem}
		Справедливы следующие эквивалетности
        \begin{multicols*}{2}
            \begin{enumerate}
                \item a $\vee$ b $\equiv$ b $\vee$ a  \textbf{симметричность}
                \item a $\wedge$ b $\equiv$ b $\wedge$ a
                \item a $\vee$ (b $\vee$ c) $\equiv$ (a $\vee$ b) $\vee$ c \textbf{ассоциативность}
                \item a $\wedge$ (b $\wedge$ c) $\equiv$ (a $\wedge$ b) $\wedge$ c
                \item a $\wedge$ (b $\vee$ c) $\equiv$ a$\wedge$b $\vee$ a$\wedge$c \textbf{дистрибутивность}
                \item a $\vee$  b $\wedge$ c $\equiv$ (a $\vee$ b) $\wedge$ (a $\vee$ c)
                \item a $\vee$ a $\equiv$ a \textbf{идемпотентность}
                \item a $\wedge$ a $\equiv$ a
                \item $\overline{(a \vee b)} \equiv \overline{a} \wedge \overline{b}$ \textbf{законы де Моргана}
                \item $\overline{(a \wedge b)} \equiv \overline{a} \vee \overline{b}$
                \item $\overline{\overline{a}} \equiv a$ \textbf{двойное отрицание}
                \item a $\vee$ a $\wedge$ b $\equiv$ a \textbf{поглощение}
                \item a $\wedge$ (a $\vee$ b) $\equiv$ a
                \item a $\vee$ $\overline{a} \wedge b \equiv a \vee b$ \textbf{слабое поглощение}
                \item a $\wedge$ $(\overline{a} \vee b) \equiv ab$
                \item $a \vee 0 \equiv a$
                \item $a \wedge 0 \equiv 0$
                \item $a \vee 1 \equiv 1$
                \item $a \wedge 1 \equiv a$
                \item $a \vee \overline{a} \equiv 1$
                \item $a\overline{a} \equiv 0$
                \item $a \rightarrow b \equiv \overline{a} \vee b$
                \item $a \leftrightarrow b \equiv \overline{a} \wedge \overline{b} \vee a \wedge b \equiv (a \rightarrow b) \wedge (b \rightarrow a)$
                \item $a + b \equiv \overline{a \leftrightarrow b} \equiv \overline{a} \wedge b \vee a \wedge \overline{b}$
                \item $a | b \equiv \overline{a \wedge b}$
                \item $a \downarrow b \equiv \overline{a \vee b}$
            \end{enumerate}
        \end{multicols*}
	\end{theorem}
    \begin{proof}
        Доказательство сводится к построению таблиц истинности для левой и правой частей каждой эквивалентности
    \end{proof}
    \subsection{Тождественно истинные (ложные) и выполнимые БФ}
    \begin{definition}
        Формула $\Phi(x_1, \dots, x_n)$ называется тождественно истинной
        (ложной), если для любого набора значений $\Phi(x_1, \dots, x_n) = 1 (или 0)$
    \end{definition}
    \begin{definition}
        Формула $\Phi(x_1, \dots, x_n)$ называется выполнимой, если
        существует набор значений, для которого $\Phi(x_1, \dots, x_n) = 1$
    \end{definition}
    \subsection{ДНФ и КНФ, алгоритмы приведения}
	\begin{definition}
		Литера - это переменная или отрицание переменной
	\end{definition}
	\begin{definition}
		Конъюнкт(элементарная конъюнкция) - это либо литера, либо конъюнкция литер
	\end{definition}
	\begin{definition}
		Дизъюнктивная нормальная форма(ДНФ) -  это либо конъюнкт, либо дизъюнкия конъюнктов
	\end{definition}
	\begin{definition}
		Дизъюнкт(элементарная дизъюнкция) - это либо литера, либо дизъюнкция литер
	\end{definition}
	\begin{definition}
		Конъюнктивная нормальная форма (КНФ) - это либо дизъюнкт, либо конъюнкция дизъюнктов
	\end{definition}
	\begin{remark}
		Алгоритм построения ДНФ(КНФ) по заданной ТИ
		\begin{enumerate}
			\item Выбрать в таблице все строки со значением функции f = 1 (f = 0)
			\item Для каждой такой строки (x, y, z) = $(a_1, a_2, a_3)$ выписать конъюнкт(дизъюнкт) по принципу: пишем переменную с отрицанием, если ее значение 0(1), иначе пишем переменную без переменную без отрицания.
			\item берем дизъюнкцию(конъюнкцию) построенных конъюнктов(дизъюнктов)
		\end{enumerate}
	\end{remark}
	\begin{remark}
		Алгоритм приведения формулы к ДНФ/КНФ методом эквивалентностей
		\begin{enumerate}
			\item Выразить все связки в формуле через конъюнкцию, дизъюнкцию и отрицание.
			\item Внести все отрицания внутрь скобок
			\item Устранить двойные отрицания
			\item Применять свойство дистрибутивности, пока это возможно
		\end{enumerate}
	\end{remark}
    \subsection{СДНФ и СКНФ, теоремы существования и единственности, алгоритмы приведения}
	\begin{definition}
		Совершенный конъюнкт от переменных $x_1, \dots , x_n$ - это конъюнкт вида $x_1^{a_1} \wedge\dots \wedge x_n^{a_n},$ где $(a_1, \dots , a_n) \in \{0, 1\}^n.$
	\end{definition}
	\begin{definition}
		Совершенный дизъюнкт от переменных $x_1, \dots , x_n$ - это конъюнкт вида $x_1^{a_1} \vee\dots \vee x_n^{a_n},$ где $(a_1, \dots , a_n) \in \{0, 1\}^n.$
	\end{definition}
	\begin{remark}
		\begin{equation*}
			x^a = 
 			\begin{cases}
   				\overline{x} &\text{если a = 0,}\\
   				x &\text{если a = 1.}
 			\end{cases}
		\end{equation*}
	\end{remark}
	\begin{definition}[СДНФ]
		Совершенная дизъюнктивная нормальная форма(СДНФ) от переменных $x_1, \dots , x_n$  - это дизъюнкция совершенных конъюнктов от $x_1, \dots , x_n$, в которой нет попарно эквивалентных слагаемых
	\end{definition}
	\begin{definition}[СКНФ]
		Совершенная конъюктивная нормальная форма(СКНФ) от переменных $x_1, \dots , x_n$ - это конъюнкция совершенных дизъюнктов от $x_1, \dots , x_n$, в которой нет попарно эквивалентных слагаемых.
	\end{definition}
	\begin{theorem}[о существовании и единственности СДНФ]
		Любая булева функция $f(x_1, \dots , x_n) \neq 0$ определяется формулой, находящейся в СЛНФ, причем эта СДНФ единственная с точностью до перестановок слагаемых и множителей в слагаемых
		\begin{proof}
			\begin{enumerate}
				\item Существование. По следствию к теореме о разложении получаем для $f(x_1, \dots , x_n) \neq 0$
					$$f = \bigvee_{(a_1, \dots , a_n) \in \{0, 1\}^n\atop{f(a_1, \dots , a_n) = 1}} x_1^{a_1} \dots  x_n^{a_n}$$
				\item Единственность. Пусть, у функции $f(x_1, \dots , x_n) \neq 0$ две СДНФ, обозначим их $\Phi$ и $\Psi$. Так как они определяют одну и ту же функцию, то $\Phi \equiv \Psi$

					Выберем в $\Phi$ произвольное слагаемое $x_1^{a_1}\dots x_n^{a_n}$. По лемме о совершенных конъюнктах это слагаемое истинно при $(x_1, \dots , x_n) = (a_1, \dots , a_n)$. Тогда и вся дизъюнкция $\Phi(a_1, \dots , a_n) = 1$, а в силу эквивалентности формул и $\Psi(a_1, \dots , a_n) = 1$

					Но тогда в $\Psi$ есть слагаемое $x_1^{b_1}\dots x_n^{b_n}$, истинное на наборе $(a_1, \dots , a_n)$. Снова по лемме это возможно только при $(a_1, \dots , a_n)  = (b_1, \dots , b_n).$

					Получаем, что все слагаемые СДНФ $\Phi$ есть в $\Psi$. Рассуждая симметрично, получаем, что и $\Psi$ содержится в $\Phi$, т.е. они равны
			\end{enumerate}
		\end{proof}
	\end{theorem}
	\begin{remark}[Лемма о совершенных конъюнктах]
		\begin{enumerate}
            \item Пусть $\Phi(x_1, \dots , x_n) =  x_1^{a_1}\dots x_n^{a_n}$ - совершенный конъюнкт. 
            Тогда для любого набора значений $(b_1, \dots , b_n) \in \{0, 1\}^n$ $$ \Phi(b_1, \dots , b_n) = 1 \leftrightarrow (b_1, \dots , b_n) = (a_1, \dots , a_n).$$
            \item Два совершенных конъюнкта от перменных $x_1, \dots , x_n$ эквивалентны 
            тогда и только тогда, когда они равны с точностью до перестановки литер.
        \end{enumerate}
	\end{remark}
	\begin{remark}
		Рассуждая двойственным образом, можно получить теорему о СКНФ
	\end{remark}
	\begin{remark}
		Алгоритм приведения формулы к СДНФ(СКНФ)
		\begin{enumerate}
			\item Строим ДНФ(КНФ) формулы.
			\item Вычеркиваем тождественно ложные(истинные) слагаемые(множители).
			\item В каждое слагаемое(множитель) добавляем переменны по правилам:

				СДНФ: $\Phi(x_1, \dots , x_n) \equiv \Phi(y \vee \overline{y}) \equiv \Phi \wedge y \vee \Phi \wedge \overline{y}$
				
				СКНФ: $\Phi(x_1, \dots , x_n) \equiv \Phi \vee y \wedge \overline{y} \equiv (\Phi \vee y) \wedge (\Phi \vee \overline{y})$
			\item Вычеркиваем повторяющиеся слагаемые(множители).
		\end{enumerate}
	\end{remark}
    \subsection{Минимизация нормальных форм (карты Карно)}
	\begin{definition}
		ДНФ $\Phi$ булевой функции называется минимальной, если в любой ДНФ этой функции количество литер не меньше, чем в $\Phi$
	\end{definition}
	\begin{definition}
		Карта Карно функции $f(x_1, \dots , x_n)$ - это двумерная таблица построенная следующим образом.
		\begin{enumerate}
			\item Разделим набор переменных $x_1, \dots , x_n$ На две части: $x_1, \dots , x_k$ и $x_{k+1}, \dots , x_n$
			\item Строкам таблицы соответсвуют всевозможные наборы нзачений переменных $x_1, \dots , x_k$, колонкам - $x_{k+1}, \dots , x_n$. При этом наборы в двух соседних строках/колонках должны отличаться не более, чем одним значением. Крайние строки/колонки считаются соседними
			\item В ячейки заносятся значения функции $f(x_1, \dots , x_n)$ на соответсвующих наборах.
		\end{enumerate}
	\end{definition}
	\begin{remark}
		Алгоритм построения минимальной ДНФ с помощью карт Карно
		\begin{enumerate}
			\item Строим карту Карно функции f
			\item В карте находим покрытие всех ячеек со значением 1 прямоугольникам со свойствами:
			\begin{enumerate}
				\item Длины сторон прямоугольника - $2^k, k \geq 0$
				\item каждый прямоугольник содержит только 1
				\item каждая ячейка с 1 покарыта прямоугольником максимальной площади
				\item количество прямоугольников минимально
			\end{enumerate}
			\item По кааждому прямоугольнику выписываем конъюнкт. Конъюнкт образуют литеры, значения которых в прямоугольнике не меняются
		\end{enumerate}
	\end{remark}
    \subsection{Полином Жегалкина, его существование и единственность. Алгоритм построения}
	\begin{definition}
		Моном от перменных $x_1, \dots , x_n$ - это либо 1, либо конъюнкт вида $x_{i_1} \cdot \dots  \cdot x_{i_k}$, где $x_{i_k}$ - переменная из списка $x_1, \dots , x_n$, без повторяющихся множителей
	\end{definition}
	\begin{definition}
		Полином Жегалкина от переменных $x_1, \dots , x_n$ - это либо 0, либо сумма мономов от переменных $x_1, \dots , x_n$ без эквивалентных слагаемых
	\end{definition}
	\begin{theorem}[о существовании и единственности полинома Жегалкина]
		Любая булева функция может быть определена полиномом Жегалкина. Полином Жегалкина буленвой функции единственный с точностью до перестановок слагаемых и множителей
	\end{theorem}
	\begin{proof}
        \begin{itemize}
            \item (Существование) Т.к. для любой булевой функции можно определить ДНФ, доказывает, что любую булеву функцию можно выразить через $\wedge, \vee, '$. Выразим  $\wedge, +, 1$ через
            $\wedge, \vee, '$. $$\overline{x} = x + 1$$ $$x \vee y = \overline{\overline{x \vee y}} = \overline{\bar{x}\bar{y}} = (x+1)(y+1) + 1 = xy + x + 1 + 1 = xy + x + y.$$
            \item (Единственность)
        \end{itemize}Количество булевых функций от n переменных $2^{2^n}$
		
		Найдем количество полиномов Жегалкина от $x_1, \dots , x_n$
		
		Сопоставим моному упорядоченный набор чисел $(a_1, .., a_n) a_i \in \{0, 1\}$, по принципу: $a_i = 1 \leftrightarrow$ переменная $x_i$ в моному есть. Это соответствие является биекцией. Таким образом, мономов от n переменных столько же, сколько наборов вида $(a_1, \dots , a_n), a_i \in \{0, 1\}$, а их $2^n$ штук.
		
		Произвольный полином Жегалкина от n переменных можно представить в виде: $p(x_1, \dots , x_n) = b_1M_1 + \dots  + b_kM_k, k = 2^n$, где $b_j \in \{0, 1\}$, а $M_1, \dots , M_k$ - все мономы от $x_1, \dots , x_n$.
		
		Сопоставим полиному p набор коэффициентов $(b_1, \dots , b_k), b_j \in \{0, 1\}.$
		
		Это снова биекция, поэтому полиномов столько же сколько таких наборов, а их $2^k = 2^{2^n}$
		
		Получили, что количество полиномов Жегалкина от n переменных равно количеству булевых функций от n переменных.
		
		Допустим теперь, что у какой-то булевой функции f два разных полинома. Тогда для какой-то другой функции g полинома не хватит. Но это противоречит тому, что каждую булеву функцию можно представить полиномом Жегалкина.
	\end{proof}
    \subsection{Суперпозиция булевых функций. Замкнутые классы булевых функций}
    \begin{definition}
        Суперпозиция булевых функций $f(x_1, \dots, x_n)$ и $f_i(x_1, \dots, x_k), i = 1, \dots, n$, — это функция 
        $F(x_1, \dots, x_k) = f(f_1, \dots, f_n).$
    \end{definition}
    \begin{definition}
        Подстановка переменной y вместо $x_i$ в булеву функцию $f(x_1, \dots, x_n)$ — это суперпозиция вида 
        $f(x_1, \dots,x_{i - 1}, y, x_{i + 1}, \dots, x_n).$
    \end{definition}
    \begin{definition}
        Замыкание класса K булевых функций (обозначение: [K]) — это наименьший класс, 
        содержащий все функции класса K, всевозможные их суперпозиции и результаты 
        подстановок переменных, суперпозиции полученных функций и т.д.
    \end{definition}
    \begin{definition}
        Замкнутый класс булевых функций — это класс, равный своему замыканию.
    \end{definition}
    \begin{example}
        $M = \{x', x\oplus y\}.$
        \begin{enumerate}
            \item $0\in [M]$, так как $0 = x\oplus x$
            \item $1\in [M]$, так как $1 = (x\oplus x)'$
            \item $x\oplus y\oplus z\in [M]$
        \end{enumerate}
    \end{example}
    \subsection{Полные системы булевых функций, базисы}
    \begin{definition}
        Система булевых функций является полной(в классе К), если ее замыкание равно классу всех булевых функций(классу К)
    \end{definition}
    \begin{example}[Примеры полных систем]
        \begin{enumerate}
            \item $M = \{\neg x, xy, x\vee y\}$ \textbf{каждая БФ может быть записана в виде ДНФ}
            \item $M = \{\neg x, x\vee y\}$ \textbf{выражаем $xy$ через отрицание и дизъюнкцию по закону де Моргана}
            \item $M = \{\neg x, xy\}$
            \item $M = \{\oplus, *, 1\}$ \textbf{полином Жегалкина}
            \item $\{\leftrightarrow, \vee, 0\}$ \textbf{навесить отрицание на функции из предыдущей системы}
            \item $M = \{x|y\}$, $\neg x \equiv x|x, xy\equiv \neg(x|y) \equiv (x|y)|(x|y)$ \textbf{аналогично стрелка Пирса}
        \end{enumerate}
    \end{example}
    \begin{definition}
        Полная (в классе К) система функций называется базисом (класса К), если никакая ее подсистема не будет полной (в классе К).
    \end{definition}
    \begin{example}[Примеры базисов]
        \begin{enumerate}
            \item $M = \{x|y\}$, $\neg x \equiv x|x, xy\equiv \neg(x|y) \equiv (x|y)|(x|y)$ \textbf{аналогично стрелка Пирса}
            \item $M = \{\&, '\}$, аналогично $\{\vee, '\}$
            Мы не могли вычеркнуть отрицание, так как $xy$ и $x\vee y\in T_0\implies [xy, x\vee y]\subseteq T_0$
            и $1\notin T_0\implies \neg x \in [xy, x\vee y]\implies$ $\{\vee, \& \}$ не полна
            \item $M = \{\oplus, *, 1\}$ \textbf{полином Жегалкина}
        \end{enumerate}
    \end{example}
    \begin{remark}
        Никакой базис не может содержать более 4 функций.
    \end{remark}
    \begin{proof}
        Из доказательства теоремы Поста $g_0(x)$ (не сохраняющая 0 функция $f(x_1, \dots, x_n)$, 
        в которую подставлили одну и ту же переменную x) либо несамодвойственна, либо немонотонна,$\implies$
        полной будет система из 4 функций.
        Этим доказано, что всякая полная система содержит полную подсистему не более чем из четырёх функций. 
        В базисе нет собственных полных подсистем, поэтому в нём не более четырёх функций.
        
        Оценку нельзя уменьшить, так как существует система $\{0, 1, xy, x\oplus y \oplus z\}$.
        Построим таблицу с классами Поста, видим, что система полна и никакая ее собственная
        подсистема не полна.
    \end{proof}
    \subsection{Классы $T_0, T_1$ (функции, сохраняющие 0 и 1)}
    \begin{definition}
        Класс $T_0 = \{f(x_1,\dots, x_n)$ | $f(0, \dots, 0) = 0\}$
    \end{definition}
    \begin{definition}
        Класс $T_1 = \{f(x_1,\dots, x_n)$ | $f(1, \dots, 1) = 1\}$
    \end{definition}
    \begin{tabular}{c|c|c|c|c|c}
        & $T_0$ & $T_1$ & S & M & L\\
        \hline
        0 & + & - & - & + & +\\
        1 & - & + & - & + & + \\
        x & + & + & - & + & +\\
        $\neg x$ & - & - & +  & - & +\\
        $xy$ & + & + & - & + & -\\
        $x \vee y$ & + & + & - & + & -\\
        $x\oplus y$ & + & - & - & - & +\\
        $x\leftrightarrow y$ & - & + & - & - & +\\
        $x\rightarrow y$ & - & + & - & - & -\\
        $x|y$ & - & - & - & - & -\\
        $x\downarrow y$ & - & - & - & - & -\\
    \end{tabular}
    \begin{remark}
        Классы $T_0, T_1$ являются замкнутыми.
    \end{remark}
    \begin{proof}
        Докажем для $T_0$. Достаточно взять булевы функции $g, g_1, \dots, g_n\in T_0$
        и доказать, что их суперпозиция из класса $T_0.$

        $g(g_1 (0, \dots, 0), \dots, g_n(0, \dots, 0)) = g(0, \dots, 0) = 0$
    \end{proof}
    \subsection{Класс S самодвойственных функций, определение двойственной БФ}
    \begin{definition}
        Булева функция $g(x_1, \dots, x_n)$ называется двойственной к БФ $f(x_1, \dots, x_n)$
        (обозначается g = f*), если $g(x_1, \dots, x_n) = f'(x_1', \dots, x_n')$.
    \end{definition}
    Из закона двойного отрицания следует, что $(f^*)^* = f$
    \begin{definition}
        Булева функция f называется самодвойственной, если $f = f^*$.
    \end{definition}
    \begin{definition}
        Класс самодвойственных функций = $\{f$ | $f = f^*\}$
    \end{definition}
    \begin{remark}
        Класс S является замкнутым.
    \end{remark}
    \begin{proof}
        Возьмем БФ $g, g_1, \dots g_k \in S$ и докажем, что их
        суперпозиция будет также из класса S. \\
        Если $F(x_1, \dots, x_n) = g(g_1(x_1, \dots, x_n), \dots, g_k(x_1, \dots, x_n))$,\\
        то $F^*(x_1, \dots, x_n) = \neg F(\neg x_1, \dots, \neg x_n) = $
        $\neg g(g_1(\neg x_1, \dots, \neg x_n), \dots, g_k(\neg x_1, \dots, \neg x_n))$.

        Так как $g_i \in S,$ то $g_i(x_1, \dots, x_n) = \neg g_i(\neg x_1, \dots, \neg x_n)$,
        что эквивалентно $ \neg g_i(x_1, \dots, x_n) = g_i(\neg x_1, \dots, \neg x_n)$.
        Следовательно, $F^*(x_1, \dots, x_n) = \neg g(\neg g_1(x_1, \dots, x_n), \dots, \neg g_k (x_1, \dots, x_n))$.\\
        Так как $g\in S$, то $\neg g(\neg g_1(x_1, \dots, x_n), \dots, \neg g_k (x_1, \dots, x_n))$
        $= (g_1(x_1, \dots, x_n), \dots, g_k(x_1, \dots, x_n))$
        $\implies f^*(x_1, \dots, x_n) = F(x_1, \dots, x_n)$
    \end{proof}
    \subsection{Класс монотонных функций}
    \begin{definition}
        Назовем два набора из 0 и 1 $a = (a_1, \dots a_n), b = (b_1, \dots b_n)$
        \textbf{соседними}, если все их координаты (кроме одной) совпадают.
    \end{definition}
    \begin{definition}
        Пусть k - номер единственной координаты, по которой отличаются соседние наборы a, b.
        Если $a_k = 0$, $b_k = 1$, то мы будем говорить, что набор a \textbf{меньше} набора b
        ($a\prec b$)
    \end{definition}
    \begin{definition}[Монотонная функция]
        БФ $f(x_1, \dots x_n)$ называется монотонной, если
        $\forall$ соседних наборов $a,b$ таких, что $a\prec b\implies f(a) \leq f(b)$
    \end{definition}
    \begin{remark}
        Класс M является замкнутым.
    \end{remark}
    \begin{proof}
        $g, g_1, \dots g_k \in M, F(x_1, \dots, x_n) = g(g_1, \dots g_k)$
        и рассмотрим два произвольных набора $a\prec b$. Пусть $c_1 = g_1(a), d_1 = g_1(b), \dots$
        $c_k = g_k(a), \dots d_k = g_k(b)$

        $g_i \in M \implies c_i\leq d_i $

        Если наборы $c = (c_1, \dots, c_k)$ и $d = (d_1, \dots, d_k)$ - соседние,
        то и $F(c)\leq F(d)$

        В противном случае легко показать, что $\exists$ цепочка

        \begin{align*}
            c \prec e_1 \prec\dots\prec e_l \prec d
        \end{align*}
        (то есть наши наборы сравнимы по определению Ашаева)
        
        и $g(c) \leq g(d)\implies F(c) \leq F(d) \implies F\in M$
    \end{proof}
    \subsection{Класс линейных функций}
    \begin{definition}
        БФ называется линейной, если ее полином Жегалкина линеен, т.е не содержит конъюнкции т.е
        его степень не выше 1.
    \end{definition}
    \begin{lemma}
        Класс L является замкнутым.
    \end{lemma}
    \begin{proof}
        При подстановке линейных функций в линейную функцию не может появиться конъюнкции. 
        $f(x_1, \dots, x_n)= a0 \oplus a_1(f_1(x_1, \dots, x_n) \dots \oplus a_m f_m(x_1, \dots, x_n)$
        $=a_0 \oplus a1(b_0^1 \oplus b_1^1 x_1 \dots \oplus b_n^1 x_n)  \dots$
        $\dots \oplus  a_m(b_0^m \oplus b_1^m x_1 \dots \oplus b_n^m x_n)=$
        $(a_0 \oplus  a_1b_0^1  \dots \oplus  a_m b_0^m) \oplus$
        $(a_1b_1^1 \oplus  \dots \oplus  a_m b_1^m)x_1 \oplus\dots  \oplus (a_1b_n^1 \oplus  \dots  \oplus  a_m b_n^m)x_n.$
    \end{proof} 
    \subsection{Леммы о несамодвойственной, немонотонной, нелинейной функциях}
    \begin{lemma}[о несамодвойственной функции]
        Если БФ $f(x_1, \dots, x_n)$ несамодвойственна, то замыкание класса $[f, \neg x]$
        содержит тождественно ложную БФ 0 и тождественно истинную БФ 1.
    \end{lemma}
    \begin{proof}
        Так как f несамодвойственна, то существует набор $a_1, \dots, a_n$
        значений аргументов такой, что \\ $f(a_1, \dots, a_n) \neq \neg f(\neg a_1, \dots, \neg a_n)$ \\
        Так как БФ принимают только значения 0 и 1, то $f(a_1, \dots, a_n) = f(\neg a_1, \dots, \neg a_n)$ \\
        Составим функцию $g(x) = f(x^{a_1}, \dots, x^{a_n})$, где 
        
        \begin{equation*}
            x^a = 
            \begin{cases}
                x & \text{если $a = 1$}  \\ 
                \text{$\neg x$} & \text{если $a = 0$} 
            \end{cases}
        \end{equation*}
        Очевидно, что $g \in [f, \neg x]$, так как является их суперпозицией.

        $g(0) = f(0^{a_1}, \dots, 0^{a_n}) = f(\neg a_1, \dots, \neg a_n)$, 
        $g(1) = f(1^{a_1}, \dots, 1^{a_n}) = f(a_1, \dots, a_n)$, 

        $g(0) = g(1)$ - $g$ - константа, $g$ и $\neg g$ принимают значения 0 и 1 чтд.
    \end{proof}
    \begin{lemma}[О немонотонной функции]
        Если $f(x_1, \dots, x_n)$ немонотонна, то $x' \in [f, 0, 1]$
    \end{lemma}
    \begin{proof}
        Из немонотонности f следует существование двух соседних наборов 
        $a = (a_1, \dots, a_n) \prec (b_1, \dots, b_n) = b$
        такие, что $f(a) > f(b)$. Б.О.О считаем, что они отличаются только в первой
        координате
        \begin{align*}
            a_1 = 0 \\
            b_1 = 1 \\
            a_i = b_i
        \end{align*}
        $\sphericalangle g(x, a_2, \dots, a_n) \in [f, 0, 1]$\\
        $g(0) = f(a) = 1\quad, g(1) = f(b) = 0 \implies g \equiv x'$
    \end{proof}
    \begin{lemma}[О нелинейной функции]
        $f(x_1, \dots, x_n) \notin L \implies xy \in [f, 0, 1, x']$
    \end{lemma}
    \begin{proof}
        $f(x_1, \dots, x_n) \notin L\implies$ полином Жегалкина функции f содержит конъюнкцию двух переменных $x_1$ и $x_2$
    
        $\implies f(x_1, \dots, x_n)=$
        $x_1x_2h_{12}(x_3, \dots x_n) + x_1h_1(x_3, \dots x_n) + h_0(x_3, \dots x_n)$
        
        $f\notin L \implies h_{12}\neq 0 \implies \exists (a_3, \dots a_n) h_{12} (a_3, \dots a_n) = 1$

        Подставим этот набор в ПЖ $f$:

        $g(x_1, x_2) = f(x_1, x_2, a_3\dots  a_n) = $
        $x_1x_2h_{12}(a_3, \dots a_n) + x_1h_1(a_3, \dots a_n) + h_0(a_3, \dots a_n)$

        $h_i\in \{0, 1\}\implies \exists 8$ вариантов того, как выглядит полином Жегалкина

        \begin{enumerate}
            \item Система функций $[g, \neg, 0, 1]$ полна и содержит конъюнкцию
            \item g - конъюнкция
            \item $xy = g(x, y') \vee xy = g(x', y)\implies xy \in $замыкание
        \end{enumerate}
        Т.к $g$ выражается через $f(x_1, \dots x_n), 0, 1$, то конъюнкция также лежит в замыкании $[f, \neg, 0, 1]$

    \end{proof}
    \subsection{Теорема Поста о полноте системы булевых функций}
    \begin{theorem}[Теорема Поста]
        Cистема БФ является полной тогда и только тогда, когда она не лежит целиком ни в одном из классов Поста.
    \end{theorem}
    \begin{proof}
        $ $\newline
        \begin{itemize}
            \item[$\Rightarrow$] Пусть все функции из 1 класса, б.о.о. они из $T_0.$
            Так как он замкнут, то замыкание этих функций не совпадает с $\mathcal{B}\implies$
            набор не полон.
            \item[$\Leftarrow$] Если набор $f_1 \dots f_k$ не содержится полностью
            ни в одном из классов Поста, то существуют БФ 
            $f_0\notin T_0 , f_1\notin T_1, f_S\notin S, f_M\notin M, f_L\notin L$

            Заменим все переменные этих функций на x и получим функцию одного аргумента
            \begin{align*}
                g_0(x) = f_0(x, x, \dots , x),
                g_1(x) = f_1(x, x, \dots , x),
                g_S(x) = f_S(x, x, \dots , x),
                g_M(x) = f_M(x, x, \dots , x),
                g_L(x) = f_L(x, x, \dots , x).
            \end{align*}
            Все БФ из замыкания этих функций $G \in [f_1, \dots, f_k]$
            (переименовали переменные).
            Докажем полноту набора $[G]$ через полноту $[\neg x, xy]$:
            
            Для $g_0, g_1: g_0(0) = 1, g_1(1) = 0$
            \begin{tabular}{c|c|c}
                $g_0(1)$ & $g_1(0)$ & \\
                \hline
                0 & 0 & $g_0 \equiv \neg x, g_1 \equiv 0$\\
                0 & 1 & $g_0 \equiv \neg x, g_1 \equiv \neg x$\\
                1 & 0 & $g_0 \equiv 1, g_1 \equiv 0$\\
                1 & 1 & $g_0 \equiv 1, g_1 \equiv \neg x$\\
            \end{tabular}
            \begin{enumerate}
                \item $[G]\ni \neg x, 0, 1$ по лемме о нелинейной функции
                содержит $xy$
                \item $[G]\ni \neg x\implies$ по лемме о несамодвойственной функции содержит 0 и 1$\implies$
                по лемме о нелинейной функции содержит $xy$
                \item $[G]\ni 0, 1\implies$ по лемме о немонотонной функции содержит $\neg x\implies$
                по лемме о нелинейной функции содержит $xy$
                \item $[G]\ni \neg x, 0, 1$ по лемме о нелинейной функции
                содержит $xy$
            \end{enumerate}
        \end{itemize}
    \end{proof}
    \subsection*{Предполные классы}
    \begin{definition}
        Предполным классом К называется неполный класс, при добавлении любой функции,
        которая не принадлежит ему, получается класс полный.
    \end{definition}
    \begin{statement}Предполный класс является замкнутым.
    \end{statement} 
    \begin{proof}
        Пусть класс A не замкнут. 
        Значит, найдется функция $f \in [A] \setminus A$. Получаем:
        $[A \cup {f}] = [A].$

        $A\neq \mathcal{B}$, но при добавлении f получаем полную систему (по определению)$\implies$ противоречие.
        Значит, A — замкнутый класс.
    \end{proof}
    \begin{statement}[Максимальные замкнутые классы]
        Классы Поста являются максимальными замкнутыми классами (предполными) и других нет.
    \end{statement}
    \begin{proof}
        $ $

        \begin{itemize}
            \item Докажем максимальность $T_0$. Пусть он не максимален, т.е существует замкнутый класс A
            такой, что $T_0\subset A \subset \mathcal{B}$, тогда $[T_0]\subseteq A$
    
            Пусть $f_0\in A\setminus T_0$, тогда $g(x) = f(x, \dots, x)\notin T_0$. Если $g(1) = 0, g \equiv \neg(x)$,
            иначе $g\equiv 1$. Так как $T_0\ni 0, xy,$ немонотонные и несамодвойственные функции, $[T_0, f] = \mathcal{B},$
            а это противоречит $[T_0, f]\subseteq A$.
            \item Докажем максимальность $T_1$. Пусть он не максимален, т.е существует замкнутый класс A
            такой, что $T_1\subset A \subset \mathcal{B}$, тогда $[T_1]\subseteq A$
    
            Пусть $f_1\in A\setminus T_1$, тогда $g(x) = f(x, \dots, x)\notin T_1$. Если $g(0) = 1, g \equiv \neg(x)$,
            иначе $g\equiv 0$. Так как $T_1\ni 1, xy,$ немонотонные и несамодвойственные функции, $[T_1, f] = \mathcal{B},$
            а это противоречит $[T_1, f]\subseteq A$.
            \item K = S. Пусть $f(x_1, \dots, x_n) \notin S.$ $x'\in S$, по лемме о несамодвойственной функции
            $0, 1\in [f, x']\subseteq[S, f]$

            Выберем в S нелинейную функцию, например, $g = xy+yz+xz$. По лемме о нелинейной функции
            $xy\in[g, 0, 1, x']\subseteq [S, f]\implies \{xy, x'\}\in [S, f]$

            $\mathcal{B} = [xy, x']\subseteq[S, f]=B$
            \item K = M, $f(x_1, \dots, x_n) \notin M$. По лемме о немонотонной функции
            $0, 1 \in M$; $x' \in [f, 0, 1]\subseteq [M, f]$

            $\{xy, x'\}\in [M, f]\implies \mathcal{B}=[xy, x']\subseteq[M, f] = B$
            \item K = L, $f(x_1, \dots, x_n) \notin L$. По лемме о нелинейной функции
            $x', 0, 1\in L;$ $xy\in  [0, 1, x', f]\subseteq [L, f]$

            $\{xy, x'\}\in [L, f]\implies \mathcal{B}=[xy, x']\subseteq[L f] = B$
        \end{itemize}
        
    \end{proof}
    \subsection{Релейно-контактные схемы: определение, примеры, функция проводимости. Анализ и синтез РКС (умение решать задачи)}
    \begin{definition}
        Реле это некоторое устройство, которое может находиться в одном из двух возможных состояний: включенном и выключенном. 
    \end{definition}
    \begin{example}
        Примеры реле: различные выключатели, термодатчики, датчики движения и т.п. 
    \end{example}

    Реле используются в построении различных электрических схем. 
    Включение или выключение реле приводит к появлению 
    или исчезновению тока на определённых участках электрической схемы. 
    


    Пусть S некоторая электрическая схема, содержащая реле $x_1, \dots , x_n$. Со схемой
    S можно связать функцию проводимости $f_S$, которая равна 1, если схема проводит
    ток при заданном состоянии реле (и $f_S$ равна 0 в противном случае). Возникает
    вопрос: а какие аргументы имеет функция $f_S$? Для определения аргументов $f_S$ мы
    будем рассматривать каждое реле $x_i$ как переменную, принимающую значения из
    множества $\{0, 1\}$ с очевидной интерпретацией: $x_i = 0$, если реле выключено, и $x_i = 1$,
    если реле включено.

    Таким образом функция проводимости $f_S(x_1, \dots , x_n)$ становится булевой функцией, 
    зависящей от текущего состояния своих реле.

    \begin{enumerate}
        \item цепь замкнута - $f_S = 1$
        \item цепь не замкнута $f_S = 0$
        \item последовательное соединение $f_S(x, y) = xy$
        \item параллельное соединение $f_S(x, y) = x\vee y$
    \end{enumerate}
    Задачи, связанные с релейно-контактными схемами можно подразделить на две
большие группы:
\begin{enumerate}
    \item дана схема, нужно построить более простую схему с такой же функцией проводимости
    \item нужно построить схему по описанию её функции проводимости.
\end{enumerate}
    \section{Логика высказываний}
    \subsection{Парадоксы в математике. Парадоксы Г. Кантора и  Б. Рассела}
    \begin{statement}[Рассел]
        Множество M будем называть нормальным, если оно не
        принадлежит самому себе как элемент. Например, множество кошек нормально, поскольку множество кошек не является кошкой. А вот каталог каталогов
        по-прежнему останется каталогом, поэтому множество каталогов, не является
        нормальным. Рассмотрим теперь множество B, составленное из всевозможных
        нормальных множеств. Формально множество B определяется так:

        $x \in  B \Leftrightarrow  x \notin  x $ (2)

        Возникает вопрос: будет ли B принадлежать самому себе как элемент? И тут
        возникает парадокс: дело в том, что если вместо x из формулы (2) подставить
        B, то возникнет явное противоречие
        $B \in  B \Leftrightarrow B \notin  B.$
    \end{statement}
    \begin{statement}[Кантор?]
        Предположим, что множество всех множеств ${ V=\{x\mid x=x\}}$ существует.
        В этом случае справедливо ${ \forall x\forall T(x\in T\rightarrow x\in V)}$, то есть всякое множество
        ${ T} $является подмножеством ${ V}$. Но из этого следует
        ${ \forall T\;|T|\leqslant |V|}$ — мощность
        любого множества не превосходит мощности ${ V}$.

        Но в силу аксиомы множества всех подмножеств, для ${ V}$, как и любого множества,
        существует множество всех подмножеств$ { {\mathcal {P}}(V)}$, и по
        теореме Кантора ${ |{\mathcal {P}}(V)|=2^{|V|}>|V|}$,
        что противоречит предыдущему утверждению. Следовательно, ${ V}$ не может существовать,
        что вступает в противоречие с «наивной» гипотезой о том, что любое синтаксически корректное логическое условие
        определяет множество, то есть что ${ \exists y\forall z(z\in y\leftrightarrow A)}$ для любой формулы ${ A}$,
        не содержащей ${ y}$ свободно.
    \end{statement}
    \subsection{Логическое следование в логике высказываний. Проверка логического следования с помощью таблиц истинности и эквивалентных преобразований.}
    \begin{definition}
    Интерпретация переменных - это отображение вида $\alpha : \{ x, x_1, \dots, x_n \} \to \{ 0, 1\} $.
    Задать интерпретацию - приписать j-той переменной значение 0, 1
    \end{definition}
    Если $\Phi$ - формула, а $\alpha$ - интерпретация, то $\Phi^\alpha$ - значение формулы, когда вместо $x_i$ подставили $\alpha (x_i)$
    
    Первый способ определить математическое понятие доказательства - логическое следование.
    \begin{definition}
        $\Gamma $ - множество формул, $\Phi$  - формула логики высказываний. Формула $ \Phi$  логически следует из $\Gamma$ ($\Gamma \models \Phi $), если для любой интерпретации
        $\alpha_k$ верно - если истинны все формулы из $\Gamma$ при этой интерпретации, то истинна и $\Phi$.

        $\forall \alpha (\forall \psi \in \Gamma$ $\psi^\alpha = 1)\implies \Phi^\alpha = 1$
    \end{definition}
    \subsubsection*{Свойства логического следования}
    \begin{enumerate}
        \item $\Phi \models \Psi, \Psi \models \Theta \implies \Phi \models \Theta$
        \item $\Gamma, \Delta $ - множество формул, $\Phi$ - формула. Если $\forall \psi\in\Delta$ $\Gamma\models\psi$ $[\Gamma \models \Delta]$ $\&\Delta\models\Phi$, то $\Gamma\models\Phi$
        \item Если $\Gamma \models \Phi, \Gamma \subseteq \Delta, \implies \Delta \models \Phi$
        \item $\models \Phi \implies \Phi \equiv 1$
        \item $\Phi_1, \dots, \Phi_n\models \Phi_1,\& \dots,\& \Phi_n$ \& 
        
        $\Phi_1,\& \dots,\& \Phi_n\models\Phi_1 \dots \Phi_n$
        \item $\Gamma, \Phi\models \Psi\Leftrightarrow \Gamma \models \Phi \to \Psi$
        \item Если $\Gamma = \{\Phi_1, \dots, \Phi_n \}$ - конечное, то $\Gamma \models \Phi \Leftrightarrow \Phi_1,\& \dots,\& \Phi_n\to \Phi \equiv 1$
    \end{enumerate}
    \begin{proof}
        \begin{enumerate}
            \item Следует из 2 [$\Delta= \{ \Psi\}, \Gamma = \{\Phi \}  $]
            \item $\Gamma \models \Delta: \forall \alpha$ - интерпретация $\forall \Psi \in \Delta [(\forall \theta \in \Gamma \quad \theta^\alpha = 1)\implies \Psi^\alpha = 1]$
            
            $\Delta \models \Phi: \forall \alpha$ - интерпретация $[(\forall \Psi \in \Delta \quad \Psi^\alpha = 1)\implies \Phi^\alpha = 1]$

            $(\forall \theta \in \Gamma \quad \theta^\alpha = 1)\implies \forall \Psi \in \Delta \quad \Psi^\alpha = 1\implies \Phi^\alpha = 1$

            $\implies \Gamma \models \Phi$

            [тупо пишем условие]
            \item $\Gamma \models \Phi\implies \Phi: \forall \alpha [(\forall \Psi \in \Gamma \quad \Psi^\alpha = 1)\implies \Phi^\alpha = 1]$
            
            $\Gamma \subseteq \Delta: \forall\Psi\in\Delta \quad \Psi^\alpha = 1\implies \forall\Psi\in\Gamma \quad \Psi^\alpha = 1\implies \Phi^\alpha = 1$
            \item $\models \Phi \Leftrightarrow \forall \alpha (\forall \Psi \in \varnothing\quad \Psi^\alpha = 1)\to \Phi^\alpha = 1 \implies \forall \alpha \quad \Phi^\alpha = 1\implies \Phi \equiv 1$
            \item $\alpha$ - инт., тогда $(\forall \Psi \in \{ \Phi_1, \dots, \Phi_n\} \Psi^\alpha = 1)\Leftrightarrow (\Phi_1, \dots, \Phi_n)^\alpha = 1\implies$
            $\Phi_1, \dots, \Phi_n\models \Phi_1,\& \dots,\& \Phi_n$.
            Обратное аналогично.
            \item \begin{itemize}
                \item[$\Rightarrow$]
                    \begin{equation}
                        \forall \alpha \text{- инт.}[(\forall \theta \in \Gamma\quad \theta^\alpha = 1\quad \&\quad \Phi^\alpha = 1)\implies \Psi^\alpha = 1]\tag{*}
                    \end{equation}    
                    пусть $\alpha: (\forall \theta \in \Gamma \quad \theta^\alpha = 1)$  
                    \begin{enumerate}
                        \item $\Phi^\alpha= 1$, тогда из (*) $\Psi^\alpha = 1 (\Phi\to \Psi)^\alpha = 1$
                        \item $\Phi^\alpha = 0\implies (\Phi\to\Psi) = 1\implies$
                    \end{enumerate}     
                    $(\Phi\to \Psi)^\alpha = 1$
                \item[$\Leftarrow$] $\Gamma \models \Phi\to \Psi:$
                
                $\alpha: (\forall \theta \in \Gamma \quad \theta^\alpha = 1)\implies (\Phi\to \Psi)^\alpha = 1$

                Из истинности всех формул из $\Gamma$ следует истинность импликации, а если добавить еще и 
                истинность $\Phi$ при той же интерпретации, то из этого будет следовать истинность посылки, то есть $\Psi$.  
            \end{itemize}
            \item Следует из п.4-п.6
        \end{enumerate}
    \end{proof}
    Проверять логическое следование можно при помощи таблиц истинности и эквивалентных преобразований, пользуясь 7 свойством
    (проверить, является ли импликация тождественно истинной функцией или нет).
    \subsection{Понятия прямой теоремы, а также противоположной, обратной и обратной к противоположной теорем}
    Многие математические теоремы имеют структуру, выражаемую формулой $X\to Y$. 
    Утверждение X называется условием теоремы, а утверждение Y — ее заключением.
    Далее, если некоторая теорема имеет форму $X\to Y$, утверждение $Y\to X$ называется \textbf{обратным} для данной теоремы.
    Это утверждение может быть справедливым, и тогда оно называется теоремой, \textbf{обратной} для теоремы $X\to Y$, которая, 
    в свою очередь, называется \textbf{прямой} теоремой.

    Для теоремы, сформулированной в виде импликации $X\to Y$, кроме обратного утверждения $Y\to X$
    можно сформулировать противоположное утверждение. Им называется утверждение вида $\lnot X\to\lnot Y$.
    Утверждение, противоположное данной теореме, может быть также теоремой, т. е. быть истинным высказыванием,
    но может таковым и не быть. 

    Теорема, обратная противоположной: $\lnot X\to\lnot Y$ (контрапозиция).
    \begin{statement}
        $A\to B \models B' \to A'$

        $A\to B, B\to A \models B' \to A', A' \to B'$

        \begin{enumerate}
            \item Из прямого следует противоположное обратному
            \item Из прямого утверждения в общем случае не следует обратное и противоположное 
            \item Если одновременно истинно и прямое, и обратное, то истинны все четыре
        \end{enumerate}
    \end{statement}
    \begin{example}
        Если формула - ДНФ, то это дизъюнкция.
        Прямое и контрапозиция верны, а противоположное и обратное нет.
    \end{example}
    \subsection{Понятия необходимых и достаточных условий}
    Если некоторая математическая теорема имеет структуру, выражаемую формулой $X\to Y$, то высказывание Y 
    называется \textbf{необходимым} условием для высказывания X (другими словами, если X истинно, то Y с необходимостью должно быть 
    также истинным), а высказывание X называется \textbf{достаточным} условием для высказывания Y (другими словами, для того чтобы Y 
    было истинным, достаточно, чтобы истинным было высказывание X).
    \subsection{Формальные системы. Выводы в формальных системах. Свойства выводов}
    \begin{definition}
        Формальная система состоит из четырех элементов:
        \begin{enumerate}
            \item алфавит (некоторое множество)
            \item набор формул (множество слов, отобранных с помощью некоторых правил)
            \item набор аксиом (множество формул, отобранных по некоторым правилам)
            \item набор правил вывода вида $\frac{\phi_1, \dots, \phi_n}{\Psi}$
            (из формул $\phi_1, \dots, \phi_n$ следует формула $\Psi$)
        \end{enumerate}
    \end{definition}
    \begin{definition}
        Вывод формулы $\phi$ из множества формул $\Gamma$ в формальной системе — это конечная 
        последовательность формул $\phi_1, \dots , \phi_n = \phi$, в которой каждая $\phi_i$
        \begin{itemize}
            \item либо аксиома формальной системы
            \item либо принадлежит множеству $\Gamma$ (является гипотезой)
            \item либо получена из предыдущих формул по одному из правил вывода.
        \end{itemize}
    \end{definition}
    \begin{definition}
        Формула $\phi$ выводитcя из множества формул $\Gamma$ (обозначение:$ \Gamma \vdash \phi$), если существует 
вывод $\phi$ из $\Gamma$.
    \end{definition}
    \begin{statement}[Свойства выводов]
        $ $\\
        \begin{enumerate}
            \item Если $\Gamma \vdash \phi$, то существует конечное подмножество $\Gamma_0 \subseteq \Gamma$ такое, что $\Gamma_0 \vdash \phi$.
            \item Если $\Gamma \vdash \phi$ и $\Gamma \subseteq \Delta$, то $\psi \vdash \Delta$.
            \item (транзитивность выводимости) Если $\Gamma \vdash \Delta$ (т.е. все формулы из $\Delta$ выводятся из $\Gamma$) и 
            $\Delta \vdash \phi$, то и $\Gamma \vdash \phi$.
        \end{enumerate}
    \end{statement}
    \begin{proof}
        $ $\\
        \begin{enumerate}
            \item $\Gamma\vdash \phi: \exists\phi_1, \dots, \phi_n = \phi$.
            Так как вывод конечный, то можно найти конечное множество гипотез, оно и будет $\Gamma_0$
            \item Есть вывод $\Gamma\vdash \phi: \phi_1, \dots, \phi_n = \phi$
            
            Гипотезы $\Gamma\subseteq$ гипотезы из $\Delta\implies\Delta \vdash \phi$
            \item $\Gamma \vdash \Delta, \Delta\vdash\psi$
            
            $\psi_{i1},\dots, \psi_{ik} = \psi_i$ - вывод $\psi_i$ из $\Gamma$ $[\Delta = \bigcup_{i}\psi_i]$

            $\theta_1, \dots, \theta_m = \phi$ - вывод $\Delta\vdash\psi$

            Построим единую последовательность $\psi_{i1},\dots, \psi_{ik}, \theta_1, \dots, \theta_m = \phi$ (проходим по всевозможным
            $\psi_i$)
        \end{enumerate}
    \end{proof}
    \subsection{Исчисление высказываний (ИВ) Гильберта. Примеры выводов}
    \begin{definition}
        Исчисление высказываний - конкретная формальная система на базе логики высказываний.
        \begin{enumerate}
            \item алфавит = символы переменных, отрицание, импликация, скобки
            \item формулы ИВ -  формулы языка ЛВ, использующие только отрицание и импликацию
            \item (схемы аксиом) аксиомы ИВ:
            \begin{itemize}
                \item[$A_1$] $A\rightarrow (B\rightarrow A)$ 
                \item[$A_2$] $(A\rightarrow (B\rightarrow C))\rightarrow ((A\rightarrow B)\rightarrow (A \rightarrow C))$ 
                \item[$A_3$] $(B'\rightarrow A')\rightarrow((B'\rightarrow A)\rightarrow B)$
            \end{itemize}
            \item силлогизм: $\frac{A, A\rightarrow B}{B}$ \textbf{modus ponens}
        \end{enumerate}
    \end{definition}
    \begin{example}
        $A, A\rightarrow B, \vdash B$
        \begin{enumerate}
            \item A
            \item $A\rightarrow B$
            \item B (MP 1, 2)
        \end{enumerate}
    \end{example}
    \begin{example}
        $A \vdash B\rightarrow A$
        \begin{enumerate}
            \item($A_1$) $A\rightarrow (B\rightarrow A)$ 
            \item A
            \item $B\rightarrow A$ (MP 1, 2)
        \end{enumerate}
    \end{example}
    \begin{remark}
        Если $\Gamma=\varnothing$, то пишем $\vdash \phi(\phi$ доказуема)
    \end{remark}
    \subsection{Теорема о дедукции для ИВ}
    \begin{theorem}
        $\Gamma$ - множество формул, A, B - формулы ИВ. Тогда
        $\Gamma, A\vdash B \Leftrightarrow \Gamma, \vdash A\rightarrow B$
    \end{theorem}
    \begin{proof}
        \begin{itemize}
            \item[$\Leftarrow$]$\Gamma \vdash A\rightarrow B$, строим $\Gamma, A\vdash B$
            
            $\Gamma,  A\vdash A, A\rightarrow B$ и $A, A\rightarrow B \vdash B (MP)$,
            По транзитивности получаем требуемое.
            \item[$\Rightarrow$] 
            доказывается индукцией по длине вывода B из $\Gamma$, A.
            \begin{enumerate}
                \item Если этот вывод — длины 1, то B — аксиома или гипотеза. Если B
                — аксиома, то имеем вывод A $\rightarrow$ B ( из $\varnothing$):
                \begin{enumerate}
                    \item B (аксиома)
                    \item B $\rightarrow$ (A $\rightarrow$ B) (аксиома A1)
                    \item A $\rightarrow$ B (1,2, MP)
                \end{enumerate}
                \item Если B $\in$ $\Gamma$, то имеем такой же вывод A $\rightarrow$ B из $\Gamma$:
                \begin{enumerate}
                    \item B (гипотеза)
                    \item B $\rightarrow$ (A $\rightarrow$ B) (аксиома A1)
                    \item A $\rightarrow$ B (1,2, MP)
                \end{enumerate}
                \item Если B = A, то A $\rightarrow$ B = A $\rightarrow$ A. Но $\vdash$ A $\rightarrow$ A:
                    \begin{enumerate}
                        \item ($A_2$) $(A\rightarrow ((A\rightarrow A)\rightarrow A))\rightarrow ((A\rightarrow (A\rightarrow A))\rightarrow (A \rightarrow A))$ 
                        \item ($A_1$) $A\rightarrow ((A\rightarrow A)\rightarrow A)$ 
                        \item (MP 1, 2) $(A\rightarrow(A\rightarrow A))\rightarrow(A\rightarrow A)$
                        \item ($A_1$) $A\rightarrow (A\rightarrow A)$ 
                        \item (MP 3, 4) $A\rightarrow A$
                    \end{enumerate}
                \item Предположим теперь, что $\Gamma$, A $\vdash$ B и утверждение ($\Rightarrow$) верно для
                всех более коротких выводов, т.е.
                для всех C, если $\Gamma$, A $\vdash$ C и вывод C из $\Gamma$, A короче, чем вывод B, то
                $\Gamma$ $\vdash$ A $\rightarrow$ C.
            \end{enumerate}
            Докажем, что $\Gamma$ $\vdash$ A $\rightarrow$ B.
            
            Рассмотрим вывод из $\Gamma$, A, который заканчивается формулой B. При
            этом B может оказаться аксиомой или гипотезой (тогда все предыдущие
            формулы для доказательства B не нужны). Но в этом случае $\Gamma$ $\vdash$ A $\rightarrow$ B
            по (1)-(3).
            Остается случай, когда B получается по MP из формул C, C $\rightarrow$ B, причем $\Gamma$, A $\vdash$ C и $\Gamma$, A $\vdash$ C $\rightarrow$ B с более короткими доказательствами. По
            предположению индукции имеем
            
            (*) $\Gamma$ $\vdash$ A $\rightarrow$ C, A $\rightarrow$ (C $\rightarrow$ B).
            С другой стороны,
            (**) A $\rightarrow$ C, A $\rightarrow$ (C $\rightarrow$ B) $\vdash$ A $\rightarrow$ B:
            \begin{enumerate}
                \item A $\rightarrow$ C (гипотеза)
                \item A $\rightarrow$ (C $\rightarrow$ B) (гипотеза)
                \item (A $\rightarrow$ (C $\rightarrow$ B)) $\rightarrow$ ((A $\rightarrow$ C) $\rightarrow$ (A $\rightarrow$ B)) (аксиома A2)
                \item (A $\rightarrow$ C) $\rightarrow$ (A $\rightarrow$ B) (2,3, MP)
                \item A $\rightarrow$ B (1,4, MP)
            \end{enumerate}
            Из (*), (**) по транзитивности получаем $\Gamma$ $\vdash$ A $\rightarrow$ B.
        \end{itemize}
    \end{proof}
    \subsection{Теорема о полноте и непротиворечивости ИВ}
	\begin{theorem}[Теорема о полноте ИВ]
		$\vdash A$ тогда и только тогда, когда A - тавтология
	\end{theorem}
	\begin{proof}
		=> Если формула А выводится из аксиом($\vdash$), то A является тавтологией
		<= Пусть A - тавтология. Тогда $\bar{A}$ - тождественно ложная. Докажем что множество Г = $\{\bar{A}\}$ является противоречивым. Если оно непротиворечиво , то по лемме 7.6 существует полное непротиворечивое множество $\bar{Г} \supset Г$. По лемме 7.7 для множества $\bar{Г}$ существует набор значений переменных $\bar{a}$, на котором все формулы из $\bar{Г}$ (в том числе и $\neg A$) принимают значение 1. Мы приходим к противоречию, поскольку формула $\neg A$ тождественно ложна. В соответсвии с теоремой ИВ Гильберта из множества Г выводится абсолютно любая формула в том числе и формула А.

		$\neq A \vdash A$. По теорема дедукции это означает, что формула $\neg A \rightarrow A$ выводится из аксиом ИВ Гильберта, то есть $\vdash \neg A \rightarrow A$. Напишем такой вывод.
		\begin{enumerate}
			\item $(\neg A \rightarrow A) \rightarrow ((\neg\neg A \rightarrow A) \rightarrow A)$
			\item $\neg A \rightarrow A$
			\item $(\neg\neg A \rightarrow A) \rightarrow A$ MP(1, 2)
			\item $\neg\neg A \rightarrow A$
			\item A MP(3, 4)
		\end{enumerate}

		Построенный вывод показывает, что формула А является теоремой ИВ Гильберта, то есть $\vdash A$. 
	\end{proof}
	\begin{remark}
		Лемма 7.6 Для каждого непротиворечивого мноэества формул Г существует полное непротиворечивое множество $Г' \supset Г$
	\end{remark}
	\begin{remark}
		Лемма 7.7 Для любого непротиворечивого полного множества формул Г сущесвует набор переменных, на которых все формулы множества Г истинны.
	\end{remark}
    \subsection{ИВ Генцена, его полнота}
    \begin{enumerate}
        \item Алфавит: $\{ x_1,\dots, x_n, \&, \vee, \to, ', (, ), \vdash, "," \} $
        \item Используем слова двух видов:
            \begin{enumerate}
                \item Формулы - формулы логики высказываний
                \item Секвенции - слова вида $\Gamma\vdash \Delta$, где $\Gamma, \Delta$ - множества формул
            \end{enumerate}
            Из всех формул $\Gamma$ вместе следует хотя бы одна формула из $\Delta$ ($\&\Gamma \to \vee\Delta$)
        \item Аксиомы: секвенции
        
        $\Gamma, \phi \vdash \Delta, \phi$
        \item Правила вывода:
        \begin{itemize}
            \item[$\vdash\;\&$] $\frac{\Gamma \vdash \phi, \Delta \quad \Gamma \vdash \psi, \Delta}{\Gamma \vdash \phi\&\psi, \Delta}$
            \item[$\vdash\;\vee$] $\frac{\Gamma \vdash \phi, \psi, \Delta}{\Gamma \vdash \phi\vee\psi, \Delta}$
            \item[$\vdash\;\to$] $\frac{\Gamma , \phi \vdash  \Delta \quad \Gamma \vdash \psi, \Delta}{\Gamma \vdash \phi\to\psi, \Delta}$
            \item[$\vdash\;'$] $\frac{\Gamma, \phi\vdash\Delta}{\Gamma \vdash \phi', \Delta}$
            \item[$\&\;\vdash$] $\frac{\Gamma, \phi, \psi \vdash\Delta}{\Gamma, \phi\&\psi \vdash\Delta}$
            \item[$\vee\;\vdash$] $\frac{\Gamma , \phi \vdash  \Delta \quad \Gamma, \psi \vdash  \Delta}{\Gamma,\phi\vee\psi \vdash  \Delta}$
            \item[$\to\;\vdash$] $\frac{\Gamma \vdash\phi ,  \Delta \quad \Gamma, \psi \vdash  \Delta}{\Gamma,\phi\to\psi \vdash  \Delta}$
            \item[$'\;\vdash$] $\frac{\Gamma \vdash\phi ,  \Delta }{\Gamma,\phi' \vdash  \Delta}$
        \end{itemize}
    \end{enumerate}
    \begin{definition}
        Секвенция $\Gamma\vdash \Delta$ доказуема, если существует конечная последовательность секвенций
        $\Gamma_1\vdash \Delta_1, \dots, \Gamma_n\vdash \Delta_n\vdash \Gamma\vdash \Delta$, в которой каждая секвенция:
        \begin{itemize}
            \item либо аксиома;
            \item либо получена из предыдущих по одному из правил вывода.
        \end{itemize}
    \end{definition}
    \begin{remark}[Алгоритм поиска контрпримера к секвенции]
        \begin{enumerate}
            \item  Взять исходную секвенцию $\Gamma\vdash \Delta$ и разместить её в корне дерева.
            \item С помощью правил вывода ИВ Генцена добавлять в дерево новые вершины.
            Правила вывода нужно применять верх ногами, то есть по имеющейся секвенции выписать секвенции, которые находятся в верхней строке правил вывода
            ИВ Генцена.
            \item Процесс построения дерева завершается, когда во всех его листьях строят секвенции без логических операций.
            \item Если во всех листьях дерева строят аксиомы ИВ Генцена, то исходная секвенция $\Gamma\vdash \Delta$ не имеет контрпримера. Иначе, у секвенции $\Gamma\vdash \Delta$ существует
            контрпример.
        \end{enumerate}
    \end{remark}
    \begin{example}
        $x\to y\vdash x' \vee y$

        Строим вывод:
        \begin{itemize}
            \item [$\to\;\vdash$] $\frac{x\vdash x, y\quad y, x\vdash y}{x\to y, x \vdash y}$
            \item [$\vdash\; '$] $\frac{x\to y, x \vdash y}{x\to y\vdash x', y}$
            \item [$\vdash\;\vee$]$\frac{x\to y\vdash x', y}{x\to y \vdash x' \vee y}$
        \end{itemize}
    \end{example}
    \begin{definition}
        $\Gamma = \{ \phi_1, \dots, \phi_n\}, \Delta = \{ \psi_1, \dots, \psi_m\} $. Секвенция
        $\Gamma \vdash \Delta$ тождественно истинна, если тождественно истинна формула
        $(\phi_1, \& \dots, \&\phi_n) \to (\psi_1\vee \dots \vee\psi_m)$
    \end{definition}
    \begin{theorem}
        Секвенция $\Gamma \vdash \Delta$ доказуема $\Leftrightarrow$ $\Gamma \vdash \Delta$ тождественно истинна
    \end{theorem}
    \begin{itemize}
        \item[$\Rightarrow$] (Корректность ИВ Генцена)
        $\Gamma \vdash \Delta$ доказуема $\Leftrightarrow$ есть вывод
        $\Gamma_1\vdash \Delta_1, \dots, \Gamma_n\vdash \Delta_n\vdash \Gamma\vdash \Delta$
        Индукция по номеру секвенции:
        докажем, что все секвенции в выводе тождественно истинны:

        Основание: k = 1.

        $\Gamma_1\vdash \Delta_1$ аксиома, т.е имеет вид $\Gamma, \Psi\vdash \widetilde{\Delta}, \Psi $ - 
        она тожд. истинная =>
        $\& \Gamma_1 \to \vee \Delta_1 \sim \phi \& \& \Gamma_1 \to (\phi\vee \vee\Delta_1)$

        Шаг индукции. Докажем для $\Gamma_k\vdash \Delta_k$, считая, что для всех предыдущих секвенций все доказано
        \begin{enumerate}
            \item $\Gamma_k\vdash \Delta_k$ аксиома - все аксиомы тожд. истинны (как и раньше)
            \item Если секвенция получена по правилу вывода из секвенций $\Gamma_i\vdash \Delta_i, \Gamma_j\vdash \Delta_j, i, j < k$
        \end{enumerate}
        По инд. допущению, они тожд. истинны.

        Осталось доказать, что любое правило вывода из тождественно истинных секвенций даст тождественно истинную секвенцию.
        Это делается перебором правил:
        [$\vdash\;\&$] $\frac{\Gamma \vdash \phi, \Delta \quad \Gamma \vdash \psi, \Delta}{\Gamma \vdash \phi\&\psi, \Delta}$

        $\Gamma \vdash \phi, \Delta$ тождественно истинны <=> $\& \Gamma \to (\phi\vee \vee\Delta)\sim 1$

        $\Gamma \vdash \psi, \Delta$ тождественно истинны <=> $\& \Gamma \to (\psi\vee \vee\Delta)\sim 1$

        \begin{enumerate}
            \item $\& \Gamma \to \vee \Delta \sim 1$ => $\& \Gamma \to (\psi\&\phi\vee\vee\Delta)\sim 1$
            \item $\& \Gamma \to \vee \Delta $не $\sim 1$
            
            $\& \Gamma \to \phi \sim 1$ и $\& \Gamma \to \psi \sim 1 => \& \Gamma \to (\psi\&\phi)\sim 1$ и поэтому $\& \Gamma \to (\psi\&\phi\vee \vee\Delta)\sim 1$
        \end{enumerate}
        Остальные правила вывода аналогично.
        \item[$\Leftarrow$] (Полнота ИВ Генцена)
        $\Gamma \vdash \Delta$ - тождественно истинна.

        Заметим, что во всех правилах верхняя секвенция содержит на одну связку меньше, чем нижняя.

        Пусть в $\Delta$ есть формула со связкой, например, $\Phi\vee \Psi$. По правилу получим:

        $\frac{\Gamma, \phi, \psi\vdash \widetilde{\Delta} }{\Gamma \vdash \phi\vee\psi,  \widetilde{\Delta}}, \phi\vee\psi, \widetilde{\Delta} = \Delta$

        Действуя аналогично, уберем в формулах из $\Delta $ все логические связки, уберем и в $\Gamma$

        Получим набор секвенций $\Gamma \vdash \Delta$ в которых $\Gamma$ и $\Delta$ состоят только из переменных.
        \begin{enumerate}
            \item Пусть есть переменная $x \in \Gamma \cap \Delta=>$ это аксиома
            \item нет $x \in \Gamma \cap \Delta=\varnothing$, пусть
            $\Gamma = \{ y_1, \dots, y_k \}, \Delta = \{ z_1, \dots, z_n \}$

            положим $y_1 = \dots = y_k = 1, z_1, = \dots = z_n = 0$, при этой интерпретации $\&\Gamma \to \vee\Delta^\alpha = 0$
        \end{enumerate}
        Перебирая правила, докажем, что при любой интепретации $\alpha$, если одна из 
        секвенций $\Gamma_1 \vdash \Delta_1$ $\Gamma_2 \vdash \Delta_2$ ложна, то и результат тоже ложь

        [$\vdash\;\&$] $\frac{\Gamma \vdash \phi, \Delta \quad \Gamma \vdash \psi, \Delta}{\Gamma \vdash \phi\&\psi, \Delta}$

        $\alpha$ - интерпретация $(\& \Gamma \to (\phi \vee \vee \Delta ))^\alpha = 0\Leftrightarrow (\& \Gamma)^\alpha = 1$

        $(\phi\vee \vee \Delta)^\alpha = 0$$\phi^\alpha = 0$ и $(\& \Delta)^\alpha = 0$, тогда 

        $(\& \Gamma \to (\phi\&\psi \vee \vee\Delta))^\alpha = 0 [\& \Gamma=1, \phi\&\psi = 0, \bigvee\Delta = 0]$ 

        Спускаясь вниз к исходной секвенции, получаем что она ложна => противоречие.
    \end{itemize}
    \subsection{Метод резолюций для логики высказываний (без обоснования корректности)}
    \section{Логика предикатов}
    \subsection{Понятие предиката и операции, их представления, примеры}
    \begin{definition}
        n-местный предикат на множестве A - это отображение вида $P: A^n \rightarrow \{0, 1\}$ 
    \end{definition}
    \begin{definition}
        n-местная операция на множестве A - это отображение вида $f: A^n \rightarrow A$ 
    \end{definition}
    Предикат можно задать как множество тех аргументов, на которых он является истинным
    \begin{example}
        $P = \{1, 3\} : P = 1 \Leftrightarrow x = 1 \vee x = 3$
    \end{example}
    \begin{example}
        $Q = \{(1, 2), (3, 4), (5, 6)\}$
    \end{example}
    Способы задания:
    \begin{enumerate}
        \item описательный
        \item множество (отношения)
        \item таблица (истинности)
        \item графы 
        
        для предиката $P(x, y)$ ребро $(x, y)$ обозначает $P(x,y) = 1$

        для операции $f(x)$ дуга $(x, y)$ обозначает $y = f(x)$
    \end{enumerate}
    \subsection{Сигнатура, интерпретация сигнатуры на множестве, алгебраические системы}
    \begin{definition}
        Сигнатура - набор предикатных, функциональных и константных символов с указанием местностей
    \end{definition}
    \begin{example}
    $\sigma = \{P^{(1)}, Q^{(2)}, f^{(1)}, g^{(2)}, c\}$
    \end{example}
    \begin{definition}
        Две сигнатуры считаем \textit{равными}, если в них одинаковое кол-во символов каждого
        сорта и местности соответствующих символов равны
    \end{definition}
    \begin{definition}
        Интерпретация сигнатуры $\sigma$ на множестве А - это отображение, которое
        \begin{enumerate}
            \item каждому n-местному предикатному символу $P^{(n)}\in \sigma$ сопоставляет n-местный предикат 
            на А
            \item каждому n-местному функциональному символу  $f^{(n)}\in \sigma$ сопоставляет n-местную операцию на А
            \item каждому константному символу сопоставляет элемент множества А
        \end{enumerate}
    \end{definition}
    \begin{definition}
        Алгебраическая система - набор, состоящий из множества А, сигнатуры $\sigma$ и интерпретации $\sigma$ на А. 
        Множество А называют основным множеством системы ($\mathfrak{a} = <A, \sigma>$)
    \end{definition}
    \subsection{Язык логики предикатов, термы, формулы логики предикатов}
    Зафиксируем сигнатуру $\sigma$. Алфавит логики предикатов сигнатуры $\sigma$ — это множество

    $\sigma_{A\text{ЛП}} = \sigma \cup \{x_1, x_2\dots, \&, \vee, \rightarrow, \leftrightarrow, \neg, \forall, \exists, (, ), =, \textbf{,} \}$
    \begin{definition}
        Терм - слово алфавита логики предикатов, построенное по правилам:
        \begin{enumerate}
            \item символ переменной - терм
            \item константный символ - терм
            \item если $t_1,\dots t_n$ - термы, $f^{(n)}\in\sigma$, то и $f(t_1,\dots, t_n)$ - терм
        \end{enumerate}
    \end{definition}
    \begin{definition}
        Атомарная формула сигнатуры $\sigma$ - это слово одного из двух видов:
        \begin{enumerate}
            \item $t_1 = t_2$, где $t_1, t_2$ - термы
            \item предикат $P(t_1,\dots, t_n), P^{(n)}\in\sigma, t_1,\dots t_n$ - термы 
        \end{enumerate}
    \end{definition}
    \begin{definition}
        Формула ЛП сигнатуры $\sigma$ - слово, построенное по правилам:
        \begin{enumerate}
            \item атомарная формула - формула
            \item если $\phi_1$ и $\phi_2$ -  формулы, то слова $(\phi_1 \& \phi_2), $
            $(\phi_1 \vee \phi_2), (\phi_1 \leftrightarrow \phi_2), (\phi_1 \rightarrow \phi_2), \neg \phi_1$
            тоже формулы
            \item если $\phi$ - формула, то слова $(\forall x \phi)$ и $(\exists x \phi)$ тоже формулы
        \end{enumerate}
    \end{definition}
    \subsection{Свободные и связанные переменные. Замкнутые формулы}
    \begin{definition}
        Вхождение переменной х в формулу $\phi$ \textbf{связанное}, если х попадает в область действия квантора $\exists x / \forall x$,
        в противном случае вхождение х \textbf{свободное}
    \end{definition}
    \begin{definition}
        Переменная х \textbf{свободна} в формуле $\phi$, если есть хотя бы одно свободное вхождение х в $\phi$,
        в противном случае она \textbf{связанная}
    \end{definition}
    \begin{definition}
        Формула замкнутая, если она не содержит свободных переменных.
    \end{definition}
    \subsection{Истинность формул на алгебраической системе}
    \begin{definition}
        Множество истинности формулы $\phi$ в алгебраической системе $\mathfrak{a}$ - это
        $A_\phi = \{(a_1, \dots, a_n) | a_i \in A, \mathfrak{a}\models \phi(a_1, \dots, a_n)\} $
    \end{definition}
    \begin{definition}
        Множество $B\subseteq A^n$ выразимо в алгебраической системе $\mathfrak{a}$, если 
        $\exists$ формула $\phi$ такая , что $A_\phi = B$

        ИЛИ ПО ШЕВЛЯКОВУ
        
        Предикат $Q(x1, \dots , x_n)$ называется выразимым на АС 
        $\mathfrak{A} =<A, \sigma>$ сигнатуры $\sigma$, если существует формула $\phi(x1, \dots , xn)$ сигнатуры $\sigma$ со свободными переменными 
        $x1, \dots , x_n$ такая, что
        $\mathfrak{A} \models \phi(a_1, \dots , a_n) \Leftrightarrow Q(a_1, \dots , a_n).$
    \end{definition}
    \begin{definition}
        Функция $f:A^n \to A$ выразима, если выразимо множество $\Gamma_f = \{ (a_1, \dots, a_n, b) | a_i, b \in A, b = f(a_1, \dots, a_n)\} $
    \end{definition}
    \begin{definition}
        Предикат $P:A^n \to \{ 0, 1\} $ выразим, если выразимо его множество истинности. !!!
    \end{definition}
    Каждый терм $t(x_1,\dots, x_n)$ определяет в системе $\mathfrak{a}$ функцию $t_{\mathfrak{a}}: A^n \rightarrow A$
    следующим образом: в терме все функциональные и 
    константные символы заменяются на их интерпретации в системе A, после чего 
    вычисляется полученная суперпозиция от входных аргументов.
    
    Пусть также $\phi(x_1 \dots, x_n)$ — формула со свободными переменными $x_1, \dots, x_n$. Определим 
    понятие истинности формулы $\phi$ на наборе элементов $a_1, \dots a_n \in \mathfrak{a}$ в алгебраической 
    системе $\mathfrak{a}$ (обозначение: $\mathfrak{a} \models \phi(a_1, \dots a_n))$ следующим образом.
    \begin{definition}
        \begin{enumerate}
            \item Пусть $\phi$ имеет вид $t_1 = t_2$. Тогда $A \models \phi(a_1, \dots a_n) \Leftrightarrow t_{1A}(a_1, \dots a_n) = t_{2A}(a_1, \dots a_n)$ (здесь $t_{iA}$ —
            функция, определяемая термом $t_i$ в системе A).
            \item Пусть $\phi$ имеет вид $P(t_1,\dots, t_k)$. Тогда 
            $A \models \phi(a_1, \dots a_n) \Leftrightarrow P_A(t_{1A}(a_1, \dots a_n), \dots, t_{kA}(a_1, \dots a_n)) = 1$, 
            где $P_A$ — интерпретация предикатного символа P в системе A.
            \item Пусть $\phi$ имеет вид $(\phi_1 \& \phi_2), (\phi_1 \vee \phi_2), (\phi_1 \rightarrow \phi_2), (\phi_1 \leftrightarrow \phi_2), \neg\phi_1$. Тогда истинность формулы 
            $\phi$ определяется по значениям $\phi_1(a_1, \dots a_n)$ и $\phi_2(a_1, \dots a_n)$ по таблицам истинности логических 
            связок.
            \item Пусть $\phi(x_1 \dots, x_n)$ имеет вид $(\forall x \phi(x, x_1, \dots x_n))$. Тогда $A \models \phi(a_1, \dots a_n) \Leftrightarrow$ для всех элементов 
            $b \in A$ выполнено $A \models \phi (b, a_1, \dots a_n)$.
            \item Пусть $\phi(x_1 \dots, x_n)$ имеет вид $(\exists x \phi(x, x_1, \dots x_n))$. Тогда $A \models \phi(a_1, \dots a_n) \Leftrightarrow$ для некоторого 
            элемента $b \in A$ выполнено $A \models \phi (b, a_1, \dots a_n)$.
        \end{enumerate}
    \end{definition}
    \subsection{Изоморфизм систем. Теорема о сохранении значений термов и формул в изоморфных системах. Автоморфизм}
    \begin{definition}
        АС $\mathfrak{a} = <A, \sigma>, \mathfrak{b} = <B, \sigma>$ сигнатуры $\sigma$ изоморфны, если существует отображение
        $F : A \to B$ со свойствами:
        1. F - биекция между основными множествами A и B;
        2. $F(c_A) = c_B$ , где $c_A, c_B$ интерпретации константного символа c $\in$ $\sigma$ в АС
        $\mathfrak{a}$ и $\mathfrak{b}$ соответственно (биекция F должна переводить константы одной АС в
        константы другой АС);
        3. $F(f_A(x_1, \dots , x_n)) = f_B(F(x_1), \dots F(x_n))$, где $f_A, f_B$ интерпретации функционального символа 
        $f \in \sigma$ в АС $\mathfrak{a}$ и $\mathfrak{b}$ соответственно (говорят, что биекция
        F сохраняет значение функции f);
        4. $P_A(x1, \dots , xn) = 1 \Leftrightarrow P_B(F(x1), \dots , F(xn)) = 1$, где $P_A, P_B$ интерпретации предикатного 
        символа $P \in \sigma$ в АС $\mathfrak{a}$ и $\mathfrak{b}$ соответственно (то есть биекция F отображает отображает область 
        истинности предиката $P_A$ на область истинности предиката $P_B$).
    \end{definition}
    \begin{definition}
        Алгебраические системы A и B изоморфны (обозначение: A $\cong$ B), если существует 
        изоморфизм A на B.
    \end{definition}
    \begin{statement}
        Отношение изоморфизма есть отношение эквивалентности.
    \end{statement}
    \begin{definition}
        Автоморфизм - изоморфизм алгебраической системы самой на себя.
    \end{definition}
    \begin{theorem}[Теорема о сохранении значений термов и формул в изоморфных системах]
        $\alpha(x)$ - изоморфизм, $\mathfrak{a} = <A, \sigma>$, на $\mathfrak{b} = <B, \sigma>$

        Тогда:
        \begin{enumerate}
            \item Для любого терма $t(x_1, \dots, x_n)$ сигнатуры $\sigma$
            
            $\forall a_1, \dots, a_n \in A: \alpha (t_\mathfrak{a}(a_1, \dots, a_n))= t_\mathfrak{b}(\alpha(a_1), \dots, \alpha(a_n))$

            \item Для любой формулы $\phi(x_1, \dots, x_n)$ сигнатуры $\sigma$
            
            $\forall a_1, \dots, a_n \in A: \mathfrak{a}\models \phi(a_1, \dots, a_n)\Leftrightarrow \mathfrak{b} \models \phi(\alpha(a_1), \dots, \alpha(a_n))$
        \end{enumerate}
    \end{theorem}
    \begin{proof}
        Индукция по построению термов. Основание: const/переменная

        \begin{enumerate}
            \item $t(x_1, \dots, x_n) = x_i\implies \alpha (t_A(a_1, \dots, a_n)) = \alpha(a_i) = t_B (\alpha(a_1), \dots, \alpha(a_n))$
            \item $t(x_1, \dots, x_n) = c \implies \alpha(t_A(a_1, \dots, a_n))= [t_A(a_1, \dots, a_n) = c_A] = c_B = t_B(\alpha(a_1), \dots, \alpha(a_n))$
        \end{enumerate}

        Шаг индукции

        Пусть утверждение теоремы доказано для термов $t_1(x_1, \dots, x_n), \dots, t_k (x_1, \dots, x_n)$
    \end{proof}
    \subsection{Элементарная теория алгебраической системы. Элементарная эквивалентность систем. Связь понятий изоморфизма и элементарной эквивалентности}
    \begin{definition}
        Пусть $\mathfrak{A}$ АС сигнатуры $\sigma$. Множество всех замкнутых формул
        сигнатуры $\sigma$, истинных на A называется теорией АС $\mathfrak{A}$ и обозначается Th($\mathfrak{A}$).
        Более формально,
        $Th(\mathfrak{a}) = \{\phi|\mathfrak{a} \models \phi\}$
    \end{definition}
    \begin{definition}
        AC $\mathfrak{a} = <A, \sigma>, \mathfrak{b} = <B, \sigma>$ элементарно эквивалентны, если
        $Th(\mathfrak{a}) = Th(\mathfrak{b})$. Обозначается $\mathfrak{a}\equiv \mathfrak{b}$
    \end{definition}
    \begin{theorem}
        Если две алгебраические системы изоморфны, то они элементарно эквивалентны
    \end{theorem}
    \begin{proof}[$A\cong B => A\equiv B$]
        $\mathfrak{a}, \mathfrak{b}$ изоморфны, берем произвольную замкнутую формулу, по теореме о сохранении изоморфизмом значений термов и формул

        $(\mathfrak{a}\models \phi \Leftrightarrow \mathfrak{b}\models \phi) \Leftrightarrow \mathfrak{a} \equiv \mathfrak{b}$
    \end{proof}
    \begin{remark}
        Обратное не верно в общем случае.
        
        $\sigma = \{P^{(2)}\}, \mathfrak{A} = <\mathbb{Q}, \sigma>, \mathfrak{B} = <\mathbb{R}, \sigma>$
        $P_A(x, y) = P_B(x, y) = \{x < y\}$

        Их элементарные теории совпадают, однако они не изоморфны ($|\mathbb{Q} |\neq |\mathbb{R} |$)

        Однако для конечных множеств выполняется следующее:
    \end{remark}
    \begin{theorem}
        Конечные АС изоморфны $\Leftrightarrow$ элементарно эквивалентны.
    \end{theorem}
    \begin{proof} [$A\cong B <= A\equiv B$]
        Построим формулу, которая кодирует операции, предикаты и константы на $\mathfrak{A} : \phi_\mathfrak{A}$
        
        $\sigma = P\cup f\cup c, |a| = n, x_1, \dots, x_n - $ пронумерованные элементы A
        
        $\phi_A = \exists x_1, \dots, \exists x_n (\neg(x_1 = x_2) \& \neg (x_1 = x_3) \dots \& \neg (x_n = x_{n-1})) \&$ [равенство / неравенство элементов A]

        $P_1(x_i), \dots P_1(x_j) \&$ [множество истинности всех предикатов]

        $(f(x_l) = x_r) \dots\&$ [значения для операций]

        $(c = x_v)\dots\&$ [значения для констант]

        $\forall x[(x = x_1) \vee \dots \vee (x = x_n)])$ [$\forall x$ зависит от кванторов существования]

        Так как $\mathfrak{A}\models \phi_A$, то из элементарной эквивалентности следует что и $\mathfrak{B} \models \phi_A$

        Это означает, что B состоит из того же кол-ва элементов, функции, предикаты, константы устроены точно так же, как и на A, поэтому они изоморфны.
    \end{proof}
    \begin{remark}
        Это док-во показывает, почему для бесконечных АС теорема не верна. Дело в том, чтобы описать бесконечное множество
        необходимо бесконечное количество переменных, а формула - конечное выражение.
    \end{remark}

    Чтобы определить, что АС элементарно не эквивалентны, необходимо сформулировать свойство, которое верно для одной АС, и ложно в другой,
    и записать свойство в виде замкнутой формулы сигнатуры.
    \subsection{Выразимость свойств в логике предикатов. Умение записать формулой различные свойства систем и элементов систем}
    \subsection{Эквивалентность формул логики предикатов}
    \begin{definition}
        Формулы $\phi(x_1, \dots, x_n)$ и $\psi(x_1, \dots, x_n)$ сигнатуры $\sigma$ эквивалентны
        в алгебраической системе $\mathfrak{a} = <A, \sigma>$ ($\phi \sim_\mathfrak{a} \psi$), если
        \begin{align*}
            \forall a_1, \dots, a_n\in A\quad\mathfrak{a}\models \psi(a_1, \dots, a_n)
        \end{align*}
    \end{definition}
    \begin{definition}
        Формулы $\phi(x_1, \dots, x_n)$ и $\psi(x_1, \dots, x_n)$ сигнатуры $\sigma$ эквивалентны
        ($\phi \sim  \psi$), если
        \begin{align*}
            \forall \mathfrak{a} = <A, \sigma> (\phi \sim_\mathfrak{a} \psi)
        \end{align*}
    \end{definition}
    \subsection{Тождественно истинные (ложные) и выполнимые формулы}
    \begin{definition}
        Формула $\phi(x_1, \dots, x_n)$ сигнатуры $\sigma$ тождественно истинная (ложна) в алгебраической 
        системе $A = <A, \sigma>$, если для всех наборов элементов $a_1\dots a_n \in A$ выполнено 
        $A \models \phi(a_1\dots a_n) (A \not\models \phi(a_1\dots a_n)).$
    \end{definition}
    \begin{definition}
        Формула $\phi(x_1, \dots, x_n)$ выполнима в алгебраической системе 
        $A = <A, \sigma>$, если для хотя бы одного набора элементов $a_1\dots a_n \in A $ выполнено 
        $A \models \phi(a_1\dots a_n)$.
    \end{definition}
    \begin{definition}
        Формула $\phi$ сигнатуры $\sigma $ тождественно истинная (ложна), если $\phi$ тождественно истинна 
        (ложна) во всех алгебраических системах сигнатуры $\sigma$. 
    \end{definition}
    \begin{definition}
        Формула $\phi$ сигнатуры $\sigma$ выполнима, 
        если $\phi$ выполнима хотя бы в одной алгебраической системе сигнатуры $\sigma$. 
    \end{definition}
    \subsection{Пренексный вид формулы}
    \begin{definition}
        Формула $\phi$ находится в пренексном виде, если она 
        \begin{itemize}
            \item либо не содержит кванторов (бескванторная)
            \item либо имеет вид $Q_1 x_1\dots Q_n x_n \psi$, где $Q_i$ - кванторы, а формула $\psi$ бескванторная.
        \end{itemize}
    \end{definition}
    \begin{theorem}[о приведении формулы логики предикатов в пренексный вид]
        Любая формула логики предикатов может быть преобразована в эквивалентную формулу в пренексном виде.
    \end{theorem}
    \begin{proof}
        На основании предложения о эквивалентностях логики высказываний 
        выразим все связки через $\&, \vee, \neg$.
        Получим эквивалентную формулу $\Psi$.

        Индукция по построению формулы $\Psi$.

        Основание - $\psi$ - бескванторная, то есть уже в ПНФ.

        Предположение индукции: допустим, теорема доказана для формул с
        $\leq k$ логическими знаками и кванторами.
        Шаг индукции: докажем теорему для формул с k + 1 логическими знаками и кванторами.
        Рассмотрим последний логический знак или квантор, входящий в формулу:
        \begin{enumerate}
            \item$ A = \neg A_1$ ,
            \item $A = A_1 \vee A_2$ ,
            \item $A = A_1 \& A_2$ ,
            \item $A = A_1 \to A_2$ ,
            \item $A = \exists xA_1(x)$,
            \item $A = \forall xA_1(x)$,
        \end{enumerate}
        причем формулы $A_1, A_2$ содержат $\leq k$ логических знака и квантора и для
        них теорема доказана. Значит, для них существуют эквивалентные формулы,
        находящиеся в пренексной нормальной форме. Обозначим их через $B_1, B_2$ :
        $A_1 \sim B_1$ и $A_2 \sim B_2$ . Можно считать, что связанные переменные, 
        входящие в формулу $B_1$ , не совпадают со связанными переменными, входящими
        в формулу $B_2$ (иначе их можно переименовать).

        Пусть $B_1, B_2$ имеют вид:

        $B_1 = Q_1y_1Q_2y_2 \dots Q_ny_nC_1(y_1, y_2, \dots , y_n, u_1, u2, \dots , u_{l_1}),$

        $B_2 = R_1z_1R_2z_2 \dots R_mz_mC_2(z_1, z_2, \dots , z_m, v_1, v_2, \dots , v_{l_2}),$

        где $C_1(y_1, y_2, \dots , y_n, u_1, u_2, \dots , u_{l_1}), C_2(z_1, z_2, \dots , z_m, v_1, v_2, \dots , v_{l_2})$

        - формулы, не содержащие кванторов. Чтобы не загромождать запись, будем писать
        просто $C_1, C_2$ , не указывая переменные.

        В каждом из 6 случаев построим формулу, эквивалентную A и 
        находящуюся в пренексной нормальной форме, используя эквивалентности 
        логики предикатов. Последняя формула в цепочке эквивалентностей находится в
        пренексной нормальной форме.
        \begin{enumerate}
            \item 
            $A = \neg A_1 \sim \neg B_1 \sim Q_1'y1 Q_2' y_2 \dots Q_n' y_n \neg C_1,$ где
            \begin{equation*}
                Q_i' = 
                \begin{cases}
                    \exists, \text{если} Q_i = \forall, \\
                    \forall, \text{если} Q_i = \exists
                \end{cases}
            \end{equation*}
            \item 
            $A = A_1 \vee A_2 \sim B_1 \vee B_2 = Q_1 y_1 Q_2 y_2, \dots, Q_n y_n C_1 \vee R_1z_1R_2z_2\dots R_mz_m C_2$

            $\sim Q_1 y_1 Q_2 y_2, \dots, Q_n y_n R_1z_1R_2z_2\dots R_mz_m (C_1 \vee C_2)$
            \item
            $A = A_1 \& A_2 \sim B_1 \& B_2 = Q_1 y_1 Q_2 y_2, \dots, Q_n y_n C_1 \& R_1z_1R_2z_2\dots R_mz_m C_2$

            $\sim Q_1 y_1 Q_2 y_2, \dots, Q_n y_n R_1z_1R_2z_2\dots R_mz_m (C_1 \& C_2)$
            \item
            $A = A_1 \to A_2 \sim B_1 \to B_2 = Q_1 y_1 Q_2 y_2, \dots, Q_n y_n C_1 \to R_1z_1R_2z_2\dots R_mz_m C_2$

            $\sim Q_1 y_1 Q_2 y_2, \dots, Q_n y_n R_1z_1R_2z_2\dots R_mz_m (C_1 \to C_2)$
            \item
            $A = \exists x A_1(x) \sim \exists x B_1(x) \sim \exists x Q_1 y_1 Q_2 y_2, \dots, Q_n y_n C_1$
            \item 
            $A = \forall x A_1(x) \sim \forall x B_1(x) \sim \forall x Q_1 y_1 Q_2 y_2, \dots, Q_n y_n C_1$
        \end{enumerate}
    \end{proof}

    Алгоритм приведения ф. ЛП в ПНФ:
    \begin{enumerate}
        \item Выразить все связки через $\&, \vee, '$
        \item Переименовать все связанные переменные так, чтобы они отличались друг от друга и от связанных переменных
        \item Действуя от внутренних подформул к внешним, выносим кванторы влево.
    \end{enumerate}
    (нельзя переименовывать свободную формулу)
    \subsection{Основные эквивалентности логики предикатов}
    \begin{statement}[Об эквивалентностях ЛВ]
        Пусть $\phi(x_1, \dots, x_n), \psi(x_1, \dots, x_n)$ от булевых переменных, $\phi_1, \dots, \phi_n$ - формулы ЛП

        Тогда если $\phi \sim\psi$  В ЛВ, то результат подстановки эквивалентен в ЛП
    \end{statement}
    \begin{proof}
        $\mathfrak{a}$ - произвольная алг. система сигнатуры $\sigma$, $a_1, \dots, a_n \in A$,
        тогда $b_i = \phi_i(a_1, \dots, a_k)\in \{ 0, 1\} $
        $\phi \sim\psi$ в ЛВ $\Leftrightarrow \forall b_1, \dots, b_n \in \{ 0, 1\} \phi(b_1, \dots, b_n) = \psi (b_1, \dots, b_n)$
    \end{proof}
    \begin{theorem}[Основные эквивалентности ЛП]
        След. пары формул эквивалентны
        [свободные переменные остаются свободными]:
        \begin{enumerate}
            \item (Перестановка одноименных кванторов) 
            \begin{alignat*}{2}
                \forall y \forall x P(x) \sim \forall x \forall y P(x)
                \quad&\quad
                \exists y \exists x P(x) \sim \exists x \exists y P(x)
            \end{alignat*}
            \item (Переименование связанных переменных) нельзя брать свободные переменные
            \begin{alignat*}{2}
                \forall x \psi(x) \sim \forall y \psi(y)
                \quad&\quad
                \exists x \psi(x) \sim \exists y \psi(y)
            \end{alignat*}
            \item (Отрицание и кванторы)
            \begin{alignat*}{2}
                \neg(\forall x \psi(x)) \sim \exists x \neg \psi(x)
                \quad&\quad
                \neg (\exists x \psi(x)) \sim \forall x \neg \psi(x)
            \end{alignat*}
            \item             
            \begin{alignat*}{2}
                (\forall x \phi(x)) \& (\forall x \psi(x)) \sim \forall x \phi(x) \& \psi(x)
                \quad&\quad
                (\exists x \phi(x)) \vee (\exists x \psi(x)) \sim \exists x \phi(x) \vee \psi(x)
            \end{alignat*}
            \item             
            \begin{alignat*}{2}
                (\forall x \phi(x)) \vee (\forall y \psi(y)) \sim \forall x \forall y (\phi(x) \vee \psi(y))
                \quad&\quad
                (\exists x \phi(x)) \& (\exists y \psi(y)) \sim \exists x \exists y  (\phi(x) \& \psi(y))
            \end{alignat*}
            \item
                \begin{align*}
                    (\forall x \phi(x)) \&/\vee (\exists y \psi(y)) \sim \forall x \exists y (\phi(x) \&/\vee \psi(y)) \sim \exists y \forall x (\phi(x) \&/\vee \psi(y))
                \end{align*}
            \item переменная х не входит свободно в $\psi$
            \begin{alignat*}{2}
                (\forall x \phi(x)) \&/\vee \psi \sim \forall x  (\phi(x) \&/\vee \psi)
                \quad&\quad
                (\exists x \phi(x)) \&/\vee \psi \sim \exists x  (\phi(x) \&/\vee \psi)
            \end{alignat*}
            \end{enumerate}
    \end{theorem}
    \begin{proof}
        Для доказательства эквивалентности необходимо показать, 
        что на любой модели, сигнатура которой содержит сигнатуру формул,
        при любых значениях свободных переменных обе формулы либо истинны, 
        либо ложны одновременно.
        \begin{enumerate}
            \item Очевидно
            \item Очевидно
            \item 
            Пусть $\neg \forall xA(x)$ истинна при заданной фиксации свободных переменных,
            тогда $\forall xA(x)$ - ложь. То есть формула $A(x)$ ложна при некотором значении
            x. Тогда при этом значении x формула $\neg A(x)$ истинна. Значит, истинна и
            формула $\exists x\neg A(x)$.

            Пусть теперь истинна формула $\exists x\neg A(x)$ при заданной фиксации свободных
             переменных. Тогда формула $\neg A(x) $ истинна при некотором значении
            x. Значит, формула $ A(x)$ ложна при этом значении x. По смыслу квантора всеобщности, ложна формула $\forall xA(x)$. Следовательно, формула $\neg \forall xA(x)$
            истинна.
            \item 
            Пусть M - модель, сигнатура которой содержит предикаты A(x) и
            B(x). Если предикаты содержат другие свободные переменные, кроме переменной x, то фиксируем произвольные значения для них.
            
            Пусть $\exists xA(x) \vee  \exists xB(x)$ - ложна при заданных значениях свободных
            переменных. Тогда ложна как формула $\exists xA(x)$, так и формула $\exists xB(x)$. По
            смыслу квантора существования, A(x) и B(x) ложны при любом значении
            x. Значит, при любом x ложна формула $A(x) \vee  B(x)$. По смыслу квантора
            существования, формула $\exists x(A(x) \vee  B(x))$ также ложна.

            Пусть $\exists x(A(x) \vee  B(x))$ ложна при заданных значениях свободных переменных. 
            Тогда $A(x) \vee  B(x)$ ложна при любом значении x. Значит, A(x) и
            B(x) ложны при любом значении x. Отсюда следует, что ложны формулы
            $\exists xA(x)$ и $\exists xB(x)$ и ложна их дизъюнкция $\exists xA(x) \vee  \exists xB(x)$
        \end{enumerate}
    \end{proof}
    \subsection{Классы формул $\Sigma_n, \Pi_n, \Delta_n$. Соотношения между классами}
    Вид кванторного префикса в ПНФ -  показатель сложности формулы
    \begin{definition}
        Класс $\Sigma_n$ (n > 0) состоит из всех формул в пренексном виде, у которых кванторный префикс 
        начинается с квантора существования и содержит (n-1) перемену кванторов.
    \end{definition}
    \begin{definition}
        Класс $\Pi_n$ (n > 0) состоит из всех формул в пренексном виде, у которых кванторный префикс 
    начинается с квантора всеобщности и содержит (n-1) перемену кванторов.
    \end{definition}
    \begin{definition}
        Класс $\Delta_n$ (n > 0) состоит из всех формул, которые можно привести как к виду $\Pi_n$, 
        так и к виду $\Sigma_n$.
    \end{definition}
    
    При n = 0 классы $\Sigma_0$ = $\Pi_0$ = $\Delta_0$ — все бескванторные формулы.

    \begin{theorem}[соотношения между классами формул]
        $i, j > 0,$формулы из $\Pi_i$ и $\Sigma_i$ можно преобразовать в $\Delta_{i+1}$, а
        формулы из $\Delta_{i}$ можно преобразовать в формулы из $\Pi_{i+1}$ и $\Sigma_{i+1}$
    \end{theorem}
    \begin{proof}
        Поскольку каждая формула первого порядка имеет ПНФ, 
        каждой формуле присваивается по крайней мере одна классификация. 
        Поскольку избыточные кванторы могут быть добавлены к любой формуле, как только 
        формуле присваивается классификация ${ \Sigma _{n}}$ 
        или ${ \Pi _{n}}$ ему будут присвоены классификации 
        ${ \Sigma _{r}}$ и ${ \Pi _{r}}$ для каждого $r > n$. 
        Таким образом, единственной релевантной классификацией, присвоенной формуле, 
        является классификация с наименьшим числом n; все остальные 
        классификации могут быть определены на ее основе.

        Из $\Delta_n$ $\Pi_n$ и $\Sigma_n$ выводятся по определению класса дельта.
    \end{proof}
    \subsection{Нормальная форма Сколема, ее построение (на примерах)}
    \subsection{Проверка существования вывода методом резолюций (алгоритм)}
    \subsection{Логическое следование в логике предикатов}
    \begin{definition}
        Пусть $\Gamma$ — множество формул логики предикатов сигнатуры $\sigma, \phi(x_1, \dots, x_n)$ -- формула 
        сигнатуры $\sigma$. Тогда формула $\phi$ \textbf{логически следует} из множества $\Gamma (\Gamma \models \phi)$, если для любой 
        алгебраической системы $\mathfrak{a} = <A, \sigma>$ и любых элементов $a_1, \dots, a_n \in A$ , если на этих элементах 
        в системе $\mathfrak{a}$ истинны все формулы из $\Gamma$, то истинна и $\phi(a_1, \dots, a_n)$.
    \end{definition}
    \subsection{Исчисление предикатов (ИП)  Гильберта. Свойства выводов}
    \subsection{Теория. Модель теории}
    \begin{definition}
        Теория сигнатуры $\sigma$ - это произвольное множество замкнутых формул сигнатуры  $\sigma$.
    \end{definition}
    \begin{definition}
        Модель теории T — это алгебраическая система A, в которой истинны одновременно все 
    формулы теории T.
    \end{definition}
    \subsection{Непротиворечивая теория. Полная теория. Свойства непротиворечивых и полных теорий}
    \begin{definition}
        Теория T противоречивая, если существует формула $\phi$ такая, что одновременно 
    $T \models \phi$ и $T \models \neg\phi$. В противном случае теория T непротиворечивая.
    \end{definition}
    \subsection{Теорема о существовании модели (без доказательства)}
    \begin{theorem}[Теорема о существовании модели]
        Каждая непротиворечивая теория имеет модель. 
    \end{theorem}
    \subsection{Теорема о связи выводимости и противоречивости}
    \subsection{Теоремы о корректности и полноте ИП}
    \subsection{Теорема компактности}
    \begin{theorem}[Теорема компактности]
        Теория имеет модель $\Leftrightarrow$ каждая ее конечная подтеория имеет модель. 
    \end{theorem}
    \subsection{Аксиоматизируемые и конечно аксиоматизируемые классы. Конечно аксиоматизируемые теории}
    \subsection{Обоснование нестандартного анализа (построение алгебраической системы, элементарно эквивалентной полю вещественных чисел, содержащей бесконечно малые элементы)}
    \subsection{Метод резолюций для логики предикатов (без доказательства корректности)}


% \end{multicols*}

\end{document}