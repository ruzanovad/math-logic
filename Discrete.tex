\documentclass[a4paper]{article}
\usepackage[utf8]{inputenc}
\usepackage[T2A]{fontenc}
\usepackage{amsfonts}
\usepackage{amsmath, amsthm}
\usepackage{amssymb}
\usepackage{hyperref}
\usepackage{multicol}

\newcommand\letsymbol{\mathord{\sqsupset}}
\usepackage[russian]{babel}
\renewcommand\qedsymbol{$\blacktriangleright$}
\newtheorem{theorem}{Теорема}[section]
\newtheorem{lemma}{Лемма}[section]
\theoremstyle{definition}
\newtheorem*{example}{Пример}
\newtheorem*{definition}{Определение}
\theoremstyle{remark}
\newtheorem*{remark}{Замечание}

\setlength{\topmargin}{-0.5in}
\setlength{\oddsidemargin}{-0.5in}
\textwidth 185mm
\textheight 250mm

\begin{document}
% \begin{multicols*}{2}
    \tableofcontents
    \pagenumbering{arabic}
    \setcounter{page}{1}
    \section{Теория булевых функций}
    \subsection{Определение булевой функции (БФ). Количество БФ от n переменных. Таблица истинности БФ}
    \begin{definition}
        Булева функция от n переменных - это отображение $\{0,1\}^n \rightarrow \{0, 1\}$
    \end{definition}

    \begin{remark}
        Количество БФ от n переменных - $2^{2^n}$
    \end{remark}
    \begin{proof}
        Каждая булева функция определяется своим столбцом значений.
         Столбец является булевым вектором длины $m=2n$, где n – число аргументов функции.
          Число различных векторов длины m (а значит и число булевых функций, зависящих от n переменных) равно $2^m=2^{2^n}$
    \end{proof}
    \subsection{Булевы функции одной и двух переменных (их таблицы, названия)}
Булевы функции одной переменной:
         \begin{tabular}{c|cccc}
        x & $f_1$ & $f_2$ & $f_3$ & $f_4$ \\
        \hline
        0 & 0 & 0 & 1 & 1 \\
        1 & 0 & 1 & 0 & 1 \\ 
        \end{tabular}
    $f_1$ - тождественный 0, $f_2$ - тождественная функция, $f_3$ - отрицание ($\neg$), $f_4$ - тождественная 1
\\ \\
    Булевы функции двух переменных
    \begin{tabular}{cc|cccccccccccccccc}
        x & y & 0 & $\wedge$ & $\rightarrow '$ & $x$ & $\leftarrow '$ & $y$ & $+$ & $\vee$ & $\downarrow$ & $\leftrightarrow$ & $y'$ & $\leftarrow$ & $x'$ & $\rightarrow$ & | & 1\\
        \hline
        0 & 0 & 0 & 0 & 0 & 0 & 0 & 0 & 0 & 0 & 1 & 1 & 1 & 1 & 1 & 1 & 1 & 1 \\
        0 & 1 & 0 & 0 & 0 & 0 & 1 & 1 & 1 & 1 & 0 & 0 & 0 & 0 & 1 & 1 & 1 & 1 \\ 
        1 & 0 & 0 & 0 & 1 & 1 & 0 & 0 & 1 & 1 & 0 & 0 & 1 & 1 & 0 & 0 & 1 & 1 \\
        1 & 1 & 0 & 1 & 0 & 1 & 0 & 1 & 0 & 1 & 0 & 1 & 0 & 1 & 0 & 1 & 0 & 1 \\
        \end{tabular}
    \begin{enumerate}
        \item $\wedge$ - конъюнкция
        \item $\leftarrow$ -  антиимпликация
        \item $\rightarrow$ - импликация
        \item $\vee$ - дизъюнкция
        \item | - штрих Шеффера
        \item $\downarrow$ - стрелка Пирса
        \item + -  взаимоисключающее или, сложение по модулю 2 (XOR)
    \end{enumerate}
    \subsection{Формулы логики высказываний. Представление БФ формулами}
    \begin{definition}
        Формула логики высказываний - слово алфавита логики высказываний,
        построенное по следующим правилам:
        \begin{enumerate}
            \item символ переменной - формула
            \item символы 0 и 1 - формулы
            \item если $\Phi_1$ и $\Phi_2$ - формулы, то слова ($\Phi_1 \& \Phi_2$),
            ($\Phi_1 \leftrightarrow \Phi_2$), ($\Phi_1 \rightarrow \Phi_2$), 
            ($\Phi_1 | \Phi_2$), $\dots$ , $\Phi_1 '$ тоже формулы
        \end{enumerate}
    \end{definition}
    \subsection{Эквивалентные формулы. Основные эквивалентности теории булевых функций}
	\begin{definition}
		Формулы логики высказываний 	$\Phi$($x_1$, $x_2$, ..., $x_n$) и $\Psi$($x_1$, $x_2$, ..., $x_n$) эквивалетные, если для всех наборов значений $a_1$, ..., $a_n\in\{$0, 1$\}$
		$\Phi$($a_1$, ..., $a_n$) = 1 $\Leftrightarrow \Psi$($a_1$, ..., $a_n$) = 1
	\end{definition}
	\begin{theorem}
		Справедливы следующие эквивалетности
		\begin{enumerate}
			\item a $\vee$ b $\equiv$ b $\vee$ a
			\item a $\wedge$ b $\equiv$ b $\wedge$ a
			\item a $\vee$ (b $\vee$ c) $\equiv$ (a $\vee$ b) $\vee$ c
			\item a $\wedge$ (b $\wedge$ c) $\equiv$ (a $\wedge$ b) $\wedge$ c
			\item a $\wedge$ (b $\vee$ c) $\equiv$ a$\wedge$b $\vee$ a$\wedge$c
			\item a $\vee$  b $\wedge$ c $\equiv$ (a $\vee$ b) $\wedge$ (a $\vee$ c)
			\item a $\vee$ a $\equiv$ a
			\item a $\wedge$ a $\equiv$ a
			\item $\overline{(a \vee b)} \equiv \overline{a} \wedge \overline{b}$
			\item $\overline{(a \wedge b)} \equiv \overline{a} \vee \overline{b}$
			\item $\overline{\overline{a}} \equiv a$
			\item a $\vee$ a $\wedge$ b $\equiv$ a
			\item a $\wedge$ (a $\vee$ b) $\equiv$ a
			\item a $\vee$ $\overline{a} \wedge b \equiv a \vee b$
			\item a $\wedge$ $(\overline{a} \vee b) \equiv ab$
			\item $a \vee 0 \equiv a$
			\item $a \wedge 0 \equiv 0$
			\item $a \vee 1 \equiv 1$
			\item $a \wedge 1 \equiv a$
			\item $a \vee \overline{a} \equiv 1$
			\item $a\overline{a} \equiv 0$
			\item $a \rightarrow b \equiv \overline{a} \vee b$
			\item $a \leftrightarrow b \equiv \overline{a} \wedge \overline{b} \vee a \wedge b \equiv (a \rightarrow b) \wedge (b \rightarrow a)$
			\item $a + b \equiv \overline{a \leftrightarrow b} \equiv \overline{a} \wedge b \vee a \wedge \overline{b}$
			\item $a | b \equiv \overline{a \wedge b}$
			\item $a \downarrow b \equiv \overline{a \vee b}$
		\end{enumerate}
		\begin{proof}
			Доказательство сводится к построению таблиц истинности для левой и правой частей каждой эквивалентности
		\end{proof}
	\end{theorem}
    \subsection{Тождественно истинные (ложные) и выполнимые БФ}
    \subsection{ДНФ и КНФ, алгоритмы приведения}
    \subsection{СДНФ и СКНФ, теоремы существования и единственности, алгоритмы приведения}
    \subsection{Минимизация нормальных форм (карты Карно)}
    \subsection{Полином Жегалкина, его существование и единственность. Алгоритм построения}
    \subsection{Суперпозиция булевых функций. Замкнутые классы булевых функций}
    \subsection{Полные системы булевых функций, базисы}
    \subsection{Классы $T_0, T_1$ (функции, сохраняющие 0 и 1)}
    \begin{definition}
        Класс $T_0 = \{f(x_1,\dots, x_n)$ | $f(0, \dots, 0) = 0\}$
    \end{definition}
    \begin{definition}
        Класс $T_1 = \{f(x_1,\dots, x_n)$ | $f(1, \dots, 1) = 1\}$
    \end{definition}
    \begin{tabular}{c|c|c|c}
        & $T_0$ & $T_1$ & S\\
        \hline
        0 & + & - & - \\
        1 & - & + & - \\
        x & + & + & - \\
        $\neg x$ & - & - & + \\
        $xy$ & + & + & - \\
        $x \vee y$ & + & + & -\\
        $x\oplus y$ & + & - & -\\
        $x\leftrightarrow y$ & - & + & -\\
        $x\rightarrow$ & - & + & -\\
        $x|y$ & - & - & -\\
        $x\downarrow y$ & - & - & -\\
    \end{tabular}
    \begin{remark}
        Классы $T_0, T_1$ являются замкнутыми.
    \end{remark}
    \begin{proof}
        Докажем для $T_0$. Достаточно взять булевы функции $g, g_1, \dots, g_n\in T_0$
        и доказать, что их суперпозиция из класса $T_0.$

        $g(g_1 (0, \dots, 0), \dots, g_n(0, \dots, 0)) = g(0, \dots, 0) = 0$
    \end{proof}
    \subsection{Класс S самодвойственных функций, определение двойственной БФ}
    \begin{definition}
        Булева функция $g(x_1, \dots, x_n)$ называется двойственной к БФ $f(x_1, \dots, x_n)$
        (обозначается g = f*), если $g(x_1, \dots, x_n) = f'(x_1', \dots, x_n')$.
    \end{definition}
    Из закона двойного отрицания следует, что $(f^*)^* = f$
    \begin{definition}
        Булева функция f называется самодвойственной, если $f = f^*$.
    \end{definition}
    \begin{definition}
        Класс самодвойственных функций = $\{f$ | $f = f^*\}$
    \end{definition}
    \begin{remark}
        Класс S является замкнутым.
    \end{remark}
    \begin{proof}
        Возьмем БФ $g, g_1, \dots g_k \in S$ и докажем, что их
        суперпозиция будет также из класса S. \\
        Если $F(x_1, \dots, x_n) = g(g_1(x_1, \dots, x_n), \dots, g_k(x_1, \dots, x_n))$,\\
        то $F^*(x_1, \dots, x_n) = \neg F(\neg x_1, \dots, \neg x_n) = $
        $\neg g(g_1(\neg x_1, \dots, \neg x_n), \dots, g_k(\neg x_1, \dots, \neg x_n))$.

        Так как $g_i \in S,$ то $g_i(x_1, \dots, x_n) = \neg g_i(\neg x_1, \dots, \neg x_n)$,
        что эквивалентно $ \neg g_i(x_1, \dots, x_n) = g_i(\neg x_1, \dots, \neg x_n)$.
        Следовательно, $F^*(x_1, \dots, x_n) = \neg g(\neg g_1(x_1, \dots, x_n), \dots, \neg g_k (x_1, \dots, x_n))$.\\
        Так как $g\in S$, то $\neg g(\neg g_1(x_1, \dots, x_n), \dots, \neg g_k (x_1, \dots, x_n))$
        $= (g_1(x_1, \dots, x_n), \dots, g_k(x_1, \dots, x_n))$
        $\implies f^*(x_1, \dots, x_n) = F(x_1, \dots, x_n)$
    \end{proof}
    \subsection{Класс монотонных функций}
    \subsection{Класс линейных функций}
    \subsection{Леммы о несамодвойственной, немонотонной, нелинейной функциях}
    \begin{lemma}[о несамодвойственной функции]
        Если БФ $f^*(x_1, \dots, x_n)$ несамодвойственна, то замыкание класса $[f, \neg x]$
        содержит тождественно ложную БФ 0 и тождественно истинную БФ 1.
    \end{lemma}
    \begin{proof}
        Так как f несамодвойственна, то существует набор $a_1, \dots, a_n$
        значений аргументов такой, что \\ $f(a_1, \dots, a_n) \neq \neg f(\neg a_1, \dots, \neg a_n)$ \\
        Так как БФ принимают только значения 0 и 1, то $f(a_1, \dots, a_n) = f(\neg a_1, \dots, \neg a_n)$ \\
        Составим функцию $g(x) = f(x^{a_1}, \dots, x^{a_n})$

    \end{proof}
    \subsection{Теорема Поста о полноте системы булевых функций}
    \subsection{Релейно-контактные схемы: определение, примеры, функция проводимости. Анализ и синтез РКС (умение решать задачи)}
    \section{Логика высказываний}
    \subsection{Парадоксы в математике. Парадоксы Г. Кантора и  Б. Рассела}
    \subsection{Логическое следование в логике высказываний. Проверка логического следования с помощью таблиц истинности и  эквивалентных преобразований.}
    \subsection{Понятия прямой теоремы, а также противоположной, обратной и обратной к противоположной теорем}
    \subsection{Понятия необходимых и достаточных условий}
    \subsection{Формальные системы. Выводы в формальных системах. Свойства выводов}
    \subsection{Исчисление высказываний (ИВ) Гильберта. Примеры выводов}
    \subsection{Теорема о дедукции для ИВ}
    \subsection{Теорема о полноте и непротиворечивости ИВ}
    \subsection{ИВ Генцена, его полнота}
    \subsection{Метод резолюций для логики высказываний (без обоснования корректности)}
    \section{Логика предикатов}
    \subsection{Понятие предиката и операции, их представления, примеры}
    \begin{definition}
        n-местный предикат на множестве A - это отображение вида $P: A^n \rightarrow \{0, 1\}$ 
    \end{definition}
    \begin{definition}
        n-местная операция на множестве A - это отображение вида $f: A^n \rightarrow A$ 
    \end{definition}
    Предикат можно задать как множество тех аргументов, на которых он является истинным
    \begin{example}
        $P = \{1, 3\} : P = 1 \Leftrightarrow x = 1 \vee x = 3$
    \end{example}
    \begin{example}
        $Q = \{(1, 2), (3, 4), (5, 6)\}$
    \end{example}
    Способы задания:
    \begin{enumerate}
        \item описательный
        \item множество (отношения)
        \item таблица (истинности)
        \item графы 
        
        для предиката $P(x, y)$ ребро $(x, y)$ обозначает $P(x,y) = 1$

        для операции $f(x)$ дуга $(x, y)$ обозначает $y = f(x)$
    \end{enumerate}
    \subsection{Сигнатура, интерпретация сигнатуры на множестве, алгебраические системы}
    \begin{definition}
        Сигнатура - набор предикатных, функциональных и константных символов с указанием местностей
    \end{definition}
    \begin{example}
    $\sigma = \{P^{(1)}, Q^{(2)}, f^{(1)}, g^{(2)}, c\}$
    \end{example}
    \begin{definition}
        Две сигнатуры считаем \textit{равными}, если в них одинаковое кол-во символов каждого
        сорта и местности соответствующих символов равны
    \end{definition}
    \begin{definition}
        Интерпретация сигнатуры $\sigma$ на множестве А - это отображение, которое
        \begin{enumerate}
            \item каждому n-местному предикатному символу $P^{(n)}\in \sigma$ сопоставляет n-местный предикат 
            на А
            \item каждому n-местному функциональному символу  $f^{(n)}\in \sigma$ сопоставляет n-местную операцию на А
            \item каждому константному символу сопоставляет элемент множества А
        \end{enumerate}
    \end{definition}
    \begin{definition}
        Алгебраическая система - набор, состоящий из множества А, сигнатуры $\sigma$ и интерпретации $\sigma$ на А. 
        Множество А называют основным множеством системы ($\mathfrak{a} = <A, \sigma>$)
    \end{definition}
    \subsection{Язык логики предикатов, термы, формулы логики предикатов}
    Зафиксируем сигнатуру $\sigma$. Алфавит логики предикатов сигнатуры $\sigma$ — это множество

    $\sigma_{A\text{ЛП}} = \sigma \cup \{x_1, x_2\dots, \&, \vee, \rightarrow, \leftrightarrow, \neg, \forall, \exists, (, ), =, \textbf{,} \}$
    \begin{definition}
        Терм - слово алфавита логики предикатов, построенное по правилам:
        \begin{enumerate}
            \item символ переменной - терм
            \item константный символ - терм
            \item если $t_1,\dots t_n$ - термы, $f^{(n)}\in\sigma$, то и $f(t_1,\dots, t_n)$ - терм
        \end{enumerate}
    \end{definition}
    \begin{definition}
        Атомарная формула сигнатуры $\sigma$ - это слово одного из двух видов:
        \begin{enumerate}
            \item $t_1 = t_2$, где $t_1, t_2$ - термы
            \item предикат $P(t_1,\dots, t_n), P^{(n)}\in\sigma, t_1,\dots t_n$ - термы 
        \end{enumerate}
    \end{definition}
    \begin{definition}
        Формула ЛП сигнатуры $\sigma$ - слово, построенное по правилам:
        \begin{enumerate}
            \item атомарная формула - формула
            \item если $\phi_1$ и $\phi_2$ -  формулы, то слова $(\phi_1 \& \phi_2), $
            $(\phi_1 \vee \phi_2), (\phi_1 \leftrightarrow \phi_2), (\phi_1 \rightarrow \phi_2), \neg \phi_1$
            тоже формулы
            \item если $\phi$ - формула, то слова $(\forall x \phi)$ и $(\exists x \phi)$ тоже формулы
        \end{enumerate}
    \end{definition}
    \subsection{Свободные и связанные переменные. Замкнутые формулы}
    \begin{definition}
        Вхождение переменной х в формулу $\phi$ \textbf{связанное}, если х попадает в область действия квантора $\exists x / \forall x$,
        в противном случае вхождение х \textbf{свободное}
    \end{definition}
    \begin{definition}
        Переменная х \textbf{свободна} в формуле $\phi$, если есть хотя бы одно свободное вхождение х в $\phi$,
        в противном случае она \textbf{связанная}
    \end{definition}
    \begin{definition}
        Формула замкнутая, если она не содержит свободных переменных.
    \end{definition}
    \subsection{Истинность формул на алгебраической системе}
    Каждый терм $t(x_1,\dots, x_n)$ определяет в системе $\mathfrak{a}$ функцию $t_{\mathfrak{a}}: A^n \rightarrow A$
    следующим образом: в терме все функциональные и 
    константные символы заменяются на их интерпретации в системе A, после чего 
    вычисляется полученная суперпозиция от входных аргументов.
    
    Пусть также $\phi(x_1 \dots, x_n)$ — формула со свободными переменными $x_1, \dots, x_n$. Определим 
    понятие истинности формулы $\phi$ на наборе элементов $a_1, \dots a_n \in \mathfrak{a}$ в алгебраической 
    системе $\mathfrak{a}$ (обозначение: $\mathfrak{a} \models \phi(a_1, \dots a_n))$ следующим образом.
    \begin{definition}
        \begin{enumerate}
            \item Пусть $\phi$ имеет вид $t_1 = t_2$. Тогда $A \models \phi(a_1, \dots a_n) \Leftrightarrow t_{1A}(a_1, \dots a_n) = t_{2A}(a_1, \dots a_n)$ (здесь $t_{iA}$ —
            функция, определяемая термом $t_i$ в системе A).
            \item Пусть $\phi$ имеет вид $P(t_1,\dots, t_k)$. Тогда 
            $A \models \phi(a_1, \dots a_n) \Leftrightarrow P_A(t_{1A}(a_1, \dots a_n), …, t_{kA}(a_1, \dots a_n)) = 1$, 
            где $P_A$ — интерпретация предикатного символа P в системе A.
            \item Пусть $\phi$ имеет вид $(\phi_1 \& \phi_2), (\phi_1 \vee \phi_2), (\phi_1 \rightarrow \phi_2), (\phi_1 \leftrightarrow \phi_2), \neg\phi_1$. Тогда истинность формулы 
            $\phi$ определяется по значениям $\phi_1(a_1, \dots a_n)$ и $\phi_2(a_1, \dots a_n)$ по таблицам истинности логических 
            связок.
            \item Пусть $\phi(x_1 \dots, x_n)$ имеет вид $(\forall x \phi(x, x_1, \dots x_n))$. Тогда $A \models \phi(a_1, \dots a_n) \Leftrightarrow$ для всех элементов 
            $b \in A$ выполнено $A \models \phi (b, a_1, \dots a_n)$.
            \item Пусть $\phi(x_1 \dots, x_n)$ имеет вид $(\exists x \phi(x, x_1, \dots x_n))$. Тогда $A \models \phi(a_1, \dots a_n) \Leftrightarrow$ для некоторого 
            элемента $b \in A$ выполнено $A \models \phi (b, a_1, \dots a_n)$.
        \end{enumerate}
    \end{definition}
    \begin{definition}
        Формула $\phi(x_1, \dots, x_n)$ сигнатуры $\sigma$ тождественно истинная (ложна) в алгебраической 
системе $A = <A, \sigma>$, если для всех наборов элементов $a_1\dots a_n \in A$ выполнено 
$A \models \phi(a_1\dots a_n) (A \not\models \phi(a_1\dots a_n)).$
    \end{definition}
    \begin{definition}
        Формула $\phi(x_1, \dots, x_n)$ выполнима в алгебраической системе 
$A = <A, \sigma>$, если для хотя бы одного набора элементов $a_1\dots a_n \in A $ выполнено 
$A \models \phi(a_1\dots a_n)$.
    \end{definition}
    \begin{definition}
        Формула $\phi$ сигнатуры $\sigma $ тождественно истинная (ложна), если $\phi$ тождественно истинна 
(ложна) во всех алгебраических системах сигнатуры $\sigma$. 
    \end{definition}
    \begin{definition}
        Формула $\phi$ сигнатуры $\sigma$ выполнима, 
если $\phi$ выполнима хотя бы в одной алгебраической системе сигнатуры $\sigma$. 
    \end{definition}
    \subsection{Изоморфизм систем. Теорема о сохранении значений термов и формул в изоморфных системах. Автоморфизм}
    \subsection{Элементарная теория алгебраической системы. Элементарная эквивалентность систем. Связь понятий изоморфизма и элементарной эквивалентности}
    \subsection{Выразимость свойств в логике предикатов. Умение записать формулой различные свойства систем и элементов систем}
    \subsection{Эквивалентность формул логики предикатов}
    \subsection{Тождественно истинные (ложные) и выполнимые формулы}
    \subsection{Пренексный вид формулы}
    \subsection{Основные эквивалентности логики предикатов}
    \subsection{Классы формул $\Sigma_n, \Pi_n, \Delta_n$. Соотношения между классами}
    \subsection{Нормальная форма Сколема, ее построение (на примерах)}
    \subsection{Проверка существования вывода методом резолюций (алгоритм)}
    \subsection{Логическое следование в логике предикатов}
    \subsection{Исчисление предикатов (ИП)  Гильберта. Свойства выводов}
    \subsection{Теория. Модель теории}
    \subsection{Непротиворечивая теория. Полная теория. Свойства непротиворечивых и полных теорий}
    \subsection{Теорема о существовании модели (без доказательства)}
    \subsection{Теорема о связи выводимости и противоречивости}
    \subsection{Теоремы о корректности и полноте ИП}
    \subsection{Теорема компактности}
    \subsection{Аксиоматизируемые и конечно аксиоматизируемые классы. Конечно аксиоматизируемые теории}
    \subsection{Обоснование нестандартного анализа (построение алгебраической системы, элементарно эквивалентной полю вещественных чисел, содержащей бесконечно малые элементы)}
    \subsection{Метод резолюций для логики предикатов (без доказательства корректности)}


% \end{multicols*}

\end{document}