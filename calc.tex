\documentclass[a4paper]{article}
\usepackage{cmap}
\usepackage[utf8]{inputenc}
\usepackage[T2A]{fontenc}
\usepackage{amsfonts}
\usepackage{amsmath, amsthm}
\usepackage{amssymb}
\usepackage{hyperref}
\usepackage{multicol}
\usepackage{tikz} 
% \usepackage{pdfpages}

\newcommand\letsymbol{\mathord{\sqsupset}}
\usepackage[russian]{babel}
\renewcommand\qedsymbol{$\blacktriangleright$}
\newtheorem{theorem}{Теорема}[section]
\newtheorem{lemma}{Лемма}[section]
\theoremstyle{definition}
\newtheorem*{example}{Пример}
\newtheorem*{definition}{Определение}
\newtheorem*{statement}{Утверждение}
\theoremstyle{remark}
\newtheorem*{remark}{Замечание}
\newtheorem*{corollary}{Следствие}

\setlength{\topmargin}{-0.5in}
\setlength{\oddsidemargin}{-0.5in}
\textwidth 185mm
\textheight 250mm

\begin{document}
    \tableofcontents
    \section{Интегралы, зависящие от параметра}
    \subsection{	Интегралы, зависящие от параметра. Принцип равномерной сходимости}
    $\letsymbol f(x,y): [a, b]\times Y$

    Для $\forall y \in Y f_y(x) = f(x,y) - \letsymbol\quad$ она $\in R([a, b])$ (интегрируема)
    
    $\implies \forall \alpha\quad$ и $\quad\beta\in[a,b]$ определена функция $F(y, \alpha, \beta) = \int_{a}^{b} f_y(x)dx = \int_{a}^{b} f(x, y)dx$

    $F(y, \alpha, \beta)$ - функция, заданная интегралом, зависящим от параметра

    [$F(y, a, b)$ - частный случай функции]

    \begin{definition}
        $X\times Y \subset \mathbb{R}^2, f(x,y)$ определена на $X\times Y$, пусть $y_0$ - предельная точка Y
   \begin{enumerate}
        \item пусть $\forall x \in X \quad \exists \lim\limits_{y\to y_0}f(x,y):=\phi(x)$
        \item пусть $\forall \epsilon >0 \exists \delta(\epsilon)$ такая что $|y - y_0|<\delta |f(x,y) - \phi(x)|< \epsilon$ для $\forall x \implies$ тогда говорят, что $f(x,y)$ равномерно сходится к $\phi(x)$ 
      \end{enumerate}
   \end{definition}

   \begin{theorem}[Свойства равномерной сходимости]
    $f:X \times Y \longrightarrow \mathbb{R}, y_0$ - предельная точка $Y$
\begin{enumerate}
    \item $f(x,y)$ равномерно на $X$ сходится к $\phi(x)$ тогда и только тогда,
     если $\forall \epsilon>0 \quad \exists \delta(\epsilon): \forall x \in X \forall y', y'' \in Y$ $|f(x, y') - f(x, y'')|<\epsilon$ [Критерий Коши]
    \item $f(x,y)$ равномерно по $X$ стремится к $\phi(x)$ тогда и только тогда, если для  $\forall\{y_n\}$ так что $y_n \longrightarrow y_0$
     - последовательность $\{f(x,y_n)\}$ равномерно сходится к $\phi(x)$ [сходимость по Гейне]
    \item Если при $\forall y$ функция $f(x,y)$ непрерывна по x (интегрируема) и $f(x,y)$ равномерно сходится к $\phi(x)$, то $\phi(x)$ -  непрерывна и интегрируема
    \item $\letsymbol x_0, y_0$ предельные точки X и Y, $f(x,y)$ равномерно по х сходится к $\phi(x)$,\hfill \break 
     $\letsymbol\forall y \in Y \exists \lim\limits_{x \to x_0}f(x,y) =: \psi(y)$, тогда $\exists \lim\limits_{x \to x_0}\phi(x) = \lim\limits_{y \to y_0}\psi(y) [= \lim\limits_{x \to x_0}\lim\limits_{y \to y_0}f(x,y)]$ 
  \end{enumerate}
\end{theorem}
\begin{proof}
    \begin{enumerate}
         \item $\triangleleft \Rightarrow  \lim\limits_{y \to y_0}f(x,y) =: \phi(y)$\hfill \break 
         $|f(x,y') - f(x,y'')| = |f(x,y') - \phi(x) - f(x,y'')+\phi(x)| \le |f(x,y') - \phi(x)| + |f(x,y'')-\phi(x)|$\hfill \break
         $\Leftarrow x \in X |f(x,y') - f(x,y'')| < \epsilon$ при
         $
         \begin{array}{l}
              |y_0 - y'|< \delta\\
              |y_0 - y''|<\delta
         \end{array}
         $
         $\Leftarrow$ при $\forall x \exists \lim\limits_{y\to y_0}f(x,y)=:\phi(x)$
    
         $|f(x,y')-f(x,y'')| < \epsilon$, $y''\rightarrow y_0$
    
         $|f(x,y')-\phi(x)| \le\epsilon$ ,$f(x,y) \rightrightarrows \phi(x)$
         \item \hypertarget{p1}{Необходимость очевидна}
         
         Достаточность: $\{y_n\} \rightarrow y_0$
    
         $\{f(x,y_n)\} \rightarrow \phi(x)$, пусть $|y_0 - y_n|< \delta = \frac{1}{n} \implies {\ y_n}\ \rightarrow y_0$ 
         
         и $|f(x,y_n) - \phi(x)|>\epsilon$; $f(x,y_n) \nrightarrow\phi(x)$ противоречие
         \item $\letsymbol \{y_n \} \rightarrow y_0, f_n(x) = f(x, y_n)$
         
         $f_n(x)$ равномерно сходится к $\phi(x)$ по \hyperlink{p1}{2}
    
         Далее $\phi(x)$  равномерный предел хороших функий $\implies \phi(x)$ хорошая

         Попа дробнее... (для последовательности функций от одной переменной)

         $|s(x_0+h) - s(x_0)| = |s(x_0+h) + s_n(x_0 + h) - s_n(x_0) - s_n(x_0+h) + s_n(x_0)- s(x_0)|$ 
         
         $\leq |s(x_0+h) - s_n(x_0 + h)| + |s_n(x_0 + h) - s_n(x_0)| + |s_n(x_0) - s(x_0)|$

         Каждое из этих слагаемых меньше $\epsilon/3$(среднее по причине непрерывности $s_n(x)$, остальные по причине равномерной сходимости)
         \item $f(x,y) \rightrightarrows \phi(x), \letsymbol \epsilon>0$, выберем $\delta >0$ такое что:
         
         $|y_0 - y'| < \delta$ и $|y_0-y''|< \delta \implies$
    
         $|f(x,y') - f(x,y'')|< \epsilon$ по к. Коши
    
         $x \to x_0 : |\psi(y') - \psi(y'')| \leq \epsilon \implies$
         
         для $\psi(y)$ верен критерий Коши $\implies$
    
         $\exists \lim\limits_{y \to y_0}\psi(y) = A = \lim\limits_{y \to y_0}\lim\limits_{x \to x_0}f(x,y)$
    
         $|f(x,y) - \phi(x)|< \epsilon, |\psi(y) - A|< \epsilon$ если $|y - y_0|< \delta$
    
         $|\phi(x)-A| \leq {|\phi(x) - f(x,y)|}_{\leq\epsilon}+{|f(x,y) - \psi(y)|}_{<\epsilon, \text{т.к дельты}} + {|\psi(y) - A|}_{\leq\epsilon} \leq 3\epsilon$
    
         при $x \to x_0 \implies \lim\limits_{x\to x_0}\phi(x) = A$ 
    \end{enumerate}
\end{proof}

    \subsection{	Теорема о коммутировании двух предельных переходов. Предельный переход под  знаком интеграла}

    $f(x,y): [a,b]\times Y\rightarrow\mathbb{R}, y_0$ - предельная точка Y и 
    $f_y(x) = f(x,y)$ - интегрируема на $[a,b]$
    
    $F(y) = \int_{a}^bf(x,y)dx$
    \begin{theorem}[О предельном переходе] \hypertarget{p2}{}
         Если кроме того, что $f(x,y)$ равномерно на $[a,b]$ стремится к $\phi(x)$ при $y\to y_0$, то
         $\lim\limits_{y\to y_0}F(y) = \lim\limits_{y\to y_0}\int_{a}^bf(x,y)dx = \int_{a}^b \lim\limits_{y\to y_0} f(x,y)dx$
    \end{theorem}
    \begin{proof}
         $\triangleleft \phi(x)$ - равномерный предел, непрерывен
    
    $f_y(x)\implies \phi(x) $ - интегрируема, $\letsymbol{} \epsilon > 0 \quad \delta(\epsilon)>0$ выбрано из
    определения равномерной сходимости
    
    $|\int_{a}^bf(x,y)dx - \int_{a}^b\phi(x)dx|$ = $|\int_{a}^b(f(x,y) - \phi(x))dx| \leq \int_{a}^b|f(x,y) - \phi(x)|dx$
    $\leq \epsilon(b-a)$ если $|y - y_0|<\epsilon$
    
    $\lim\limits_{y\to y_0}\int_{a}^b f(x,y)dx = \int_{a}^b \phi(x) dx$
    \end{proof}

    \subsection{	Теорема о непрерывности интеграла, зависящего от параметра}
    \begin{theorem}[Непрерывность]

        $f(x,y) - $непрерывна, $f: [a,b]\times [c,d]\rightarrow \mathbb{R} \implies$
   
        $f(y) = \int_{a}^b f(x,y)dx $ непрерывна на $[c,d]$
   \end{theorem}
   \begin{proof}
        $\triangleleft [a,b]\times [c,d]$ компакт $\implies f(x,y)$ равномерно непрерывна на компакте
   
   $\forall \epsilon>0:$
   $
        \begin{array}{l}
             |x - x'|< \delta\\
             |y - y'|<\delta
        \end{array}
        $
        $\implies |f(x,y) - f(x', y')|<\epsilon$
   
        $x' = x, y' = y_0$
   
        $|f(x,y) - f(x,y_0)|<\epsilon$ при $|y - y_0|< \delta(\epsilon)$
   
        $f(x,y) \rightrightarrows f(x, y_0) = \phi(x)$ равномерный предел не зависит от х 
   
        по теореме о предельном переходе:
        
        $\lim\limits_{y\to y_0} F(y) = \lim\limits_{y\to y_0} \int_{a}^b f(x,y)dx = \int_{a}^b \phi(x) dx = \int_{a}^b f(x,y_0)dx  = F(y_0)\implies$
        $F$ непрерывна в $y_0 \in [c,d]\implies F$ непрерывна на $[c,d] $
   \end{proof}
    \subsection{	Дифференцирование под знаком интеграла. Правило Лейбница}
    \begin{theorem}[О дифференцируемости интеграла, зависящего от параметра]\hypertarget{theorem4}{}
     $f(x,y)$ - определена в $[a, b]\times[c, d]$ при $\forall y \in [c, d]$
     функция $f_y(x) = f(x, y)$ непрерывна по х, $\letsymbol{} f'_y(x,y)\exists$
     и непрерывна в прямоугольнике, тогда

     $F(y) = \int_{a}^{b}f(x,y)dx$ и $F'(y) = \int_{a}^{b}f'_y(x,y)dx$
\end{theorem}
\begin{proof}
$\triangleleft$ в силу непрерывности $f(x,y)$ по х, определена $F(y) = \int_{a}^{b}f(x,y)dx$

$y_0 \in [c,d], F(y_0) = \int_{a}^{b}f(x,y_0)dx$

$F(y_0+\triangle) = \int_{a}^{b}f(x,y_0 + \triangle)dx$

$\frac{F(y_0+\triangle) - F(y_0)}{\triangle} = \int_{a}^{b}\frac{f(x,y_0+\triangle) - f(x,y_0)}{\triangle}dx$

По теореме Лагранжа, $\exists \theta \in (0, 1)$ т.ч 

$\frac{f(x,y_0+\triangle) - f(x, y_0)}{\triangle} = f'_y(x, y_0+\theta\triangle)$

т.к F непрерывна $\implies$ равномерно непрерывна  $\implies$
для $\epsilon>0 \exists \delta>0$
$\begin{array}{l}
|x' - x''|< \delta\\
|y' - y''|<\delta
\end{array}
$
$\implies |f'_y(x', y') - f'_y(x'', y'')|$


$x' = x'' = x, y' = y_0 + \triangle\theta, y'' = y_0,$если $\triangle<\delta$

$|\frac{f(x,y_0+\triangle) - f(x, y_0)}{\triangle} - f_y'(x,y_0)| = |f'_y(x, y_0+\theta\triangle) - f_y'(x,y_0)|<\epsilon$ т.к $\delta(\epsilon)$

неравенство не зависит от точек, т.е

$\frac{f(x,y_0+\triangle) - f(x, y_0)}{\triangle}\rightrightarrows f_y'(x,y_0)$ равномерно по х

В силу теоремы \hyperlink{p2}{о предельном переходе}, получаем что $\int_a^b \frac{f(x,y_0+\triangle) - f(x, y_0)}{\triangle}dx \rightarrow \int_a^b f'_y(x, y_0)dx$

$\frac{F(y_0+\triangle) - F(y_0))}{\triangle} \rightarrow F'_y(y_0)$
\end{proof}
    \subsection{	Интегрирование под знаком интеграла}
    \begin{theorem}[О интегрируемости F(y)] 
     $\letsymbol{}f(x,y)$ непрерывна в $[a, b]x[c, d]$, тогда имеет место равенство

     $\int_c^d (\int_a^b f(x,y)dx)dy = \int_a^b (\int_c^d f(x,y)dy)dx$
\end{theorem}
\begin{proof}
     $\triangleleft \letsymbol{} \eta \in [c, d]$, покажем, что
$\int_c^\eta (\int_a^b f(x,y)dx)dy = \int_a^b (\int_c^\eta f(x,y)dy)dx$

$\int_c^\eta F(y)dy = \mathcal{F}(\eta) -  \mathcal{F}(c), \mathcal{F}' = F$

Производная левой части по $\eta = F(\eta) = \int_a^b f(x,\eta)dx$

$\phi (\eta) := \int_c^\eta f(x,y)dy$ непрерывна по x

$\phi (x, \eta) \rightarrow \phi_\eta'(x, \eta)$

$\letsymbol{} \Phi(x, \eta)'= \phi(x,\eta), \Phi'(x,\eta) = f$

$\phi(x,\eta) = \Phi(x,\eta) - \Phi(x, c)$

$\phi'_\eta = \Phi'_\eta  = f \quad \phi'_\eta(x, \eta) = f(x, \eta)$

По \hyperlink{theorem4}{предыдущей теореме}
$(\int_a^{b}\phi(x, \eta)dx)'_\eta = \int_a^b \phi_\eta'(x, \eta)dx$
$= \int_a^b f(x,\eta)dx = F(\eta)\implies$
левая и правая часть могут отличаться лишь на const, но при $\eta = c$
обе части равны 0 $\implies C = 0$
\end{proof}
    \subsection{	Непрерывность и дифференцируемость интеграла с переменными пределами интегрирования}
    \begin{theorem}
     $\letsymbol{} f(x,y)$ определена и непрерывна в прямоугольнике $[a, b]\times[c, d]$

     $x = \alpha(y); x = \beta(y)$ непрерывны и не выходят за пределы прямоугольника

     Тогда $F(y) = \int_{\alpha(y)}^{\beta(y)}f(x,y)dx$ непрерывен
\end{theorem}

\begin{proof}
     $\triangleleft y_0\in[c,d]$

     $F(y) = \int_{\alpha(y_0)}^{\beta(y_0)}f(x,y)dx + \int_{\beta(y_0)}^{\beta(y)}f(x,y)dx - \int_{\alpha(y_0)}^{\alpha(y)}f(x,y)dx$

     т.к $\beta(y_0), \alpha(y_0) = C$, то 

     $\int_{\alpha(y_0)}^{\beta(y_0)}f(x,y)dx \stackrel{\rm{def}}{=}\widetilde{F}(y)\rightarrow \int_{\alpha(y_0)}^{\beta(y_0)}f(x,y_0)dx = \widetilde{F}(y_0)$

     $|\int_{\beta(y_0)}^{\beta(y)}f(x,y)dx| \leq \int_{\beta(y_0)}^{\beta(y)}|f(x,y)|dx \leq M |\beta(y)-\beta(y_0)|\to 0$, где
     $M \leq |f(x,y)|$, при $y\to y_0$

     при $y \to y_0 \quad F(y) \to \widetilde{F}(y)$

     $F(y)\to \widetilde{F}(y) \to \widetilde{F}(y_0) = F(y_0)$
\end{proof}

\begin{theorem}
     $\letsymbol{}f(x,y)$ определена в $[a,b]\times[c,d]$ имеет в ней непрерывную производную $f'_y(x,y)$

     $\alpha'(y)$ и $\beta'(y)$ - непрерывны, тогда $F'_y(y) = \int_{\alpha(y_0)}^{\beta(y_0)}f'_y(x,y)dx + \beta'(y)f(\beta(y), y) - \alpha'(y)f(\alpha(y), y)$
\end{theorem}

\begin{proof}
     $F(y) = \int_{\alpha(y_0)}^{\beta(y_0)}f(x,y)dx + \int_{\beta(y_0)}^{\beta(y)}f(x,y)dx - \int_{\alpha(y_0)}^{\alpha(y)}f(x,y)dx$

     $(\int_{\alpha(y_0)}^{\beta(y_0)}f(x,y)dx)'_y = \int_{\alpha(y_0)}^{\beta(y_0)}f'_y(x,y)dx$ т.к пределы постоянные

     $\frac{\int_{\beta(y_0)}^{\beta(y)} f(x,y)dx - 0}{y-y_0} = \frac{f(\widetilde{x}, y) (\beta(y) - \beta(y_0))}{y-y_0} [\widetilde{x}$ между $\beta(y)$ и $\beta(y_0)]$

     при $y  \to y_0 \frac{\int_{\beta(y_0)}^{\beta(y)} f(x,y)dx}{y-y_0} \to f(\beta(y_0), y_0)\beta'(y_0)$, т.е
     
     $(\int_{\alpha(y_0)}^{\beta(y_0)}f(x,y)dx)'_y = f(\beta(y), y)\beta'(y)$, аналогично со вторым интегралом
\end{proof}
    \subsection{	Равномерная сходимость интегралов. Достаточные признаки равномерной сходимости}

    $\int_a^\omega F(x)dx$ - несобственный, если $\omega = \pm\infty$ или $f(x)$ не ограничена в окрестности $\omega$

$\letsymbol{} f(x,y)$ определена на множестве $[a, \omega)\times Y$

Для всех $y\in Y$ функция $f_y(x) = f(x,y)$ несобственно интегрируема на $[a, \omega)$, тогда $F(y) = \int_a^\omega f(x,y)dx = \lim\limits_{b\to\omega} \int_a^b f(x,y)$

\begin{definition}
     $f(b, y) = \int_a^b f(x,y) dx$, тогда сходимость F(y) равносильна существованию предела $\lim\limits_{b\to\omega}F(b, y) = F(y) = F(\omega, y)$
\end{definition}
\begin{definition}
     F(y) называется равномерно сходящейся относительно y на Y, если $\forall\epsilon \quad\exists\delta(\epsilon): \forall y\in Y \quad \forall b \in (a,\omega) |b-\omega|< \delta \implies |F(b, y) - F(y)| < \epsilon$

     $F(b, y)\rightrightarrows_{b\to \omega} F(y)$
\end{definition}
\begin{remark}
     $\letsymbol{} - \{b_n\}$ - последовательность сходится к $\omega$ согласно свойствам равномерной сходимости

     $F(b,y) \rightrightarrows F(y) \leftrightarrow F(b_n, y) \rightrightarrows F(y)$

     $a_n y \stackrel{\rm{def}}{=} \int_{b_n}^{b_{n+1}}f(x,y)dx, b_1 = a, b_j \geq a$

     Тогда $F(y) = \sum_{n = 1}^{\infty} a_n(y) $

     Равномерная сходимость F(y) равносильна равномерной сходимости ряда
\end{remark}

\begin{theorem}[Признаки равномерной сходимости интеграла]
     \begin{enumerate}
          \item(Вейерштрасса) $f(x,y)$ определена на $[a, \omega)\times Y, \omega$ - особая точка f(x,y) и f(x,y) интегрируема на $[a,b]\subset[a, \omega)$
          Если $\exists \phi(x) такая что |f(x,y)| \leq\phi(x)\quad \forall x \in [a,\omega)\forall y\in Y$ и $\int_a^\omega \phi(x)dx$ сходится, то $\int_a^\omega f(x, y)dx = F(y)$
          \item(Дирихле) $F(y) = \int_a^\omega f(x, y)g(x,y)dx, g(x,y)$ монотонно по х -> $\omega$ равномерно по y стремится к 0
          и для $\forall$ отрезка $[a,b]\subset[a,\omega)$

          $|\int_a^b f(x, y)dx|\leq L$, тогда F(y) сходится равномерно
          \item (Абель) $F(y) = \int_a^\omega f(x, y)g(x,y)dx$
          
          Если $\int_a^\omega f(x, y)dx$ сходится равномерно $g(x,y)$ монотонно по х равномерно по у сходится к своему пределу 
     \end{enumerate}
\end{theorem}
\begin{proof}
     \begin{enumerate}
          \item очевидно
          Для F(y) используем критерий Коши
          \item $\int_{b'}^{b''}f(x,y)g(x,y)dx = g(b', y)\int_{b'}^\xi f(x,y)dx + g(b'', y)\int_{\xi}^{b''} f(x,y)dx, \xi \in (b', b'')$
          
          $g(b, y)\to 0$ равномерно по y $\implies\exists B$ такое что $\forall b', b'' > B$

          $|g(b', y)|< \frac{\epsilon}{2L}\quad |g(b'', y)|< \frac{\epsilon}{2L}\implies F(y)$ сходится равномерно
          \item $\int_a^\omega f(x,y)dx$ сходится равномерно
          $\forall \epsilon>0 \exists\delta \quad \forall b', b'' > B |\int_{b'}^{b''}f(x,y)dx| \widetilde{\epsilon}$

          т.к $g(x,y)$ равномерно сходится к G(y)

          $|g(x,y)|\leq M$ при х близком к $\omega$

          $\widetilde{\epsilon} = \frac{\epsilon}{2M}$, $|\int_{b'}^{b''}f(x,y)g(x,y)dx|\leq M\frac{\epsilon}{2M} +M\frac{\epsilon}{2M} = \epsilon\implies F(y)$ сходится равномерно
     \end{enumerate}
\end{proof}
    \subsection{	Предельный переход в несобственном интеграле, зависящем от параметра}
    \begin{theorem}[О предельном переходе] \hypertarget{p4}{}
     $\letsymbol{}f(x,y)$ определена на $[a,\omega)\times Y$ для $\forall y\in Y$, интегрируема на $[a,b]\subset[a, \omega]$ равномерно относительно у сходится к функции $\phi(x)$ при $y\to y_0$ если $F(y) = \int_a^\omega f(x,y)dx$ 
     сходится равномерно относительно $y\in Y$
     $\lim_{y \to y_0}  \int_a^\omega  f(x,y)dx = \int_a^\omega  \phi(x)dx = \int_a^\omega  \lim_{y \to y_0}  f(x,y)dx$ 
\end{theorem}

\begin{proof}
     $F(b, y) = \int_a^b f(x,y)dx$ это несобственный интеграл и для него верна теорема о \hyperlink{p2}{о предельном переходе}

     $\lim_{y \to y_0}F(b, y) = \int_{a}^{b} \phi(x) \,dx $

     $\lim_{b\to \omega}F(b, y) = \int_a^\omega f(x,y)dx$ - равномерно

     $F(b,y)$ - для этой функции верны условии о перемене предельных переходов $\implies$

     $\lim_{y \to y_0} \lim_{b \to \omega} \int_a^b f(x,y)dx = lim_{y \to y_0} \int_a^\omega f(x,y)dx$
\end{proof}
Следствие:
Если $f(x,y)$ монотонно по y $\lim_{y \to y_0}  f(x,y) = \phi(x)$ - непрерывны, тогда

$\int_a^\omega \phi(x)dx\rightrightarrows\int_a^\omega f(x,y)dx$ сходится равномерно

$\lim_{y \to y_0} F(y) = \int_a^\omega \phi(x)dx$
\begin{proof}
     $f(x,y)\to \phi(x)\quad y\to y_0 \quad \forall\epsilon>0\exists\delta: |y - y_0|< \delta\implies |f(x,y) - \phi(x)|< \epsilon$

     $\letsymbol{} f(x,y)$ возрастает по y, тогда $F(b, y) = \int_a^b f(x,y)dx$ возрастает по у

     но $f(x,y)\leq\phi(x)\implies F(b, y)\leq \int_a^b \phi(x)dx \leq \int_a^\omega \phi(x)dx\implies \lim_{b \to \omega} F(b,y) = \int_a^\omega f(x,y)dy$ - сходится

     Равномерность по Вейерштрассу
\end{proof}
    \subsection{	Дифференцирование  по параметру несобственного интеграла}
    \begin{theorem}[О непрерывности интеграла] \hypertarget{p5}{}
     $\letsymbol f(x,y)$ - определена на $[a,\omega)\times[c, d]$ и непрерывна

     $F(y) = \int_a^\omega f(x,y)dx$ сходится равномерно относительно у на [c,d]

     Тогда F(y) - непрерывная функция на [c,d]
\end{theorem}
\begin{proof}
     $\triangleleft$ Пусть $y_0\in[c,d]$
     
     $F(x,y)\rightarrow_{y\to y_0} \phi(x), [a, b]\subset [a,\omega)$
     
     $[a,b] \times [c,d]$ - компакт $\implies f(x,y)$ равномерно сходится на $[a,b] \times [c,d]$
     
     $f(x,y)\rightrightarrows_{y\to y_0} \phi(x)$ равномерно по х
     
     $\forall \epsilon > 0 \exists \delta>0: |y-y_0|< \delta \forall x\in [a,b] |f(x,y) - \phi(x)|< \epsilon$
     
     
     $\forall \epsilon > 0 \exists \delta_1>0:$
     $
               \begin{array}{l}
                    |x' - x''|< \delta_1\\
                    |y' - y''|<\delta_1
               \end{array}
               $
               $|f(x',y') - f(x'',y'')|<\epsilon$
     
     $\forall \epsilon \exists \delta_2(x)>0: |y - y_0|<\delta_2 \implies |f(x,y)-\phi(x)|<\epsilon$
     
     Фиксируем $x_0, \delta_1 = \delta_2(x_0)$
     
     $\forall x \in [a, b]: |x - x_0|< \delta_1\implies$ если $|y-y_0|< \delta_2 \implies|f(x,y) - \phi(x)|\leq |f(x,y) -f(x_0,y)|+|f(x_0, y) - \phi(x_0)|\leq\epsilon$
     
     Выбираем конечное подпокрытие [a,b] такими окрестностями $x_0\pm \delta(x_0),$ выбираем наименьшее $\delta_2$
     
     Тогда если $|x' - x''|<\delta_2\implies\exists x_0: |x' - x_0|<\delta_2$ и $|x'' - x_0|< \delta_2$
     
     Тогда $|f(x',y) - \phi(x')|\leq |f(x',y) - f(x_0,y)|+|f(x_0, y) - \phi(x_0)|+|\phi(x') - \phi(x_0)| < \epsilon+\epsilon+\epsilon$
     
     Тогда $F(y, b) = \int_a^b f(x,y)dx $ - непрерывна по у
     
     $\lim_{b \to \omega} F(y,b)  = \int_a^\omega f(x,y)dx = F(y)$
     
     $\lim_{y \to y_0} \int_a^\omega f(x,y)dx = \int_a^\omega \lim_{y \to y_0} f(x,y)dx  = \int_a^\omega\phi(x)dx = F(y_0$
     
     $\phi(x) = f(x,y_0) = \lim_{y\to y_0} f(x,y)$
     
     \end{proof}
     Следствие: Если f(x,y)>=0, то из непрерывности F(y) следует равномерная сходимость $\int_a^\omega f(x,y)dx$

\begin{proof}
     $F(b,y) = \int_a^b f(x,y) dx$ неубывает с ростом b

     Предельная функция - F(b, y) это F(y)

     $\forall \epsilon > 0 \exists B \quad b', b'' \in (\omega -B, \omega) \forall y |\int_{b'}^{b''}f(x,y) dx|\leq \epsilon$

     $F(b,y)\rightarrow F(y)$

     $\forall \epsilon\exists B: \forall b' \in (\omega - B, \omega) |\int_{b'}^\omega f(x,y)dx |< \epsilon$

     $|\int_{b'}^{b''}f(x,y) dx|\leq|\int_{b'}^\omega f(x,y)dx |< \epsilon$
\end{proof}
\begin{theorem}[О дифференцируемости несобственных интегралов]
     Пусть f(x,y) непрерывна по х на $[a,b] \times [c,d]$, ее производная по y непрерывна на этом множестве

     Пусть $\forall y F(y) = \int_a^\omega f(x,y )dx$ сходится, и сходится равномерно $\int_a^\omega f'_y(x,y )dx$ по у

     Тогда F(y) дифференцируемо на [c,d] и $F'(y) = \int_a^\omega  f'_y(x,y)dx$
\end{theorem}

\begin{proof}
     $\frac{F(y_0+\Delta) - F(y_0)}{\Delta} = \int_a^\omega\frac{f(x, y_0 +\Delta) - f(x, y_0)}{\Delta}dx$

     на $[a,b] \subset [a,\omega) \frac{f(x, y_0 +\Delta) - f(x, y_0)}{\Delta}\rightrightarrows_{\Delta\to 0}f'_y(x,y_0)$

     $\int_a^\omega f'_y(x,y )dx$ по у cходится равномерно по условию

     $\forall \epsilon > 0\exists \delta > 0$
     $
     \begin{array}{l}
          |b - \omega|< \delta\\
          |b'' \omega|<\delta
     \end{array}
     $
     $\implies |\int_{b'}^{b''} f'_y(x,y)dx|< \epsilon$

     Пусть $\Phi(y) = \int_{b'}^{b''}f(x,y)dx$, тогда $\Phi'(y) = \int_{b'}^{b''}f'_y(x,y)dx\implies |\Phi'(y)|<\epsilon$

     $\frac{\Phi(y+\Delta) - \Phi(y)}{\Delta} = \Phi'(\eta ), \eta \in (y, y+ \Delta)$

     $|\int_{b'}^{b''} \frac{f(x, y +\Delta) - f(x, y)}{\Delta}dx|<\epsilon$ сходится равномерно

     Тогда имеем право перейти к пределу $\Delta \to 0$

     $\lim_{\Delta\to 0}\frac{F(y + \Delta) - F(y)}{\Delta} = \lim_{\Delta\to 0} \int_a^\omega \frac{f(x, y+\Delta) - f(x,y) }{\Delta}dx=\int_a^\omega f'_y(x,y)dx = F'(y)$


\end{proof}
    \subsection{	Интегрирование по параметру несобственного интеграла}
    \begin{theorem} \hypertarget{p3}{}
     пусть $f(x,y)$ определена и непрерывна на $[a,\omega)\times[c,d]$ и $F(y)=\int)a^\omega f(x,y)dx$ - сходится равномерно, тогда

     $\int_c^d F(y)dy = \int_a^\omega dx(\int_c^d f(x,y)dy)$
\end{theorem}

\begin{proof}
     $\letsymbol{}b \in (a, \omega)$ тогда $\int_c^d dy \int_a^b f(x,y)dx = \int_a^b dx (\int_c^d f(x,y)dy)$
     по теореме о интегрировании собственного интеграла

     $F(b, y) = \int_a^b f(x,y)dx\rightrightarrows_{b\to\omega} F(y)$

     $\int_c^d F(b,y)dy = \int_c^d F(y)dy $ по $b\to\omega$

     $\int_c^d F(y)dy = \lim_{b\to\omega} \int_c^d dx\int_a^b f(x,y)dx = \lim_{b\to\omega}\int_a^b dx(\int_c^d f(x,y)dy)$

     $\int_c^d F(y)dy = \int_a^\omega dx (\int_c^d f(x,y)dy)$
\end{proof}

\begin{theorem}[о несобственном интегрировании несобственного интеграла]
     f(x,y)- определена и непрерывна на $[a,\omega')\times[b, \omega'')$

     Пусть $\int_a^{\omega'} f(x,y)dx$ и $\int_b^{\omega''} f(x,y)dy$ сходится равномерно относительно у и х в любом промежутке
\end{theorem}
\begin{proof}
     $\letsymbol{} \exists \int_a^{\omega'}dx\int_c^{\omega''}|f(x,y)|dy$

     Для $\forall d\in (c, \omega'') на [c,d] F(y)$ интегрируема

     $\int_c^d\int_a^\omega f(x,y) = \int_a^\omega dx \int_c^d f(x,y)dy$  \hyperlink{p3}{(Предыдущая теорема)}

     $G(d,x) = \int_c^d f(x,y)dx$ - непрерывна и при $d\to\omega''$ стремится к
     $\int_c^{\omega''}| f(x,y)|dx$ равномерно относительно х 
     $\quad|\int_a^{\omega'}dx \int_c^d|f(x,y)|dy|$ сходится $\implies \int_a^{\omega'}dx \int_c^d|f(x,y)|dy$ сходится равномерно по d

     $\int_c^d dy \int_a^{\omega'}dx |f(x,y)|$ сходится равномерно по d

     Тогда $b\to\omega' $ и применяем \hyperlink{p4}{теорему о предельном переходе}

     $\int_c^{\omega''}\int_a^{\omega'}|f(x,y)|dx = \int_a^{\omega'}dx\int_c^{\omega}dy$
\end{proof}

Следствие: Если f(x,y) интегрируема и неотрицательна и $\int_a^{\omega'}f(x,y)dx$ и $\int_c^{\omega''}f(x,y)dy$ сходится равномерно (Достаточно непрерывности)

Тогда из $\exists$ одного из интегралов следует существование второго и их равенство

\begin{proof}
     $\int_a^{\omega'} dx \int_c^{\omega}f(x,y)dy$, $\int_c^{\omega}dy \int_a^{\omega'}f(x,y) dx $ следует существование второго и их равенства

$\int_a^{\omega'}f(x,y)$ - непрерывна и f>= 0, \hypertarget{[p5]}{то по следствию теоремы о непрерывности интегралов } $\int_a^{\omega'}f(x,y)dx$сходится равномерно
\end{proof}
    \section{Кратные интегралы}
    \subsection{ Двоичные разбиения. Двоичные интервалы, полуинтревалы, кубы. Свойства двоичных инервалов, кубов}
    \begin{definition}
     $f(x):\mathbb{R}^n \to \mathbb{R} $ называется ступенчатой, если можно указать
     конечный набор n-мерных непересекающихся кубов так, что на $\forall$ кубе $f(x) = c$
\end{definition}
\begin{definition}
     Мера(Объем) n-мерного куба Q, обозн. $\mu_n(Q)$

     Если ребро куба равно a, то $\mu_n(Q) = a^n$
\end{definition}
\begin{definition}
     Интегралом $f:\mathbb{R}^n\to\mathbb{R}$ из пространства $R^n$ н-ся число
     $\int_{\mathbb{R}^n}f(x)dx = \sum_{i = 1}^{\infty} f_i\mu_n(Q_i), \quad f_i$ постоянное значение f на кубе $Q_i$
\end{definition}
\begin{example}
     $\letsymbol{} M \subset \mathbb{R}^n, M \neq \emptyset, f, g: M\to \widehat{\mathbb{R}} $и $A\subset M$
\end{example}
\begin{definition}
     Будем говорить, что $f\leq g$ на А, если $\forall x\in A\quad f(x)\leq g(x)$

     $f\leq g\implies f\leq g$ на M
\end{definition}
\begin{remark}
     $f\leq g$ отношение порядка 
\end{remark}
\begin{definition}
     $\{f_n\}$ последовательность неубывает $\Leftrightarrow f_{n+1}> f_n$ невозрастает
\end{definition}
\begin{remark}
     Если $|f| = \sup_{x\in M} |f(x)|, \{f_n\}\to f$
\end{remark}
\begin{definition}
     Если $\{f_n\}$ не возрастает и сходится к f, то $f_n \searrow f$ сходится сверху
     
     $\{f_n\}$ не убывает и сходится к f $\implies f_n\nearrow f$(снизу)
\end{definition}
\begin{definition}
     $f:M\to\mathbb{R} $

     $f^{+} = \max{0, f(x)}, f^{+}, f^{-}\leq 0$

     $f^{-} = \max{0, -f(x)}, f^{+}, f^{-}\leq 0$

     $f = f^{+} - f^{-}$
\end{definition}
$\letsymbol x\in M,$если $f(x)\leq 0$

$f^{+}(x) = f(x), f^{-}(x) = 0\implies f^{+}(x) - f^{-}(x) = f(x)$

Если f(x)<0

$f^{+}(x) = 0, f^{-}(x) = -f(x)\implies f^{+}(x) -f^{-}(x) = f(x)$

$|f|(x) = |f(x)| = f^{+}(x) + f^{-}(x)$

\begin{definition}
     $\letsymbol A\subset \mathbb{R}^n, $, функция

     
     \begin{equation*}
          X_A(x) = 
           \begin{cases}
             1 &\text{$x\in A\quad$} \\
             0 &\text{$x \notin A\quad$}
           \end{cases}
     \end{equation*}

Индикатор множества А, характеристическое изложение
\end{definition}
\begin{remark}
     $X_A(x)\equiv 0\Leftrightarrow A=\emptyset$

     $X_A(x)\equiv 1\Leftrightarrow A=\mathbb{R}^n $
\end{remark}
\begin{lemma}
     $A, B\subset M$
     
     $A\subset B \Leftrightarrow X_A \leq X_B$

     Если $\{A_n\}\subset M$ и $A ={\cup}_{n=1}A_n,$ то $X_A\leq \sum_{n = 1}^{\infty} X_{A_n},$ если $\{A_n\}$
     попарно не пересекаются, то равенство
\end{lemma}

\begin{proof}
     очевидно
\end{proof}

     \begin{definition}
          $\alpha = <a_i, b_i> \times <a_n, b_n> прямоугольник в \mathbb{R}^n, b_j>a_j, l_j = b_j - a_j - $длина ребра 
     \end{definition}
     $\mu(\alpha) = \Pi_{j=1}^n (b_j - a_j) -$мера, объем

     $\alpha =[a_1, b_1)\times[a_n, b_n)$ - полуоткрытый прямоугольнике

     \begin{definition}
          Двоичный полуинтервал - полуинтервал вида [a, b), где a = $\frac{s}{2^r}$, b = $\frac{s+1}{2^r}$, r - ранг полуинтервала

          $\mu([a,b)) = \frac{1}{2^r}$
     \end{definition}
     \begin{definition}
          Двоичный брус - это произведение двоичных интервалов одного ранга r - ранг бруса
     \end{definition}
     \begin{remark}
          Если f - ступенчатая, то существуют числа $f_1,\dots f_n$ и прямоугольники $\alpha_1, \dots\alpha_n$ т.ч

          $f(x) = \sum_{n=1}^n f_k \chi_{\alpha_k}(x)$
     \end{remark}
     \begin{remark}
          Любой полуинтервал - объединение двоичных полуинтервалов
     \end{remark}

     Предложение (Свойства двоичных полуинтервалов)
     \begin{enumerate}
          \item $\alpha$ и $\beta$ - двоичные полуинтервалы ранга r и s соответственно
          \begin{proof}
               $r\leq s, $тогда если они пересекаются => ($\alpha \cap \beta\neq\emptyset$), то $\beta \subset \alpha$

          $\alpha = [\frac{n}{2^r}, \frac{n+1}{2^r})$, $\beta = [\frac{m}{2^s}, \frac{m+1}{2^s}))$

          пусть они пересекаются => х общая точка

          $\frac{n}{2^r}\leq x < \frac{n+1}{2^r}\quad \frac{m}{2^s}\leq x < \frac{m+1}{2^s} | 2^s$

          $n2^{s-r}\leq x 2^r < (n+1)2^{s-r}\quad m \leq x 2^s < m+1$

          $n 2^{s-r}< m+1 \implies n 2^{s-r}\leq m\implies \frac{n}{2^r}\leq \frac{m}{2^s}$

          $(n+1)2^{s-r}> m\implies (n+1)2^{s-r}\geq n+1\quad \frac{n+1}{2^r}\leq \frac{m+1}{2^s}$

          \end{proof}
          \item Если $\alpha$и $\beta$ двойные полуинтервалы и их ранги равны, то они либо не пересекаются, либо совпадают
          \begin{proof}
               
          \end{proof} 
          \item Если $n\in \mathbb{N} $то каждая точка из $\mathbb{R} $принадлежит ровно одному полуинтервалу ранга r
          
          \begin{proof}
               $\forall x\in\mathbb{R} \exists m \in \mathbb{N} : x\tau^r\in[m, m+1)$
          \end{proof}
          \item Если [a,b) - двоичный полуинтервал ранга r, $c = \frac{a+b}{2}$, то [a,c), [c, b) двоичные полуинтервалы ранга r+1
          \begin{proof}
               
          \end{proof}
     \end{enumerate}
     \begin{remark}
          Двоичные полуинтервалы фиксированного ранга r образуют разбиение $\mathbb{R} $ на непересекающиеся классы (множества)

     \end{remark}
     \begin{remark}
          все свойства двоичных полуинтервалов переносятся на брусы
     \end{remark}
     \begin{remark}
          $\letsymbol{}\alpha = [a_1, b_1)\times\dots\times[a_n, b_n)$ - некоторый прямоугольник, причем каждые из чисел $a_i, b_i$ имеют вид $\frac{p}{2^q}, p\in\mathbb{Z}, q\in\mathbb{N}$, тогда
          $\alpha$ конечное объединение брусов фиксированного ранга
     \end{remark}
          \begin{proof}
               r - макс q т.ч $a_ib_i = \frac{p}{2^q}$

               $\letsymbol[a_i, b_i)$прямоуг = $[\frac{p_1}{2^{q_1}}, \frac{p_2}{2^{q_2}}), q_1, q_2\leq r$

               очевидно можно $[a_i, b_i)$ записать в виде $[\frac{m}{2^r}, \frac{k}{2^r})$

               $[\frac{m}{2^r}, \frac{k}{2^r}) = [\frac{m}{2^r}, \frac{m+1}{2^r})\cup\dots\cup[\frac{k-1}{2^r}, \frac{k}{2^r})$

               $\alpha - $конечное произведение конечных объединений (конечное объединение брусов)
          \end{proof}
          \begin{lemma}
               A - компактное множество в $\mathbb{R}^n\implies\exists$конечное множество брусов ранга r, покрывающих A
          
          \end{lemma}
          \begin{proof}
               A -  ограничено $\implies M\subset N, A\subset [-M, M]^n$
          
               $[-M, M]$ подходит под условия предыдущей леммы
          
               $M = \frac{M}{2^0} = \frac{2M}{2^1}\implies [-M, M]^n - $конечное объединение брусов
               ранга r
          \end{proof}
    \subsection{ Ступенчатые функции. Интеграл от ступенчатой функции (естественное и индуктивное определения). Теорема о совпадении определений}
    \begin{definition}
     $f:\mathbb{R} ^n \to \mathbb{R} $ ступенчатая функция, если f - лин. комбинация конечного числа двоичных брусов
\end{definition}
\begin{remark}
     f образуют линейное пространство ступенчатых функций $\mathcal{L}(\mathbb{R} ^n) $

\end{remark}
\begin{remark}
     Если $f\in \mathcal{L}(\mathbb{R} ^n),$ то f можно представить в виде 
     линейной комбинации конечного числа брусов одного ранга
\end{remark}
\begin{proof}
     очевидно
\end{proof}
\begin{lemma}
     f - ступенчатая функция, $f\in \mathcal{L}(\mathbb{R} ^n)$

     $\sum_{k=1}^m f_k \chi_{\alpha_k}$, $\alpha_k$ -  не пересекающиеся кубы, тогда $|f| = \sum_{k=1}^m f_k \chi_{\alpha_k}$, $\alpha_k$
\end{lemma}
\begin{proof}
     $\alpha = \bigcup\alpha_k$

     Если $x\notin\alpha\implies\forall k, x\in\alpha_k\implies f(x) = 0$

     $|f|(x) = |f(x)| = 0 = \sum_{k=1}^m |f_k |\chi_{\alpha_k}$, $\alpha_k$

     Если $x\in \alpha\implies\exists!\alpha_j : x\in \alpha_j$

     $f(x) = f_j, |f|(x) = |f(x)| = |f_j|$

     $\sum_{j=1}^m |f_j |\chi_{\alpha_j} = |f_j|\implies |f| = \sum |f_j |\chi_{\alpha_j}$
\end{proof}
Для всякой $f\in \mathcal{L}(\mathbb{R} ^n)$ можно указать число, которое будем называть
интегралом от f по $\mathbb{R} ^n$ и обозначать

$\int_{\mathbb{R}^n}f(x)dx$
\begin{remark}
     n = 1 $f\in \mathcal{L}(\mathbb{R})$ все функции оттуда ограничены и имеют лишь конечное число точек разрыва => определен интеграл Римана

     $\int_{-\infty}^{+\infty} f(x)dx = \int_{-\infty}^{+\infty}\sum_{k=1}^m f_k \chi_{\alpha_k}(x)dx =$
     $\sum_{k=1}^m f_k \int_{-\infty}^{+\infty}\chi_{\alpha_k}(x)dx = \sum_{k=1}^m f_k \int_{a_k}^{b_k}1dx = \sum_{k=1}^m f_k \mu(\alpha_k)$

     r = 1 разумно считать, что $\int_{\mathbb{R}^n}f(x)dx = \int_{-\infty}^{+\infty} f(x)dx =\sum_{k=1}^m f_k \mu(\alpha_k)$
\end{remark}
\begin{definition}
     $f\in\mathcal{L}(\mathbb{R} ^n)$ и $f = \sum_{k=1}^m f_j \chi_{\alpha_j}$

     $\alpha_j$ - попарно не пересекаются, тогда

     $\int_{\mathbb{R}^n}f(x)dx = \sum_{k=1}^n f_k \mu_n(\alpha_k)$
\end{definition}
\begin{remark}
     Вообще говоря, нужно доказывать независимость интеграла от представления функции в виде линейных компонент
     индикатора

     Мы не будем доказывать корректность, по-другому определим интеграл, а затем покажем, что новое определение совпадет со старым
\end{remark}
\begin{proof}
     \begin{enumerate}
          \item n = 1, $\int_\mathbb{R} f(x)dx = \int_{-\infty}^{+\infty}f(x)dx = \sum_{k=1}^n f_k \mu_n(\alpha_k)$
          \item $f\in\mathcal{L}(\mathbb{R^{n+1}})$
          
          $x\in\mathbb{R^{n+1}} = \mathbb{R^{n}}\times\mathbb{R} = (y, z)$ и

          $f(x) = \sum_{k =1}^m f_k \chi_{\alpha_k}(x)$, $\alpha_k$ - куб в $\mathbb{R} ^{n+1}$

          $\alpha_k = [a_1, b_1)\times\dots\times[a_{n+1}, b_{n+1}) = \beta_k\times\gamma_k$

          $\mu_{n+1}(\alpha_k) = \mu_n(\beta_k)\mu_k(\gamma_k)$

          $y\in\beta$ тогда $f_y(z) = f(y, z): \mathbb{R} \to\mathbb{R} $

          $f_y(z) = \sum_{k=1}^m f_k \chi_{\beta_k}(y)\chi_{\gamma_k}(z) \in \mathcal{L}(\mathbb{R})$

          $\int_\mathbb{R} f_y(z)dz = \sum_{k=1}^m f_k \chi_{\beta_k}(y)\mu_1(\gamma_k)$

          $F(y) = \int_\mathbb{R} f_y(z)dx$ -  ступенчатая из $\mathcal{L}(\mathbb{R})\implies$
          $\int_\mathbb{R} F(y)dy = \sum_{k=1}^n f_k \mu_1(\gamma_k) \mu_n(\beta_k) = \sum_{k=1}^n f_k \mu_{n+1}(\alpha_k)$

          положим что $\int_{\mathbb{R} ^{n+1}}f(x)dx = \int_R F(y)dy = \int_{R^n} dy (\int_R f_y(z)dz) = \int_{R^n}(\int_R f(y,z)dz)dy$

          т.о мы определим интеграл для $\forall n$
     \end{enumerate}
\end{proof}

    \subsection{ Свойства интеграла от ступенчатой функции (линейность интеграла, положительность, оценка интеграла)}
    \begin{remark}
     $\letsymbol{}f(x)\in \mathcal{L}(\mathbb{R} ^n)$ тогда для любого хорошего отрезка [a, b]
     верно равенство $\int_a^b f(x)dx = \int_\mathbb{R} f(x)\chi_{[a,b]}(x)dx$

     $[a,b]$ - объединение конечных приращений
\end{remark}
\begin{theorem}
     $\letsymbol{}f, g\in\mathcal{L}(\mathbb{R} ^n), \lambda, \mu \in \mathbb{R} $ 
     тогда

     $\int_{\mathbb{R}^n }(\lambda f + \mu g)(x)dx=\lambda\int_{\mathbb{R}^n }f(x)dx + \mu \int_{\mathbb{R}^n }g(x)dx$
\end{theorem}
\begin{proof}
     $f = \sum_{i = 1}^m f_k \chi_{\alpha_i}, g=\sum_{i = 1}^k f_j \chi_{\beta_j}$

     $(\lambda f + \mu g) = \sum_{n=1}^m \mu f_i \chi_{\alpha_i} + \sum_{j = m+1}^{m+k}\mu g_{j-k}\chi_{\beta_{j-k}}$

     $\int_{\mathbb{R} ^n} (\lambda f + \mu g)(x)dx = \sum_{i=1}^m \mu f_i \mu(\alpha_i)+ \sum_{j = m+1}^{m+k}\mu g_{j-k}\mu(\beta_{j-k})$
    $=\lambda\int_{\mathbb{R} ^n}f(x)dx + \mu\int_{\mathbb{R} ^n}g(x)dx$ 

\end{proof}
\begin{theorem}
     Если $f\in\mathcal{L}(\mathbb{R} ^n)$ и $f(x)>0$ для $\forall x\in\mathbb{R} ^n$, то 
     $\int_{\mathbb{R} ^n} f(x)dx\leq 0$
\end{theorem}
\begin{proof}
\end{proof}
\begin{lemma}
     $\letsymbol{}f \in \mathcal{L}(\mathbb{R} ^n)$ и $|f(x)|\leq L$ на $\mathbb{R} ^n$, Пусть 
     $P = [a_1, b_1)\times\dots\times[a_n, b_n)$ такой, что $\forall x\notin P \implies$
     $f(x) = 0,$ тогда 

     $|\int_{\mathbb{R} ^n}|\leq L \mu_n(P) = L \prod_{k = 1}^n (b_k - a_k)$
\end{lemma}
\begin{proof}
     База n =1, $|\int_\mathbb{R} f(x)dx| = |\int_{a_1}^{b_1}f(x)dx|\leq L(b_1 - a_1)$

     Переход $P_{n+1} = P_n \times P_1$
     Проводя те же рассуждения, что и для определения интеграла

     $|\int_{\mathbb{R} ^{n+1}}| = |\int_{\mathbb{R} }dz \int_{\mathbb{R} ^b}f(y, z)dy| \leq|\int_{\mathbb{R} ^n} f(y, z) dy||b_{n+1} - a_{n+1}|\leq$
     $L\prod_{k=1}^n (b_k-a_k)(b_{n+1} - a_{n+1})$
\end{proof}
    \subsection{ Теорема о пределе интегралов убывающей последовательности функций, поточечно сходящейся к нулю}
    \begin{theorem}
     $\{f_n\}$ убывающая последовательность функций определена на $[a,b]\subset \mathbb{R} $

     если каждая $f_n(x)$ интегрируема на $[a,b]$ и $f_n(x)\to 0, n\to\infty$ в основном, тогда
     $\int_a^b f_n(x)dx \to 0, n\to\infty$
\end{theorem}
\begin{proof}
     $\letsymbol{}, \Delta = [\alpha, \beta]\subset (a, b)$

     $F_n(\Delta) = \int_\alpha^\beta f_n(x)dx$ функции отрезка

     так как $\{f_n\}^\alpha$ убывает и $f_n\to0$, $f_n\leq 0$ в основном, тогда $F_n(\Delta)\leq 0$

     $F_{n+1}(\Delta) \leq F_n(\Delta)\quad \forall n \forall \Delta$

     $\{f_n(\Delta)\}$ убывает, ограничена снизу
     $\implies\exists\lim_{n\to\infty}F_n(\Delta) = F(\Delta)\geq 0$

     Если $\Delta_1, \Delta_2$ - два отрезка, $\Delta_1\cap\Delta_2$ состоит
     не более чем из 1 точки
     
     это верно для $F_n$

     $F_n(\Delta_1 \cup\Delta_2) = F_n(\Delta_1)+F_n(\Delta_2)$

     $F_n(\Delta)$ непрерывные функции и $0\leq F(\Delta) \leq F_n(\Delta)$

     $F(\Delta)$ тоже непрерывна

     $M_0 - $множество $x\in(a, b),$ что $f_n(x)\nrightarrow 0$ не более чем счетно

     $\Delta F_n(x) = \lim_{\Delta\to x} \frac{F_n(\Delta)}{|\Delta|}[норма]$

     $\Delta F_n$ в основном равно $\delta_n, \quad t\in(0, 1)$

     $\frac{F_n(\Delta)}{|\Delta|} = \frac{\int_\alpha^\beta f_n(x)dx}{\beta - \alpha} = \frac{(\beta-\alpha)f_n(\alpha + t(\beta-\alpha))}{\beta-\alpha}$
     $\to f_n(x)$

     $M_n - \{x\in(a,b):\lim_{\Delta\to x}\frac{F_n(\Delta)}{|\Delta|}\}\neq f_n(x)$ не более
     чем счетно
     
     $M = M_0\cup M_1\cup \dots $не более чем счетно

     $x\in(a, b)\diagdown M\quad f_n(x)\to 0 n\to\infty\quad \epsilon>0 \quad \exists n_0 \forall n > n_0\implies$
     $f_n\leq \frac{\epsilon}{2}$

     $x\in M \implies\Delta F_n(x) = f_n(x)\leq \frac{\epsilon}{2}$ начиная с $n_0$

     найдем $|\frac{F_n(\Delta)}{|\Delta|} - f_n(x)|< \frac{\epsilon}{2}$ начиная с $n_0$ [$\delta >0, |\Delta|<\delta$]

     $\frac{F_n(\Delta)}{|\Delta|}\leq f_n(x) + \frac{\epsilon}{2}< \epsilon,$ то $F_n(\Delta)\geq F(\Delta)\implies$
     $0\leq \frac{F(\Delta)}{|\Delta|}<\epsilon,$ если $|\Delta|<\delta\implies\Delta F(x) = 0 \quad \forall x\notin M$
     
     $F(x)$ почти везде постоянна

     $F(\Delta) = 0$ иначе $\frac{F(\Delta)}{|\Delta|}<\epsilon$ не верно

     $F(\Delta) = 0\implies F([a, b]) = 0$

     $F([a,b]) = \lim_{n\to\infty}\int_a^b f_n(x)dx = 0$
\end{proof}
    \subsection{ Теорема о пределе интегралов убывающей последовательности ступенчатых функций, поточечно сходящейся к нулю}
    \begin{theorem}[О пределе интегралов убывающей последовательности функций поточечно сходящейся к 0]
     $\letsymbol{}\{f_m\}$ убывающая последовательность функций из $\mathcal{L} (\mathbb{R} ^n)$
     поточечно сходится к 0, тогда $\int_{\mathbb{R} ^m}f_m(x)\to0 \quad n\to\infty$
\end{theorem}
\begin{proof}
     Индукция по n
     
     База m = 1 => применяем предыдущую теорему

     Переход $\letsymbol{}$ это верно для m: $\{f_n\}\subset\mathcal{L} (\mathbb{R} ^{m+1}), x = (y, z),$
     $x\in \mathbb{R} ^{n+1}, y\in \mathbb{R} ^n, z\in\mathbb{R} $

     при фиксированном у определена ступенчатая функция

     $F_m(y) = \int_\mathbb{R} f_m(y, z)dz$

     $F_m(y) = \int_\mathbb{R} f_m(y, z)dz\geq\int_\mathbb{R} f_{m+1}(y, z)dz = F_{m+1}(y)$

     $F_m(y)\geq 0$

     $\lim_{m\to\infty}F_m(y) = \lim_{m\to \infty}\int_\mathbb{R} f_n(y, z)dz = 0$

     $\{F_m\}$ к ним применить индукционное предположение

     $\lim_{m\to\infty}\int_{\mathbb{R} ^n}f_m(x)dx = \int_{\mathbb{R} ^n}dy \int_\mathbb{R} f_m(y,z)dz\to 0, m\to\infty$
\end{proof}
    \subsection{ Системы с интегрированием. Основной пример. Свойства систем с интегрированием}
    $\letsymbol{} M -$ некоторое подмножество $\mathbb{R} ^n, M\neq\emptyset$

$\mathcal{F} - $множество функций из M в $\mathbb{R} \quad dom{\mathcal{F} } = M$

$I:\mathcal{F} \to \mathbb{R} $ функционал, т.е для $\forall f\in \mathcal{F} $определено число I(f)
\begin{definition}
     $(M, \mathcal{F} , I)$ система с интегрированием, если
     \begin{enumerate}
          \item{R1} $\mathcal{F} $ - лин пространство
          \item{R2} $f\in\mathcal{F}$, то $|f|\in \mathcal{F} $
          \item{R3} Функционал I линеен: $I(\alpha f + \beta g) = \alpha I(f)+\beta I(g)$
          \item{R4} Если $f\in \mathcal{F} $ и $f\leq0$ на М, то $I(f)\geq0$
          \item{R5} Если $\{f_n\}$ послед. убывает $f_n(x)\to0\quad n\to\infty, \forall x\in M,$ то $\lim_{n\to\infty}I(f_n) = 0$ 
     \end{enumerate}
\end{definition}
\begin{remark}
     $(M, \mathcal{F} , I)$, система с интегрированием, тогда М - базисное, $\mathcal{F} -$ множество
     основных или простых функций

     $I(f)$ - интеграл от f по М

     $I$ интеграл системы
     
     $R1-R5$ аксиомы
\end{remark}
\begin{example}
     $M = \mathbb{R} ^n, \mathcal{F} -$ ступенчатая на $\mathbb{R} ^n$

     интеграл от f $I = \int_{\mathbb{R} ^n}f(x)dx$

     Очевидно, что $(M, \mathcal{F} , I)$ - система с интегрированием
\end{example}
\begin{remark}
     Если $(M, \mathcal{F} , I)$ - система с интегрированием

     $\forall f\in \mathcal{F}, f^{+}, f^{-}\in \mathcal{F} $
     
     \begin{equation*}
          \text{$f^+ = \frac{f+|f|}{2} = $}
           \begin{cases}
             f(x) &\text{$f(x)\leq 0$} \\
             0 &\text{$f(x)< 0$}
           \end{cases}
     \end{equation*}

     $f^- = \frac{-f+|f|}{2} \in \mathcal{F}$

     \begin{equation*}
          \text{$max\{f, g\} = $}
           \begin{cases}
             f(x) &\text{$f(x)\leq g(x)$} \\
             g(x) &\text{$g(x)\leq f(x)$}
           \end{cases}
     \end{equation*}

     $f, g\in\mathcal{F} \implies max\{f, g\}, min\{f, g\}\in \mathcal{F} $

     $min\{f, g\}(x) = (f - (f-g)^+)(x)$

     $max\{f, g\}(x) = (g+(f-g)^+)(x)$
\end{remark}
\begin{lemma}
     Если f и g - простые функции и $f\leq g$, тогда $I(f)\leq I(g)$
\end{lemma}
\begin{proof}
     $h = f-g\leq 0 \implies R4 \implies I(h)\leq 0$

     $I(h) = I(f) - I(g)\leq 0$
\end{proof}
\begin{lemma}
     Если $f\in \mathcal{F} $, то $|I(f)|\leq I(|f|)$
\end{lemma}
\begin{proof}
     $|I(f)| = |I(f^+ - f^-)| = |I(f^+) - I(f^-)|\leq|I(f^+)| + |I(f^-)| = I(f^+) + I(f^-) = I(f^+ + f^-) = I(|f|)$
\end{proof}
\begin{lemma}
     f и $\{f_n\}\in \mathcal{F} $, $f_n$ возрастает

     если для любых х из M $f(x)\leq \lim_{n\to\infty}f_n(x)$, то $I(f)\leq \lim_{n\to\infty}I(f_n)$
\end{lemma}
\begin{proof}
     Т.к $f_n$ возрастает, то $I(f_n)$ тоже возрастает, имеет предел (возможно $\infty$)

     $\lim_{n\to\infty}I(f_n)=\sup_{n}I(f_n), V(x) = \lim_{n\to\infty}f_n(x)$

     $\{f-f_n\}$ убывает, $u_n = (f-f_n)^+$, $u_n$ тоже убывает

     Если $u_n(x)=0\implies (f-f_n)(x)<0\implies$

     $\forall m>n (f-f_m)(x)< 0\implies u_m(t) = 0$

     Если $u_n(x)>0\implies (f-f_n)(x)>0$

     $\forall m > n (f-f_m)(x)\leq (f-f_n)(x)\implies u_m = (f-f_m)^+\leq u_n$

     $f - f_n\leq u_n, \quad I(f-f_n)\leq I(u_n)$

     $I(f)\leq I(f_n)+I(u_n)$

     $V(x) = \lim_{n\to\infty}f_n(x)\geq f(x)\implies f-v\leq 0$

     $\lim_{n\to\infty}u_n(x) = \lim_{n\to\infty}(f-f_n)^+(x) = (f - v)^+(x) = 0$

     $\{u_n\}\to 0, n\to \infty$ применяем аксиому R5$\implies\lim I(u_n) = 0$

     $\lim I(f)\leq \lim I(f_n) + \lim I(u_n)$

     $I(f)\leq \lim_{n\to\infty}I(f_n)$
\end{proof}

\subsubsection{Пример системы с интегрированием}
Пусть $[a,b]\in \bar{R}, \omega: [a,b]\to R $, $\omega $ положительна и интегрируема на (a,b), M = [a,b]

$f: M \to \bar{R}$ -  финитная, если $\exists [c,d]\subset M$ т.ч $\forall x\notin [c,d] f(x) = 0$

$\mathcal{F}$ -  множество всех непрерывных, финитных функций, это линейное пространство $\implies R1, R2$ верна 

Положим, $I(f) = \int_a^b f(x)\omega(x)dx$

Очевидно, что I - линейный и $I(f)\geq 0$ если $f\geq 0\implies R3, R4$ верны

Пусть $\{f_n\}$ убывающая последовательность функции 

$0 \leq f_n \leq f_1 $ все $f_n$ равны 0 все отрезка [c,d], все которых $f_1  = 0$

т.к $f_n(x )\to 0$ и [c,d] конечен $\implies f_n\rightrightarrows 0$

$\int_a^b f_n(x)\omega (x)dx = \int_c^d f_n(x)\omega (x)dx \leq \int_c^d \omega(x)dx =\sup_{x\in [c,d]}f_n\to 0$ т.к $f_n\rightrightarrows 0\implies$

$0\leq \int_a^b f_n(x)\omega (x)dx\leq 0\implies R5 $ верно

$(M, \mathcal{F} , I)$- система с интегрированием

Интеграл в этой системе называется интегралом Лебега относительно веса $\omega$ (Лебега-Стилтьеса)

В частности, можно взять вместо $[a,b] = \bar{R}$, $\omega = 1$
\begin{remark}
     В этой системе функция Дирихле интегрируема и I(D) = 0
\end{remark}

    \subsection{ L1 норма. Множество L1*($\Sigma$). L1-норма как интеграл от модуля функции}
    $(M, f, I) - $система с интегрированием

$f:M\to\bar{R},$ положим 0*$\infty$ = 0

$\letsymbol{}f : M \to \bar{R} и \{f_n\}$ последовательность функций из M в R

Будем говорить, что $\{f_n\}$ - мажорирует f если 

\begin{enumerate}
     \item $f_n\leq 0$
     \item $\{f_n\}$ возрастает
     \item $|f(x)| = \lim_{n\to\infty}f_n(x)\forall x\in M$ 
\end{enumerate}
\begin{remark}
     Если $f_n\in \mathcal{F} $ и $\{f_n\}$ возрастает, то $\{I(f_n)\}\implies \lim_{n\to\infty}I(f_n)-\exists$
\end{remark}
\begin{definition}
     $f:M\to\bar{R}, h\in\bar{R},$будем говорить, что h - верхнее число
     функции f, если $\exists\{f_n\}\subset \mathcal{F} $, кот. мажорирует
     f т.ч $h = \lim_{n\to\infty}I(f_n)$

     $W(f) - $множество всех верхних чисел для f
\end{definition}
\begin{definition}
     $L_1$ нормой f будем называть $\inf W(f)$ и обозначать

     $||f||_{L_1(M, F, I)}, ||f||_{L_1(\Sigma)}, ||f||_{L_1}$
\end{definition}
\begin{remark}
     Если W(f) пусто, т.е нет посл. мажорирующих f, то $||f||_{L_1} = \infty$
\end{remark}
\begin{remark}
     $||f||_{L_1}\leq 0$
\end{remark}
\begin{definition}
     $L_1^*(\Sigma)$ множество всех функций т.ч $||f||_{L1}\in \mathbb{R} $
\end{definition}
\begin{lemma}
     Если $f\in\mathcal{F} $, то $||f||_{L1} = I(|f|)$
\end{lemma}
\begin{proof}
     $\letsymbol{}\{f_n\} $произвольная мажорирующая последовательность f
     $|f(x)|\leq \lim_{n\to\infty}f_n(x)\implies I(f)\leq \lim_{n\to\infty}I(f_n)$

     $I(|f|)>||f||_{L1} = \inf (\lim I(f_n))$

     Рассмотрим $f_n = |f|$ и $\forall n$ мажорирует функцию f

     $|f(x)|\leq f_n(x) = |f(x)|$

     $\lim I(f_n) = \lim I(|f|) = I(|f|) - $верхнее число f

     $||f||_{L1}\leq I(|f|)$
\end{proof}
Следствие. Если $f(x)\equiv 0$, то $||f||_{L1} = 0$
\begin{proof}
     $I(f) = I(0*f) = 0I(f) = 0$ [$f(x)\in\mathcal{F} $]
     
     $I(f) = I(|f|) = ||f||_{L1}$
\end{proof}
    \subsection{ Свойства L1 нормы ("линейность", норма функции равной нулю почти всюду и т.д.)}\begin{lemma}
     $\letsymbol f-$функция $f\in L_1^*(\Sigma), \alpha\in R$, тогда $||\alpha f||_{L_1} = |\alpha|||f||_{L_1}$

\end{lemma}
\begin{proof}
     $\alpha \neq 0$

     Норма конечна => $\exists \{f_n\}$который мажорирует f 
     
     $\{|\alpha|f_n\}$ мажорирует $\alpha f$

     $|\alpha f|\leq |\alpha|\lim_{n\to\infty} f_n\implies$

     $||\alpha f||_{L1}\leq |\alpha|\lim_{n\to\infty} I(f_n)\implies ||\alpha f||_{L1}\leq |\alpha|$

     $||f||_{L1}, g = \alpha f, \beta = 1\alpha$

     $||\beta g||_{L1}\leq |\beta|||g||_{L1}$

     $||f||_{L1}\leq \frac{1}{|\alpha|}, \quad ||\alpha f||_{L1} \geq |\alpha|||f||_{L1}$

     Если $\alpha = 0$, тогда $\alpha f\equiv 0\implies ||\alpha f||_L1 = 0$

     $\alpha ||f||_{L1} = 0$
\end{proof}
Следствие. Для $\forall f$функции $||f||_{L1} = ||-f||_{L1}$

Следствие. $||v - u||_{L1} = ||u - v||_{L1}$
    \subsection{ Субаддитивность L1-нормы}
    \begin{lemma}[Субаддитивность нормы L1]
     $\letsymbol{}f, \{f_n\} - $функции $M\to \bar{R}$
     для $\forall x\in M$ верно

     $|f(x)| \leq \sum_{n = 1}^{\infty} |f_n(x)| $

     Тогда $||f||_{L1} =\sum_{n = 1}^{\infty} ||f_n||_{L1} $
\end{lemma}
\begin{proof}
     Будем считать, что $||f_n||_{L1}< +\infty$

     $\letsymbol{}\{f_n\}_m$ - последовательность функций из $\mathcal{F} $ мажорирующих $f_n$

     $\lim_{x \to \infty} I(f_{n_m})< ||f_n||_{L1} + \frac{\epsilon}{2^n m}$

     $g_m = \sum_{j = 1}^{m}  f_{j_m}$

     $g_{m+1} = \sum_{j = 1}^{m+1} f_{j_{m+1}} \geq \sum_{j = 1}^{m} f_{j_{m+1}}\geq\sum_{j = 1}^{m}  f_{j_m} = g_m$

     $f_{j_{m+1}}\geq f_{j_m}$ 

     $g_m(x) = \sum_{j = 1}^{m} f_{j_m}\to\implies$ верно при $m\leq 0$

     $\lim_{n\to\infty} g_m(x)\leq \sum_{j = 1}^{n}  |f_j(x)| \quad (|f_j|\leq \lim_{m \to \infty} f_{j_m})$
     
     предел по $n\to\infty$

     $\lim_{n \to \infty}  (g_m(x))\geq \sum_{j = 1}^{\infty} |f_j(x)| \geq |f(x)|$

     $g_m$ мажорирует f

     $\{g_m\}\subset \mathcal{F}, \quad ||f||\leq \lim_{n \to \infty} I(g_m) $

     $I(f_{n_m})\leq \lim_{n \to \infty} I(f_m) < ||f_n||_{L1}+\frac{\epsilon}{2^n}$

     $I(g_m) = \sum_{j = 1}^{n}  I(f_{j_m})< \sum_{j = 1}^{m} ||f_j||_{L1} +\epsilon(1 - \frac{1}{2^m})$

     $\lim_m (g_m)\leq \sum_{j = 1}^{\infty} ||f_j||_{L1} +\epsilon$

     $||f||_{L1}\leq \sum_{j = 1}^{\infty} ||f_j||+\epsilon$

     $||f||_{L1}\leq \sum_{j = 1}^{\infty} ||f_j||_{L1}$
\end{proof}
Следствие. $f_1, \dots f_n, f : M\to \bar{\mathbb{R} }$ и $|f(x)|\leq \sum_{j = 1}^{n}  |f_j(x)|$

$\forall x \in M $ то $||f||_{L1}\leq \sum_{j = 1}^{n} ||f_j||_{L1} $

Следствие. Если $|f(x)|\leq |g(x)| \forall x \in M,$ то $||f||_{L1}\leq ||g||_{L1}$

Следствие. Для $\forall f, g$

$||f+g||_{L1}\leq ||f||_{L1} + ||g||_{L1}$ (неравенство для L1 нормы)

$||f||_{L1}\leq ||f-g||_{L1} + ||g||_{L1}$

$||g||_{L1} \leq ||f-g||_{L1}+||f||_{L1}$

Следствие. $||f||_{L1} = |||f|||_{L1}$
\begin{proof}
     $f\leq |f|\implies ||f||\leq |||f|||_{L1}$

     $||f||\leq |f|$
     $|||f|||_{L1}\leq ||f||_{L1}$
\end{proof}
Следствие. $|||f|||_{L1}  - |||g|||_{L1}\leq ||f-g||_{L1}$
\begin{proof}
     $||f-g||_{L1}\geq ||f||_{L1} - ||g||_{L1}$

     $||f - g||_{L1}\geq ||g||_{L1} - ||f||_{L1}$

     $||f-g||_{L1}\geq| ||f||_{L1} - ||g||_{L1}|$
\end{proof}
    \subsection{ Сходимость в смысле L1}
    $\Sigma = (M, \mathcal{F} , I) -$ система с интегралом
\begin{definition}
     $\{f_n\}$ последовательность всюду определенных функций на M, $f_n: M\to \mathbb{R} $

     $f_n$ сходится к f в смысле L1 нормы, Если
     \begin{enumerate}
          \item $\forall n ||f_n = f||_{L1}< +\infty$
          \item $\lim_{n \to \infty} ||f_n - f||_{L1} = 0 $
     \end{enumerate}
\end{definition}
    \subsection{ Определение понятие интеграла и интегрируемой функции}
    \begin{definition}
     $f: M\to \bar{\mathbb{R} }$ интегрируема, если $\exists \{f_n\}\subset \mathcal{F} $

     $f_n\to_{L1}f$
\end{definition}
\begin{definition}
     Множество всех интегрируемых функций:$L_1(\Sigma)$ или $L_1$
\end{definition}
     \begin{remark}
         f, g, h -  функции на М, причем g, h всюду конечна$\implies$
         $\forall x |g(x) - h(x)|\leq |g(x) - f(x)| +|f(x)- h(x)|$
     \end{remark}
     \begin{proof}
          \begin{enumerate}
               \item f(x) -  конечно, то верно (неравенство треугольника)
               \item f(x) = $\infty\implies ||g-h||_{L1}\leq ||g-f||_{l1}+ ||f-h||_{L1}$
          \end{enumerate}
     \end{proof}
    \subsection{ Свойства интеграла и интегрируемых функций}
    \begin{lemma}
     $f\in L_1(\Sigma)\implies \exists \lim_{g:||f-g||_{L1}\to 0} I(g)< +\infty $
\end{lemma}
\begin{proof}
     \begin{enumerate}
          \item $g, h\in\mathcal{F} \implies |I(g) - I(h)| = |I(g-h)|\leq I(|g-h|) = ||g-h||_{L1}$
          $\leq ||g-f||_{L1} + ||f-h||_{L1}$
          \item $\forall \epsilon >0 \exists \delta (|g-f|< \delta \implies$
          $||f-g||_L1<\frac{\epsilon}{2})\implies |I(g) - I(h)|<\epsilon\implies $ по критерию Коши I(g) имеет предел
     \end{enumerate}
\end{proof}
$I^*(f) = \lim_{g:||g-f||_{L1}\to 0} I(g)$

     Если $f\in\mathcal{F} \implies|I(g) - I(f)|\leq||f-g||_{L1}\to 0\implies$
     $\lim_{g:||g-f||_{L1}\to0} I^*(f) = I(f) = I^*(f)$

     $\forall f\in\mathcal{F} (I^*_(f) = I(f))$

     \begin{remark}
     $\forall f_n \subset \mathcal{F}  (||f_n - f||_{L1}\to 0 (n\to\infty))$ и $f\in L1(\Sigma)\implies I(f) = \lim_{n\to\infty} I(f_n)$
     \end{remark}
     \begin{lemma}
          $f_n, f_n: M\to R$ и $\exists c>0 : |f_n|\leq C$ и $f_n\to f [n\to \infty , L1], f: M\to \bar{R}\implies$
          \begin{enumerate}
               \item $||f||_{L1} = +\infty \implies \forall n ||f_n||_{L1} = +\infty$
               \item $||f||_{L1}< +\infty\implies ||f||_{L1} = \lim_{n\to\infty}||f_n||_{L1}$
               \item $\forall \alpha \in \mathbb{R} \alpha f_n\to^{L1} \alpha f$
               \item $|f_n\to^{L1} |f|$
          \end{enumerate}
     \end{lemma}
     \begin{proof}
          $||f||_{L1}\leq ||f-f_n||_{L1} + ||f_n||_{L1}\implies ||f_n||_{L1} = +\infty$

          $|||f||_{L1} - ||f_n||_{L1}|\leq ||f-f_n||_{L1}\to_{n\to \infty} 0\implies ||f||\to_{L1, n\to\infty} ||f||_{L1}$

          $||\alpha f_n - \alpha f||_{L1} = |\alpha| ||f_n - f||_{L1}\to 0 \implies \alpha f_n \to_{L1, n\to \infty} \alpha f$

          $||f_n(x)| - |f(x)||\leq|f_n(x) - f(x)|\implies|||f_n| - |f|||\leq ||f_n - f||_{L1}\to_{n\to \infty} 0 \implies$
          $|f_n| \to^{L1} |f|$
     \end{proof}
     \begin{lemma}
          $f_n, f_n\to^{L1} f, g_n, g_n\to^{L1} g \implies \{f_n+g_n\}:$
          $f_n+g_n\to^{L1} f+g$
     \end{lemma}
     \begin{proof}
          $\forall x\in M |f_n(x) g_n(x) - f(x) - g(x)|\leq |f_n(x) - f(x)| + |g_n(x) - g(x)|\implies$
          $||f_n+g_n - f - g||_{L1}\leq ||f_n - f||_{L1} + ||g_n - g||_{L1}$
     \end{proof}
\begin{theorem}
     \begin{enumerate}
          \item $\forall (f\in L1(\Sigma)\implies \alpha f\in L1(\Sigma): I(\alpha f) = \alpha I(f))$
          \item $(f, g \in L1(\Sigma)\implies f+g\in L1(\Sigma): I(f+g) = I(f)+ I(g))$
     \end{enumerate}
\end{theorem}
\begin{proof}
     $f\in \mathcal{F} \implies I^*(f) = I(f)$

     применяем лемму
\end{proof}
Следствие.
     $(f, g\in L1(\Sigma)and f\geq g)\implies(I(f)\geq I(g))$

     Резюме:

     $\Sigma = (M, \mathcal{F} , I)$

     определена L1 норма на всех функциях из М

     $f_n\subset \mathcal{F} and \lim_{n\to\infty}f_n =^{L1} f$

     $I^*(f) = \lim I(f_n)$ (если есть пробел)
     \begin{remark}
          $\Sigma = (M, \mathcal{F}, I) -$система с интегрированием
     \end{remark}
     \begin{definition}
          $\phi(x)$мажорирует f(x), если $\phi(x)\geq |f(x)|$
     \end{definition}
     \begin{definition}
          $||f||_{\mathbb{R} } = \inf_{\phi \in \mathcal{F}, \phi \geq |f|} (I(\phi))$ есть норма Римана
     \end{definition}
     \begin{definition}
          f интегрируема в смысле Римана, если $\exists f_n\subset \mathcal{F} $т.ч $f_n\to^{||\cdot||_\mathbb{R} } f$,
          $I^*(f) = \lim I(f_n)$
     \end{definition}
     \begin{remark}
          для нормы Римана не выполнено свойство субаддитивности
     \end{remark}
    \subsection{ Множества меры ноль. Свойства функций совпадающих почти всюду}
    \begin{definition}
     $f: M\to \bar{R}$ называется пренебрежимой, если $||f||_{L1} = 0$
\end{definition}
\begin{definition}
     $E\subset M$называется пренебрежимым (множеством меры 0), если $\chi_E$ суть пренебрежимая функция
\end{definition}
\begin{remark}
     $\emptyset - $множество меры 0, так как индикатор тождественно равен нулю, как и мера L1
\end{remark}
\begin{remark}
     ($R^n. \mathcal{L} (\mathbb{R} ^n), \int_{\mathbb{R} ^n}$)в любое одноточечное множество пренебрежимо. 
\end{remark}
\begin{proof}
     $E = \{p\}$

     $E_r - $двоичный куб, содержащий p, ребро - r

     $||\chi_E||_{L1}\leq ||\chi_{E_r}||_{L1} = \int_{\mathbb{R} ^n}\chi_{E_r}(x)dx = \frac{1}{(2^r)^n}\to 0, n\to \infty$
\end{proof}
\begin{theorem}
     $||f||_{L1}< +\infty\implies \mu(\{x | f(x) = \infty\}) = 0$
\end{theorem}
\begin{proof}
     \begin{enumerate}
          \item $E =\{x|f(x)=+\infty\}$ 
          \item $\forall n \in \mathbb{N} (0\leq n \chi_E(x)\leq |f(x)|)\implies \forall n ||n\chi_E||_{L1} \leq ||f||_{L1}\implies$
          $\forall n ||\chi_E||_{L1}\leq \frac{1}{n}||f||_{L1}< +\infty\implies ||\chi_E||_{L1} = 0$
     \end{enumerate}
\end{proof}
\begin{lemma}
     $f:M\to\bar{\mathbb{R} }, S(f) = \{x | f(x)\neq 0\}$

     f - пренебрежима тогда и только тогда, когда $S(f)$ меры 0
\end{lemma}
\begin{proof}
     \begin{enumerate}
          \item $f_n = |f|\implies |\chi_{S(f)}(x)| = \chi_{S(t)} (x)\leq \sum_{n = 1}^{\infty} f_n(x) $
          
          $x\notin S(f)\implies 0\leq 0, ok$

          $x\in S(f)\implies \chi_{S(t)}(x) = 0 \lor \sum_{n = 1}^{\infty} f_n(x) = \sum_{n = 1}^{\infty} |f| = +\infty $
          $\implies $по свойству субаддитивности$||\chi_{S(f)}||\leq ||f_n||_{L1}$

          f - пренебрежима $\implies ||f||_{L1} = 0\implies |||f|||_{L1} = ||f|| = 0\implies$
          $||\chi_{S(f)}|| = 0,$т.е S(f) имеет меру 0
          \item S(f) имеет меру 0
          
          $f_n(x)=\chi_{S(f)}(x)\implies ||f(x)||\leq \sum_{n = 1}^{\infty} f_n(x) $

          $||f||_{L1}\leq \sum_{n = 1}^{\infty} ||f_n||_{L1} = 0 \implies ||f||_{L1} = 0$
     \end{enumerate}
\end{proof}
\begin{remark}
     $f, g : M\to \bar{R}$(f = g почти всюду)$\implies ||f||_{L1} = ||g||_{L1}\land ||f-g||_{L1} = 0$
     $\land f\in \mathcal{L}_1(\Sigma)\implies g\in \mathcal{L} (\Sigma): I(f) = I(g)$
\end{remark}
\begin{proof}
     \begin{equation*}
          h(x) = 
          \begin{cases}
               0 & \text{$f(x) = g(x)$}\\
               \text{$+\infty$} & \text{$f(x)\neq g(x)$}
          \end{cases}
     \end{equation*}
     $||h||_{L1} = 0$

     $|f(x)|\leq |g(x)+h(x)|\land |g(x)|\leq |f(x) +h(x)|, \quad |f(x)|\leq |g(x)|+|h(x)|$

     $\implies ||f||_{L1}\leq ||g||_{L1} +||h||_{L1} = ||g||_{L1}, \quad$ аналогично $||g||_{L1}\leq ||f||_{L1}\implies ||f||_{L1}=||g||_{L1}$

     $\{f_n\}\subset\mathcal{F} : f_n\to^{L1}_{n\to\infty} f$

     $f_n - f$ и $f - g$ совпадают почти всюду$\implies ||f-f_n||_{L1} = ||g - f_n||_{l1}\implies$
     $f_n\to^{L1}_{n\to\infty} g\implies g\in \mathcal{L}_1{\Sigma} \land  I(f) = I(g)$
\end{proof}
\begin{remark}
     Если f интегрируемая функция, то при изменении значения функции f на множестве меры 0, то $||f||_{l1}$ на $I(f)$ не изменяется
\end{remark}
\begin{lemma}
     Если A - множество меры 0, и $E\subset A$ то E - множество меры 0
     
     Пусть $E_1...E_n$ -  множества меры 0, тогда их объединение - множество меры 0.
\end{lemma}
\begin{proof}
     $||\chi_A||_{L1} = 0$, $E\subset A\implies |\chi_E|\leq |\chi_A|\implies ||\chi_E||_{L1}\leq ||\chi_A||_{L1} = 0$

     $|\chi_E| = \sum_{n = 1}^{\infty} \chi_{E_n}, \quad ||\chi_E||_{L1} \leq\sum_{n = 1}^{\infty} ||\chi_{E_n}||_{L1} = 0$
\end{proof}

\begin{remark}
     любое не более чем счетное подмножество $\mathbb{R} $ имеет меру $\mathbb{R} $
\end{remark}
\begin{example}
     $D(x) = 1, x\in Q; 0, x\notin Q, ||D||_{L1} = 0$, тк мера Q равна 0
\end{example}
Следствие. Если $\{P_n(X)\}$ - семейство условий, верных почти всюду, тогда почти всюду выполняются все $\{P_n\}$
\begin{proof}
     $E_n$ = множество тех x: P(x) верно.

     объединение $E_n$ -  множество тех х, что кто-то из $P_n$ не выполнен. Мера $E_n$ равна $0\implies$ мера объединения $E_n$ равна 0
\end{proof}
\begin{theorem}
     Если f почти всюду определена и интегрируема, то $f^+, f^-$ тоже всюду интегрируемы.

     Если еще и g почти всюду определена и интегрируема, то $\max{f, g}$ и $\min{f, g}$ также интегрируемы
\end{theorem}
\begin{proof}
     
     \begin{equation*}
          \text{$\widetilde{f}(x) = $}
          \begin{cases}
               f(x) & \text{если f(x) определена} \\
               0 & \text{иначе}
          \end{cases}
     \end{equation*}
     

     $ f = \widetilde{f}$ почти всюду $\implies \widetilde{f}$ интегрируема и $I(f) = I(\widetilde{f})$

     $f^+ = \widetilde{f}^+$почти всюду, аналогично с минусом

     Но $\widetilde{f}^+$ и $\widetilde{f}^-\implies f^+ $ и $f^-$ интегрируемы 

     $max\{f, g\}(x) = (g+(f-g)^+)(x)$ - интегрируема
\end{proof}
\subsubsection*{Теорема о предельном переходе над знаком интеграла}
$(M, \mathcal{F} , I) $ - система с интегрированием
\begin{remark}
     Если f совпадает с g почти всюду и существуют их L1 нормы, то они равны. 

     Это позволяет определить L1-норму для функций, определенных почти всюду.
\end{remark}
\begin{lemma}
     Пусть $f: M\to R$ определена почти всюду
     
     $\{f_n\}$ последовательность интегрируемых функций, такая, что $||f - f_n||_{L1}\to 0 $(сходится в смысле нормы l1)$ n\to \infty$

     Тогда f интегрируема и $I(f) = \lim_{n\to\infty} I(f_n)$
\end{lemma}

\begin{proof}
     Можно считать, что f и $f_n$ всюду определены и конечны.

     Для любого n $\exists g_n\in\mathcal{F} : ||f_n-g_n||_{L1}\leq \frac{1}{n}$

     $||f - g_n||_{L1}\leq ||f-f_n||_{L1} + ||f_n- g_n||_{L1}\leq ||f-f_n|||_{L1}+\frac{1}{n} (||f-f_n|||_{L1}\to 0)$

     $\implies g_n$ сходится к $f \implies$

     f - интегрируема и $I(f) = \lim I(g_n) = \lim I(f_n)$
\end{proof}
    \subsection{ Нормально сходящиеся ряды. Теорема о нормально сходящихся рядах}
    \begin{definition}
     Пусть $\sum_{n = 1}^{\infty} f_n$ = функциональный ряд. Будем говорить, что ряд сходится нормально, если сходится ряд$\sum_{n = 1}^{\infty} ||f_n||_{L1}  $
\end{definition}
\begin{theorem}
     пусть $\sum_{n = 1}^{\infty} f_n$ - нормально сходящийся ряд определенных почти всюду функций,

     для почти всех х из М $f_n(x)$ определены для каждого n

     Кроме того, числовой ряд $\sum_{n = 1}^{\infty} f_n(x)$ сходится

     Если $F(x)=\sum_{n = 1}^{\infty} f_n(x), $тогда

     $||F - \sum_{k =1}^{n} f_k ||_{L1}\leq \sum_{k = n+ 1}^{\infty} ||f_k||_{L1} $

     Если все $f_n$ интегрируемы, то F - интегрируема и 
     $I(f) = \sum_{n = 1}^{\infty} I(f_n) $
\end{theorem}
\begin{proof}
     Пусть $E_n = \{x\in M : f_n$ не определено$\}$
     
     $E_n$ имеет меру ноль, их объединение имеет меру 0

     Тогда для любого x не из E все $f_n(x)$ определены
     
     $\Phi(x) =\sum_{n = 1}^{\infty} |f_n(x)| \implies ||\Phi(x)||_{L1}\leq \sum_{n = 1}^{\infty} ||f_n|| < \infty$

     Множество всех х: $\Phi(x)=\infty$ имеет меру $0\implies\sum_{n = 1}^{\infty} |f_n(x)|$ сходится почти всюду$\implies$
     $\sum_{n = 1}^{\infty} f_n(x)$ сходится почти всюду

     Положим F = $\sum_{n = 1}^{\infty} f_n(x)$ если сходится, 0, иначе.

     $R_n(x) = F(x) - \sum_{k = 1}^{n} f_k(x)  = \sum_{r = n+1}^{\infty}  f_k(x)$

     $|R_n(x)|\leq \sum_{n = 1}^{\infty} |f_n(x)|\implies ||R_n||_{L1}\leq \sum_{k = 1}^{\infty} ||f_k||_{L1}\to 0\implies ||R_n||_{L1} \to 0$
     
     пусть все $f_n$ интегрируемы, $F_n =\sum_{k = 1}^{n}\quad  f_k, F_n$ интегрируемы
     
     И по первой части $||F- F_n||_{L1} \to 0$

     F - интегрируема 

     $I(F) = \lim_{n\to \infty} I(F_n)= \lim_{n\to\infty}\sum_{j = 1}^{n} I(f_k) = \sum_{k = 1}^{\infty} I(f_k) \quad(|I(f_k)|\leq ||f_k||_{L1}) $
\end{proof}
    \subsection{ Теоремы Леви для функциональных рядов и последовательностей}
    Следствие.{Теорема Леви для функциональных рядов}

Если $\sum_{n = 1}^{\infty} f_n$ функциональный ряд, и все $f_n$ неотрицательные и интегрируемые, тогда если 
сходится$\sum_{n = 1}^{\infty} I(f_n),$ то для почти всех х определена $F(x) = \sum_{n = 1}^{\infty} f_n(x)$ и

F -  интегрируема, $I(F) = \sum_{n = 1}^{\infty} I(f_n)$

\begin{proof}
     т.к $f_n$ неотрицательна, то $f_n  = |f_n|$ и $||f_n||_{L1} = |||f_k|||_{L1} = I(|f_n|) = I(f_n) \implies \sum_{n = 1}^{\infty} f_n $сходится нормально, применяем 
     теорему, все хорошо 
\end{proof}
Следствие.(Теорема Леви для последовательностей)

Пусть $\{f_n\}$ последовательность функций, интегрируемых и определенных почти всюду (за исключением множества E меры 0)

$f_n(x)$ монотонна для всех х, кроме х из Е

Тогда если $I(f_n)$ ограничена, то почти для всех x из M определена $f(x) = \lim_{n\to\infty}f_n(x)$, причем
f - интегрируема и $I(f) = \lim_{n\to\infty} I(f_n)$
\begin{proof}
     пусть $f_n$ возрастает, тогда $\sum_{n = 1}^{\infty} f_{n+1}(x)-f_n(x)$ состоит из положительных функций 

     $\forall n \sum_{k = 1}^{n} (f_{k+1}(x) - f_k(x)) = f_{n+1}(x) - f_n(x)$

     $\sum_{k = 1}^{n}I(f_{k+1} - f_k) = I(f_{n+1}) - I(f_n)$ - ограничена 
     
     $|I(f_n)|< A \implies \sum_{k = 1}^{n} I(f_{n+1} - f_n)$ - ограничена и возрастает по n $\implies \sum_{k = 1}^{n} I(f_{k+1}) - I(f_k)  $

     Применяем теорему Леви для рядов

     $f_{n+1}(x) - f_n(x)$ - определена почти для всех х и

    $ G(x) = \sum_{n = 1}^{\infty}  f_{n+1}(x) - f_n(x)$

    $ G(x) = \lim \sum_{k = 1}^{n} (f_{k+1} - f_k(x) = \lim (f_{n+1}(x) - f_n(x)) = \lim f_n(x) - f_1(x)\implies f(x) = \lim f_n(x) = G(x) + f_1(x)$
     
     $I(G) = \lim\sum_{k = 1}^{n} I(f_{k+1} - f_k) = \lim I(f_{n+1}) - I(f_1)$
     
     $\lim I(f_n) = I(G) + I(f_1) = I(f)$
\end{proof}
    \subsection{Огибающие для последовательности интегрируемых функций. Нижний и верхний предел последовательности}
     \begin{definition}
          Пусть дана последовательность $(f_\nu)_{\nu\in \mathbb{N}}$ вещественных функций, 
          каждая из которых определена почти всюду в M. Найдем множество 
          $W \subset M$, состоящее из всех $x \in M$, для которых $f_\nu(x)$ не определено 
          хотя бы для одного значения $v \in N$. Для всякого $x \notin E$ определены 
          величины 

          $g(x) = \inf_{\nu \in N}{f_\nu(x)}, h(x) =\sup_{\nu \in N}{f_\nu(x)}.$ 

          Определенную таким образом функцию g будем называть нижней 
          огибающей последовательности $(f_\nu)_{\nu\in \mathbb{N}}$. Функция f называется  
          верхней огибающей последовательности. Будем писать 
          $g = \inf_{\nu \in N}{f_\nu}, h =\sup_{\nu \in N}{f_\nu}.$ 
     \end{definition}
     \begin{lemma} \hypertarget{l4.2}{}
          Пусть E - произвольное множество. Тогда для  
          всякой функции $F: E \to \mathbb{R}$ имеют место равенства

          \hypertarget{4.2}{$\inf_{\xi \in E}{F(\xi)} = -\sup_{\xi \in E}{-F(\xi)}$ } (4.2)

          \hypertarget{4.3}{$\sup_{\xi \in E}{F(\xi)} = -\inf_{\xi \in E}{-F(\xi)}$ } (4.3)
     \end{lemma}
     \begin{proof}
          Пусть $p = - \sup_{\xi \in E}(-F(\xi))$. Тогда для всякого 
          $\xi \in E$ имеем $-F(\xi) \leq -p$ и, значит, $F(\xi) \geq p$ для всех $\xi \in E$, т. е. 
          p является нижней границей функции F. 

          Пусть $p'$ - произвольная другая нижн. гран. функции F.  
          Тогда для любого $f \in E$ выполняется $F(\xi) \geq p'$, откуда следует, что для 
          всех $\xi \in E$ выполняется $-F(\xi) \leq -p'$.

          Мы получаем, таким образом, что $-p'$ есть верх.гран.  
          функции $-F$. Так как $-p$ есть точная верх.гран. функции $-F$ на E, 
          то $-p \leq -p'$, откуда получаем, что 

          \hypertarget{4.4}{$p \geq p'$}

          Таким образом, p есть нижн. гран. функции F, и для любой 
          другой ее нижней границы выполняется неравенство выше.По  
          определению, это и означает, что $p = \inf_{\xi \in E} F(\xi)$. 
          Этим доказано \hyperlink{4.2}{равенство}.

          Функция F в \hyperlink{4.2}{равенстве} совершенно произвольна.
          Заменяя в нем 
          F на - F, получим 

          $\inf_{\xi \in E}{-F(\xi)} = - \sup_{\xi \in E}{F(\xi)}$

          откуда, очевидно, следует \hyperlink{4.3}{(4.3)}. 
     \end{proof}
     \begin{lemma}
          Пусть дана последовательность $(x_n \in \mathbb{R})_{n\in N}$.  
          Определим по индукции последовательности $(p_\nu)_{\nu \in N}$ и 
          $(q_\nu)_{\nu \in N}$ полагая 
          $p_1 = q_1 = x_1$. Если для некоторого $n in N$ числа $p_n$ и $q_n$ определены, 
          то $p_{n+1} = \min{p_n,x_{n+1}}, q_{n+1} = \max{q_n, x_{n+1}}$
          
          Тогда, $(q_\nu)_{\nu \in N}$ есть 
          возрастающая последовательность, $(p_\nu)_{\nu \in N}$ — убывающая  
          последовательность и 
          $\lim_{n\to \infty} = \inf_{n\in N}x_n, \lim_{n\to \infty}q_n = \sup_{n\in N} x_n$
     \end{lemma}
     \begin{proof}
          Из определения следует, что при каждом n  
          выполняются неравенства $p_{n+i} \leq p_n, q_{n+1} \geq q_n$. Это доказывает, что 
          $(q_\nu)_{\nu \in N}$ есть возрастающая последовательность, a $(p_\nu)_{\nu \in N}$ —  
          убывающая. 
          
          Пусть $L = \lim_{n\to \infty}q_n$. При каждом п имеем $x_n \leq q_n \leq L$ и, значит, L
          есть верх.гран. последовательности $(x_n \in \mathbb{R})_{n\in N}$.

          Пусть $L'$ — произвольная другая верх.гран.  
          последовательности $(x_n \in \mathbb{R})_{n\in N}$. 
          Докажем, что для всех n выполняется неравенство $q_n \leq L'$. Для 
          $n = 1$ это, очевидно, верно. 

          Предположим, что для некоторого n неравенство $q_n \leq L'$  
          выполняется. Так как $x_{n+1} \leq L'$, то также и $q_{n+1} = \max\{q_n, x_{n+1}\}\leq L'$. 
          
          Из доказанного, очевидно, следует, что $q_n \in L'$ для всех n и,  
          значит, $L = \lim_{n\to \infty}q_n\leq L'$. Таким образом, L есть верх.гран. 
          последовательности $(x_n \in \mathbb{R})_{n\in N}$ и для любой другой ее верхней границы L' 
          выполняется неравенство $L \leq L'$. По определению, это и означает, что 
          $L = \sup_{n\in N}x_n$. Для последовательности $(p_\nu)_{\nu \in N}$ соответствующее 
          утверждение доказывается аналогично. 

          (Формально можно получить его как 
          следствие доказанного, используя результат предыдущей леммы)
     \end{proof}
     \begin{theorem}
          Пусть $(f_\nu)_{\nu\in \mathbb{N}}$ есть произвольная  
          последовательность интегрируемых функций. Предположим, что существует  
          интегрируемая функция $\phi$ такая, что при каждом $\nu \in N$ выполняется 
          $f_\nu(x)\geq \phi(x)$ для почти всех $x \in M$. Тогда нижняя огибающая  
          последовательности функций $(f_\nu)_{\nu\in \mathbb{N}}$ интегрируема. Если существует  
          функция $\psi$ такая, что при каждом $\nu \in N$ выполняется $f_\nu(x) < \psi(x)$ для 
          почти всех $x \in M$, то верхняя огибающая последовательности функций 
          $(f_\nu)_{\nu\in \mathbb{N}}$ есть интегрируемая функция. 
     \end{theorem}
     \begin{proof}
          Предположим, что при каждом $\nu$ выполняется 
          $f_\nu(x)\geq \phi(x)$ для почти всех $x\in M$, где $\phi \in L_1(\Sigma)$. Пусть g есть 
          нижняя огибающая последовательности $(f_\nu)_{\nu\in \mathbb{N}}$. Пусть $E_\nu$ есть  
          множество меры нуль, состоящее из всех точек $x\in M$, для которых либо 
          одна из величин $f_\nu(x), \phi(x)$ не определена, либо они обе определены, 
          но неравенство $f_\nu(x) \geq \phi(x)$ не выполняется. 
          
          Положим E = $\bigcup_{\nu = 1}^\infty E_\nu$.
          Множество E является пренебрежимым.

          Построим некоторую последовательность функций $(u_\nu)_{\nu\in \mathbb{N}}$,  
          полагая $u_\nu(x) = \infty$ для любого $x\in E$ при всех $\nu \in N$. Для $x\notin E$ 
          последовательность $u_\nu(x)_{\nu\in \mathbb{N}}$ определим из условий $u_1(x) = f_1(x)$, 
          и если значение $u_\nu(x)$ определено для некоторого $\nu in N$, то 
          $u_{\nu+1}(x) = \min\{u_\nu(x), f_{\nu+1}(x)\}. $

          Из определения последовательности $(u_\nu)_{\nu\in \mathbb{N}}$ видно, что она  
          является убывающей. Функция $u_1$ интегрируема, и $u_1(x) \geq \phi(x)$ для всех 
          $x\notin E$. Пусть $\nu in N$ таково, что функция $u_\nu$ для данного $\nu$
          интегрируема, причем $u_\nu(x) \geq \phi(x)$ для всех $x\notin E$. В силу свойств  
          интегрируемых функций, установленных ранее, из определения 
          функции $u_{\nu + 1}$ следует, что тогда функция $u_{\nu+i} = \min\{u_v, f_{v+i}\}$ 
          также интегрируема. 
          
          Так как, по условию, $f_{v+1}(x) \geq\phi(x), u_\nu(x) \geq \phi(x)$ для всякого 
          $x\notin E$, то также и $u_{v+1}(x) \geq \phi(x)$ для любого $x\notin E$. 
          В силу предыдущей леммы для всякого $x\notin E$ имеем $g(x) = \lim_{\nu\to \infty} u_\nu(x)$, так 
          что функции $u_\nu$ сходятся к функции g почти всюду в M. При каждом 
          $\nu \in N$ имеем $\phi \leq u_v \leq u_1$ почти всюду в M и, значит,

          $I(\phi)\leq I(u_\nu)\leq I(u_1)$ 

          Последовательность интегралов $I(u_\nu)_{\nu\in \mathbb{N}}$ таким образом,  
          является ограниченной. В силу теоремы Леви для последовательностей 
          отсюда вытекает, что функция g 
          интегрируема, что и требовалось доказать. 

          Утверждение, касающееся верхней огибающей последовательности 
          функций $(f_\nu)_{\nu\in \mathbb{N}}$ может быть доказано аналогичными рассуждениями. 
          Формально это следует из доказанного. Именно, пусть h есть  
          верхняя огибающая последовательности $(f_\nu)_{\nu\in \mathbb{N}}$. Тогда в силу \hyperlink{l4.2}{леммы}
          -h является нижней огибающей последовательности $(-f_\nu)_{\nu\in \mathbb{N}}$.
          Если при каждом $\nu in N$ для почти всех $x\in M$ выполняется $f_\nu(x)\leq \psi(x)$, где 
          $\psi\in L_1$, то $-f_\nu(x)\geq -\psi(x)$ для почти всех x.

          Функция $-\psi$ интегрируема, и, значит, по доказанному, также  
          интегрируема функция $-h$, т. е. имеет место равенство 
          $-\sup_{\nu \in N}{f_\nu} =- h = \inf_{\nu \in N}{-f_\nu}$ 

          Отсюда следует интегрируемость h. Теорема доказана. 
     \end{proof}
     \begin{definition}[Нижнее число]
          Число $H\in \bar{\mathbb{R}} $
          называется нижним числом последовательности $(x_\nu \in \bar{\mathbb{R}})_{\nu \in N}$, если существует 
          номер $\bar{\nu} $ такой, что для всех $\nu \geq \bar{\nu} $ выполняется неравенство $x_v \geq H$. 
     \end{definition}
     Множество нижних чисел непусто, так как $-\infty$ является нижним 
     числом любой последовательности рассматриваемого вида. 
     \begin{definition}[Нижний предел]
          Точная верхняя граница множества всех нижних чисел  
          последовательности называется ее нижним пределом и обозначается символом 
          $\underline{\lim}_{\nu \to \infty}x_\nu$. 
     \end{definition}
     \begin{definition}[Верхнее число]
          Число $H\in \bar{\mathbb{R}} $
          называется нижним числом последовательности $(x_\nu \in \bar{\mathbb{R})}_{\nu \in N}$, если существует 
          номер $\bar{\nu} $ такой, что для всех $\nu \geq \bar{\nu} $ выполняется неравенство $x_v \leq H$. 
     \end{definition}
     Множество верхних чисел непусто, так как $\infty$ является верхним
     числом любой последовательности рассматриваемого вида. 
     \begin{definition}[Верхний предел]
          Точная верхняя граница множества всех нижних чисел  
          последовательности называется ее нижним пределом и обозначается символом 
          $\overline{\lim} _{\nu \to \infty}x_\nu$. 
     \end{definition}

     \begin{definition}[Предел последовательности]
          $L\in\bar{\mathbb{R}}$ - предел последовательности $x_\nu\in \bar{\mathbb{R}}$, если
          
          $L = \varliminf\limits_{\nu \to \infty}x_\nu = \varlimsup\limits_{\nu \to \infty}x_\nu$
     \end{definition}
     \begin{lemma}
          $X_\nu \in \bar{\mathbb{R}}$ - последовательность.

          $\forall \nu \in N:\quad  N_\nu = \inf\limits_{\mu \geq \nu}X_\mu, V_\nu = \sup\limits_{\mu \geq \nu}X_\mu$.

          Тогда последовательности $N_\nu$ - возрастающая, $V_\nu$ - убывающая и

          $\lim\limits_{\nu \to \infty} N_\nu = \varliminf\limits_{\nu \to \infty} X_\nu, \quad$
          $ \lim\limits_{\nu \to \infty} V_\nu = \varlimsup\limits_{\nu \to \infty} X_\nu$
     \end{lemma}
     \begin{proof}
          Пусть $G_\nu = \{\mu \in N | \mu \geq \nu\}.$ Тогда

          $N_\nu = \inf\limits_{\mu\in G_\nu}X_\mu, V_\nu = \sup\limits_{\mu\in G_\nu}X_\mu$

          $\forall \nu \in N: \quad G_\nu \supset G_{\nu+1} \implies$ по свойствам sup inf функции
          $N_\nu \leq N_{\nu+1}, V_\nu \geq V_{\nu+1} $ (осталось доказать равенства, пределы существуют)

          $P' = \lim\limits_{\nu \to \infty} N_\nu, P = \varliminf\limits_{\nu \to \infty} X_\nu\qquad$
          $Q' = \lim\limits_{\nu \to \infty} V_\nu, Q = \varlimsup\limits_{\nu \to \infty} X_\nu$

          $\forall \nu \in N: \quad X_\mu \geq N_\nu \quad \forall \mu \geq \nu \implies N_\nu$ - нижнее число $X_\nu$
          $\implies \forall \nu \in \mathbb{N} \quad N_\nu \leq P \implies$

          $P' \leq P = \sup \{$ множество всех нижних чисел $X_\nu,$ обозначим $N(X)\} = N(X)$

          $\letsymbol{} l \in N(X) \implies\exists a_l\in \mathbb{N}: \forall \nu \geq a \quad\quad X_\nu \geq l\implies$

          $P'\geq N_a\geq l\implies$ из произвольности l $P' = \sup N(X)\implies P' \geq P\implies P = P'$.

          Аналогично $Q = Q'$.
     \end{proof}
    \subsection{ Теорема Фату о предельном переходе. Следствие из теоремы Фату}
    \subsection{ Теорема Лебега о предельном переходе}
    \subsection{ Лемма о приближении стуенчатой функции с помощью непрерывных финитных}
    \subsection{ Теорема о приближении интегрируемой функции с помощью непрерывных финитных}
    \subsection{ Измеримые функции. Свойства пространства измеримых функций. Измеримые множества}
    \subsection{ Теорема об интегрируемости измеримой функции}
    \subsection{ Теорема об измеримости предела измеримых функций}
    \subsection{ Теорема об интегрируемости предела возрастающей
    последовательности положительных измеримых функций}
    \subsection{ Обобщенно измеримые функции. Измеримые множества, мера множества. Теорема об измеримости объединения и пересечения измеримых множеств}
    \subsection{ Счетная аддитивность интеграла и меры}
    \subsection{ Измеримые множества в Rn. Внешняя мера множества. Лемма о представлении открытого множества как объединения кубов. Теорема об измеримости открытых и замкнутых множеств в Rn}
    \subsection{ Теорема о внешней мере множества}
    \subsection{ Лемма о приближении неотрицательной вещественной функции ступенчатыми функциями. Следствие об измеримости непрерывной почти всюду функции}
    \subsection{ Теорема о совпадении интералов Римана и Лебега}
    \subsection{ Теорема Фубини и следствия из нее}
    \subsection{ Теорема Тонелли и следствия из нее}
    \subsection{ Диффеоморфизмы и их свойства. Теорема о замене переменной в кратном интеграле (формулировка)}
    \subsection{ Лемма о замене переменной при композиции диффеоморфизмов}
    \subsection{ Лемма о сведении замены переменной в общем случае к случаю индикатора двоичного куба}
    \subsection{ Лемма о представлении диффеоморфизма в виде композиции диффеоморфизмов специального вида}
    \subsection{ Теорема о замене переменной в кратном интеграле}
    
\end{document}