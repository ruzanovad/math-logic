\documentclass[a4paper]{article}
\usepackage{cmap}
\usepackage[utf8]{inputenc}
\usepackage[T2A]{fontenc}
\usepackage{amsfonts}
\usepackage{amsmath, amsthm}
\usepackage{amssymb}
\usepackage{hyperref}
\usepackage{multicol}
\usepackage{tikz} 

\newcommand\letsymbol{\mathord{\sqsupset}}
\usepackage[russian]{babel}
\renewcommand\qedsymbol{$\blacktriangleright$}
\newtheorem{theorem}{Теорема}[section]
\newtheorem{lemma}{Лемма}[section]
\theoremstyle{definition}
\newtheorem*{example}{Пример}
\newtheorem*{definition}{Определение}
\theoremstyle{remark}
\newtheorem*{remark}{Замечание}
\newtheorem*{corollary}{Следствие}

\setlength{\topmargin}{-0.5in}
\setlength{\oddsidemargin}{-0.5in}
\textwidth 185mm
\textheight 250mm

\begin{document}
    \tableofcontents
    \section{Интегралы, зависящие от параметра}
    \subsection{	Интегралы, зависящие от параметра. Принцип равномерной сходимости}
    $\letsymbol f(x,y): [a, b]\times Y$

    Для $\forall y \in Y f_y(x) = f(x,y) - \letsymbol\quad$ она $\in R([a, b])$ (интегрируема)
    
    $\implies \forall \alpha\quad$ и $\quad\beta\in[a,b]$ определена функция $F(y, \alpha, \beta) = \int_{a}^{b} f_y(x)dx = \int_{a}^{b} f(x, y)dx$

    $F(y, \alpha, \beta)$ - функция, заданная интегралом, зависящим от параметра

    [$F(y, a, b)$ - частный случай функции]

    \begin{definition}
        $X\times Y \subset \mathbb{R}^2, f(x,y)$ определена на $X\times Y$, пусть $y_0$ - предельная точка Y
   \begin{enumerate}
        \item пусть $\forall x \in X \quad \exists \lim\limits_{y\to y_0}f(x,y):=\phi(x)$
        \item пусть $\forall \epsilon >0 \exists \delta(\epsilon)$ такая что $|y - y_0|<\delta |f(x,y) - \phi(x)|< \epsilon$ для $\forall x \implies$ тогда говорят, что $f(x,y)$ равномерно сходится к $\phi(x)$ 
      \end{enumerate}
   \end{definition}

   \begin{theorem}[Свойства равномерной сходимости]
    $f:X \times Y \longrightarrow \mathbb{R}, y_0$ - предельная точка $Y$
\begin{enumerate}
    \item $f(x,y)$ равномерно на $X$ сходится к $\phi(x)$ тогда и только тогда,
     если $\forall \epsilon>0 \quad \exists \delta(\epsilon): \forall x \in X \forall y', y'' \in Y$ $|f(x, y') - f(x, y'')|<\epsilon$ [Критерий Коши]
    \item $f(x,y)$ равномерно по $X$ стремится к $\phi(x)$ тогда и только тогда, если для  $\forall\{y_n\}$ так что $y_n \longrightarrow y_0$
     - последовательность $\{f(x,y_n)\}$ равномерно сходится к $\phi(x)$ [сходимость по Гейне]
    \item Если при $\forall y$ функция $f(x,y)$ непрерывна по x (интегрируема) и $f(x,y)$ равномерно сходится к $\phi(x)$, то $\phi(x)$ -  непрерывна и интегрируема
    \item $\letsymbol x_0, y_0$ предельные точки X и Y, $f(x,y)$ равномерно по х сходится к $\phi(x)$,\hfill \break 
     $\letsymbol\forall y \in Y \exists \lim\limits_{x \to x_0}f(x,y) =: \psi(y)$, тогда $\exists \lim\limits_{x \to x_0}\phi(x) = \lim\limits_{y \to y_0}\psi(y) [= \lim\limits_{x \to x_0}\lim\limits_{y \to y_0}f(x,y)]$ 
  \end{enumerate}
\end{theorem}
\begin{proof}
    \begin{enumerate}
         \item $\triangleleft \Rightarrow  \lim\limits_{y \to y_0}f(x,y) =: \phi(y)$\hfill \break 
         $|f(x,y') - f(x,y'')| = |f(x,y') - \phi(x) - f(x,y'')+\phi(x)| \le |f(x,y') - \phi(x)| + |f(x,y'')-\phi(x)|$\hfill \break
         $\Leftarrow x \in X |f(x,y') - f(x,y'')| < \epsilon$ при
         $
         \begin{array}{l}
              |y_0 - y'|< \delta\\
              |y_0 - y''|<\delta
         \end{array}
         $
         $\Leftarrow$ при $\forall x \exists \lim\limits_{y\to y_0}f(x,y)=:\phi(x)$
    
         $|f(x,y')-f(x,y'')| < \epsilon$, $y''\rightarrow y_0$
    
         $|f(x,y')-\phi(x)| \le\epsilon$ ,$f(x,y) \rightrightarrows \phi(x)$
         \item \hypertarget{p1}{Необходимость очевидна}
         
         Достаточность: $\{y_n\} \rightarrow y_0$
    
         $\{f(x,y_n)\} \rightarrow \phi(x)$, пусть $|y_0 - y_n|< \delta = \frac{1}{n} \implies {\ y_n}\ \rightarrow y_0$ 
         
         и $|f(x,y_n) - \phi(x)|>\epsilon$; $f(x,y_n) \nrightarrow\phi(x)$ противоречие
         \item $\letsymbol \{y_n \} \rightarrow y_0, f_n(x) = f(x, y_n)$
         
         $f_n(x)$ равномерно сходится к $\phi(x)$ по \hyperlink{p1}{2}
    
         Далее $\phi(x)$  равномерный предел хороших функий $\implies \phi(x)$ хорошая

         Попа дробнее... (для последовательности функций от одной переменной)

         $|s(x_0+h) - s(x_0)| = |s(x_0+h) + s_n(x_0 + h) - s_n(x_0) - s_n(x_0+h) + s_n(x_0)- s(x_0)|$ 
         
         $\leq |s(x_0+h) - s_n(x_0 + h)| + |s_n(x_0 + h) - s_n(x_0)| + |s_n(x_0) - s(x_0)|$

         Каждое из этих слагаемых меньше $\epsilon/3$(среднее по причине непрерывности $s_n(x)$, остальные по причине равномерной сходимости)
         \item $f(x,y) \rightrightarrows \phi(x), \letsymbol \epsilon>0$, выберем $\delta >0$ такое что:
         
         $|y_0 - y'| < \delta$ и $|y_0-y''|< \delta \implies$
    
         $|f(x,y') - f(x,y'')|< \epsilon$ по к. Коши
    
         $x \to x_0 : |\psi(y') - \psi(y'')| \leq \epsilon \implies$
         
         для $\psi(y)$ верен критерий Коши $\implies$
    
         $\exists \lim\limits_{y \to y_0}\psi(y) = A = \lim\limits_{y \to y_0}\lim\limits_{x \to x_0}f(x,y)$
    
         $|f(x,y) - \phi(x)|< \epsilon, |\psi(y) - A|< \epsilon$ если $|y - y_0|< \delta$
    
         $|\phi(x)-A| \leq {|\phi(x) - f(x,y)|}_{\leq\epsilon}+{|f(x,y) - \psi(y)|}_{<\epsilon, \text{т.к дельты}} + {|\psi(y) - A|}_{\leq\epsilon} \leq 3\epsilon$
    
         при $x \to x_0 \implies \lim\limits_{x\to x_0}\phi(x) = A$ 
    \end{enumerate}
\end{proof}

    \subsection{	Теорема о коммутировании двух предельных переходов. Предельный переход под  знаком интеграла}

    $f(x,y): [a,b]\times Y\rightarrow\mathbb{R}, y_0$ - предельная точка Y и 
    $f_y(x) = f(x,y)$ - интегрируема на $[a,b]$
    
    $F(y) = \int_{a}^bf(x,y)dx$
    \begin{theorem}[О предельном переходе] \hypertarget{p2}{}
         Если кроме того, что $f(x,y)$ равномерно на $[a,b]$ стремится к $\phi(x)$ при $y\to y_0$, то
         $\lim\limits_{y\to y_0}F(y) = \lim\limits_{y\to y_0}\int_{a}^bf(x,y)dx = \int_{a}^b \lim\limits_{y\to y_0} f(x,y)dx$
    \end{theorem}
    \begin{proof}
         $\triangleleft \phi(x)$ - равномерный предел, непрерывен
    
    $f_y(x)\implies \phi(x) $ - интегрируема, $\letsymbol{} \epsilon > 0 \quad \delta(\epsilon)>0$ выбрано из
    определения равномерной сходимости
    
    $|\int_{a}^bf(x,y)dx - \int_{a}^b\phi(x)dx|$ = $|\int_{a}^b(f(x,y) - \phi(x))dx| \leq \int_{a}^b|f(x,y) - \phi(x)|dx$
    $\leq \epsilon(b-a)$ если $|y - y_0|<\epsilon$
    
    $\lim\limits_{y\to y_0}\int_{a}^b f(x,y)dx = \int_{a}^b \phi(x) dx$
    \end{proof}

    \subsection{	Теорема о непрерывности интеграла, зависящего от параметра}
    \begin{theorem}[Непрерывность]

        $f(x,y) - $непрерывна, $f: [a,b]\times [c,d]\rightarrow \mathbb{R} \implies$
   
        $f(y) = \int_{a}^b f(x,y)dx $ непрерывна на $[c,d]$
   \end{theorem}
   \begin{proof}
        $\triangleleft [a,b]\times [c,d]$ компакт $\implies f(x,y)$ равномерно непрерывна на компакте
   
   $\forall \epsilon>0:$
   $
        \begin{array}{l}
             |x - x'|< \delta\\
             |y - y'|<\delta
        \end{array}
        $
        $\implies |f(x,y) - f(x', y')|<\epsilon$
   
        $x' = x, y' = y_0$
   
        $|f(x,y) - f(x,y_0)|<\epsilon$ при $|y - y_0|< \delta(\epsilon)$
   
        $f(x,y) \rightrightarrows f(x, y_0) = \phi(x)$ равномерный предел не зависит от х 
   
        по теореме о предельном переходе:
        
        $\lim\limits_{y\to y_0} F(y) = \lim\limits_{y\to y_0} \int_{a}^b f(x,y)dx = \int_{a}^b \phi(x) dx = \int_{a}^b f(x,y_0)dx  = F(y_0)\implies$
        $F$ непрерывна в $y_0 \in [c,d]\implies F$ непрерывна на $[c,d] $
   \end{proof}
    \subsection{	Дифференцирование под знаком интеграла. Правило Лейбница}
    \subsection{	Интегрирование под знаком интеграла}
    \subsection{	Непрерывность и дифференцируемость интеграла с переменными пределами интегрирования}
    \subsection{	Равномерная сходимость интегралов. Достаточные признаки равномерной сходимости}
    \subsection{	Предельный переход в несобственном интеграле, зависящем от параметра}
    \subsection{	Дифференцирование  по параметру несобственного интеграла}
    \subsection{	Интегрирование по параметру несобственного интеграла}
    
    \section{Кратные интегралы}
    \subsection{ Двоичные разбиения. Двоичные интервалы, полуинтревалы, кубы. Свойства двоичных инервалов, кубов}
    \subsection{ Ступенчатые функции. Интеграл от ступенчатой функции (естественное и индуктивное определения). Теорема о совпадении определений}
    \subsection{ Свойства интеграла от ступенчатой функции (линейность интеграла, положительность, оценка интеграла)}
    \subsection{ Теорема о пределе интегралов убывающей последовательности функций, поточечно сходящейся к нулю}
    \subsection{ Теорема о пределе интегралов убывающей последовательности ступенчатых функций, поточечно сходящейся к нулю}
    \subsection{ Системы с интегрированием. Основной пример. Свойства систем с интегрирование}
    \subsection{ L1 норма. Множество L1*($\Sigma$). L1-норма как интеграл от модуля функции}
    \subsection{ Свойства L1 нормы ("линейность", норма функции равной нулю почти всюду и т.д.)}
    \subsection{ Субаддитивность L1-нормы}
    \subsection{ Сходимость в смысле L1}
    \subsection{ Определение понятие интеграла и интегрируемой функции}
    \subsection{ Свойства интеграла и интегрируемых функций}
    \subsection{ Множества меры ноль. Свойства функций совпадающих почти всюду}
    \subsection{ Нормально сходящиеся ряды. Теорема о нормально сходящихся рядах}
    \subsection{ Теоремы Леви для функциональных рядов и последовательностей}
    \subsection{ Огибающие для последовательности интегрируемых функций. Нижний и верхний предел последовательности}
    \subsection{ Теорема Фату о предельном переходе. Следствие из теоермы Фату}
    \subsection{ Теорема Лебега о предельном переходе}
    \subsection{ Лемма о приближении стпенчатой функции с помощью непрерывных финитных}
    \subsection{ Теорема о приближении интегрируемой функции с помощью непрерывных финитных}
    \subsection{ Измеримые функции. Свойства пространства измеримых функций. Измеримые множества}
    \subsection{ Теорема об интегрируемости измеримой функции}
    \subsection{ Теорема об измеримости предела измеримых функций}
    \subsection{ Теорема об интегрируемости предела возрастающей
    последовательности положительных измеримых функций}
    \subsection{ Обобщенно измеримые функции. Измеримые множества, мера множества. Теорема об измеримости объединения и пересечения измеримых множеств}
    \subsection{ Счетная аддитивность интеграла и меры}
    \subsection{ Измеримые множества в Rn. Внешняя мера множества. Лемма о представлении открытого множества как объединения кубов. Теорема об измеримости открытых и замкнутых множеств в Rn}
    \subsection{ Теорема о внешней мере множества}
    \subsection{ Лемма о приближении неотрицательной вещественной функции ступенчатыми функциями. Следствие об измеримости непрерывной почти всюду функции}
    \subsection{ Теорема о совпадении интералов Римана и Лебега}
    \subsection{ Теорема Фубини и следствия из нее}
    \subsection{ Теорема Тонелли и следствия из нее}
    \subsection{ Диффеоморфизмы и их свойства. Теорема о замене переменной в кратном интеграле (формулировка)}
    \subsection{ Лемма о замене переменной при композиции диффеоморфизмов}
    \subsection{ Лемма о сведении замены переменной в общем случае к случаю индикатора двоичного куба}
    \subsection{ Лемма о представлении диффеоморфизма в виде композиции диффеоморфизмов специального вида}
    \subsection{ Теорема о замене переменной в кратном интеграле}
    
\end{document}