\documentclass[a4paper]{article}
\usepackage{cmap}
\usepackage[utf8]{inputenc}
\usepackage[T2A]{fontenc}
\usepackage{amsfonts}

\usepackage{amsmath, amsthm}
\usepackage{amssymb}
\usepackage{mathrsfs}
\usepackage{hyperref}
\usepackage{multicol}
\usepackage{xcolor}

\newcommand\letsymbol{\mathord{\sqsupset}}
\usepackage[russian]{babel}
\renewcommand\qedsymbol{$\blacktriangleright$}
\newtheorem{theorem}{Теорема}[section]
\newtheorem{lemma}{Лемма}[section]
\newtheorem*{axiom}{Аксиома}
\theoremstyle{definition}
\newtheorem*{definition}{Определение}
\newtheorem*{statement}{Утверждение}
\theoremstyle{remark}
\newtheorem*{remark}{Замечание}
\newtheorem*{example}{Пример}

\setlength{\topmargin}{-0.5in}
\setlength{\oddsidemargin}{-0.5in}
\textwidth 185mm
\textheight 250mm

\begin{document}
\tableofcontents
\section{Вероятностное пространство}

\begin{definition}[Алгебра]
    Семейство $\mathcal{A}$ подмножеств множества $\Omega$ называется
    алгеброй, если выполнены след. аксиомы:
    \begin{enumerate}
        \item $\varnothing \in \mathcal{A}$
        \item $A\in\mathcal{A} \implies \overline{A}\in \mathbb{A}$
        \item (аддитивность) $A_1, \dots, A_n \in \mathbb{A} \implies A_1 \cup \dots \cup A_n\in \mathbb{A}$ 
    \end{enumerate}
\end{definition}
\begin{definition}[$\sigma$-алгебра]
    Алгебра называется $\sigma$-алгеброй, если 
    \[A_1, \dots, A_n \in \mathcal{A} \implies \bigcup\limits_{k = 1}^\infty  A_k \in \mathcal{A}\]
\end{definition}
\begin{definition}[мера]
    $\mu: \mathcal {A} \to [0; \infty)$ - мера, если 
    \[A_1, ..., A_n \in \mathcal{A}, A_i\cap A_j = \varnothing, i\neq j: \quad \mu(\bigcup\limits_{n = 1}^\infty  A_n) = \sum\limits_{n = 1}^\infty \mu (A_n)\quad \text{счетная аддитивность}\]

    Мера конечная, если $\mu(\Omega) < \infty$

    Мера вероятностная, если $\mu (\Omega) = 1$
\end{definition}
\begin{definition}[Вероятностное пространство]
    Тройка $(\Omega, \mathcal{A}, P)$, где 
    \begin{enumerate}
        \item $\Omega$ - пространство элементарных событий;
        \item $\mathcal{A}$ - $\sigma-$алгебра подмножеств $\Omega$ (события);
        \item P - вероятностная счетно-аддитивная мера на $\mathcal{A}$ (вероятность);
        называется вероятностным пространством.
    \end{enumerate}
\end{definition}
Все элементарные исходы равновозможны
\begin{multicols*}{2}
    \begin{definition}[Классическая вероятность]
        Модель вероятностного пространства (A - событие)
        \begin{enumerate}
            \item $\Omega = \{\omega_1, \dots, \omega_n\}$ - конечное пространство
            \item $\mathcal{A}$ все подмножества $\Omega$
            \item $P(A) = \sum\limits_{\omega\in A}p_\omega = \frac{|A|}{|\Omega|}$
        \end{enumerate}
    \end{definition}
    \vfill\null\columnbreak
    \begin{definition}[Геометрическая вероятность]  
        $V\in \mathbb{R}^n$
        \begin{enumerate}
            \item $\Omega = V$
            \item $\mathcal{A}$ - борелевская $\sigma-$алгебра 
            (минимальная $\sigma-$алгебра, содержащая все компакты) подмножеств $V$
            \item $P(A) = \frac{\mu(A)}{\mu(V)}$
        \end{enumerate}
    \end{definition}
\end{multicols*}

\subsection{Некоторые следствия аксиоматики}
\begin{enumerate}
    \item \begin{axiom}[Аксиома непрерывности]
        Если $A_1\supset A_2, \dots, \supset A_n \supset \mathcal{A}, \bigcap\limits_{i = 1}^\infty A_i = \varnothing$, то
        \[\lim\limits_{n\to \infty} P(A_n) = 0\]
    \end{axiom}
    \begin{proof}
        Пусть $B_n\downarrow \varnothing$. Тогда обозначим $A_n = B_n \setminus B_{n+1}, n = 1, \dots, .$. $A_n$ попарно несовместны и 
        \[B_1  =    \sum\limits_{n = 1}^\infty A_n \quad B_n  =  \sum\limits_{k = n}^\infty A_k, \]
        поэтому из счетной аддитивности меры следует сходимость ряда \[P(B_1)=\sum\limits_{n = 1}^\infty P(A_n),\] и сумма остатка ряда \[P(B_n) =  \sum\limits_{k = n}^\infty P(A_k) = 0.\] 
    \end{proof}
    \item (Формула включений и исключений)
    \[P(\bigcup\limits_{k = 1}^\infty A_k)  = \sum_{k = 1}^n P(A_k) - \sum_{i < j}^n P(A_i \cap A_j) + \dots + {(-1)}^{n-1} P(A_1 \cap ... \cap A_n)\]
    \begin{proof}
        Выводится через обычную формулу включений и исключений для множеств по индукции \[P(A \cup B) = P(A) + P(B) - P(AB)\]+ 
        \[\begin{cases}
            A \cup B = A + (B\setminus AB) \\ 
            \text{Счетная аддитивность} \\ 
            P(B \setminus AB) = P(B) - P(AB) (\text{также по счетной аддитивности})
        \end{cases}
        \]
    \end{proof}
\end{enumerate}
\subsubsection{Индикатор}
\begin{definition}
    Индикатор события А - это функция $I_A(\omega) = \begin{cases}
        1,  & \omega \in A \\
        0,  & \omega \notin A
        \end{cases}$
\end{definition}
\paragraph{Свойства индикатора}
\begin{enumerate}
    \item $I_{\bar{A}} = 1 - I_A$
    \item $I_{A_1 \cap A_2} = I_{A_1}I_{A_2}$
    \item $I_{A_1\cup\dots\cup A_n} = 1 - I_{\bar{A_1} \cap \dots\cap \bar{A_n}} = 1  -  I_{\bar{A_1}}\dots I_{\bar{A_n}} = 1 - (1 - I_{A_1})\dots(1 - I_{A_n})$
\end{enumerate}
\section{Условные вероятности и независимость}
\begin{definition}[Условная вероятность]
    Пусть $P(B)>0$. Условной вероятностью $P(A|B)$ события А при условии, что произошло событие B (или просто: при условии B), назовем отношение
    $$P(A|B) = \frac{P(AB)}{P(B)}$$
    Применяется также обозначение $P_B(A)$
\end{definition}
\begin{theorem}[Теорема умножения]
    Пусть события $A_1, \dots, A_n$ таковы, что $P(A_1, \dots, A_{n-1)}>0$. Тогда $$P(A_1, \dots, A_n) = P(A_1)P_{A_1}(A_2)\dots P_{A_1, \dots, A_{n-1}}(A_n)$$
\end{theorem}
\begin{proof}
    Из условия теоремы вытекает, что существуют все условные вероятности из формулы.
    
    База индукции $P(AB) = P(B)P_B(A)$.

    Переход: $B=A_1, \dots, A_{n-1}, A = A_n$, применим формулу выше
\end{proof}
\begin{definition}[Разбиение]
    Систему событий $A_1, \dots, A_n$ будем называть конечным разбиением (в дальнейшем - просто разбиением),
    если они попарно несовместны и 
    \[A_1+\dots A_n = \Omega\]
\end{definition}
\begin{theorem}[Формула полной вероятности]
    Если $A_1, \dots, A_n$ - разбиение и все $P(A_k)>0$, то для любого события B имеет
    место формула
    \[P(B) = \sum_{k = 1}^n P(A_k)P(B|A_k)\]
\end{theorem}
\section{Случайные величины}
\begin{definition}[Случайная величина]
    Случайной величиной (СВ) $X(\omega)$ называется функция элементарного события $\omega$ с областью определения $\Omega$
и областью значений $\mathbb{R}$ такая, что событие $\{\omega : X(\omega) \leq x\}$ принадлежит $\sigma$ -алгебре $\mathcal{F}$ при любом действительном $x \in \mathbb{R}$ . Значения x функции $X(\omega)$ называются реализациями СВ $X(\omega)$.    
\end{definition}
\begin{definition}[Алгебра, порожденная случайной величиной]
    Пусть $x_1< \dots< x_k$ - значения, принимаемые случайной величиной $\xi$. Каждая такая
    величина определяет разбиение из событий $A_i = \{\omega : \xi(\omega) = x_i\}$.
    Т.к $x_i\neq x_j$, то $A_i A_j = \varnothing$. Сумма - достоверное событие $\Omega$.

    Разбиение порождает алгебру событий \[\{\xi \in B =\} = \{\omega: \xi (\omega) \in B\} \],
    B - числовое множество. 
\end{definition}
\begin{definition}[Закон распределения]
    Любое правило (таблица, функция), позволяющее находить вероятности всех возможных событий, связанных со случайной величиной.
\end{definition}

\paragraph*{Примеры законов распределения}
\begin{enumerate}
    \item Биномиальный закон
    \item Гипергеометрическое распределение
    \item Равномерное распределение
\end{enumerate}

\begin{definition}[Математическое ожидание]
    Математическое ожидание случайной величины $\xi = xi(\omega)$ обозначается $M\xi$ и определяется как сумма
    \[M\xi = \sum_{\omega \in \Omega} \xi(\omega) p(\omega)\]
\end{definition}
\paragraph*{Свойства мат. ожидания}
\begin{enumerate}
    \item $M I_A = P(A)$
    \begin{proof}
    \[M I_A = \sum_{\omega \in \Omega} I_A(\omega) p (\omega) = \sum_{\omega \in A} p(\omega)  = P(A)\]
    \end{proof}
    \item Аддитивность: $M(\xi + \eta) = M\xi + M\eta$
    \begin{proof}
        
    \end{proof}
    Из этого также следует конечная аддитивность.
    \item Для любой константы C \[M(C\xi) = cM\xi,\quad MC = C\]
    \item Математическое ожидание $\xi$ выражается через закон распределения случайной величины $\xi$ формулой
    \[M\xi = \sum_{i = 1}^k x_k P \{\xi = x_i\}\]
\end{enumerate}
Подставляя в числовую функцию случайную величину, мы также получаем случайную величину. Например, если $\eta = g(\xi)$, то \[M\eta  = M g(\xi)  = \sum_{i = 1}^k g(x_i) P\{\xi = x_i\}\]
При этом \[g(x_i) = \sum_{i = 1}^k g(x_i) I_{\xi = x_i}\]
\begin{definition}[n-ый момент случайной величины]
    Математическое ожидание $M\xi^n$ называется n-ым моментом (или моментом n-ого порядка) случайной величины $\xi$ (или ее закона распределения).
\end{definition}
\begin{definition}[Абсолютный n-ый момент]
    Математическое ожидание $M {|\xi|}^n$.
\end{definition}
\begin{definition}[Центральный момент n-ого порядка]
    $M(\xi-M\xi)^n$
\end{definition}
\begin{definition}[Абсолютный центральный момент n-ого порядка]
    $M|\xi-M\xi|^n$
\end{definition}
\begin{definition}[Дисперсия]
    $D\xi = M(\xi - M\xi)^2$
\end{definition}
\begin{definition}[Среднее квадратическое отклонение (стандартное отклонение)]
    $\sqrt(D\xi)$
\end{definition}
\paragraph*{Свойства дисперсии}
\begin{enumerate}
    \item $D\xi = M\xi^2 - (M\xi)^2$
    \item $D\xi\leq 0$ и $D\xi = 0$ тогда и только тогда, когда существует такая константа c, что $P\{\xi = c\} = 1$
    \item Для любой константы c $D(c\xi) = c^2 D\xi, \quad D(\xi +c) = D\xi$
\end{enumerate}
\begin{theorem}[Неравенство Иенсена]
    Если числовая функция $g(x)$, то для любой случайной величины $\xi$
    \[Mg(\xi)\leq g(M\xi)\]
\end{theorem}
\begin{theorem}[Неравенство Ляпунова]
    Для любых положительных $\alpha \leq \beta$
    \[(M|\xi|^\alpha)^{1/ \alpha}\leq (M|\xi|^\beta)^{1/ \beta}\]
\end{theorem}
\begin{theorem}[Неравенство Коши-Буняковского]
    
\end{theorem}
\paragraph*{Джентльменский набор}
\begin{enumerate}
    \item Равномерное дискретное распределение
    \[P\{\xi=k\} = \frac{1}{N}, \quad M\xi = \frac{1+N}{2}, \quad D\xi = \frac{N^2-1}{12}\]
    \item Биномиальное (распределение Бернулли)
    \[P\{n=k\}=C_n^k p^k (1-p)^{n-k}, \quad M\xi = np, \quad D\xi = np(1-p)\]
    \item Геометрическое распределение
    \[P\{n=k\}=(1-p)p^k, \quad M\xi = \frac{p}{1-p}, \quad D\xi = \frac{p}{(1-p)^2}\]
    \item Распределение Пуассона
    \[P\{n=k\}=\frac{\lambda^k}{k!}e^{-\lambda}, \quad M\xi = \lambda, \quad D\xi = \lambda\]
    
\end{enumerate} 
\subsection{Многомерные законы распределения}
\subsection{Независимость случайных величин}
\begin{definition}[Независимость случайных величин]
    $\xi_1, \dots, \xi_n$ называются независимыми, если порожденные ими алгебры \[\mathcal{A}_{\xi_1}, \dots, \mathcal{A}_{\xi_n}\] независимы.
\end{definition}
\begin{definition}[Независимость случайных величин]
    $\xi_1, \dots, \xi_n$ называются независимыми, если для любых $x_{1_{j_1}}\dots, x_{x_{j_n}}$
    \[P\{\xi_1 = x_{1_{j_1}}, \xi_n = x_{1_{j_n}}\} = \prod_{i = 1}^n P\{\xi_i = x_{1_{j_i}}\}\]
\end{definition}
\begin{theorem}
    Если случайные величины \(\xi_1, \dots \xi_n\) независимы, а \(g_i(x)\) - числовые функции, то случайные величины \(\eta_1 = g_1(\xi_1), \dots \eta_n = g_n(\xi_n) \) также независимы.
\end{theorem}
\begin{theorem}[Мультипликативное свойство математических ожиданий]
    Если случайные величины \(\xi_1, \dots \xi_n\) независимы, то 
    \[M \xi_1, \dots, \xi_n = \prod_{i = 1}^n M \xi_i\]
\end{theorem}
\begin{theorem}[Аддитивное свойство дисперсии]
    Если случайные величины \(\xi_1, \dots \xi_n\) независимы, то 
    \[D(\xi_1 + \cdots + \xi_n) = D\xi_1 + \cdots + D\xi_n\]
\end{theorem}
\paragraph*{Евклидово пространство случайных величин}
Условное мат. ожидание $M (\xi | \eta)$- ортогональная проекция $\xi$ на линейное подпространство $\eta$
\paragraph*{Условные математические ожидания}
\begin{definition}[Условная вероятность]
    Условная вероятность \(P(B|\mathscr{A}(\alpha))\) относительно \(\mathscr{A}(\alpha)\) как случайную величину, которая принимает значение \(P(B|A_k)\) при \(\omega \in A_k\).
\end{definition}
\begin{definition}[Условный закон распределения]
    Условный закон распределения $\eta $ при заданном значении $\xi = x_k$ назовем набор условных вероятностей
    \[P\{\eta = y_t | \xi = x_k\} = \frac{P(\eta = y_t, \xi = x_k)}{P(\xi = x_k)}, \quad t = 1, \dots, m\]
\end{definition}
\begin{definition}[Условное мат.ожидание]
    Условное мат.ожидание $\eta$ при заданном значении $\xi = x_k$
    \[M\{\eta | \xi = x_k\} = \sum_{t = 1}^{m} P\{\eta = y_t| \xi = x_k\} = \frac{\sum_{t = 1}^{m}y_t P(\eta = y_t, \xi = x_k)}{P(\xi = x_k)}\]
\end{definition}
Условное мат.ожидание является функцией от $\eta$. Случайная величина $M(\eta | \xi)$ - условное мат.ожидание при заданном $\xi$
\begin{definition}
    \[M[M(\eta | \xi)] = \sum_{k = 1}^{n} P\{\xi  = x_k\}M\{\eta | \xi = x_k\}\]
\end{definition}
\begin{theorem}
    \[M[M(\eta | \xi)] = M\eta\]
\end{theorem}
\begin{theorem}[Неравенство Чебышева]
    Для любого $x>0$ имеют место неравенства:
    \[P\{|\xi|\geq x\} \leq \frac{M|\xi|}{x}\]
    \[P\{|\xi - M\xi| \geq x\} \leq \frac{D \xi}{x^2}\]
\end{theorem}
\section{Случайные величины (общий случай)}
\begin{definition}
    Числовая функция $\xi = \xi (\omega)$ от элементарного события $\omega \in \Omega$ называется случайной величиной, если для любого числа x
    \[\{\xi \leq x\}=\{\omega : \xi(\omega)\leq x\} \in \mathscr{A}\]
\end{definition}
\begin{definition}[Функция распределения случайной величины $\xi$]
    \[F(x) = F_\xi (x) = P \{\xi \le x\}\], определенная при всех $x\in R$
\end{definition}
При помощи этой функции можно выразить вероятность попадания $\xi$ в интервалы.
\[P(x_1 < \xi \le x_2) = F(x_2) - F(x_1)\]
\[\{\xi < x\}: \quad \sum_{n=1}^{\infty} \{x - \frac{1}{n-1}<\xi \le x - \frac{1}{n}\} \]
\[P({\xi = x}) = F(x) - F(x - 0)\]
\[P({x_1 \le \xi \le x_2}) = F(x_2) - F(x_1 - 0)\]
\[P({x_1 < \xi < x_2}) = F(x_2 - 0 )  - F(x_1)\]
\[P (x_1 \le \xi < x_2) = F(x_2 - 0) - F(x_1 - 0)\]
\begin{theorem}[Свойства функции распределения]
    Функция распределения $F(x)$ обладает следующими свойствами:
    \begin{enumerate}
        \item F(x) не убывает
        \item F(x) непрерывна справа
        \item $F(+\infty) = 1$
        \item $F(-\infty) = 0$
    \end{enumerate}
\end{theorem}
\begin{definition}[Борелевская $\sigma$-алгебра]
    $\sigma$-алгебра $\mathcal{A}$ числовых множеств, порожденная всевозможными интервалами вида $x_1<x\leq x_2$, называется борелевской; множества A, входящие в $\mathcal{A}$, называются борелевскими.
\end{definition}
\begin{definition}[$\sigma-$алгебра, порожденная случайной величиной $\xi$]
    Совокупность $\xi^{-1}(B)$ для всех борелевских множеств борелевской алгебры.
\end{definition}
\paragraph*{Примеры дискретных распределений}
\begin{enumerate}
    \item Биномиальное
    \item Пуассоновское
    \item Геометрическое
\end{enumerate}
\begin{theorem}
    Если $\xi$ - случайная величина, а $g(x)$ - борелевская функция, то $\eta = g(\xi)$ есть случайная величина
\end{theorem}
\begin{definition}[Распределение вероятностей]
    $P_\xi(B)$, определенная для всех $B\in \mathscr{B}$, называется распределением вероятностей случайной величины $\xi$
\end{definition}
\begin{definition}[величина с дискретным распределением]
    величина имеет дискретное распределение, если в точках разрыва
    функции распределения вероятности таковы, что их сумма $\sum_{k = 1}^{\infty} p_k  = 1$

\end{definition}
% \begin{theorem}[Каратеодори]
    
% \end{theorem}
\begin{definition}[Плотность распределения]
    $p(x) = p_\xi(x)$ - плотность распределения случайной величины $\xi$, если для любых $x_1 < x_2$
    \[P\{x_1 < \xi < x_2\} = \int_{x_1}^{x_2}p_\xi (x)dx\]
\end{definition}
\paragraph*{Свойства}
\[p(x)\le 0, \int_{-\infty}^{\infty} p(x)dx = 1\]
\section{Математическое ожидание}
\begin{definition}[Простая случайная величина]
    Случайная величина простая, если она представима в виде
    \[\xi = \xi (\omega) = \sum_{j = 1}^m x_j I_{A_j}(\omega)\]
    где события $A_1, \dots, A_m$ составляют разбиение, т.е $A_i A_j = \varnothing$ при $i \neq j$ и $\sum_{j = 1}^m A_i = \Omega$
\end{definition}
\begin{definition}[Мат. ожидание простой случайной величины]
    \[M\xi = \sum_{j = 1}^m x_j P(A_j)\]
\end{definition}
\begin{definition}[Мат. ожидание неотрицательной случайной величины]
    \[M\xi = \lim_{n\to \infty} M\xi^n\]
\end{definition}
\begin{definition}[Мат. ожидание в общем случае]
    \[\xi = \xi^+ - \xi^-,\] где \(\xi^+ = \xi I_{\{\xi\geq 0\}}\), \(\xi^+ = |\xi| I_{\{\xi< 0\}}\)
\end{definition}
\paragraph*{Свойства мат. ожидания}
\begin{enumerate}
    \item Свойство линейности
    \item Свойство положительности
    \item Свойство конечности
\end{enumerate}
\paragraph*{Джентльменский набор абсолютно непрерывных распределений}
\begin{enumerate}
    \item Нормальное (гауссово распределение)
    \begin{definition}
        Случайная величина $\xi$ имеет нормальное распределение с параметрами $(a, \sigma)$, $-\infty < a < \infty, \sigma > 0$, если она имеет плотность
        \[p_\xi(x) = \frac1{\sqrt{2\pi}\sigma} e^{-\frac{(x-a)^2}{2\sigma^2}}\]
        Нормальное распределение с параметрами (0, 1) называется стандартным.
        \[p(x) = \frac1{\sqrt{2\pi}}e^{-\frac{x^2}2}\]
        Для плотности истинно условие
        \[\int_{-\infty}^{\infty}p(x) dx = \frac1{\sqrt{2\pi}} \int_{-\infty}^{\infty} e^{-\frac{x^2}2}dx = |t = \frac{x}{\sqrt{2}}|= \frac1{\sqrt{\pi}} \int_{-\infty}^{\infty}  e^{-t^2}dt = |\text{гауссов интеграл}| = 1\]
    \end{definition}
    \item Равномерное распределение
    \begin{definition}
        Случайная величина $\xi$ имеет равномерное распределение на отрезке [a, b] если ее плотность имеет вид:
        \[p_\xi (x) = \begin{cases}
            C & \text{при } a \le x \le b \\
            0 &\text{при } x < a \text{ или } x > b
        \end{cases}\]
        Так как 
        \[\int_{-\infty}^{\infty}p(x) dx = C \int_{a}^{b}dx = C(b-a) = 1,\]
        то $C = b - a.$
    \end{definition}
    \item Гамма-распределение 
    \begin{definition}[гамма распределение]
        \[p_\xi (x) = \begin{cases}
            0 & x< 0 \\
            \frac{\lambda^\alpha x^{\alpha-1}}{\Gamma(\alpha)}e^{-\lambda x} & x\ge 0,
        \end{cases}\]
        где $\alpha > 0, \lambda > 0$ - параметры
    \end{definition}
    При $\alpha = 1$ имеем показательное распределение
    \[p_\xi(x) = \begin{cases}
        0 & x< 0 \\
            \lambda e^{-\lambda x} & x\ge 0
    \end{cases}\]
    \[\int_{-\infty}^{\infty}p(x) dx  = \int_{0}^{\infty} \lambda e^{-\lambda x} dx = |-\lambda x = t, dt =  - \lambda dx| = -\int_{-\infty}^{0} \frac{e^t }{-\lambda} dt = e^t {|}_{-\infty}^0 = 1 - 0 = 1\]
\end{enumerate}
\begin{center}
    \begin{tabular}{ |c c c c| }
        \hline
        $p_\xi(x)$ & $F_\xi(x)$ & $M(\xi)$ & $D(\xi)$ \\ 
        \hline
        $\frac1{\sqrt{2\pi}\sigma} e^{-\frac{(x-a)^2}{2\sigma^2}}$ & $\frac12 [1 + erf(\frac{x-a}{\sqrt{2\sigma^2}})]$ & a  & $\sigma^2$\\ 
        
        ${\displaystyle\left\{{\begin{matrix}{\dfrac {1}{b-a}},&x\in [a,b]\\0,&x\not \in [a,b]\end{matrix}}\right..}$ &
        ${\displaystyle \left\{{\begin{matrix}0,&x<a\\{\dfrac {x-a}{b-a}},&a\leqslant x<b\\1,&x\geqslant b\end{matrix}}\right..}$ &
        $\frac{a+b}2$ & $\frac{{(b-a})^2}{12}$ \\


        $\displaystyle\left\{{\begin{matrix}x^{{\alpha-1}}{\frac  {e^{{-x\lambda }}}{\lambda ^{-\alpha}\,\Gamma (\alpha)}},&x\geq 0\\0,&x<0\end{matrix}}\right.$ & $\dots$ & $\alpha \lambda^{-1}$ & $\alpha \lambda^{-2}$ \\
        
     \hline
    \end{tabular}
    \end{center}
    \[\operatorname {erf}\,x={\frac  {2}{{\sqrt  {\pi }}}}\int \limits _{0}^{x}e^{{-t^{2}}}\,{\mathrm  d}t.\]
\subsection{Правила для вычисления}
\[M\xi = \int_{-\infty}^{\infty} x dF_\xi(x)\]
Для непрерывных случайных величин:
\[M\xi = \int_{-\infty}^{\infty} x p_\xi (x)dx\]
\[M g(\xi) = \int_{-\infty}^{\infty} g(x) p_\xi (x) dx\]
\section{Производящие функции}
\begin{definition}[Целочисленная случайная величина]
    Дискретная случайная величина $\xi$, принимающая только целые неотрицательные значения.

    Закон распределения: \[p_n = P\{\xi = n\}, n = 0, 1\dots, \quad \sum_{n = 0}^{\infty} p_n = 1\]
\end{definition}
\begin{definition}[Производящая функция]
    \[\phi_\xi(s) = M s^\xi = \sum_{n = 0}^{\infty} p_n s^n\]
    Ряд абсолютно сходится при $|s|\le 1$
\end{definition}
\paragraph*{Джентльменский набор}
\begin{enumerate}
    \item Равномерное дискретное распределение
    \[P\{\xi=k\} = \frac{1}{N}, \quad M\xi = \frac{1+N}{2}, \quad D\xi = \frac{N^2-1}{12}, \quad \phi(s) = \sum_{n=1}^{\infty} \frac{s^n}{n}=-\ln(1-s)\]
    \item Биномиальное (распределение Бернулли)
    \[P\{n=k\}=C_n^k p^k {(1-p)}^{n-k}, \quad M\xi = np, \quad D\xi = np(1-p), \quad \phi(s) = \sum_{m = 0}^{\infty} C_n^m p^m {(1-p)}^{n-m} = {(ps +1-p)}^n\]
    \item Геометрическое распределение
    \[P\{n=k\}=(1-p)p^k, \quad M\xi = \frac{p}{1-p}, \quad D\xi = \frac{p}{{(1-p)}^2}, \quad \phi(s) = \sum_{n=1}^{\infty}p^k (1-p) s^n =\frac{p}{1-(1-p)s} \]
    \item Распределение Пуассона
    \[P\{n=k\}=\frac{\lambda^k}{k!}e^{-\lambda}, \quad M\xi = \lambda, \quad D\xi = \lambda, \quad \phi(s) = \sum_{n = 0}^{\infty} \frac{\lambda^n s^n}{n!}e^{-\lambda}=e^{\lambda (s-1)}\]
\end{enumerate} 
\section{Характеристические функции}
129-137 
\[\xi(t) = \xi_1(t)+ i\xi_2(t)\]
\[|M\xi| \le M |\xi|\]
\begin{definition}
    Функция $f_\xi(t)$ называется характеристической функцией случайной величины $\xi$, если она имеет вид
    \[f_\xi(t)  = Me^{it\xi}\]
\end{definition}
Если $\xi$ - целочисленная случайная величина, то $\phi_\xi (z) = M z^\xi$
\[f_\xi (t) = M e^{it\xi} = M (e^{it})^\xi = \phi_\xi (e^{it})\]
(Свойства х.ф.)
\begin{enumerate}
    \item $|f_\xi(t)|\le 1, f_\xi(0) = 1$
    \item $f_\xi$ - равномерно непрерывна по t
    \item $f_{a\xi+ b}(t) = M e^{it(a\xi)}e^{itb} = e^{itb} M e^{i\xi(at)} = e^{itb} f_\xi(at)$
    \item $\xi_1, \dots, \xi_n$ - независимы, тогда
    \[f_{\xi_1+ \dots+ \xi_n}(t) = \prod_{i = 1}^n f_{\xi_i}(t)\]
    \[M e^{it (\xi_1+ \dots+ \xi_n)} = M \prod_{i = 1}^{n} e^{it \xi_i} = \prod_{j = 1}^{n} M e^{it \xi_j} = \prod_{j = 1}^n f_{\xi_j} (t)\]
    \item $f_\xi(-t) = \overline{f_\xi(t) }$
    \[M e^{-it\xi} = M \overline{ e^{it\xi}} = \overline{M e^{it\xi}}\]
    \item $\exists m_1, \dots, m_n = M \xi^n$ (существуют первые n моментов), тогда 
    \[f_\xi(t) = \sum_{k = 0}^{n}\frac{(it)^k}{k!}m_k + R_n(t),\]
    где $R_n(t) = o(t^n)$ при $t \to 0$
    \item \[\zeta  \begin{cases}
        \xi, с вероятностью p \\
        \eta, с вероятностью 1 - p,
        \end{cases}
        p \in (0, 1)\]
        
        \[f_\zeta (t)  = p f_\xi (t) + (1 - p) f_\eta(t)\]
\end{enumerate}
\begin{example}
    \begin{enumerate}
        \item $\cos(t)$

        \textsubscript{мы не знаем косинус\dots}
\[\xi = \begin{cases}
    -1 & p = 1/2 \\
    1 & p = 1/2
\end{cases}\quad \textsubscript{Бернуллиевская случайная величина}\]
$M e^{it\xi} = \frac12 e^{-it} = \frac12 e^{-it} = \cos t$
        \item $\cos^3(t)$

        свойство про независимость
        \item $\frac{\cos (t) + \cos(2t)}2$
        
        свойство про независимость, свойство про выпуклую комбинацию (3)
        \item $e^{-t^4}$
        
        шестое свойство, по формуле Тейлора
        \[e^{-t^4} = 1 - t^4 + o(t^4)\]

    \end{enumerate}
\end{example}
функция обращения - слишком тяжко, проверяем по свойствам а потом мучаемся (Фурье, Лаплас?)

$\xi = C $ с вероятностью 1
\[Me^{i\xi t} = e^{iCt}\]
\begin{remark}
    В силу 6 свойства, можно обобщить - если моменты до второго равны 0, то уже не характеристическая функция. (сравниваем с х.ф тождественного нуля, а $t^4$ высоко)
\end{remark}
\begin{enumerate}
    \item Стандартное распределение
    \[f_\xi(t) = e^{-t^2/2}\]
    можно получить нормальное при помощи 3 свойства.
    \item равномерное на [a, b]:
    \[f_\xi = \frac{e^{itb} - e^{ita}}{it(b-a)}\]
    \item Гамма распределение с параметром $\alpha$
    \[f_\xi (t) = (1-it)^{-\alpha}\]
    \textsubscript{многозначная функция - нужно выделять ветвь}
\end{enumerate}
\paragraph*{Абсолютно непрерывный случай}
\[f_\xi(t) = \int_{-\infty}^{\infty} e^{itx}p_\xi(x)dx \quad \textsubscript{Преобразование Фурье}\]
\[p_\xi(x) = \frac1{2x} \int_{-\infty}^{\infty} e^{-itx}f_{\xi}(t) dt\quad \textsubscript{Обратное преобразование Фурье}\]
\end{document}