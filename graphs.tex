\documentclass[a4paper]{article}
\usepackage{cmap}
\usepackage[utf8]{inputenc}
\usepackage[T2A]{fontenc}
\usepackage{amsfonts}
\usepackage{amsmath, amsthm}
\usepackage{amssymb}
\usepackage{hyperref}
\usepackage{multicol}
\usepackage{tikz} 

\newcommand\letsymbol{\mathord{\sqsupset}}
\usepackage[russian]{babel}
\renewcommand\qedsymbol{$\blacktriangleright$}
\newtheorem{theorem}{Теорема}[section]
\newtheorem{lemma}{Лемма}[section]
\theoremstyle{definition}
\newtheorem*{example}{Пример}
\newtheorem*{definition}{Определение}
\newtheorem*{statement}{Утверждение}
\theoremstyle{remark}
\newtheorem*{remark}{Замечание}
\newtheorem*{corollary}{Следствие}

\setlength{\topmargin}{-0.5in}
\setlength{\oddsidemargin}{-0.5in}
\textwidth 185mm
\textheight 250mm

\begin{document}
    \tableofcontents
    \pagenumbering{arabic}
    \setcounter{page}{1}
    \section{Определение графа. Примеры графов. Степени вершин графа. Лемма о рукопожатиях}
    \begin{definition}
        Граф (неориентированный) состоит из непустого конечно множества V и конечного множества
        E неупорядоченных пар элементов из V (записывается G =  (V, E)).
    \end{definition}
    Элементы множества $V = V_G$ называются \textbf{вершинами}, а элементы множества $E = E_G$ - \textbf{ребрами} графа G.
    Те и другие называются \textbf{элементами} графа.
    \begin{definition}
        Если $\{u, v\}\in E$, то будем записывать $e = uv$ и говорить, что 
        вершины u и v смежны, а вершина u и ребро e инцидентны (так же, как вершина
        v и ребро e). Два ребра называются смежными, если они имеют общую вершину.
    \end{definition}
    \begin{definition}
        Степенью вершины v в графе G называется число ребер, инцидентных вершине v (обозначается $d_G(v) = d(v)$).
    \end{definition}
    Вершина степени 0 - изолированная, вершина степени 1 - висячая.
    Минимальная и максимальная степени вершин графа G обозначаются $\delta(G), \Delta(G)$.

    Последовательность степеней вершин графа G, выписанных в порядке неубывания называется степенной
    последовательностью или вектором степеней графа G.
    \begin{definition}
        Кратные ребра - два и более ребра, соединяющие одну и ту же пару вершин.
    \end{definition}
    \begin{definition}
        Петли - ребра, соединяющие вершины сами с собой.
    \end{definition}
    \begin{definition}
        Мультиграф - граф с кратными ребрами
    \end{definition}
    \begin{definition}
        Обыкновенный граф - граф без петель и кратных ребер.
    \end{definition}
    Примеры графов:
    \begin{enumerate}
        \item Граф G = (V, E) с n вершинами и m ребрами называется (n, m)-графом,
        (1, 0)-граф называется тривиальным.
        \item Пустой граф - $O_n$
        \item Полный граф $K_n, C_n^2 = \frac{n(n-1)}{2}$
        \item Двудольный граф $G = (V_1, V_2; E)$
        \item Полный двудольный граф - $K_{p,q}$
        \item Звезда - полный двудольный граф $K_{1,q}$
        \item Простой цикл $C_n$
        \item Регулярный (однородный) граф - граф, все вершины которого имеют
        одну и ту же степень. 
        Кубические графы - 3-регулярные
        \item Графы многогранников 
    \end{enumerate}
    \begin{lemma}[О рукопожатиях]
        Сумма степеней всех вершин произвольного графа $G = (V, E)$ - четное число, равное удвоенному
        числу его ребер:
        $\sum_{v\in V} d_G(v) = 2|E|$
    \end{lemma}
    \begin{proof}
        Индукция по числу ребер.\\
        База: если в графе G нет ребер, то $\sum_{v\in V} d_G(v) = 0$.
        Предположим, что формула верна для любого графа, число ребер в котором не превосходит $m\leq 0$.\\
        Пусть $|E| = m+1$. Рассмотри произвольное ребро $e = uv\in E$ и удалим его из графа G. Получим граф
        $G' = (V, E'), |E'| = m$. По предположению индукции $\sum_{v\in V} d_{G'}(v) = 2 |E'| = 2m$\\
        Тогда $\sum_{v\in V} d_{G}(v) = \sum_{v\in V} d_G(v) + 2 = 2m + 2 = 2|E|$.
    \end{proof}
    Теорема имеет место быть и для мультиграфов.
    \begin{corollary}
        В любом графе число вершин нечетной степени четно.
    \end{corollary}
    \section{Маршруты, цепи, циклы. Лемма о выделении простой цепи. Лемма об объединении 
    простых цепей}
    \section{Эйлеровы графы. Критерий существования эйлерова цикла (теорема Эйлера)}
    \section{Гамильтоновы графы. Достаточные условия существования гамильтонова цикла (теоремы 
    Оре и Дирака)}
    \section{Изоморфизм графов. Помеченные и непомеченные графы. Теорема о числе помеченных 
    n-вершинных графов}
    \section{Проблема изоморфизма. Инварианты графа. Примеры инвариантов. Пример полного 
    инварианта}
    \section{Связные и несвязные графы. Лемма об удалении ребра. Оценки числа ребер связного 
    графа}
    \section{Плоские и планарные графы. Графы Куратовского. Формула Эйлера для плоских графов}
    \subsection*{Графы Куратовского}
    \begin{remark}
        Графы $K_{3,3}$ и $K_5$ непланарны
    \end{remark}
    \begin{proof}
        $K_{3, 2}$ - плоский, в нем по формуле Эйлера 3 грани независимо от способа изображения.
        Пытаемся добавить 6 вершину, подставляя ее в каждую грань, получаем каждый раз противоречие - 
        невозможность соединить вершину с необходимыми.
        Аналогично для $K_5$.
    \end{proof}
    \begin{theorem}[Формула Эйлера для плоских графов]
        Для любого связного плоского графа G = (V, E) верно $n - m + l = 2$, где n = |V|, m = |E|,
        l - число граней 
    \end{theorem}
    \begin{proof}
        Рассмотрим две операции перехода от связного плоского графа G к его связному 
        плоскому подграфу, не изменяющие величины $n - m + l$
        \begin{enumerate}
            \item удаление ребра, принадлежащего сразу 2 граням (одна из которых может быть внешней) \textbf{уменьшает m и l на 1}
            \item удаление висячей вершины (вместе с инцидентным ребром) \textbf{уменьшает m и n на 1}
        \end{enumerate}
        Очевидно, что любой связный граф после этих операций может быть приведен к тривиальному, а для него формула верна $\implies$
        верна и для данного 
    \end{proof}
    \section{Деревья. Теорема о деревьях (критерии)}
    \begin{theorem}[о деревьях №1]
        Для (n, m)-графа G следующие определения эквивалентны:
        \begin{enumerate}
            \item G - дерево
            \item G - связный граф и $m = n - 1$
            \item G - ациклический граф и $m = n - 1$
        \end{enumerate}
    \end{theorem}
    \begin{proof}
        \begin{itemize}
            \item[$1 \to 2$] Дерево - связный, планарный граф (имеет 1 грань) $\implies$ $n - m + 1 = 2\implies m = n - 1$
            \item[$2 \to 3$] Пусть граф не ациклический $\implies$ есть цикл и e - циклическое ребро.
            Тогда по лемме об удалении ребра граф $G - e$ также связен и имеет m - 1 = n - 2 ребер
            $\implies$ противоречие оценке числа ребер связного графа $\implies$ граф ациклический
            \item[$3 \to 1$] Обозначим число компонент связности - k. Пусть $T_i$ - iтая компонента,
            является $(n_i, m_i)$-графом. Т.к $T_i$ - дерево, то по ранее доказанному ($1 \to 2$) $m_i = n_i - 1, i = \overline{1, k}$
            $\implies n - 1 = m = \sum_{i = 1}^k m_i = \sum_{i = 1}^k n_i - k = n - k \implies k = 1\implies$ граф связный 
        \end{itemize}
    \end{proof}
    \begin{theorem}[о деревьях №2]
        Для (n, m)-графа G следующие определения эквивалентны:
        \begin{enumerate}
            \item G - дерево
            \item G - ациклический граф и если $\forall$ пару несмежных вершин соединить ребром, то
            полученный граф будет содержать ровно 1 цикл
            \item $\forall$ 2 вершины графа G соединены единственной простой цепью
        \end{enumerate}
    \end{theorem}
    \begin{proof}
        \begin{itemize}
            \item[$1\to 2$] В связном графе все несмежные вершины соединены простой цепью \textbf{(по лемме о выделении простой цепи)}
            $\implies$ добавление ребра e = uv приведет к образованию цикла, а два цикла образоваться не может в силу свойства циклов
            \item[$2 \to 3$] любые две несмежные вершины u,v графа G соединимы, иначе при добавлении ребра uv не появится цикл$\implies$
            в силу леммы о выделении простой цепи любые две вершины соединены простой цепью. А она единственная, иначе по лемме об 
            объединении простых цепей в графе G был бы цикл.
            \item[$3\to1$]из условия следует, что граф связен, а существование цикла противоречит условию единственности цепи$\implies$
            граф ациклический.
        \end{itemize}
    \end{proof}
    \begin{lemma}[О листьях дерева]
        В любом нетривиальном дереве имеется не менее двух листьев
    \end{lemma}
    \begin{proof}
        $\forall v\in V$ $d(v)\geq 1$

        $\sum_{v\in V} = 2|E|=2m=2(n-1)=2n-2$

        Если 2 листа - то у 2 вершин степень 1 и у остальных n-2 как минимум 2,
        а для меньшего количества листьев оценка суммы неверна

        $\sum_{v\in V} \leq 2 + (n-2)2 = 2n - 2$
    \end{proof}
    \section{Перечисление деревьев. Теорема Кэли о числе помеченных n-вершинных деревьев}
    \section{Центр дерева. Центральные и бицентральные деревья. Теорема Жордана}
    \section{Изоморфизм деревьев. Процедура кортежирования. Теорема Эдмондса}
    \section{Вершинная и реберная связность графа. Основное неравенство связности}
    \section{Отделимость и соединимость. Теорема Менгера}
    \section{Реберный вариант теоремы Менгера}
    \section{Критерии вершинной и реберной k-связности графа (без доказательства)}
    \section{Ориентированные графы. Основные понятия. Ормаршруты и полумаршруты. 
    Ориентированые аналоги теоремы Менгера}
    \section{Ориентированные графы. Достижимость и связность. Три типа связности. Критерии 
    сильной, односторонней и слабой связности орграфа}
    \section{ Основные структуры данных для представления графов в памяти компьютера. Их 
    достоинства и недостатки}
    \section{Влияние структур данных на трудоемкость алгоритмов (на примере алгоритма 
    отыскания эйлерова цикла)}
    \section{Задача о минимальном остовном дереве. Алгоритм Прима}
    \section{Задача о кратчайших путях. Случай неотрицательных весов дуг. Алгоритм Дейкстры}
    \section{Потоки в сетях. Увеличивающие пути. Лемма об увеличении потока}
    \section{Алгоритм Эдмондса-Карпа построения максимального потока}
    \section{Разрезы. Лемма о потоках и разрезах. Следствие}
    \section{Теорема Форда-Фалкерсона}
    \section{Два критерия максимальности потока.}
    \section{Приложения теории потоков в сетях. Задачи анализа структурно-надежных 
    коммуникационных сетей}
    \section{Задачи комбинаторной оптимизации. Массовая и индивидуальная задачи. 
    Трудоемкость алгоритма. Полиномиальные и экспоненциальные алгоритмы}
    \section{Задачи распознавания свойств. Детерминированные и недетерминированные 
    алгоритмы. Классы P и NP. Проблема “P vs NP”}
    \section{Полиномиальная сводимость задач распознавания. Свойства полиномиальной 
    сводимости}
    \section{NP-полные задачи распознавания. Теорема о сложности NP-полных задач. Примеры 
    NP-полных задач}
\end{document}